\documentclass[UTF8]{ctexart}
\usepackage[perpage]{footmisc}
\usepackage{geometry}
\usepackage[bookmarksnumbered=true]{hyperref}
\CTEXoptions[today=old]
\CTEXsetup[name={第,章},number={\chinese{section}}]{section}
\geometry{a4paper,left=1.27cm,right=1.27cm,top=1.27cm,bottom=1.27cm}
\hypersetup{
colorlinks=true,
linkcolor=black
}
\pagestyle{plain}
\title{吹响吧!上低音号\\ \Large{第三卷\,最大的危机}}

\date{April 18, 2015}
\author{武田綾乃}
\begin{document}
    \maketitle
    \section*{简介}
    即使肩负来自父母的压力、友谊的苦恼,

    还是忘不了我们夏天在宇治的约定:

    要是能和妳一起努力就好了!

    北宇治高中管乐社终于要进军全国大赛,却在眼看距离比赛只剩不到三周的时间,传出重要成员必须退社的震撼消息。

    历经京都大赛、关西大赛,北宇治高中管乐社即将踏上梦想中的最高舞台。就在众人奋力不懈、努力练习时,明日香却被迫退出社团。身为管乐社的灵魂人物,明日香总是促使大家前进的动力。失去明日香的管乐社,仿佛变成了一盘散沙,不知该何去何从……

    父母的决定就是对的吗?孩子是否有选择自己人生的权利?进军全国大赛是一生难得的机会,为什么父母却不能全力支持?这群青春的高中生,要如何在家人与梦想之间取得平衡?

    久美子在努力跟上课业进度与准备全国比赛的紧凑日子里,还身负「带回明日香学姊大作战」的重责大任。她不仅撞见了老师的过去、也发现了明日香的秘密,而久美子期待已久的爱情,好像也在这个紧张的时刻翩然降临。

    在这初秋的日子里,她们一定要携手站在梦想的舞台上,吹响自己的未来! 

    \tableofcontents
    \setcounter{secnumdepth}{-2}
    \section{前言}
    年幼的明日香眼前摆着一个黑色大盒子,皮革表面闪烁着鲜艳光泽,撩拨着她的好奇心。明日香抚摸细致的提把,检查寄件人的名字:「进藤正和」。 

    明日香听过这个名字,但不记得他的长相。她把脸贴在有些陈旧的盒子上,表面冰冰凉凉的,让明日香火热的皮肤冷却下来。 

    「……先生。」 

    没有人回应明日香的呼唤。妈妈去上班了,这个狭小的空间里只剩下自己。明日香缓缓站起,往快递刚才送来的纸箱探头探脑。大大的纸箱里只有两样东西,不知道里面是什么的黑色盒子和一本老旧的笔记本。 

    明日香悄悄拿起笔记本,旧纸的味道扑鼻而来,是时间被太阳晒黄的味道。 

    「可以打开吗?」 

    明日香自言自语,又是一个没有人回答的问题。自己的声音回荡在狭小空间里。她心跳得好快。不知怎地,紧贴在耳膜上的寂静令她喘不过气。明日香咽了咽口水,小心翼翼伸手探向盒子上的锁头,小巧的手指滑过银色扣环。喀嚓!清脆的声音在房里显得异常响亮。明日香屏住呼吸,慢慢掀起上盖。一把她从未看过的乐器正静静躺在覆有软毛的盒子里。她注视着乐器的银色表面,只见自己的脸往左右延伸,正困惑地回看着自己。 

    「这是什么?」 

    她先戳戳比较粗的部分,食指指纹清楚浮现在光滑的银白色表面,明日香感觉好像做了什么坏事,赶紧用手指擦了擦乐器,不擦还好,擦了之后,反而好像更脏了。明日香不知该怎么办,不敢再碰乐器。 

    里头还有什么?明日香仔细查看盒子里面,发现有个银色零件塞在左边角落。她拔出来一看,仿佛是流星倒过来的形状,形状好特别。 

    「好奇怪噢!」 

    她敲敲看,没有声音。这到底是用来做什么的?明日香的脑子里充满问号,摸索着盒子其他部分。右边有个貌似口袋的构造,她掀起上盖,里面有一张纸,看来是乐器行的广告传单,上头印有琳琅满目的乐器照片,其中有她也认识的小号及长笛。明日香把传单摊开在地板上,食指在每种乐器上游移,来到最底下的狭小栏位时,终于看到那个乐器。 

    「啊,是这个。」 

    手指像被吸住般停在纸上。黑白照片里的乐器,显然跟明日香眼前的一样。她小声地念出印得超小字的乐器名称。 

    「……上低音号。」 

    那一瞬间,明日香第一次知道上低音号这个乐器的存在。   

    \setcounter{secnumdepth}{3}
    \section{吹响吧!小号}
    白色指挥棒在空中勾勒出优美线条,泷修长的手指行云流水地挥舞,社员配合他的动作,一同开始演奏。金色的号口震动,吐出发自丹田的低音。久美子的指尖在乐谱上移动,视线再度回到指挥者身上。黑色的小蝌蚪在五线谱上游来游去。 

    「大家好!接下来是管乐社的演奏时间。」 

    在明日香一声令下,北宇治高中文化祭的管乐社演奏节目开始了。每年体育馆的舞台表演皆以管乐社的演奏画下句点。这时候全校学生都会集合在体育馆,校长会在管乐社结束演奏后,开始致辞。管乐社分配到的时间是三十分钟,文化祭的选曲是三年级的特权,整个夏天学长姐指定了许多乐谱。 

    田中明日香的声音透过麦克风传来。三年级的她是管乐社副社长,同时也是久美子所属低音组的组长,负责吹奏上低音号。身材修长高挑,是很适合用长相标致、眉清目秀等成语来形容的学姐,只可惜个性不太好。 

    「时间虽短,还请各位欣赏我们的演奏。」 

    这句话让会场响起了掌声。她的声音十分清亮,听起来很悦耳。要是能改掉平常像机关枪一样的说话方式,明明就是个正常的漂亮学姐。久美子抱着上低音号,事不关己地想着。 

    「明日香学姐!」明日香对台下传来的欢呼会有何反应呢?北宇治高中管乐社共有八十一位社员,不可能让所有人都站上体育馆狭小的舞台,因此社员分成两部分,分别在舞台上下各就各位。久美子抱着乐器,坐在舞台下的后方,除非她回头,否则看不到台上的动静。她只能直直盯着指挥者泷的方向,因为只要有一个人的动作不自然,就会很突兀。 

    「谢天谢地,我们管乐社今年可以参加全国大赛了。」 

    「恭喜!」「加油!」此起彼落的加油声,令社员的表情在不知不觉中变得柔和。心里痒痒的,不知该如何是好。在加油声静止前,久美子只能拼命忍住别笑得太忘形。 

    全日本管乐大赛。 

    这场比赛是国内最大规模、最具传统的音乐盛事,全国管乐社员都会齐聚一堂。评审将分别给予角逐的团体金奖、银奖、铜奖的名次,各个团体得先在地方预赛、都道府县复赛胜出,才能参加全国大赛。 

    京都府立北宇治高中曾经是管乐强校,但自从当时的顾问调到别的学校以后,实力就一口气衰退,最近十年皆不曾留下足以说嘴的成绩。然而,拜今年新任顾问泷升的指导所赐,实力逐渐提升,两周前,北宇治高中在八月二十五日举行的关西管乐大赛中,顺利抢下前往全国大赛的门票。 

    等到会场安静下来,明日香再度开口:「我们之所以能努力走到这一步,都要感谢各位老师及家长,以及把教室让出来给我们练习的各位同学。为了表达平常对各位的感激之意,我们今天会尽全力演奏。」 

    这句话让会场内响起温暖的掌声。久美子每次看到一群人一起拍手的场面,都会很感动。她深深吸气,下意识轻轻按压活塞,指腹的触感每次都让她觉得任重道远。 

    「那么,接下来继续为各位演奏。首先为各位介绍刚才演奏的开场曲〈搭乘A号列车〉,是很有名的爵士曲风。第二首曲子是管乐社员都很熟悉的〈唱,唱,唱〉,这首曲子也曾用于电影中,大家肯定都听过,欢迎配合演奏打拍子同乐。」 

    明日香的说明又引来一阵掌声。久美子翻开乐谱,望向泷。社员全都抬起头,迫不及待地等着泷的指示。久美子觉得所有人的视线都朝着相同方向的这一幕,非常珍贵。 

    大部分社员皆已在恢复成上课状态的音乐室里就座,藏青色的水手服聚集在教室中央。挂在墙上的相框里,装饰着连续参加关西大赛、获得全国大赛金奖等褪色照片。入社当时欠缺保养、布满尘埃的相框,曾几何时也擦得亮晶晶的。 

    「啊,大家都到齐了吗?」 

    社长小笠原从前面发下讲义。她负责吹奏上低音萨克斯风,过于温柔的性格其实不太能胜任必须带着大家往前冲的社长一职,听说是因为明日香不愿当社长,才由她接下社长一职。大家恐怕是看准了小笠原好欺负,就把工作都推到她头上,要她当社长。 

    泷顾问默默站在社长旁边。最近社团内部开会时,都是由社长负责主持,泷不太发言。他刚上任时,会做出巨细靡遗的指示,但是随着时光流逝,社员已经对各自扮演的角色有所自觉,不需要泷的指示,也能自动自发完成分内工作。久美子边往后传讲义,边默默观察四周。或许是刚结束文化祭的演奏,大家脸上还残留着些许疲惫的神色。 

    「今天各位辛苦了。」小笠原说道。 

    「大家辛苦了。」众人异口同声地回答。 

    「虽然才刚演奏完,但现在才四点,所以今天就跟平常一样,要继续留下来分组练习。不过,我想刚演奏完还是很累,所以今天会比平常早一点解散。还有,刚才发下去的是这个月的行事历,请仔细确认。要请假、会迟到的话,请事先通知各声部的组长。」 

    「是。」 

    「九月底将举行车站大楼音乐会,B部门的人请以那边的练习为主。行事历上打两个圈的日期是那场音乐会的练习时间,B部门的人完成当天的基础练习后,请直接加入合奏。」 

    「是。」 

    「还有,这是最高机密……」小笠原压低嗓门。这句话让大家开始窃窃私语。只见她的视线落在纸面上,有些兴奋地接着说:「事实上,不只北宇治,清良女中也会参加车站大楼音乐会。」 

    清良女中。这所学校太有名了,大家都忘了还在开会,七嘴八舌地表达感想。坐在隔壁的绿辉手舞足蹈地高呼万岁。 

    清良女中是位于福冈县的私立高中,校内的管乐社是全国大赛的常胜军,实力超级坚强。其所演奏的CD上市时,还能挤进排行榜前几名,定期演奏会的门票也总是开卖当天就抢购一空。久美子心想,能厉害到这种地步,说是专业的演奏家也不为过。 

    「为什么又来京都?是为了这场演奏会,大老远从九州过来吗?」三年级吹萨克斯风的学姐问道。 

    小笠原摇摇头回答:「不是那样的,好像是被邀请来当京都府某庆祝仪式的特别来宾。是主办单位拜托她们顺便让我们参加演奏会,而对方也答应了。」 

    「可以现场听到清良女中的演奏,真是太棒了!」 

    强校迷绿辉兴奋地握紧拳头。川岛绿辉是毕业自强校圣女中学的一年级生,个子娇小,却负责拉低音大提琴。她对自己的名字感到很自卑,所以都要别人喊她小绿。 

    「有这么厉害吗?」 

    坐在绿辉前面的叶月转过头来问道。加藤叶月国中时是网球社,高中才加入管乐社。起初想吹小号,但名额已满,只好转而负责低音号。刚拿到乐器时,连声音都吹不出来,如今已经能灵活吹出各式各样的曲子。 

    叶月和绿辉及久美子同班,都是一年三班的学生。级任老师松本美知惠也是管乐社的副顾问,以非常严厉的性格著称,还因此被学生取了军曹老师的绰号。 

    「前阵子才上过电视不是吗?明明都是女生,演奏出来的声音却很宏亮,是赫赫有名的学校。」 

    「听得不是很懂,只知道非常厉害。」 

    叶月对绿辉的说明频频点头。明日香在吵吵闹闹的音乐教室前方,轻轻拍了一下手。 

    「好了,安静下来。不好意思打断各位的热烈讨论,别忘了现在还在开会。」 

    这句话让交头接耳的社员连忙转向前面,绿辉也捂住自己的嘴巴。确定教室恢复安静后,明日香朝小笠原一瞥。小笠原看了讲义一眼,回头看着泷。 

    「老师,请问还有什么事要交代吗?」 

    「啊,对了。」 

    泷若有所思地摩挲着下巴,脸上浮现出一如往常的温和笑容。 

    「清良女中是非常优秀的学校,能现场听到她们的演奏,对各位而言,肯定是很珍贵的体验。不要只是听过就算了,请抱着对接下来的演奏会有帮助的心情仔细欣赏。」 

    「是!」 

    大家精神抖擞的回答令泷有些惊讶地张大了眼睛。或许是受到学生兴奋之情的影响,音乐教室里的空气带了点热度。关西大赛结束后,大家对音乐的执着又多了几分。久美子刚进社团时,根本无法想象会看到社员随时抓紧时间练习的身影。 

    久美子认为泷是很优秀的指导者,只花了不到半年的时间,就让腐败的社团改头换面。自从他当上顾问,北宇治高中管乐社几乎没有假日可言,但就连起初对此表示不满的学长姐,至此也不再有任何人抗议。因为他真的实现了进军全国大赛这个有如白日梦的目标。 

    泷具有强烈的明星特质,让人觉得相信他准没错。可是,久美子有时会想,泷为何要对他们鞠躬尽瘁到这个地步?每个假日都来加班,不惜扮黑脸也要指导他们,究竟是为了什么? 

    久美子将吹嘴抵住嘴唇,偷偷看了泷一眼。他温柔的眼神给人柔和的印象,但大家都很清楚,只要一扯到音乐,他的眼神就会变得非常锐利。泷为什么要玩音乐?又为什么要当顾问?久美子还没机会提出这些疑问,会有可以问的一天吗?久美子边想边挺直弯腰驼背的身体。 

    「谢谢老师。」站在前面的小笠原向泷道谢。「不客气。」泷微笑回答。 

    「那么,请各位努力演奏出不比清良女中逊色的水准。距离车站大楼音乐会只剩不到一个月的时间,请利用准备全国大赛的空档好好练习。」 

    「是。」 

    众人中气十足地回应小笠原的叮咛。久美子凝视手中的讲义。暑假结束了,学校下学期的课程也开始了。为了迎战全国大赛,几乎被社团活动填满的生活,也必须再度面对课业的存在,前阵子只要考虑社团活动的日子已经结束了。 

    「那么今天就讨论到这里,接下来开始分组练习。各声部的组长请留下来开组长会议,其他的社员请各自练习。」 

    「是。」 

    听从小笠原的指示,社员各自移动到自己的分部练习教室。久美子看了一遍发下来的讲义,大大叹了一口气。 

    「车站大楼音乐会啊……」 

    比起车站大楼音乐会,久美子更想专心于全国大赛的练习,她会有这种想法是因为只想到自己吗? 

    「A部门的人如果还要练一般曲子,不会觉得很着急吗?不会想赶快进行全国大赛的练习吗?」 

    夏纪看着车站大楼音乐会用的乐谱,侧着头不可思议地问道。平常使用的分组练习教室因为还要处理文化祭的善后工作,不能使用,今天低音组的分组练习只好破例在和音乐教室有段距离的烹饪实习室进行。大概是哪个班级在这里烤过摆摊用的饼干,狭小的教室里弥漫着甜腻的味道。 

    「嗯……小绿倒不怎么着急,因为表演时很开心!」 

    绿辉天真无邪地回答。小巧的手里握着低音大提琴的弓,上头紧绷的平滑白毛以马尾制成。 

    「我只要能跟大家一起表演就很开心了,因为人家不能参加A部门的演奏。」 

    叶月露出雪白的牙齿咧嘴一笑,低音号放在大腿上。「说的也是。」一旁的梨子有些不知所措地低眉敛眼。 

    二年级的梨子和夏纪是性格相反的学姐。个性温和的梨子负责低音号,比较有攻击性的夏纪和久美子一样,都是上低音号。 

    久美子边用指尖描摹乐谱夹的边缘,边望向夏纪。 

    管乐比赛只有编制比较大的部门,也就是所谓的A部门才会举行全国大赛。然而,举行比赛的部门可不只有A部门,还有以分部或都道府县为单位、小学部门、小编制部门、联合部门等也都会举行比赛,几乎所有的管乐团体只要有意愿都能参加比赛。人数众多的学校管乐社通常会拆成好几个单位参加比赛,北宇治高中每年也都分成A部门和B部门参加比赛。 

    今年北宇治高中以选拔方式选出A部门的成员,导致低音组一年级的叶月和二年级的夏纪被编到B部门。B部门只能比到京都大赛,因此比赛结束后,B部门的成员只能练习演奏会的曲目或协助A部门的成员。 

    「……比赛和演奏会两边都很重要。」 

    负责吹低音号的卓也一脸正色地说。二年级的副组长卓也是低音组唯一的男社员,沉默寡言、正经八百,与饶舌的明日香正好相反。 

    「啊,是是是,知道了啦!」 

    夏纪缩着脑袋说。久美子又朝吹嘴吹进一口气。以她个人的喜好来说,自己喜欢演奏会更胜于比赛,因为可以吹奏各式各样的曲子,观众的反应也很热烈。但是如果问她对何者投入比较多的感情,当然是比赛而不是演奏会。因为比赛的结果将得到金奖、银奖、铜奖的明确评价。 

    「很好!就是这股气势!」 

    这句话随着嘎啦嘎啦的开门声飘进教室里,全体组员一起望向声音的主人。只见明日香抱着银色的上低音号,急惊风似地走进教室。 

    「啊,学姐,组长会议辛苦了。」 

    绿辉说道,冲着明日香一笑。她带点浅棕色、宛如猫毛的头发随着动作轻轻晃动。明日香把上低音号放在讲桌角落,瞥了绿辉一眼,红框眼镜的镜框闪过一道凌厉光芒。 

    「关西大赛才刚结束,就马上想着比赛的事,真令人感动。虽然比赛前还有车站大楼音乐会,但也不能忘了全国大赛的事,必须牢牢锁定真正的目标。」 

    「因为我们学校的社团居然能参加全国大赛,简直跟奇迹没两样呢!」 

    「嗯,嗯。」夏纪点头如捣蒜。「就是说呀!」明日香大表赞同。 

    「居然是我们获选为关西代表,而不是三强之一的秀大附中,真的很幸运。因为打进关西大赛的二十三所高中里,只有三所学校能被选为关西分部的代表。往年都是明静工业高中、大阪东照高中、秀塔大学附属高中所谓的『三强』代表关西进军全国。」 

    「我们的努力终于开花结果了。」 

    夏纪对兴奋到握紧拳头的绿辉嗤之以鼻。 

    「当然也是因为我们拼命练习,但最大的关键还是秀大附中的独奏失误吧!要是没有那个失误,肯定还是他们会去全国大赛。」 

    「评审委员的给分好像非常接近。」 

    梨子伤脑筋地搔了搔脸颊,侧脸倒映在大大的金色号口上。卓也在一旁皱眉,大概是对这番讨论不太服气,有些不满地嘀咕:「……可是,进军全国的还是北宇治,不是秀大附中。」 

    这句话让明日香的表情顿时散发光彩,发出咯咯咯的愉快笑声,以「就是这么回事!」的夸大动作站在卓也那一边。 

    「能不能晋级下一场比赛,终究还是取决于正式上场时的表现呢!我们只管堂堂正正地过关斩将就行了。」 

    明日香如是说,志得意满地笑眯了眼。 

    「今年管乐比赛的参加团体,光是A部门就多达一千五百七十七所学校,再加上B部门的话,总共有三千两百三十七个团体。其中A部门只有二十九所学校能进军全国大赛。想当然尔,能晋级全国大赛的学校都不是等闲之辈。评审委员要听完所有学校的演奏,以A、B、C,亦即金、银、铜的三种等级评分。每位评审会给予每所学校A或B或C的评价,获得过半数评审给予A评价的学校即为金奖,得到过半数C评价的学校为铜奖,除此之外则为银奖。换句话说,可以得到金奖或银奖的学校数量并没有严格规定,每年都会不一样。」 

    叶月听完这番说明,战战兢兢地举手发问:「请问每所学校各自会得到几个ABC的评价?」 

    「依参赛的管乐社团数量而异,以高中管乐社来说,假设有十五个团体参赛,则每位评审能给予的ABC评价各有五个。」 

    「也就是说,前半场和后半场只会各有五所学校获得金奖吗?」 

    「不对,必须得到超过评审委员人数一半的A评价,才能获得金奖。所以票数一旦分散,自然就会少于五所学校。要在全国大赛拿下金奖的门槛其实非常高。」 

    「只要北宇治也拿下金奖就好了!小绿开始紧张起来了。」 

    绿辉兴高采烈地说,双手捧住自己的脸颊。她泛着红晕的脸颊让人联想到染成绯色的枫叶。 

    「小绿从国中就是全国大赛的常客吧,正式上场还会紧张吗?」久美子问道。 

    绿辉就读的圣女中学是全国首屈一指的管乐强校。只见她抱着胳膊,思索了一下。 

    「嗯……比起紧张,更多的是兴奋期待。你想想,过去的比赛都必须想着要打进下一关对吧?可是全国大赛完全不用担心晋级的事,可以比平常更自由自在地演奏。」 

    「圣女曾经连续三年在全国大赛拿下金奖呢,强校真的好厉害。」 

    梨子佩服地猛点头。「欸嘿嘿!」绿辉害羞地搔搔头。明日香兴奋地从讲桌上探出身子。 

    「那真的很厉害,因为国中的学校数量远比高中多太多了。以今年比赛的A部门为例,有一千五百七十七所高中参赛,但国中的A部门有两千九百五十一所学校。换句话说,参赛学校多了一倍,但可以进入全国大赛的名额还是一样,所以单从机率来说,国中部要打进全国困难多了。」 

    「国中管乐的竞争也很激烈呢!我念国中时,大家好像也都拼命练习。」 

    国中时期是网球社的叶月也一脸严肃地附和。久美子上的北中虽然没有圣女那么厉害,但实力也不差,还曾与同样位于宇治市的南中争夺进军关西大赛的门票。不过,北中和南中皆不曾打进全国大赛。北中虽以全国大赛为目标,但国中最后一场比赛,最后止步于在京都大赛拿下金奖。 

    「要是北宇治也能像以前那样,在全国大赛拿下金奖就好了。」叶月眯着眼睛说道。其他人都对这句话表示同意,用力点头。 

    虽说暑假已经结束了,但放学回家路上,天气还是很热。久美子拨弄着短袖水手服袖口,望向山间。夕阳西下,只剩微微红光残留在天空下摆。苍白的月色淡淡浮现在蓝色天空里,宣布夜晚已经来临。久美子撑着宇治桥的栏杆,随意望着水底,阴暗水面微微反射着商店的灯光。 

    「辛苦了。」 

    背后传来的声音令久美子猛然回头,看到秀一站在自己面前,他肩上挂着看起来很重的体育用品袋。 

    冢本秀一的母亲和久美子的母亲是朋友,所以从他们小时候两家就互有往来。小学才从东京搬来京都的久美子与秀一住在同一栋公寓里,国中、高中又加入同一个管乐社,秀一负责吹长号,和久美子同样获选为A部门的成员,每天练习他们都会碰面。 

    「辛苦了。今天的社团活动比平常还早结束,真幸运。」久美子说道。 

    秀一嘴角浮现淡淡的笑意,表示赞同。他好像趁着夏天又长高了。久美子望了他变短的裤管一眼,心里漫上异样的感觉。 

    「话说回来,清良女中要来的事,真是吓了我一跳,没想到她们真的会来。」 

    「对呀,这将是第一次听到她们现场演奏。」 

    「藤城老师老是要我们参考这种学校的演奏呢!」 

    他口中的藤城老师是他们国中时的管乐社顾问,肚子圆滚滚,感觉很温柔,从外表绝对看不出来他的指导十分严厉。 

    「明明连一年都还没过完,却感觉国中时期好像已经是很久以前的事了。」 

    「对呀,上了高中以后,每天都忙得不可开交,让人转眼就忘了国中的事。」 

    「啊,我突然想起,听说女篮队的文川同学和科学社的岸部同学交往了。」 

    「欸,真的吗?」 

    「真的,梓告诉我的,而且还是文川同学主动告白。」 

    梓是和久美子、秀一念同一所国中的女生,毕业后就读以行进乐队闻名的立华高中。 

    「不过,岸部也真有一套,我还以为那家伙绝对交不到女朋友。」 

    两人边聊着这些琐事,慢条斯理地往前走。桥上的人行步道有一排长方体的路灯,散发出暖色系的灯光。 

    「对了,你生日已经过了,对吧?」 

    「咦?」 

    秀一没头没脑的问话让久美子不自觉仰望他的脸。久美子的生日是八月二十一日,不懂他为何到了今天才又提起这件事。 

    「怎么突然问这个?」 

    「没什么,是我妈一直吵着要我送你礼物。」 

    「咦?伯母吗?为什么?」 

    「我、我哪知道。」 

    秀一丢下这句话后便转过脸,逃避久美子的视线。 

    一踏入平等院通,就传来远处焙茶的香味。久美子顺着石板路往前走,看着秀一的脸。小学时还很圆润的脸颊,曾几何时已经变得棱角分明。久美子不经意地从他那绝对称不上俊秀,但隐含着几分坚毅的侧脸上,感受到漫长岁月的流逝。 

    「你想要什么礼物?」 

    秀一依旧看着别处问道。久美子抱着胳膊,陷入沉思。绿辉送她可爱的发圈,还从叶月那边得到一百个巧克力的礼盒。叶月送给她的巧克力已经在午餐时间与班上同学一起吃掉了,害她长出满脸痘痘,真是伤脑筋。 

    秀一偷偷看着默不作声的久美子。 

    「啊,上低音号的CD如何?」 

    「丽奈已经送我进藤先生的演奏作品了。」 

    「真不愧是高坂,很懂你的喜好呢!」 

    秀一喃喃自语地表示佩服。 

    高坂丽奈和秀一、久美子一样,都是北中的毕业生,也都是管乐社的成员,从小就对小号一往情深。成绩优秀,人长得又漂亮,是一百分的美少女,唯性格强悍是她的优点,也是缺点。明明可以去念比北宇治更好的学校,但是为了接受泷的指导,所以来念这所高中。 

    「呵呵,因为我从小学就是进藤先生的粉丝嘛,收到时超高兴的。」 

    久美子噗哧一笑,忆起从丽奈手中收下包装很精美的CD时的心情。进藤正和是足以代表日本的上低音号演奏家。 

    「可是,如此一来我不就没东西可送了吗?」 

    「这是你要自己思考的问题吧!请送我会喜欢的东西,这样我也会考虑回礼。」 

    「你会喜欢的东西?好难啊!」 

    「其实什么都可以啦!」 

    久美子随口打发干劲十足的秀一。或许是不满意她的反应,秀一噘嘴抗议:「哇!你一点也不期待耶!等着瞧,你一定会喜欢的。」秀一不知怎地充满了干劲。 

    「好的、好的。」久美子笑着回答。 

    这么说来,他们是从什么时候开始不再交换礼物的。明明小学时从不曾缺席彼此的庆生会。升上国中后,两人不知不觉疏远了,觉得说起话来很尴尬,开始避着对方。想到那时候的事,不禁庆幸还好有管乐这个共通的社团活动,现在才能轻松聊天。万一自己参加的是其他社团,大概就不会再跟秀一说话了。想到会有自己以外的女生站在他旁边,久美子不由得身体一震,感觉很不开心。 

    「怎么了?」 

    或许是察觉到她的反常,秀一歪着脖子问道。带点淡淡棕色的瞳孔直瞅着她。久美子不想老实回答这个问题,往前跨出一大步。 

    「没什么!」 

    秀一则是一脸莫名其妙。 

    回到家,厨房传来的香味掠过久美子鼻尖。 

    「哎呀,你回来啦!」 

    久美子往锅子里一探,母亲正在炒绞肉,光是香味就刺激着她饥肠辘辘的肠胃。油脂在锅里弹跳,滋滋作响声回荡在狭小的厨房里。 

    「今天是文化祭吧?如何?」 

    「嗯,很开心喔!」 

    久美子从冰箱拿出果汁,倒进玻璃杯里,她心不在焉地看着橘色液体慢慢盈满杯子。电视上眼熟的主播正平淡地念着新闻稿。 

    「你刚上高中时,我还担心你不晓得会变成怎样,但是最近看你都很开心,真是太好了。社团活动也很努力吧?」 

    「还好啦!」 

    久美子躺在沙发上,一双赤脚搁在沙发扶手,脱下的袜子揉成一团扔在地板。母亲目瞪口呆地说:「你也太没规矩了。」 

    久美子闭上双眼,感觉累积到现在的疲劳一口气涌上来,直接躺在沙发上伸直手臂,肩膀底部的肌肉整个拉开,发出啪叽的清脆声响。 

    文化祭的演奏会也表现得很好,之后还能听到清良女中的演奏。大家都很有干劲,感觉好像一切都会很顺利,再来只要专心准备全国大赛即可。不知不觉间,久美子抱着抱枕的手愈来愈用力。心脏之所以扑通扑通地跳,想必是因为对未来充满期待。她在目前的生活中,感受到春天那时还没有的充实感。没问题,现在的我无所不能。为了掩饰激昂的心跳,久美子紧紧地缩成一团。 

    久美子侧耳倾听从厨房传来的母亲切菜声,静静闭上双眼。 

    「接着是下周的天气预报。」 

    电视里传来主播四平八稳的播报声。久美子的意识逐渐朦胧,脑浆开始溶解在梦境中。不带感情的播报声静静落在充满平静气氛的客厅里。 

    「强烈台风十四号从东海北上,方向逐渐转往东边,明天恐怕会从九州上陆,预测之后会从四国接近近畿地方,外出时请务必小心。接着播报下一则新闻……」 

    拿着乐器的人群以熟练的动作聚集成正方形,几十双脚踩在地板上,统一的雪白鞋子整齐画一地重复着相同的动作。轻快的音乐撼动耳膜,久美子侧耳倾听音乐,身体探出观众席的扶手。乐器反射着光线,闪闪发亮。 

    「啊,姐姐在那里。」 

    久美子在队伍中发现熟悉的面孔,对站在一旁的母亲耳语。从观众席看到的姐姐,跟平常的她完全不一样,身上穿着玩具兵般的大红色衣服,肩膀上有个威风凛凛的铜管乐器,名称好像是长号来着。姐姐加入小学的铜管乐队后,每天都忙于练习,不陪久美子玩。明明觉得「都是长号害的!」「最讨厌长号了!」可是看姐姐落落大方地吹奏长号,又觉得好羡慕,真是不可思议。 

    「妈妈,久美子也想玩那个。」 

    久美子边说边拉扯母亲的衬衫,专心摄影的母亲被打败似地低头看着久美子。 

    「那个是指行进乐队吗?久美子还太小了。」 

    「行进乐队?」 

    听都没听过的名字令久美子大惑不解。母亲仍然盯着摄影机的观景窗,小声说明。 

    「行进乐队是指边走边演奏乐器的乐队,不只听觉,也是视觉的享受喔……你看,这次变成星星的形状了。」 

    母亲微微指着舞台。久美子连忙踮起脚尖,看着下面的舞台。乐队的每个人都往经过精密计算的定点移动,在宽敞的体育馆一楼描绘出大红色的星星,由上往下看就成了星星的形状。他们移动的过程中,演奏连一秒都没有中断。 

    「久美子也想加入,我也要跟姐姐一样。」 

    「久美子才一年级吧?要升上四年级才能加入铜管乐队,所以再忍耐一下。」 

    「那等我升上四年级就可以参加吗?」 

    母亲拿她没办法似地回答:「嗯,对呀!等你长大就可以开始了。」 

    「太棒了!」 

    久美子情不自禁地高举双手,母亲也宠溺地对女儿天真无邪的狂喜模样露出苦笑。 

    「久美子真的很喜欢姐姐耶!」 

    久美子对母亲的话点头如捣蒜。 

    「因为,姐姐真的很棒嘛!」 

    翌日,持续好几天的晴天突然变脸,天气变得好差。久美子在嘈杂的雨声中惊醒,她梦到好久以前的事。那是小学一年级的时候吧!回忆起崇拜姐姐、拼命模仿姐姐的过去,久美子下意识揉乱自己的刘海。大概是睡着的时候流了汗,刘海湿答答的。 

    她起身静静望向窗外,隔着玻璃看到的雨珠令久美子忍不住叹息。她讨厌下雨天,因为头发会乱翘。久美子无奈地将头发绑成马尾,系上绿辉送给她当生日礼物的发圈,检查自己倒映在镜子里的模样。从镜中望向自己的少女,脸色似乎有点憔悴。 

    「早安。」 

    久美子上学途中遇到丽奈,她那白底水蓝色圆点的伞缘还缀着小巧的荷叶边。久美子将自己的伞斜向与丽奈相反方向,雨水哗啦一声倾泻流下。 

    「早安。天气糟透了。」 

    「台风好像要来了,明天可能会停课。」 

    「太好了。」 

    久美子忍不住欢呼起来,丽奈冷冷地睨了她一眼。 

    「你肯定还没准备英文单字测验吧?」 

    「唔!」 

    久美子身体一僵,心思被说中。丽奈目瞪口呆地叹气。 

    「万一台风改变路线怎么办?」 

    「不会,上帝一定不会抛弃我的。」 

    「最好是。」 

    丽奈说道,揶揄地莞尔一笑。不断从天空倾泻而下的雨水在水泥地上敲打出呆板的声响。久美子把伞柄夹在腋下,浮夸地反复搓手。 

    「啊,上帝求求祢,让台风往这边过来吧!」 

    「又在说蠢话了。」 

    「因为最近都忙着准备文化祭嘛!丽奈准备好英文测验了吗?」 

    「那当然。」丽奈不假思索的回答。 

    久美子无言以对。丽奈对她投以窥探的眼神。 

    「范围多达一百页,来得及吗?」 

    「应、应该来得及。」 

    久美子在脑海中衡量单字卡的厚度,垂头丧气地回答。打在伞上的雨声隔着薄薄的塑胶布,吵得令人心烦。 

    「谁叫你整个暑假都不念书。」丽奈目瞪口呆地说。 

    高中A部门的全国大赛将于十月二十六日在名古屋举行。过去原本在东京的会场举行,但是因为会场的耐震强度不足,所以换到名古屋。这个相当于管乐甲子园的场所,在当时是管乐社员的圣地。欣赏以前的比赛影片,不难发现地板黝黑发光。那个时候,管乐社员的目标就是在那个会场的独特黑色舞台上演奏。 

    京都离名古屋有段距离,因此北宇治高中必须前一天就前往名古屋,在当地过夜。如果是位于冲绳或北海道等离会场千里远的学校,光是移动就很辛苦。 

    「接着从头合奏指定曲。难得低音管和上低音萨克斯风都到齐了,请注意同一个小节。」 

    「是。」 

    分组练习中,众人异口同声回答明日香的指示。自从确定可以去全国大赛后,明日香对练习的态度愈发认真。其他三年级也一样,他们散发出剑拔弩张的紧张感,对学弟妹无疑是种压力。全国大赛结束后,三年级就得退休。虽然二月还有定期演奏会,但三年级可以选择要不要参加,忙着准备考试的社员肯定不能参加吧!一思及此,全国大赛是最后一次可以与这些成员一起演奏的机会了。一年级和二年级还有明年,但是三年级没有明年了。正因为如此,他们才这般焦虑。 

    「距离比赛已经没有时间了,请打起十二万分的精神练习,禁止聊天。」明日香推了推自己的眼镜说,薄薄的镜片闪烁着严峻的光芒。 

    她是出了名的品学兼优,不知道要考哪所学校,有小道消息说她要考京大。那是久美子无法想象的世界,要是明日香真的要考那么难的大学,继续带领社团活动不要紧吗?冷不防,久美子想起退出社团的学姐。 

    斋藤葵是负责次中音萨克斯风的三年级,与久美子是青梅竹马,以前经常玩在一起。她以准备考试为由,在京都大赛前退出了社团。参加社团活动的确可以学到很多东西,但是对考试而言无异是绊脚石。 

    因为当大家都在念书时,也必须花时间参加社团活动,就算是明日香,也会被其他考生拉开差距吧! 

    「很好,接下来从F开始。」 

    听到明日香的指示,久美子赶紧看着谱面。明日香转动节拍器,喀嚓喀嚓的单调声响在教室里回荡。 

    明日香往四周看了一圈说:「一、二、三、四。」 

    配合她一声令下,久美子用力朝乐器里吹气。 

    雨愈下愈大,晚上就连待在房间里,也能清楚听见激烈的雨声。 

    「宇治川不要紧吧!」 

    母亲忧心忡忡地喃喃自语。宇治川流域有天濑水库的庇荫,不太容易淹水,即便如此,还是经常因为豪雨导致水位上升。顺带一提,由天濑水库形成的人工湖名为凤凰湖,如今已成观光名胜。 

    「还没发出警报,我想应该不要紧吧!」 

    久美子心不在焉地盯着电视,回应母亲的担忧。 

    「是吗?」母亲依旧一脸不安。「总觉得有股不祥的预感。」 

    「你想太多了啦!」 

    久美子话还没说完,对讲机就响起。久美子下意识看了时钟一眼,已经十一点了,谁会在这时间上门。母亲大概也有相同的疑问,一脸诧异地侧着头。 

    「是不是让爸爸去应门比较好。」 

    如果是正常的访客,通常不会这么晚才来。万一是小偷怎么办?久美子还在思考这个问题时,母亲已经去书房叫父亲了。剩下自己独自一人,客厅突然变得好安静。久美子感到害怕,战战兢兢地站起来,走向玄关观察情况。就在久美子打算从猫眼往外看的刹那,门锁缓缓转动。喀嚓。久美子的身体因恐惧而动弹不得,她屏气凝神地注视着门口。果然是小偷! 

    一头棕发从微微开启的门缝探进来。 

    「……姐?」 

    出现在眼前的人物令久美子大吃一惊。只见应该在东京上大学的麻美子浑身湿透地站在门口,行李箱也覆满了水珠,在门前形成一方小小的水洼。 

    「麻美子,你怎么这副德性……」 

    前来看状况的母亲瞪大了双眼。姐姐看也不看久美子一眼,从她身旁走过。廉价的香水味和雨天特有的潮湿气味混合在一起,久美子不由得皱起眉头。门外传来夹杂在雨声里的呼啸风声。 

    「等一下,我去拿毛巾过来。」 

    母亲丢下这句话,啪嗒啪嗒地冲进浴室。父亲也一脸呆若木鸡地站在稍远处。 

    「三更半夜这是怎么回事?学校呢?」 

    「九月还在放暑假喔!」 

    这是麻美子开口的第一句话。久美子茫然地望着姐姐的背影,这才回过神来,锁上门。站在这种地方讲话会被隔壁邻居听到。 

    「喏,毛巾给你。真是的,怎么淋得湿答答。」 

    母亲抱着浴巾,不由分说地包住麻美子的身体。后者脸上闪过一丝寒光,但也没有抗议。父亲和久美子只能愣愣地注视着眼前的画面。 

    「到底怎么了?怎么突然回来。」 

    麻美子默不作声地凝视着还在抱怨「至少也先打通电话回来嘛」的母亲,手慢慢地伸向毛巾的边缘。尽管天气还与夏天无异,淋成落汤鸡还是很冷吧!只见她的身体正瑟瑟地发抖。 

    「……妈。」 

    「嗯?」 

    母亲停下正帮她擦身体的手,抬头仰望麻美子的脸。麻美子吸了一口气,以仿佛从声带挤出来的音量说:「我不想念大学了。」 

    第二天,久美子一早醒来,第一件事就是打开电视,检视萤幕中的台风讯息,找寻发布警报的地区。 

    「太好了!」 

    整个京都都发布了台风警报,只要能撑到九点,今天就不用上课,也不用考英文了。 

    拉开窗帘,天色依旧昏暗,上空积了厚厚一层雨云,细线般的雨丝不停从云层间落下。这下子确定停课了。久美子高兴得不得了,从冰箱里拿出盒装牛奶。 

    「你不用去社团吗?」 

    突然有人叫住她,久美子吓得回头,只见穿着黑色睡衣的麻美子正盯着自己。从语气不难察觉她的心情糟到不能再糟。 

    「今天放台风假,也不用晨练了。」 

    「是吗?」 

    麻美子不感兴趣地丢下这句话,躺在沙发上。久美子偷眼打量她。染成棕色的头发微微烫鬈,原本正经八百的姐姐已经不复存在。细心涂成红色的指甲与修剪得十分细致的眉毛,全都是高中时期的麻美子不可能会做的打扮。 

    麻美子小学就开始在铜管乐队吹奏乐器,为了考国中才退出乐团,后来考上私立高中,从此过着埋首书堆的生活,每天都要补习,几乎没有机会和久美子共进晚餐。她大学联考没考上第一志愿,决定去东京念吊车尾考上的大学。老实说,当她说要搬出去自己住的时候,久美子松了一口气。因为与姐姐相处的时光实在称不上舒服惬意。 

    「姐,你要退学吗?」 

    回想昨晚姐姐说的话,久美子终究无法按捺自己的好奇心。姐姐没回答她的问题。 

    昨晚,母亲并未要求麻美子交代始末,只叫她去洗澡。「有什么事改天再说。」这句话让麻美子卸下心中大石,听话照做。 

    「一天到晚啰哩啰唆叫别人念书,结果你反而要退学?」久美子说。 

    麻美子的脚趾动了一下,对她的问题做出反应。看样子姐姐听见自己的问话了。 

    久美子得理不饶人地继续追问:「你老是说光靠社团活动是考不上大学的,那你为什么要退学?你不就是为了去好的学校、要进好的公司才努力学习吗?一旦退学不就全部白费了吗?」 

    久美子的语气之所以带刺,是因为姐姐截至目前的态度所致,谁叫她动不动就嘲笑久美子。「光靠社团活动是考不上大学的」是姐姐的口头禅。 

    麻美子扬起脸,目光如炬地瞪着她。黑眼圈比八月见到她时还更严重,脸颊也瘦得凹进去了,藏不住疲惫的模样令久美子倏地倒抽一口凉气。麻美子用修长的指甲撩起刘海,大大地叹了一口气。 

    「跟你没关系。」 

    咬牙切齿的语气明摆着拒人于千里之外。久美子不敢再多说什么,默默把牛奶盒放回冰箱。冰凉的触感紧贴着手指表面,令人喘不过气来。 

    九点过后,台风警报尚未解除,因此学校确定停课,但久美子的心情还是很忧郁。都是麻美子害的,光是她待在家里,就足以让久美子感到心情沉重。父母都出门上班了,家里只剩她们姐妹俩。麻美子一直占据着客厅,也没特别在做什么,就只是心不在焉地盯着电视。 

    过了傍晚,台风似乎已离开京都,雨势小了很多,排水管流出的雨量也逐渐减少,转变成只要撑伞就能出门的天气。久美子不想待在家里,换上轻便的T恤和短裤漫无目的地出门去了。 

    受到台风的影响,在外面行走的人不多。石板路上有一滩积水,落下的雨滴在水面上激荡出一圈又一圈的涟漪。踩在水面上,球鞋鞋底感受到水面的弹力。 

    久美子穿过平等院通,漫无目的地走向宇治商店街。「抹茶可乐饼热卖中!」手写的海报吸收了湿气,字都晕开了。 

    「……咦。」 

    久美子顿时停下脚步。因为有个熟悉人物站在商店街上的花店前。 

    「泷老师。」 

    他穿着跟平常一样的西装,正和店员说话。看来老师连停课也得去学校。既然如此,那他大概是正从学校回家吧!灰色的西装裤管被雨水打湿,看起来黑黑的。确定店员消失在店里,久美子鼓起勇气,往前跨出一步。 

    「老师!」 

    出声呼唤后,泷吓了一跳,望向这边,双眼柔和地眯起。 

    「是你啊,黄前同学。这种天气不可以出门喔!」 

    「啊,对不起。」 

    久美子下意识道歉,泷为之苦笑。 

    久美子还撑着伞,往店内窥探。摆在店头的娇艳花朵映入眼帘,扑鼻而来的独特香味,大概是眼前的鲜花散发出来的。 

    「老师在这里做什么?」 

    「我来买花。」 

    泷说完这句话,不解地反问:「你的伞,不嫌麻烦吗?」 

    久美子闻言,赶紧收伞。成串的水珠从伞的尖端滴落。久美子叩叩叩地在地板上敲了几下,柏油路面转眼间就积了一滩水。 

    「老师,你没带伞吗?」 

    久美子对两手空空的泷抛出直率的疑问。「我开车来的。」泷回答。 

    「不过台风还是很讨厌呢!鞋子里都湿了。」 

    「因为雨是斜斜打下来嘛!」 

    「可是对黄前同学来说,这场台风来得正是时候吧?学校因此停课了。」 

    「呃……这个嘛,嗯……没错。」 

    久美子不好意思老实回答,有些迟疑地承认。见她承认,泷愉悦地点点头。 

    「无妨,我还是学生时,每次台风来的时候也都很期待放假。可惜十次有九次都是早上就脱离暴风圈,解除台风警报。」 

    「就是说啊,很少停课。」 

    「可是成为老师以后,就算台风来也得去学校,所以希望台风尽可能不要来。」 

    泷耸耸肩说道。有道理,今天早上的天气还很糟糕,但爸妈还是得去工作。等到自己出社会,也必须顶着暴风雨出门上班吗?那还真讨厌啊!久美子不由得皱紧了眉头。 

    「话说回来,黄前同学怎么会在这里?」 

    泷不解地侧着头问道。久美子一时答不上来。这趟门出得毫无意义,就只是不想待在家里,并没有特别的目的。可是好像也不用据实以告,于是久美子给了一个不痛不痒的答案。 

    「呃,我出来散步。」 

    「散步吗?这种雨天?」 

    「啊,那个,我喜欢下雨。」 

    「这样啊!」 

    泷虽然接受她的说词,瞳孔中却明确倒映出久美子尴尬的表情,仿佛看穿她的内心世界。久美子撇开视线,下意识用指腹摩挲伞柄,曲线状的塑胶表面十分光滑。 

    「让您久等了。」 

    女店员招呼着朝他们走来,怀里抱着一束包装好的花。花束只用上小朵的向日葵,白色的花瓣带了点淡淡的黄色。好特别的向日葵。 

    或许是察觉到久美子的存在,店员惊讶地睁大双眼。 

    「这位该不会是您女儿吧?」 

    泷对店员回以苦笑。 

    「不是,是我的学生。刚好遇到。」 

    「啊,这样啊!是我误会了。」 

    大概是觉得很不好意思,店员胀红了脸。久美子悄悄仰望站在身边的泷。他一察觉到久美子的视线,腼腆地抓了抓自己的头发。久美子在他看似平凡无奇的左手无名指上发现一样平常没有的东西,不禁屏住呼吸。他手上确实套着一枚银色的细致戒指。 

    「老师平常有戴戒指吗?」 

    久美子问道。自己是否表现得够平静呢?只见泷露出一抹苦涩的笑容回答:「今天比较特别。」 

    他的语气温柔得令人心惊,又同时隐含着不容过问的冷酷。久美子握紧雨伞把手,咽下后面的问题。那束花是要送给谁的?冲到喉咙口的询问转瞬变成泄了气的皮球。 

    泷转向店员,掏出黑色钱包。再待下去就不识相了,久美子撑开伞,对接过花束的泷道别:「老师,我差不多该回去了。」 

    才刚走到屋檐外,雨势就一口气变成滂沱大雨。敲打在塑胶伞布上的雨声惊心动魄。泷眯起眼,叮咛久美子:「回家路上要小心喔,万一受伤就糟了。」 

    「是!」 

    久美子基于平常的习惯,不小心大声回答,店员吓了一跳地看过来。久美子不由得面红耳赤。泷微微一笑,轻轻点了点头。久美子低头致意,积在伞上的水随着她的动作一口气往同一个方向倾泻而下。 

    与泷分开后,久美子一个人沿着来时路往回走,脑海中浮现方才在泷的无名指上闪闪发光的戒指,还有丽奈的脸庞。丽奈喜欢泷,而且不只是崇拜或尊敬的感情,也包含爱情。 

    「泷自从妻子去世以后,就一直郁郁寡欢。」桥本集训时说的话不经意在久美子的耳边响起。 

    桥本是泷的好朋友,也是专业的打击乐演奏家,曾经来北宇治高中指导管乐社。据他所说,泷的妻子在五年前去世了。久美子怕这个事实会伤害丽奈,所以还没告诉她这件事,在她面前一直装作什么都不知道。 

    该告诉丽奈这个秘密吗?还是继续保持沉默呢?久美子陷入沉思,心不在焉地往前走。什么才是最好的选择?除了自己以外,没有人可以帮她选择。 

    第二天一早,久美子拖着沉重的步伐走向学校,打算跟平常一样进行晨练,霙已经在音乐教室里了,发现到久美子的存在,她放开双簧管的吹嘴,打了声招呼。 

    「……早。」 

    「早安。」 

    久美子边立起谱架、边回答,她朝上低音管吹气,发出咻的一声,那是空气流经管身的声音。 

    铠冢霙是管乐社唯一吹双簧管的成员,才二年级就负责指定曲的独奏。去年和她同年级的成员大举退出社团时,在她心里烙下深刻的伤痕,幸亏有小号组的优子陪在她身边,顺利在关西大赛前克服了心灵的阴影。 

    「昨天刮台风呢!」 

    久美子说,霙面无表情地略略点头。这位学姐不太会表现出情绪。 

    「不用上课,太好了。」 

    「我也不用考英文,真幸运。」 

    「可是下周好像要补考。」 

    背后传来的声音令久美子回过头去,她定睛一看,丽奈正一手提着小号乐器盒往这边看。久美子今天早上没和丽奈一起上学。 

    「早啊,丽奈。」 

    「早……啊,学姐早安。」 

    丽奈朝久美子挥挥手,然后向霙行了礼。霙也点头致意。剪得整整齐齐的黑色短发配合她的动作,顺着下巴轻微晃动。 

    「学姐还是这么早呢!」 

    丽奈佩服地说。霙微微笑开。 

    「因为要是退步,会被希美笑。」 

    伞木希美是去年退出社团的长笛组二年级学生。她的存在是霙继续玩管乐的意义。虽然在关西大赛前引起很多风波,经过一番曲折,希美也回到管乐社了。 

    「我也得好好加油,不能输给学姐。」 

    泷交代的作业堆积如山。久美子不由得握紧拳头。霙沉默寡言,不善于表达情感,但是在别人看不到的地方,却比谁都努力。每次看到她的努力,久美子都会告诫自己不能轻敌。 

    「最近大家都好认真,尤其是三年级,真不是盖的。」 

    久美子猛点头,对丽奈这句话表示赞同。 

    「既然都走到这一步了,大家都想在全国大赛拿下金奖吧!」 

    「明明一开始的目标只是要进军全国,真是太厉害了。」 

    霙的视线瞥向挂在墙上的相框,眨了好几次眼睛。刚进这所学校的管乐社时,泷提出的目标是「参加全国大赛」。经历过关西大赛,这个目标其实已经达成了。可是全体社员并不以此为满足,大家都朝向另一个在全国大赛取得金奖的目标,拼命练习。 

    「北宇治一旦变成强校,就能招募到更多优秀的新生,只要能引起良性循环,北宇治也能成为全国大赛的常胜军喔!」 

    久美子满怀期待地说,丽奈自豪地告诉她:「只要有泷老师在,这就是当然的结果。优秀的学生都会聚集在优秀的老师身边。」 

    「就像丽奈这样?」 

    「这我倒不否认。」丽奈笑着说。 

    丽奈对自己的技术那股绝对的自信,久美子除了佩服以外,没有别的感想。 

    放学后,久美子接获指令,要将收回来的笔记本送到办公室。教室角落有个人称笔记本回收箱的小纸箱,每到了考试前,数学老师就会要求大家交出笔记本,由当天的值日生负责回收,今天刚好是久美子轮值。 

    「那我们先走了。」 

    「待会见。」 

    久美子笑着目送叶月和绿辉离去后,就从底部捧起纸箱。单一本笔记本并不重,但是全班的笔记本加起来就很重了。她在搬运乐器时发现了一个好方法,只要用手臂从底下支撑,就能减少负担。 

    她走近办公室,四周一片喧哗扰攘,路过的学生都一脸狐疑地看着办公室门口。久美子重新撑住纸箱底部,稍微推开门,从狭窄的门缝发现老师们正起了争执。可以进去吗?久美子有点裹足不前,又不能把纸箱放在门口,只好蹑手蹑脚,小心无声地溜进办公室,同样有事来办公室的学生全都一脸不安凝视着眼前的情况。 

    「万一这孩子考不上大学,你们要怎么负责?」 

    近似悲鸣的尖叫声,撕裂了室内空气。气氛太过剑拔弩张,久美子不禁停下脚步。 

    有名披头散发的女士站在办公室正中央咆哮,另一个头发很长、个子很高的女学生正拼命抓住她的肩膀安抚她。 

    「不好意思,家母她……」 

    熟悉的声音让久美子下意识凝视对方的脸。明日香学姐。久美子多想赞美险些出声、但没真正发出声音的自己。明日香口中的家母一脸凶相地瞪着女儿。 

    「你干么道歉!该道歉的是他们吧?」 

    「妈,你先冷静下来。」 

    「谁还能冷静得下来!」 

    明日香的母亲破口大骂,不依不饶地用脚跺地。仔细一看,泷和训导主任正站在明日香和她母亲面前。相较于泷一脸忧心地看着明日香,训导主任则不断拭汗,始终保持卑躬屈膝的姿态。明日香的母亲丝毫不掩饰她的焦躁,恶狠狠地瞪着泷。 

    「从事教育的人,应该知道对小孩来说,什么才是最重要的事吧!如果目标大学需要社团活动的推荐也就算了,但这孩子要参加的是一般考试。想也知道社团活动对我女儿来说是多么扯后腿的事。」 

    不由分说的叱责,训导主任频频低头道歉。 

    「真的是,您说的一点都没错。」 

    「话说回来,我答应她继续吹管乐的条件就是她要在暑假后退出社团活动。往年这个时候,管乐社的三年级早就该退休了吧!为何直到现在还每天都要他们练习?未免也太没常识了。这种事只要稍微想一下就能明白,不是吗?」 

    「嗯,的确是这样没错。这种时候还要参加社团活动真的很辛苦,尤其是像田中同学这种要报考前几志愿的人,真的很辛苦没错!」 

    「既然如此,为什么不受理她的退社申请?我不是说田中明日香从今天起就退出社团活动吗?」 

    「退出社团活动。」这几个字让久美子的脑海倏地一片空白。葵离开社团的背影掠过大脑深处。久美子捧着纸箱,下意识望向明日香,后者似乎没留意到久美子的存在。 

    定睛一看,明日香的母亲手里抓着一张皱巴巴的纸,该不会是明日香的退社申请吧?明日香把手搭在母亲肩膀上,试图安抚她的情绪。 

    训导主任拼命用手帕擦汗。 

    「呃,可是啊,今年的管乐社真的非常努力,希望您也能肯定令嫒的努力……」 

    「不管发生什么事,我都不会收下这张退社申请。」 

    泷打断训导主任说话,这是他第一次开口。「泷老师!」训导主任制止他。明日香的母亲死瞪着泷。 

    「为什么?我听说你答应让吹萨克斯风的三年级退出社团。」 

    她指的是葵。泷静静叹息,眼睛直盯着明日香的母亲。 

    「斋藤同学对我说,退出社团是她自己的意思,所以我才收下她的退社申请。」 

    泷的语气十分平静,模样乍看之下与平常无异,但是一起度过这个夏天之后,久美子看得出来他很生气。 

    「我怎么也看不出田中同学是自己想退出社团。这张退社申请应该不是田中同学的意思,而是您的意思吧?」 

    「那又怎样?」 

    明日香的母亲不以为然地打断泷的话。「妈……」明日香试图阻止母亲大放厥词,但明日香的母亲接着往下说。 

    「她是我女儿,是我独力抚养到这么大的女儿,没靠过任何人,就我一个人。所以女儿的将来由我决定。我女儿要去优秀的大学,和优秀的人结婚,过着幸福的人生。社团活动对这孩子来说只是绊脚石。」 

    听完这一长串慷慨激昂,宛如连珠炮的说词,久美子觉得头晕目眩。她说得俨然明日香是自己的所有物。明日香也真是的,为什么不回嘴?换作是平常的她,绝对不会容许别人说出这么侮辱人的话。她为何要忍气吞声地任由母亲辱骂呢? 

    训导主任苦着一张脸,万分抱歉的模样,他夹在泷和明日香的母亲中间,肯定很为难吧! 

    「嗯,您说的对。嗯,我也有个正值青春期的女儿,所以很能体会您的心情。可是啊,北宇治高中的校风向来以尊重学生本身的意志为目标,没错。距离比赛结束只剩一个月左右,可以请您看一下样子再说吗?」 

    训导主任边试探对方的反应,边拼命拣选着词汇。以前一直以为他是个长舌的啰唆老头,但他也努力想在别让明日香退出社团的情况下,让事情圆满落幕。一思及此,久美子感觉自己的喉头一下子变得好热,好想对明日香的母亲说:「明日香学姐想继续参加社团活动!」但是学姐本人一句话也不说,导致她也不好强出头 

    泷看了明日香一眼,斩钉截铁地说:「我尊重本人的意思,只要田中同学不想退出社团,我就不会受理这份退社申请,没有商量的余地。」 

    「泷老师,你这句话说得有点……」 

    「我自认已经说得很委婉了。田中同学身为副社长,很称职地帮我把社员们整合起来。参加全国大赛是全体社员的心愿,这个梦想再过一个月就可以实现了。这段期间就不能好好支持她吗?」 

    明日香的母亲被泷堵得哑口无言。办公室里充斥着异样的寂静,其他老师也都捏着一把冷汗,静观事情的发展。 

    明日香的母亲大大吐出一口气,静静扳开明日香放在自己肩膀上的手。「妈。」明日香轻声细语地呼唤她。明日香的母亲笔直地面向她,缓缓开口:「明日香,现在就说你想退出社团。」 

    「咦?」 

    「说!说你现在就要退出社团。」 

    近似悲鸣的尖叫声,让明日香一口气哽在喉咙。很少看她不知所措成这样。明日香用力咬紧下唇,仿佛在忍耐什么。久美子看到她的表情,感觉自己的指尖倏地变得冰凉,手指十分紧绷,感觉正在消失。久美子目不转睛地对明日香投以「学姐,千万不要说你要退出社团」的乞求目光。明日香开口:「我不想退出社团——」 

    她的话还没说完,耳边响起啪的一声巨响。久美子反复眨眼,不确定发生了什么事。只见明日香挨了母亲一巴掌,捂着脸颊倒在地上。久美子花了一点时间,才理解眼前的状况。 

    周围一片慌乱,明日香伸手阻止大家,口吻清晰平静地说:「不要紧的。」明明体格比母亲高大,却完全不打算抵抗。久美子咽喉深处紧紧的,眼前是令人非常不愉快的光景。 

    明日香的母亲对还倒在地上的女儿劈头痛骂:「你为什么不听我的话!你是有多讨厌我,才会这样忤逆我!才要这样伤我的心!」 

    「不是那样的。」 

    「你是故意气我才选择那种乐器吧?才不愿意退出社团吧?你想让我受苦。」 

    「妈,不是那样的。」 

    「那你为什么不听我的话!」 

    明日香的母亲歇斯底里地呼天抢地,那样子颇不寻常。在为此感到不快之前,对未知的恐惧先在久美子心里扩散开来,单纯只觉得害怕。这女人好可怕。她为什么要用言语束缚女儿到这个地步?久美子完全无法理解,也不想理解。 

    明日香站起来,以优雅的姿势拍去沾在裙子上的灰尘,挺直背脊,将长发塞到耳后,慢慢吐出一口气。她的脸颊红肿,看起来好痛。 

    「妈,不要在这里吵好吗?你看,给大家添麻烦了。」 

    明日香婉言相劝,手放在母亲的肩膀上。原本失去理智,气冲冲的母亲至此终于回过神来,视线瞥向满脸惊惧的学生。看样子,「给大家添麻烦」这句话见效了,母亲难为情地闭上嘴。明日香的手指稍微用力。 

    「社团的事改天再说好吗?你突然跑来学校,其他老师都不能工作了。」 

    明日香的母亲直勾勾地凝视着她的脸,看着、看着,表情逐渐扭曲,指尖提心吊胆地伸向明日香红肿的脸颊,以颤抖的声线说:「对不起啊,明日香。妈妈又气到失去理智了。」 

    「没关系,我没事。这不是妈的错。」 

    「明明不可以在女孩子脸上留下伤痕的。像我这样的母亲,你会讨厌我也是正常的。」 

    对不起啊、对不起啊……明日香的母亲不断重复说着同一句话。很难想象她那萎靡不振的模样与刚才疯狂叫嚣的是同一人物。周围的人全都看得目瞪口呆,明日香轻轻地向大家低头道歉。 

    「不好意思,家母打扰大家工作了。」 

    「别这么说。」训导主任边擦汗边回答。过程中,泷始终一脸严肃地注视着明日香。或许是有意忽略他的视线,明日香打死也不看泷一眼,手绕到母亲背后,像哄小孩似地对她说:「妈,我没事。不管发生任何事,我都不会讨厌你。」 

    「真的吗?」 

    「嗯。所以你不用担心。」 

    明日香的语气里充满了宽慰的味道,仿佛在说给三岁小孩听。女儿这句话让母亲如释重负地放松了表情。两人互相依偎的身影甚至让人感觉温馨,与方才的落差简直是天壤之别。这种关系实在太扭曲了,久美子无言以对。明日香的半边脸颊还残留着怵目惊心的鲜红色。 

    「老师,对不起。我今天就先和母亲一起回去了,可以让我早退吗?」 

    「这倒是无所谓。」 

    「谢谢。我明天会参加社团活动,请不用担心。造成老师的困扰真的很抱歉。」 

    明日香一气呵成地说到这里,她低头行礼,表情与平常几乎没什么两样,这样反而让久美子心里一阵骚动。泷的表情苦涩不堪,却什么也没说。 

    两人离开后,停滞的时间终于又开始转动。所有教职员仿佛六神归位地又开始工作,久美子也记起自己怀里的纸箱重量。数学老师连忙走向她,从久美子手中接过那一箱笔记本。 

    走出办公室之前,久美子望向刚才明日香他们僵持不下的地方,训导主任正火冒三丈地对泷抱怨着什么。有张纸掉落在他们脚边,是明日香的退社申请,纸上满是撕破又重新黏好的痕迹。泷无视于训导主任还在说教,弯下腰,默默捡起那张纸。 

    泷瞥了那张纸一眼,毫不犹豫地扔进垃圾桶。 

    久美子一进分组练习的教室,立刻被社员团团包围。绿辉抱着面纸盒,叶月气得柳眉倒竖,卓也和梨子一脸不安地看着她,夏纪双手扠腰站在久美子面前。 

    绿辉以带着鼻音的声音问难掩困惑的久美子:「久美子,明日香学姐真的要退出社团吗?」 

    单刀直入的质问令久美子一时半刻说不出话来。或许绿辉当她的反应是默认了,脸上乌云密布。 

    「果然还是要退出社团啊!」不知不觉间,绿辉的大眼睛浮现泪光。 

    周围的人同声叹气,久美子连忙否认:「没有,学姐一定不会退出社团!只是为了要不要退出社团起争执而已。」 

    「明日香学姐的母亲杀到教职员办公室的传言果然是真的。」 

    夏纪目光悠远,梨子心灰意冷地一屁股坐下。 

    「不过,这也是没办法的事,明日香学姐要考好的大学嘛!」 

    「呃,又还没决定要退出社团。」 

    「我不要学姐退出社团。」卓也一脸凝重地自言自语。 

    叶月垂头丧气地说:「没想到最后一次大赛竟会以这种方式结束。」 

    「听我说,学姐没有要退出社团啦!」 

    久美子使出全身的力量叫嚷,众人总算恢复理智看着她。扯着嗓门大喊,一点也不像自己会做的事,久美子觉得好丢脸,刻意清了清喉咙说:「别担心,泷老师并没有受理退社申请,明日香学姐不会退出社团。」 

    卓也无精打采地低喃:「也就是说,学姐递出退社申请了。」 

    这下子教室真的炸开了锅。「果然还是要退出社团嘛!」叶月嚷嚷。「我讨厌这样!」绿辉大哭。二年级的人都一脸严肃,交头接耳地不知在讨论什么。到底该怎么说明才好?正当久美子不知所措时,走廊上传来兵荒马乱的脚步声,教室门被用力推开。 

    「明日香要退出社团是真的吗?」 

    边问边冲进来的是小号组的组长香织,后面跟着其他组别的三年级生。事情变得无法收拾,久美子苦恼极了。 

    社员陆续在低音组的练习室集合。到底是怎么回事?发生什么事了?无数的疑问砸向久美子,她拼命重复同样的说明。久美子脑海中浮现出明日香早退的背影,不由得深深叹息。 

    结果,这一天的练习几乎整个泡汤,感觉却比平常的练习还要疲劳好几倍。久美子筋疲力尽坐在车站月台的长椅上,她将包包放在身边,发出了沉重的声响。 

    「辛苦了。」 

    肩膀被拍了一下,久美子抬头看,丽奈正对自己露出苦笑,纤长的睫毛上上下下轻盈搧动。不知怎地,久美子感觉松了好大一口气。 

    丽奈在久美子的身旁坐下,月票的颜色从包里撞进眼帘。 

    「今天好像搞得人仰马翻呢!」 

    「嗯,就是说啊!」 

    「刚才要回家的时候,三年级吵成一团。明日香学姐真的要退出社团吗?」 

    丽奈因为要上英文的特别课程,无法参加放学后的社团活动。久美子对这个问题表现出模棱两可的回应。 

    「嗯……该怎么说呢,她本人是说她不会退社啦!」 

    「所以是旁边的人要她退社?」 

    「嗯,好像是她母亲要她退社。」 

    「哼……」 

    电车滑进对侧的月台,迎面而来的风撩拨着丽奈的头发,洗发精的香甜气味轻柔拂过久美子的脸颊,令她有些脸红心跳。久美子怔忡望着丽奈微微颤抖的低垂睫毛,平交道警报器发出的声音听起来特别刺耳。 

    「好讨厌啊,这种事。」 

    「什么事好讨厌?」 

    「我最讨厌父母插手子女的事了。」 

    藏青色的裙子底下露出雪白的大腿。久美子的指尖探进有些移位的黑袜里,一股作气往上拉,或许是因为被松紧带箍住,她的皮肤上残留着凹凸不平的痕迹。 

    「社团活动不是由父母决定的吧?为什么需要父母的许可才能做自己想做的事?」 

    「可是啊,事实上就是有很多父母会对子女想做的事啰哩啰唆地指手画脚。」 

    「升学或就业也是如此,为什么不让子女做自己想做的事呢?」 

    「父母也有他们的考量吧?」 

    「话是这么说没错啦!」 

    丽奈跷着二郎腿,没好气地说。她被黑发遮住的耳朵很白,稍微尖尖的。 

    「丽奈考虑过未来的事吗?」 

    「怎么突然问这个?」 

    「看到明日香学姐的前例,不由得担心起来了。」 

    久美子抱着膝盖,坐在长椅上,额头往膝盖上一搁,发出咯吱的清脆声响,有点痛。 

    「是稍微思考过。」丽奈说道。她冷不防伸出手来,无意识地轻抚久美子的刘海,微微一笑。 

    「你其实也想过这个问题吧?」 

    久美子被她这么一问,脑海中浮现出麻美子的身影。姐姐一年级的时候就决定要念哪所学校,为了考上那所学校,每天用功读书,可惜她的目标终究未能实现。既然如此,自己该怎么办才好。久美子一点也不想象姐姐那样满脑子只有念书,也不觉得自己用功得来,但又没有特别想做的事,也没有想实现的梦想,只是拼命地追赶自顾自流逝的每一天。 

    久美子扬起头,偷偷望了丽奈一眼,后者正乐不可支地帮久美子编辫子,她细致的指尖微微碰触到久美子的脸颊。 

    丽奈眼中的世界是什么样的世界呢?即使距离如此靠近,久美子依旧不知道她在想什么。丽奈的眼眸在长长的睫毛下散发出宛如宝石般的色彩。久美子很想知道在她那双闪闪发光的眼睛里,自己是什么模样,但又同时害怕知道。久美子一向很怕听到别人的真心话。 

    「未来啊……」 

    久美子喃喃自语,丽奈则一言不发。当当当当,远方隐约传来平交道的警示音。丽奈望向月台对侧,放开久美子的头发,原本编得死紧的三股辫松松地散开。久美子轻轻用指尖绕着恢复原状的头发,觉得有点可惜。 

    隔天的假日练习,明日香也没来。自久美子加入社团,这还是明日香第一次没来社团。即使在合奏中,众人也都心浮气躁地瞥向久美子隔壁的空位。 

    「到此为止。」 

    合奏到一半,泷要求大家停止演奏,他傻眼叹气,静静把指挥棒放在讲台上。 

    「这是怎么回事?」 

    自从初次合奏以来,这是第二次听到这句话。第一次为迎接太阳祭的合奏时,泷对大家说出毫不留情的狠话。或许是想到当时的事,社员们全都悚然一惊。泷环顾众人的脸,平静地问:「各位可以集中精神吗?」 

    泷的语气还算温和,表示他不像春天合奏时那么生气。「可以。」大家七零八落地回答。无法回答得像平常那样精神抖擞,显然是心里有鬼。久美子悄悄看了旁边的空位一眼。 

    泷眯起眼,拉开关紧的窗帘。光线从厚厚的云层缝隙透出来,照亮众人的脸。乐器反射着阳光,熠熠生辉。 

    「我知道大家都很担心田中同学,但这种事以后也可能发生。正式比赛当天,可能会有人生病也说不定,可能会有人受伤、无法吹奏乐器也说不定,难道各位每次都要精神涣散地演奏得有气无力吗?」 

    没人有办法反驳。吹法国号的学姐咬紧下唇,双眼直盯着泷的脸,瞳孔表面仿佛随时都要滴出水来。为了掩饰在眼眶中摇曳的水面,学姐悄悄闭上眼睛。久美子似乎可以察觉到学姐的不甘心。 

    「今天的合奏就到这里结束。既然各位无法集中注意力,再练也是浪费时间。接下来各自分组练习吧!」 

    泷说到这里,一如往常地望向小笠原。 

    「社长,可以吗?」 

    「不可以。」 

    小笠原斩钉截铁地回答。众人全都大吃一惊看着她,音乐教室的气氛顿时沸腾起来。因为这是她第一次反抗顾问,教室中的视线同时落在她身上。小笠原站起来,用力低下头去,绑好的头发甩了好大一下,嘴唇微微颤抖。 

    「请让我们合奏。」 

    其他三年级也随社长这句话起立,椅子和地板摩擦的声音此起彼落地响起。 

    「我们还能努力下去。」 

    「拜托老师,请让我们继续合奏。」 

    学长姐说道,一起低头请求。一、二年级也赶紧站起来,学三年级低头请求。「拜托老师。」社长发自肺腑的声音响彻整个音乐教室。 

    这样的姿势不知维持了多久,泷站在教室前面,轻声叹息,打破绷得死紧的寂静。 

    「不是这样的。」 

    泷解释。反应出乎大家意料,众人一脸诧异地抬头。还以为他要不是骂人,就是说些难听的话。但泷抱着胳膊,字斟句酌地说:「我没有生各位的气。现在大家与春天的时候不同,各位都很有干劲,专心听我说话,努力回应我的要求。我充分感受到各位的士气。」 

    「可是,」泷接着说:「有些事光有士气也无可奈何。照这样下去,再有干劲也只是空转。我不是要大家别去想田中同学的事,可是如果不能区分清楚练习与担心,只会一直合奏得心不在焉,就连平常吹得好的部分也吹不好,失误变多……所以我判断与其每次练习都比上次退步,在这种情况下,干脆先停止合奏比较好。」 

    泷的言辞间,隐约可听出他的苦恼。他没有生气,也没有不耐烦,只是觉得自己身为指导者很没用,而非对学生有什么不满。久美子看到他的眼神不甘心地闪烁着,感觉自己的喉头燃起一股灼烧般的热度,眼前一片模糊。久美子为了压抑满溢的热度,咬紧牙关。 

    泷的说词很理智,反而令人无法反驳,因此没有人说得出话来。大家只是低着头,呆站着不动。 

    「接下来进行分组练习,可以吧?」 

    社员回答这个问题的声音有如呻吟,其中还有夹杂着哽咽的声音。 

    泷走出音乐教室后,大家都还动弹不得。教室里此起彼落地传来女生啜泣的声音。泪水之所以一直掉下来,并非像以前那样害怕泷的淫威,而是单纯的懊恼,懊恼自己的没用。对他们来说,让泷失望是比什么都痛苦的事。因为自己精神太软弱而让顾问烦恼的事实,让大家的情绪更低落了。 

    想努力的心情不是假的,但泷说的也没错。大家都对明日香不在感到不安。她是这个社团的精神支柱,万一她真的退出社团,这个社团到底会变成什么样呢?不祥的预感蚕食鲸吞地侵蚀众人的意识。教室内的气氛沉重到令人喘不过气来。 

    「……我们都太依赖明日香了。」坐在前面的小笠原自言自语地呢喃,周围的三年级诚惶诚恐地抬起头。 

    「因为她不在,大家就感到不安,军心涣散成这样,绝不只是因为正式比赛前少了一个人吹上低音号,是因为少了『明日香』。大家都觉得万一没有明日香,这个社团就完了。」 

    这句话让学弟妹全都难为情地别开脸。明日香擅长处理人际关系,尽管性格利己的她只是为了自己才负责居中协调社团里的人际关系,但这个角色非常重要。以希美与霙的关系来说好了,倘若没有明日香,一定会变得更复杂。久美子看着双簧管的座位,咽了口水。霙依旧面无表情凝视着小笠原的脸。 

    「可是,」社长继续说。她的脸上透露出一股坚毅的感觉,这个人已经下定决心了。眼前的小笠原让人感觉到她的坚强。「社团活动不是那样的关系,不该让责任全部落在一个人身上。为什么要有这么多社员,不就是为了在这种时候可以互相扶持吗?」 

    她的语气起初还有些不确定,如今也逐渐慷慨激昂起来,变成具有中心思想,四平八稳的声音。她握紧拳头,轮流看着所有人。 

    「我知道自己是个不中用的社长,也知道大家比起我,更依赖明日香。因为我们总是在一起,我从一年级就知道明日香比我优秀太多了。我想请辞社长一职想过无数次,总是想全部都给明日香做就好了。之所以没真的请辞,除了觉得一旦请辞就真的输了以外,主要还是因为明日香身为副社长,一直支持着我。我希望得到明日香的肯定。虽然同为三年级,但我认为明日香和我不一样,那家伙是特别的,所以只要她称赞我,我就会继续努力。」 

    「可是啊……」小笠原说到这里,深深吸进一口气,声带闷闷地发出声音。久美子的脑中浮现出在社团里听过无数次的台词:因为明日香学姐是特别的。 

    「明日香并不特别。」小笠原不容置疑地说,仿佛要点醒过去的自己。「只是我们一厢情愿认为她是特别的、一厢情愿画出楚河汉界,认为她肯定没问题,都没有人替她着想。仔细想想,明日香肯定也很辛苦,因为聪明,一定得考上好大学,也因此必须比其他人更用功,我们却先入为主认为明日香完美地身兼副社长和声部组长和鼓乐队队长数职是理所当然的事。因为她从不示弱,我们就以为她没问题。」 

    所有人都哑口无言,只是默默地听社长说话。 

    「这次轮到我们支持明日香了。目前还不确定她是不是真的要退出社团,但是为了让她随时都能回来,就不能让她担心社团活动。因为以前发生过的事,我想有些一、二年级会对三年级感到不服气。或许也有人觉得除了明日香以外,社团里都是些靠不住的学长姐。但就算是这样,也请跟随我们一同前进。」 

    小笠原低头恳求,语气十分真诚,听得出来她是认真的。学弟妹不知该有何反应才好,一脸困惑地面面相觑。在这样的情况下,优子率先打破僵局。她站在小笠原面前,轻轻碰了碰小笠原的手臂。 

    「社长。」 

    小笠原慢慢抬起头来,优子的眼睛红通通的,眼角还残留着用力揉搓的痕迹。久美子认为,不愿让人知道自己哭过,很有优子的风格。 

    优子大大吸进一口气,慢条斯理地开口。「请不要小看我们。」 

    出乎意料的台词令久美子呆若木鸡。小笠原也差不多,一脸讶异地看着眼前的学妹。优子眉头紧蹙,表示她很不高兴。「那个笨蛋。」卓也在教室角落伤透脑筋地说。 

    优子从鼻子里哼了一声,不依不饶地用力踩踏地板。 

    「这种事不用你说,我们也会跟三年级同进退。话说回来,得知全国大赛有希望的那一刻,大家就都是玩真的。应该没人会那么无聊,因为讨厌学长姐就偷懒,或因为看学弟妹不顺眼就漫不经心。现在还说这种见外的话,只会让我们觉得你是在小看我们。」 

    优子说完自己想说的话,表情很痛快。小笠原还没回神,不由得露出苦笑。绷得死紧的气氛至此终于放松下来。 

    小笠原擦了擦眼角,挖苦优子:「你的问题是对喜欢的学姐挟带太多私情呢!」 

    这句话让其他三年级哄堂大笑,学弟妹也跟着笑出来,唯有香织面红耳赤,捂住自己的脸娇嗔:「真是的。」丽奈忍住笑意,在一旁看戏。 

    或许是被优子的一席话点醒了,小笠原轻拍自己的双颊,空气中响起「啪!」的一声。她抬头挺胸轮流看着社员的脸。 

    「那好,接下来的练习要全神贯注。明天一定要让泷老师指导我们合奏。」 

    久美子闻言,用力吸气。就像练习腹式呼吸时那样,耳边传来大家呼吸的声音。仿佛要吐出熊熊燃烧的干劲,久美子中气十足地回答:「是!」 

    团结一致的声音回荡在音乐教室里。 

    回家路上,久美子的脚步比昨天轻盈,感觉鞋底踩在柏油路上的声响也轻快许多。天空已染上夜色,室外灯倾泻的光线,照亮了被雨淋湿的柏油路面。从弥漫着湿气的味道可以得知刚才的天气。 

    吸进夜晚带着热度的空气,久美子看了走在身旁的两人一眼。情绪有些亢奋的叶月和绿辉正七嘴八舌讨论明日香的事。 

    「小绿想直接杀去明日香学姐家。」 

    「不行不行,不能这么冒昧。」 

    「可是,人家又不晓得明日香学姐的手机号码。」 

    「连手机号码都不知道,是要怎么去她家。」 

    叶月戳中她的盲点,于是绿辉抱着胳膊,陷入沉思。 

    「这个嘛,嗯……问香织学姐如何?那两个人感情很好。」 

    「欸,她会告诉我们吗?」 

    「那,利用中午休息时间去明日香学姐班上找她?」 

    「明日香学姐周一会来上课吗?」 

    「欸,要是她不来上课就没办法了。」 

    久美子心不在焉地聆听她们讨论,也陷入沉思。万一明日香真的退出社团,自己能做什么?她回想明日香和她母亲的对话,感觉胸口一紧,舌头内侧涌起苦涩的情绪,突然觉得好想哭。明日香面对歇斯底里的母亲,居然一句话也没反驳,久美子简直不敢相信,多希望明日香能像平常那样,以正气凛然的大道理向母亲抗议。那才是久美子认识的明日香。她不想看见学姐被欺负成那样,不想觉得自己尊敬的学姐没出息。 

    「啊!你们看,是向日葵。」 

    绿辉突然大呼小叫。陷入沉思的久美子这才猛然回过神来。叶月在一旁傻眼地说:「向日葵有这么稀奇吗?」 

    「欸,已经九月了,当然很稀奇。」 

    「九月也会开吧!毕竟现在的天气还跟夏天没两样。」 

    绿辉指着民宅的花坛。用红色和橘色的砖块隔开的小长方形,呈现出仿佛西洋童话里才会出现的形状。绿意盎然的前端有一簇小花,花瓣的颜色是带点黄色的白色,温和纯净。久美子为植物的生命力大受感动,不经意想起。 

    「这是泷老师买的花。」 

    台风那天,泷买了一束花。因为是很特殊的向日葵,久美子对其特征记得很清楚。绿辉听见久美子的自言自语,双眼闪闪发亮。 

    「那是送给女朋友的礼物吗?」 

    「女朋友?泷老师有女朋友?」叶月挑眉。 

    「这我就不知道了。」绿辉不当回事地回答。 

    「不过,这种花的花语很浪漫。会送这种花给对方,肯定是对他很特别的人。」 

    「花语……你看太多少女漫画了吧?」 

    「跟少女漫画无关。」 

    「随便啦,人家对花语又没兴趣。」 

    「那只是你没兴趣而已。」 

    「因为总觉得『花语等于女孩子』这样。」 

    「干么画地自限!」 

    绿辉鼓着脸颊,驳斥叶月说的话。久美子打断两人仿佛要持续到地老天荒的抬杠。 

    「这种花叫什么名字?」 

    「咦,不是向日葵吗?」叶月没好气地回答。 

    「不是喔。」绿辉立刻加以否定。「这是一种意大利白向日葵。」 

    绿辉洋洋得意地开始说明,叶月一脸打从心底觉得无所谓,不带感情漫应一声:「是噢。」意大利白向日葵。花瓣很接近白色,的确符合这个名称。为了看清楚,久美子把脸凑向花坛。绿辉兴高采烈地问她:「久美子该不会对花有兴趣吧?」 

    「没,倒也不是有兴趣……只是觉得这种花好漂亮啊!」 

    「对吧对吧,所有的花里,小绿最喜欢向日葵了,因为很好吃。」 

    「咦,向日葵可以吃吗?」 

    叶月一脸难以置信地看着绿辉。 

    「对呀!」绿辉天真无邪地点点头。「向日葵的种子很香,非常好吃。好想再吃一次在北海道吃到的向日葵霜淇淋啊!」 

    「真的有那么好吃吗?好难以相信。」 

    久美子听着两人有一搭没一搭的对话,想象绿辉啃食向日葵种子的模样……小绿的确很像仓鼠,或许一点也不奇怪。久美子不动声色想着这件事时,绿辉抓住她的手臂。 

    「久美子想知道花语吗?」 

    「咦?嗯。」 

    「好奸诈,居然硬拉久美子站在你那边。」 

    「才不奸诈呢!」 

    久美子还在发呆的时候,她们又拌起嘴来。她连忙把意识拉回两人身上。 

    「那种花的花语是什么?」 

    绿辉得意洋洋地回答:「意大利白向日葵的花语是——永远想着你。」   

    \section{吹响吧!长号} 
    「姐,你为什么不吹喇叭了?」 

    麻美子不高兴地低头看着久美子。对于小学一年级的久美子而言,六年级的姐姐永远是有点可怕的存在。麻美子比久美子大很多岁,知道很多她不知道的事。不管是算数道具,还是粉彩蜡笔,久美子在学校用的东西永远都是麻美子用过的,久美子很羡慕姐姐总能央求爸妈买些亮晶晶、莫名其妙的宝贝给她。 

    「跟你没关系。」 

    麻美子不屑一顾地说,活像不想看见久美子的脸。当时她对久美子的态度就很不客气。尽管如此,久美子还是很崇拜姐姐,因为年幼的她知道姐姐其实很温柔。 

    「久美子好喜欢姐姐吹的喇叭。妈妈说,等我升上四年级就能吹喇叭了,所以我也想吹长号,想跟姐姐一起吹长号。」 

    久美子拼命想表达自己的想法,可是她愈说,麻美子的心情愈恶劣。麻美子臭着一张脸,从书包拿出珠算工具,面向书桌,完全当久美子不存在,但她还是亦步亦趋地追着姐姐跑。 

    「姐,你是为了考上好学校才放弃吹喇叭吗?不能不用功吗?姐姐那么聪明,就算不放弃喇叭也没问题的。」 

    耳边传来铅笔窸窸窣窣在笔记本上摩擦的声音,久美子不死心地继续缠着麻美子。 

    「姐,不要放弃嘛!久美子好喜欢吹喇叭的姐姐,像是你在运动会吹喇叭的时候,早纪也说久美子的姐姐好帅,所以不要放弃嘛!」 

    久美子说得快哭出来。对于小学一年级的久美子而言,姐姐要不要继续参加铜管乐团是死活问题。她很喜欢姐姐吹喇叭的样子,很喜欢帅气的姐姐。 

    「……姐姐最近一点都不帅气。」 

    久美子絮絮叨叨地说,麻美子的背影抖了一下。 

    「老是板着一张脸,好恐怖。我不喜欢现在的姐姐,你吹喇叭的时候帅气多了。」 

    这句话让麻美子突然站起来,一言不发地低头看着她。久美子藏不住内心的恐惧,伸手捂住自己的嘴巴。自己说得太过分了吗?麻美子抓住久美子扑簌簌发抖的手臂,不由分说将她推倒在地上。久美子呼吸困难,她真的觉得自己会死掉。 

    麻美子低声咆哮:「你懂什么!」 

    她从声带挤出来的声音听起来好委屈。久美子双目圆睁,凝视着姐姐近在咫尺的脸。姐姐乌溜溜的大眼睛目光闪烁,宛如夜晚的大海。看在久美子眼中,那是悲伤的颜色,是非常不帅气的颜色。 

    麻美子以粗鲁的动作放开久美子,不屑地冷哼一声。压迫肚子的感觉消失了,久美子总算能呼吸到空气。姐姐居高临下看着咳个不停的妹妹,冷冷撂下一句:「下次再乱说话,就把你的嘴巴缝起来。」 

    这很明显只是恐吓她的狠话,久美子还是老实地点头。因为姐姐看着自己的表情实在太哀伤了。这个人好可怜,久美子淡漠地想。 

    即使过了九月中,麻美子还是待在家里。大学早就已经开学了,她却窝在自己房间,偷偷摸摸不晓得在做什么。父母起初还一脸困惑地对她说教,后来大概是死心了,不再说什么。回到家就会看到姐姐的生活令久美子感到窒息,幸好她平常大部分时间都用来从事社团活动,不太需要和姐姐独处。 

    「早安。」 

    「早。」 

    久美子向丽奈打招呼,丽奈微微一笑。早晨的上学路上,空气很新鲜,视线范围也很清晰。原本灼热的空气开始带着凉意,翠绿的叶子逐渐转成红色或黄色。冷风让人感觉到秋天的来临,令久美子打了一个冷颤。最近天气一口气变冷了,夏天时还以为大热天会永远持续下去,但季节总在不知不觉中偷偷变换。 

    「明日香学姐不知道怎么样了。」 

    丽奈看着久美子说道,手中握着英文单字卡,字卡的角落贴满了五颜六色的便条纸。 

    「嗯……会怎么样呢?」久美子模棱两可地漫应一声。 

    教职员办公室那场骚动的隔周,明日香就回来参加社团活动了。久美子怔忡地回想当时的状况。 

    「明日香!」 

    她一出现在音乐室,其他社员就一起围上去,久美子也想靠近那堵人墙,但又没有勇气推开三年级走到她身边,只好从几步之外注视着明日香秀丽的侧脸。曾经肿到令人心痛的脸颊已恢复正常。 

    「虽然有奇怪的谣言,但你不会退出社团吧?」小笠原逼问明日香。 

    「抱歉。」明日香有些困扰地撇成八字眉。「好像给你添了很多麻烦,我从今天起都会来参加社团活动。」 

    「问题已经解决了吗?可以相信你真的不会退出社团吗?」 

    香织脸上浮现出不安,纤细的手抓住明日香的水手服下摆。是怕她逃走吗?抓着制服的手十分用力。深蓝色与香织白皙的肌肤形成美丽的对比。明日香眯起眼,以缓慢但确实的动作拨开她的手。薄唇勾勒出与往常无异的笑容,那是温柔的拒绝。久美子领悟到,这里就是界线,明日香不允许其他人跨越这条线。 

    「别担心,我会来参加社团活动。」 

    她没说自己要退出社团,然而,也没说不会退出社团。 

    从那一天起,明日香请假不来社团的日子变多了。大概是带乐器回家练习,她的演奏水准并没有退步,社员还是无法隐藏自己的不安。泷和明日香在办公室谈话的情景也被目击过好几次。大家都知道缺席社团活动并非她自己的意思,但不管谁去问,明日香都不肯说清楚,总是挂着一如往常的笑容,避开问题的核心,这点让香织非常伤心。 

    「小笠原社长真的非常努力呢!」 

    「大概是想连明日香学姐的份也一起努力吧!」 

    自那次合奏以后,小笠原开始积极地主动找社员说话。多亏有她,社团总算不至于分崩离析,但是任谁都看得出来,那是建立在钢索上的平衡。明日香本人也察觉到了,有时会表现出焦躁,令久美子印象深刻。 

    「我猜明日香学姐大概和泷老师商量好怎么不穿帮,瞒着她母亲来参加社团活动,或许还骗母亲说自己已经退出社团了。」 

    「为什么明日香学姐的母亲要这样扯她后腿呢?我不能理解。一般来说,要是自己的女儿能打进全国大赛,通常都会全力支持吧!」 

    「她母亲好像很讨厌管乐。从她母亲生气的感觉,似乎不只是气她参加社团。」 

    你是故意气我才选择那种乐器吧? 

    当时她母亲确实这么说了。故意气她是什么意思?玩管乐为什么是故意气她?久美子怎么都想不明白。 

    「话说回来,明日香学姐为什么要参加社团活动?」丽奈不解地问道。 

    「学姐的个性其实有些尖锐的地方吧,应该说是利己主义吗?就是类似只要自己好就好那样。」 

    「这么说倒也是。」 

    明日香的性格非常八面玲珑,但是偶尔不小心窥看到她的内心世界,会觉得头皮发麻。久美子无法明确地揣测明日香在想什么、看到什么,可有时候会因为察觉到她的敏锐,而觉得喘不过气来。明日香轻易割舍了青春期特有的柔软自我意识。这么毫不留情地割舍,反而让久美子有些畏怯。丽奈对明日香的感想大概也跟自己相去无几,正因为清楚掌握明日香的性格,才会对她做出利己主义的评语。 

    「这种人通常不会这么热心从事社团活动,不如像斋藤学姐那样退出社团,对准备考试绝对比较有利。」 

    「因为要玩社团就没什么时间念书了嘛!」 

    「既然如此,她为何对管乐这么执着呢?就连社团乱成一团的时候,也一直专注练习,肯定有什么理由。」 

    「单纯只是因为喜欢乐器吧?明日香学姐有几分乐器宅的味道。」 

    「是那样的吗?」丽奈皱着眉头思索。 

    虽然久美子不完全同意她的观点,但自己对明日香也有相同的疑问,只是没勇气直接问本人,就算问了,明日香也只会跟平常一样笑着带过。 

    「距离全国大赛只剩不到一个月,真不想还要为了演奏以外的事烦心。」 

    久美子默默点头附议。比赛在即,真不想为音乐以外的事情分心,只想全力演奏,抱着金奖回家。久美子握紧自己的裙子,看着丽奈,后者收起单字卡,抓住久美子的手说:「算了,现在说这些也没用,赶紧练习吧!」 

    她的掌心十分冰凉,久美子静静地垂下眼帘。 

    因为明日香的事,差点忘了车站大楼音乐会的脚步一天天逼近。打开行事历,日期上画着两个圈。今天是车站大楼音乐会的练习日,学校因为是假日不用上课,但比赛前的管乐社员没有假日可言。早上在音乐教室集合,进行基础练习,等泷进入音乐教室,开始当天的练习。由于是音乐会的练习,今天从指挥的方向看过来依序是明日香、夏纪、久美子三人排排坐,这种座位分配是上低音号组的基本就座顺序。附带一提,比赛时的座位是明日香比较靠近舞台中央,然后才是久美子。上低音号组基本上都固定这样坐,其他组别的成员则多半会依乐谱换位置,例如法国号或小号会依乐谱分成一部、二部、三部的声部,以北宇治高中来说,是从舞台的左侧往右侧依序站成一排。前述的乐谱分配依曲式而异,因此演奏会或比赛时经常会换座位。 

    基本上,小号的一部负责主旋律,其他人则负责相同的旋律或该旋律的和音。即使是同一种乐器也会分配到不同的任务。丽奈在比赛中负责一部,但是在演奏会上也会负责二部或三部。除了比赛以外的声部分配皆由三年级作主,也因此经常会导致学弟妹的不满。 

    「这里的长号和小号对抗起来了,要倾听整体的平衡,再稍微降低一点音量,不然实在太吵了。」 

    「是。」 

    被泷指出问题的社员回应他的指示。没时间为车站大楼音乐会练习新曲目,所以直接沿用文化祭吹奏过的乐谱,尤其是〈唱,唱,唱〉很有知名度,无论在哪个演奏会上都能炒热气氛。每次演奏会都会选择当年度流行的曲风或受欢迎的流行歌等观众比较熟悉的曲子,其中又以演歌意外受到热烈回响。由于是远比久美子他们出生更早之前流行过的曲子,不是很清楚当时到底有多受欢迎,但是对于上了年纪的听众来说,似乎比夹杂着英文的流行歌更顺耳。 

    「机会难得,上低音萨克斯风的独奏站出来吹吧!」 

    「咦?」 

    泷翻着〈搭乘A号列车〉的乐谱说道。小笠原胆怯地缩起身子。 

    「要站到前面吗?」 

    「小号的独奏也会站出来,所以你也站出来会比较帅气喔!」 

    「啊,好的,我明白了。」 

    说是这么说,但小笠原频频按着自己的脖子。她不太擅长站在人前演奏,想必很惊慌失措。 

    「那么开始练习下一首曲子。」 

    「是。」 

    久美子连忙翻开乐谱,进社团时买的乐谱夹塞了太多东西,鼓得快要炸开。她的手指在伤痕累累的透明夹上滑动,心想差不多该把不必要的乐谱移到别的文件夹了。 

    曲风愈来愈快,众人拼命吹出乐谱上的音符。泷的视线反复落在手边的谱面,指出他觉得需要改进的地方。指挥的总谱上记载所有声部的动作。久美子等人也收到比赛用的总谱,放学后,会请与自己负责的乐器吹奏同一个部分的其他成员到教室一同练习。春天时,泷还会针对练习方法进行细节的指导,如今经常放手让学生自己练习。 

    「请暂停一下,这里怪怪的。」 

    久美子闻言放开吹嘴。泷眉头微蹙,看着社员。 

    「请从F开始跟法国号同样动作的人一起吹来听听。长号、次中音萨克斯风的同学,麻烦你们了。」 

    咦?久美子盯着自己的乐谱。上低音号也是同样的动作,但没有被点到名,所以不用吹吗?久美子满头问号的同时,明日香已经举手发问:「抱歉,这里上低音号也是同样的动作。」 

    泷再看了乐谱一次,有些羞赧地莞尔一笑。 

    「啊,不好意思,我看漏了。也麻烦上低音号的同学。」 

    不知道为什么,上低音号的存在很容易被忽略。是因为乐器太没存在感吗?还是因为自己是上低音号组的人,才会特别记得被忽略的事? 

    要是上低音号也能更有存在感就好了。久美子边想边拿好乐器,只见低音管的学姐回过头来用嘴形说:「我们也好不到哪里去。」一旁的夏纪露出苦笑。 

    合奏结束后,自然而然分成留下来练习的社员与回家的社员。绝大部分的社员都会留下来继续练习,要上补习班或有事的社员则火速离开音乐教室。难得来参加假日练习的明日香也一溜烟提着乐器盒回家去了。 

    「咦,学姐今天也要留下来吗?」 

    乐谱夹在腋下的夏纪被久美子的问题吓到僵住,希美拿着长笛站在一旁。 

    「欸,啊,嗯,是打算留下来。」 

    「学姐最近经常留下来练习。」 

    夏纪要上补习班,所以暑假很少留下来练习,但最近几乎每天留下来练习。就连一向对社团活动有些冷淡的夏纪也终于萌生干劲了吗? 

    希美口齿伶俐地代替沉默不语的夏纪回答:「我请她陪我练习。毕竟我们明年就升三年级了,得努力成为靠得住的学姐才行。」 

    「这样啊。可是很少在音乐教室看到你们。」 

    「因为这里人太多了,我们都在三年级的教室练习。」 

    希美露出洁白的牙齿笑着解释。有道理,在音乐教室练习的话,因为有其他社员的声音,听不见自己的声音。像希美她们那样在别的教室练习也是个好方法,自己也去别的教室练习吧! 

    「对了,你不准来喔!」 

    夏纪仿佛看穿久美子的想法,先声夺人。或许是觉得她的话说得太严厉了,希美打圆场地接着说:「因为我们不会练习比赛的曲目,不想拖累你,就让我们两个人自己练习吧!」希美说完,对夏纪眨了眨眼睛。 

    夏纪抱着乐谱夹的指尖异常用力,久美子虽然觉得有点不太对劲,但也没有再追问下去。 

    「嘣——嘣嘣嘣,嘣——。嘣、嘣嘣——」叶月哼着低音号的音阶。 

    「叶月心情很好呢。」绿辉啃着三明治,笑着说。一旁的丽奈正用吸管喝柠檬汁。 

    第二个假日练习的中午休息时间,四个人聚集在低音组的练习教室里,悠闲地共进午餐。 

    「与其说心情好,不如说是期待。」叶月眉开眼笑地说,豪爽咬下一口红豆面包。 

    「期待什么?」久美子问道。 

    叶月摇晃着双脚回答:「周末的车站大楼音乐会!」 

    「这么说来,就是这个周末了。」 

    丽奈从小皮包里拿出行事历,蓝色封面清爽大方,角落还贴着小号的贴纸。 

    「我好期待能和大家一起参加演奏会。」叶月说完,将剩下的红豆面包放进嘴巴里。 

    京都站的车站中,JR西日本的乌丸中央口称为「京都站大楼」,夹在东侧与西侧之间的中央穿堂是个十分宽敞的挑高空间,由使用了四千片玻璃的建筑正面与巨大的屋檐构成。西侧的大楼梯也用来做为音乐会或活动会场,久美子国中时经常在那里演奏。 

    「演奏会真的好开心,小绿也兴奋起来了。」绿辉笑着说。然而,丽奈的表情蒙上一层阴影。 

    「这是三年级最后的演奏会,结束以后,就只剩比赛了。」 

    听到这句话,四个人不约而同沉默了下来。各自的心里大概都浮现出学长姐的脸吧!久美子默默喝了一口茶。丽奈说的没错,三年级将在全国大赛后退休。这也是在明日香旁边吹上低音号最后的机会了。 

    「怎么突然这么安静?」 

    「哇啊!」 

    久美子突然从背后被一把抱住,连忙将便当盒放回桌上。回头一看,希美的手臂正挂在久美子肩膀上。 

    「希美,你突然发什么神经。」夏纪目瞪口呆地说。 

    不知何故,优子居然在一旁猛点头。 

    「我懂我懂。三年级一旦不在了,简直是世界末日呢!啊,万一再也见不到香织学姐,我要靠什么活下去呢?」 

    「香织学姐不在了以后,你这喳喳呼呼吵死人的德性或许也能收敛一点吧!」 

    「什么?我哪里吵了?」 

    霙瞥了开始在后面针锋相对起来的夏纪与优子一眼,小碎步走到希美身旁。先是直勾勾盯着久美子,然后眨了好几下眼睛。 

    「正式比赛,要加油喔!」 

    「啊,嗯,我会加油。」 

    受她影响,久美子也眨了眨眼睛。这么一来,夏纪叹息着说:「社团活动也很重要,但你们的功课没问题吧?」 

    受到学姐的忠告,叶月苦着一张脸说:「不要提醒我还有模拟考这种讨厌的事啦!啊,又要被我妈骂了。」 

    「小绿也很不妙!久美子如何?」看到久美子无言以对的反应,叶月露出坏心眼的笑容。 

    「难不成久美子也不太妙?」 

    「数学有点……」 

    「耶!同病相怜。」 

    看着互相舔舐伤口的低音三人组,丽奈傻眼地叹了一口气。 

    转眼间就到了车站大楼音乐会当天。社员将打击乐器堆上卡车后,搭电车前往会场。低音号和低音大提琴都搬上卡车了,但因为容量的问题,上低音号上不了车,久美子只好徒手搬运自己的乐器到会场。整群人都穿着表演用的服装,非常显眼,车厢内不断朝他们射来好奇的视线,久美子稍微缩了缩身体。 

    「明日香学姐今天好像要先走,说她表演完就得马上回去。」 

    站在一旁的丽奈微微皱眉,手指静静抚摸吊环表面。 

    「听说了。大概是不早点回去的话,会被她母亲识破吧!」 

    「话说回来,明日香学姐真的能参加比赛吗?」 

    「咦?」 

    久美子对丽奈的问题咽了一口口水。丽奈将头发塞到耳后,压低声音,不让周围的人听到。 

    「你想想嘛,光是在这里的演奏会都有时间限制了,全国大赛可是在名古屋举行喔,不可能瞒着父母去吧!」 

    「这、这倒也是……」 

    「今天的音乐会还有夏纪学姐,所以就算明日香学姐不在,上低音号也还有两个人。可是,万一明日香学姐不来比赛,上低音号就只剩你了,人数根本不够。」 

    丽奈一语道破,久美子为之愕然。久美子一直以来都很担心明日香的决定,但都是以她会参加比赛为前提,像是再缺席练习下去,比赛时能否继续保持高水准的演奏。久美子的担忧皆与明日香正式上场的表现有关,完全没有她不能参加比赛的想法,或许是下意识不愿意去想这个可能性也说不定。 

    「明日香学姐不参加比赛吗?」久美子的嘴唇很干,询问的声线微微颤抖,听起来非常窝囊。丽奈理所当然地点头。 

    「大家就是怕最后变成这样吧!不过就算问她,她也不会正面回答。老实说,是因为她过去对社团有很多贡献,大家才对她睁一只眼、闭一只眼,换成明日香学姐以外的人,肯定没有这种待遇。」 

    「这倒是。」 

    「泷老师对这件事又是怎么想呢?我不认为他会这样得过且过拖到正式比赛那天。我希望在最完美的状态下进军全国,不想被扯后腿,这么想是不是太无情了?」 

    「可、可是明日香学姐吹得那么好,不至于扯后腿啦!」久美子反驳。 

    丽奈冷冷看了她一眼,深深叹了一口气。 

    「明日香学姐的确很厉害,可是她不来练习的话,也会影响到全体的士气。不能全员到齐一起练习,演奏的完成度也会下降。如此一来,明日香学姐的存在对整个社团反而是扣分的。」 

    久美子这下子无法反驳了,她低头看着脚边的黑色乐器盒,再看看丽奈的脸。丽奈的神情十分真摰,并没有要数落明日香之类的负面情绪,她说的没错,只是戳破了久美子不想面对的事实。 

    「或许吧!」 

    久美子移开目光,丽奈再次叹息。正当她开口想继续说些什么的时候,车掌告知站名的广播声响起。京都站到了,大批乘客从车内涌向月台。丽奈望了窗外的指标一眼,面向久美子。 

    「……走吧!」久美子无言点头。 

    为了不挡住通路,社员在用路人比较少的走道上打开乐器盒。上低音号不要拿太久就不会觉得重,但是长时间提着的话,提把会陷进掌心里,痛得要死。 

    「这个乐器未免也太重了吧!」叶月从卡车上搬下低音号,气喘如牛地抱怨。 

    「话说回来,观众挺多的耶,就连楼梯都挤满人。」绿辉兴奋地说。装在黑色软壳盒子里的低音大提琴看起来比平常还要巨大。「今天也要加油喔!乔治。」绿辉兴高采烈地对乐器说话。对了,她口中的乔治是绿辉为乐器取的名字。 

    其他的团体也在走道后面忙着准备。久美子转头望着那个方向,喃喃自语:「我记得今天参加的学校有我们和清良女中,还有……」 

    「久美子!」 

    熟悉的声音打断了久美子的话,她下意识回过头去,梓穿着如立华注册商标的水蓝色制服朝她冲过来。绿辉与叶月一脸好奇宝宝的样子,凝视着梓。 

    「梓,好久不见。对了,立华今天也要上场吗?」 

    「对呀!立华、清良女中、北宇治。嗯,真是豪华的阵容啊!」 

    梓发出咯咯咯的愉快笑声说道。梓国中和久美子同样是管乐社的成员,后来考上管乐强校立华高中。上次和久美子见面是在关西大赛的时候,当时立华高中是银奖,平日相当坚强的梓也和社员一起大哭。久美子每次想起她悔恨的表情,胸口就会掠过一股尖锐的痛楚。 

    「北宇治要去全国了呢!哎呀,真是想不到啊,恭喜你了。」 

    「啊,谢谢。立华的行进乐队比得如何?大赛是上周对吧?」 

    「行进乐队完全没问题,今年也要进军全国!」 

    梓喜上眉梢地笑开怀。久美子也忍不住绽放笑意。 

    「真有你的。」 

    「还好啦,暑假不眠不休的努力总算有价值了。」 

    「立华的练习真的很辛苦吗?」 

    「不是开玩笑的,学姐也超凶的。」 

    梓装模作样地抖了抖身体,久美子回忆立华高中在太阳祭时的表演。的确,要达到那种程度的运动量,想必练习也很耗费体力。北宇治大都坐着演奏,几乎不用做行进的练习,但是要同时参加管乐比赛和行进比赛的学校真的很辛苦。 

    「就算再辛苦,只要能参加全国大赛,就完全不是问题。话说回来,我就是为了接受严格的训练才去立华的。」 

    「哦,立华的练习果然很辛苦啊!」 

    从刚才就掩不住兴奋的绿辉突然插入两人的对话。梓吓了一跳地看着绿辉,脸凑向久美子问道:「你朋友?」 

    「嗯,这位是……」 

    绿辉——话到嘴边,久美子赶紧吞回去。因为绿辉不喜欢自己的名字。 

    「这位是?」梓不解地反问。 

    久美子连忙掌心朝上,介绍两人给她认识。 

    「呃,这位是负责拉低音大提琴的小绿,吹低音号的是叶月。她们也都是一年级。还有,小绿是圣女毕业的。」 

    「欸,好厉害!」 

    梓瞪大双眼。绿辉一如往常地浮现出人畜无害的笑容,冲到梓身边说:「小绿最喜欢立华的演奏了。近年来,尤其喜欢四年前在府大赛上的演奏。在太阳祭的时候也大饱眼福,对吧?叶月。」 

    「嗯、嗯。」 

    相较于笑容可掬的绿辉,叶月的表情则有些阴郁,指尖拽着自己的蓝色西装外套,将笔挺的布料抓出皱褶。 

    「梓,该走喽!」 

    梓面向呼唤自己的方向。定睛一看,立华的学生在远处喊她的名字,手里拿着亮晶晶的长号,大概是同一组的人。 

    「抱歉,有人在叫我,我得回去了。我很期待你们的表现。」 

    「嗯,我也很期待你们的表演,加油。」 

    「久美子也是。」 

    梓挥挥手,跑向立华的集团。望着梓的背影,绿辉欣羡地说:「好好噢,立华的衣服好可爱。」 

    「确实很可爱。」 

    「不过,我也很喜欢北宇治的衣服就是了。」 

    绿辉抓住西装外套的下摆,转了一圈。过程中,叶月始终一言不发。久美子硬是盯着她低垂的脸追问:「你从梓出现以后就怪怪的,怎么啦?」 

    「没、没什么。」 

    叶月逃避地背过脸。绿辉一把抱住她正要打开乐器盒的腰。 

    「欸,才不是没什么呢,发生什么事了?」 

    绿辉目不转睛地直视叶月的双眼。「都说没什么了。」叶月又摇摇头,大概是拗不过死活不肯放开自己的绿辉,微微耸了耸肩,手指不知所措地搔着脸颊。 

    「因为人家不像你们是老手,觉得有点无地自容。」 

    「无地自容?」 

    「小绿不懂你的意思。」 

    久美子和绿辉都感到不解,叶月苦笑着解释:「我是说,我不像小绿或久美子那样,从国中就参加管乐社,对管乐也不是很有研究。该怎么说呢,担心加入你们的对话,可能会让人觉得我不懂装懂。只是很无聊的原因,所以才说什么都没有,别放在心上。」叶月一口气说完,掰开绿辉抱住自己的手。绿辉不满地鼓起脸,拍拍她的背。 

    「叶月好傻。」 

    「你、你说我傻……」 

    叶月一脸困惑,绿辉不当一回事地大声说:「老手不见得比新手厉害,也不是有经验的人就会演奏得比较好。努力的人只要半年就能变得很厉害,反之要是偷懒,有再多经验也不会进步。叶月在意的点真的很无聊。对吧?久美子。」 

    「嗯、嗯。」久美子突然被问到,连忙点头。 

    「再说了,」绿辉抱着胳膊,继续说教。「叶月太在意自己是初学者的事了。问题是每个人都是从初学者开始的。何况叶月已经是低音部引以为傲的战力,可以抬头挺胸以自己为荣!懂不懂?」 

    绿辉的脸凑近叶月。大概是身高不够,绿辉拼命踮起脚尖,身体不住抖动。叶月目瞪口呆地看着绿辉,再看看久美子。久美子莞尔一笑,站到绿辉旁边。 

    「没错没错,叶月可以更有自信一点喔!」 

    「瞧!久美子也这么想。」 

    「嗯,我也这么想。」 

    「看吧!」绿辉乐不可支地说,不由分说抓住叶月的手。手指紧紧陷进叶月被太阳晒黑的掌心。 

    叶月轮流注视她们的脸,有些腼腆地微微一笑。 

    北宇治高中排在立华高中后面出场。平常边吹乐器边动个不停的立华表现得非常厉害。在后台预备的时候,感觉会场上还残留着方才演奏的余韵,椅子上也还残留着刚才坐在上头的女学生的余温。 

    车站大楼音乐会的舞台设置于大楼梯中段的广场,观众坐在阶梯上欣赏演奏。或许是因为立华和清良女中都来了,楼梯上挤满人,其中还可以看到中小学生的身影,恐怕是为了欣赏优秀学校的演奏才整团整团地带过来吧!也有很多观众是碰巧路过,受到演奏的吸引上前围观。 

    北宇治高中管乐社共八十一名成员,分别站在广场的舞台上下预备。久美子等人拿着乐器,站在舞台下面的后方,小号及长号的成员在舞台上一字排开。一开始负责独奏的小笠原大概很紧张,扭扭捏捏地站在泷身边,看起来坐立不安。 

    「别担心。」或许是注意到久美子的视线,明日香对她说。 

    「别看她那样,她毕竟是社长。」 

    既然明日香都这么说了,肯定没问题。久美子移开看着站在前方的社长的视线,重新面向乐谱。 

    泷对观众一鞠躬,场上的喧闹扰攘慢慢沉淀下来,会场鸦雀无声。泷举起指挥棒的同时,久美子等人也拿好乐器。白色指挥棒微微一震。久美子吸气,将气息送进吹嘴。 

    乐曲始于轻快的旋律,慢慢加快节奏,悦耳的音符在舞台上弹跳,营造出非常活泼的气氛。观众席自然而然打起拍子,老实说,完全跟不上节奏。在演奏会上,观众打拍子和乐曲的节奏各自为政是常有的事。 

    曲子行云流水地来到上低音萨克斯风的独奏。小笠原吹出第一个音符的瞬间,久美子整个人呆掉了。因为那个音充满了放克风味,小笠原演奏的音乐光灿耀眼,远胜于过去在比赛或练习时的每一次演奏。该怎么说呢,总之是帅气逼人。 

    上低音萨克斯风的低沉音色让会场内的空气为之震动,观众发出愉快的呼声。说不定爵士乐才是小笠原的拿手好戏。交织着即兴演出的演奏炒热气氛,久美子不仅佩服,还有些目瞪口呆。偶尔会看到一演奏就变了个人的演奏者,没想到小笠原也是其中之一。久美子望着情绪沸腾的观众,拼命吹出自己分配到的部分。 

    车站大楼音乐会的编制与文化祭时几乎一模一样,由明日香介绍曲目,再进行演奏。演奏受到观众热烈的回响,以如雷的掌声欢送久美子等人退场。还有小学生高喊「全国大赛要加油喔!」好像是打击乐学姐的妹妹。 

    「请问我姐姐在吗?」 

    泷宣布解散后,那个妹妹悄悄出现在广场上,可爱的模样令学长姐们欢声雷动。 

    「小悠,你妹来了。」 

    「欸,不会吧!」 

    其他的三年级帮忙叫人,被称为小悠的学姐冲向妹妹。刚把乐器搬上卡车的学姐额头上冒出一层薄薄的汗水。妹妹一看到姐姐的身影,就露出满脸的笑容。 

    「你一个人来?很危险耶!」 

    「不是啦,人家是跟爷爷一起来的。因为人家说想看姐姐帅气的样子,爷爷就带我来了。」 

    「这样啊,可是你现在就一个人落单啦,爷爷呢?」 

    「不知道,爷爷该不会迷路了吧?」 

    「欸……那我们一起去找他。」 

    学姐说完,牵起妹妹的手。妹妹或许是很以穿着演奏服装的姐姐为荣,兴高采烈地笑个不停。久美子仿佛在她们身上看到以前的自己,不由自主地垂下眼。 

    「晴香,我先走了。」 

    「没问题。已经解散了,你可以直接回家。」 

    「谢啦。」 

    姐妹俩手牵着手下楼。久美子怔忡望着她们的背影,冷不防有人从背后拍她肩膀。 

    「久美子,快点去看清良女中的演奏。」 

    回头一看,丽奈正提着装有小号的乐器盒看着自己。绿辉和叶月也在她身后。低音号和低音大提琴都已经上了卡车,所以两人手里只拿着皮包。 

    「啊,抱歉,不小心发呆了。」 

    久美子说道,抓紧乐器盒的提把,提把因乐器的重量陷进皮肤里,掌心掠过轻微的痛楚。 

    清良女中的学生全都穿着白色的西装外套搭长裤,系着黑色的细领带,长发扎成马尾,打扮得十分干净俐落。因为是女子高中,社员全都是女生。 

    久美子认得坐在舞台后面吹法国号的少女,是上次出现在电视上的清良女中管乐社长。久美子下意识伸直背脊,想看清楚她。舞台前的楼梯已经人山人海,久美子她们只能站着看,从广场上可以勉强看到舞台。绿辉个子太矮,几乎什么都看不见,有个中年妇女大概是觉得她很可怜,让给她一个好位置。 

    立华及北宇治的学生都想听演奏,所以广场前的走道上人满为患。再加上是大老远从九州前来演奏的管乐强校,观众人数比前面演奏的两所学校还多。强校的光环果然不同凡响,社员抬头挺胸,目光如炬地看着指挥者,让看的人感受到某种压迫感。社员人数恐怕是北宇治的两倍以上,多到快要从舞台上满出来。从闪闪发光的号口传送过来的声音十分厚实,毫不留情地展现出与北宇治的实力差距。 

    「那么,接下来是今天这场音乐会的最后一首曲子。」 

    担任司仪的女学生用麦克风向观众席报告。「欸……」周围发出不满的声浪,司仪笑着回应:「全力以赴演奏到现在,各位觉得如何?能有机会在京都演奏,全体社员都觉得非常光荣。感谢各位今天来听我们演奏。」 

    女学生说到这里,低头行礼。观众对她送上热烈的掌声。她抬起头来,脸上浮现出甜美的笑容。 

    「最后这首曲子是〈马多克的最后一封家书〉。」 

    女学生语声未落,顾问紧接着缓缓挥下指挥棒,低沉柔和的音乐交织成静谧的旋律,温润的旋律流淌在会场内。 

    这首〈马多克的最后一封家书〉由管乐社员熟悉的樽屋雅德作曲。马多克是豪华客船《铁达尼号》的一等船长,同时也是直到沉船的最后一刻,都还努力拯救乘客的船员之一。他在航海时,每天都会写信给家人,这首曲子就是把他的信编成爱尔兰风的旋律。 

    慢板的曲调再加上单簧管及铜管,音乐的规模一口气变得壮阔,每个音符都璀璨生辉,孕育出悦耳动听的旋律。清良女中演奏的音乐明显与其他学校不同。久美子不知道理由何在,但是清良女中的乐器释放出来的每一个音符无疑都华美无双,会场内充满了让人心荡神驰的甜美音色。 

    随着铃鼓刻画出的节奏,曲风摇身一变,打击乐器与轻快的铃鼓重复着你来我往的竞演,营造出活泼的气氛。充满节奏感的铃鼓贴着旋律共舞。铜管的音色十分协调,渗入以上的旋律里,让音乐显得热闹非凡。然后再由木管静静加入渐弱的音符,场上逐渐趋于寂静。 

    法国号的声音响彻会场,以其余韵为背景音乐,单簧管温润的旋律响起。然后换长笛变成曲子的主角,银色的乐器织出忧伤的曲调。众人配合指挥者的暗示吸气,曲风迎向崭新的局面。钟声响起,众人的视线都集中在指挥者身上,只见指挥棒朝着打击乐器轻盈挥动。快节奏的音乐与钹和定音鼓一起流淌在会场里,猝不及防地打破平静的空间,与具有特色的主旋律融为一体,袭向观众。撼动人心的音浪带着分量十足的声响,确实地传送到久美子她们所在的广场。气贯丹田的沉稳重低音一波接着一波,再叠上单簧管轻盈的声音,蹦蹦跳跳地加速前进。这时,曲子突然戛然而止,只剩中音鼓激烈的鼓声在会场中回荡。乘着那个节奏,其他乐器的音色一口气变得朝气蓬勃,重复刚才的旋律,渐渐加速。木管演奏者的手指快得令人眼花缭乱,撩起听众的焦躁情绪。久美子的呼吸不知怎地变得困难,她紧紧揪住左侧胸口的制服。 

    激昂的曲子瞬间归于寂静,四周逐渐充满了轻声细语般的平静音符。双簧管、单簧管各自演奏出感情丰沛的音色,交织成让听众安心的柔美旋律。随着指挥的动作愈来愈大,音乐也变得愈来愈厚重。声音的体积愈来愈大,逐渐合而为一,整合成完美的音乐。 

    最后一个音符响起的瞬间,走廊上的人无不停下脚步。转瞬的沉默后,会场上响起几乎掀开屋顶的掌声,掌声响遍整栋车站大楼,在建筑物内回荡。久美子俯瞰舞台,深深地叹息,这时才发现自己竟然看到忘了呼吸。 

    坐在楼梯上的人群中,也有不少人正用手帕按着眼角。久美子则用指尖压紧灼热的眼头。 

    「太神奇了。」 

    站在一旁的叶月用面纸大声擤鼻涕。她眼睛红通通,脸上还残留着几缕泪痕。 

    「明明是听都没听过的曲子,泪水却莫名其妙地自己流下来。演奏技巧高超的学校,真的好厉害啊!」 

    叶月说完又擤了擤鼻涕。〈马多克的最后一封家书〉不是流行乐,一般观众并不熟悉,即便如此,还是能让这么多人像这样泪流满面,献上赞赏的掌声,除了曲子本身很好听以外,也多亏清良女中空前绝后的表现能力。就算吹奏同一首曲子,北宇治也无法让人感动到这个程度。就算使用同一种乐器,吹奏同一首歌,完成的曲风也截然不同。久美子认为这就是音乐有趣的地方,同时也对这个事实感到有些害怕,需要一点勇气才能直视自己与别人力量的落差。 

    「这就是全国大赛的金奖水准呢!」丽奈喃喃自语。 

    绿辉一脸兴奋地从不远处冲过来。强校迷的她似乎对刚才的演奏满意极了,双颊泛着红晕,猫毛似的头发也有些凌乱。 

    「啊,今天真是完美的一天!」绿辉说道,心满意足地捧着自己的脸颊。 

    叶月将面纸揉成一团,塞进皮包里的塑胶袋。 

    「清良女中今年也要参加全国大赛吧?老实说,我一直觉得北宇治A部门的人都吹得非常好,可是,正所谓人外有人。」叶月语重心长地嘟嚷。 

    久美子也望向舞台。清良女中的学生已经迅速开始准备撤退,动作丝毫不拖泥带水。那就是每年在全国大赛拿下金奖的人。久美子悄悄叹了一口气。 

    她们沐浴在聚光灯下,看起来好成熟,实在难以想象与自己同年。 

    「接下来是基础练习的三号曲。」 

    「是。」 

    这是车站大楼音乐会后的第一次合奏。今天是平日,但提早下课,所以可以合奏。明日香一如往常站在前面,指示社员。合奏时,社员会先进行基础练习,然后再由泷指导。A部门和B部门的成员都要参加基础练习。众人皆已熟悉春天发下来的大量基础练习用乐谱,不用看谱就能回答哪首曲子是几号。 

    「一、二、三、四。」 

    打击乐器的成员配合节拍器刻画出规律的节奏,这个乐谱要意识到运用嘴唇的圆滑音,久美子对此很不擅长。因为有很多明明是不同的声音,却要运用相同指法的地方,所以要吹出圆滑音时,只能运用腹肌及运气方式硬生生改变音色才行。上低音号不像直笛那样指法随音阶而异,所以并非光用手指就能发出随心所欲的声音。 

    「接着是七号曲。」 

    「是。」 

    七号曲是运舌法的乐谱。利用舌头让空气瞬间停止,清楚切断每一个音符。如果以缓慢的节奏进行,倒不是太难,一旦换成快节奏的曲子,难度就会一口气提高。明日香经常苦口婆心地提醒她,上低音号的声音轮廓很容易晕开,所以练习发出干净的声音就显得格外重要。 

    明日香之后也继续云淡风轻地做出指示。当天的基础练习要演奏哪个谱面,全由明日香定夺。社员遵照她的指示继续演奏,一句怨言也没有。 

    泷来这所学校已经过了半年,随着练习次数增加,起初不习惯的基础练习也进行得愈来愈顺利,除此之外,在各方面也有许多事物都因为社员的自动自发而变得制度化,社员已经养成独立思考的习惯,即使没有顾问的指示,社团活动也不会停滞不前,或许到了下一代,练习就能进行得更顺畅吧!从摸索开始的各种尝试,教会他们选择有效的作法、舍弃没效的,只把有效的作法传承给下一届。北宇治高中几年后肯定会跟其他强校一样,成为以学生为主体,由学生主动参与的社团。 

    「接下来以踮脚法练习十三号曲。」 

    「好的。」 

    她口中的踮脚法,是由外聘的指导者桥本引进的方法,意指稍微踮起脚尖来吹奏乐器的练习法,这时必须让重心落在腹部,支撑全身的重量。对久美子而言,这种练习比仰卧起坐更吃力,一直持续相同的动作,肌肉会发热,脚尖还会开始发抖。久美子内心暗藏着疑问,这么吃力的练习真的有效吗?因为姿势十分标准的明日香吹奏同样的乐器时,看起来毫不费劲,真是不可思议。 

    「那么,基础练习到此为止。」 

    「谢谢副社长。」 

    众人一起向明日香道谢。之后,B部门的成员抱着乐器移动到视听教室。明日香随人潮走向久美子旁边的座位,银色的上低音号擦得光可照人,轮廓分明地倒映出低着头的明日香。久美子下意识按住自己的上低音号活塞阀,望向明日香。 

    「那个,明日香学姐。」 

    「嗯?什么事?」 

    感觉周围的视线正往这边集中,一旁吹法国号的学姐咕嘟地咽了口水,其他的学长姐则一脸若无其事地继续练习。明日香的神情与往常无异,眼睛底下浮现出淡淡的黑眼圈。 

    明日香学姐真的能参加比赛吗? 

    丽奈说过的话在久美子脑海中苏醒。久美子紧紧抓住自己的裙摆,手背渗出一层薄薄的汗水。 

    「我那天不小心看到学姐和令堂争执的过程了,学姐真的不会退出社团吧?」 

    明日香眯起眼,修长的手指轻抚上低音号的表面,剪得很整齐的粉红色指甲闪烁着暧昧的光泽。 

    「那天是指我妈来学校那天?」 

    「是、是的。」 

    「嗯哼,原来如此。」 

    明日香摩挲着自己的下巴,陷入沉思,唇畔流泄出深深的叹息。是不是问了不该问的问题?久美子吓得浑身僵硬。明日香看也不看她一眼,以跟平常没两样的声音回答:「关于这点不用担心,我也有我的打算。」 

    「真的吗?可以相信你吗?」 

    「嗯,没问题、没问题。」 

    明日香夸下海口,朝她挥挥手。她的回答太过轻佻,久美子反而感到不安,但也不敢继续追问,只能凝视着眼前的学姐。明日香瞥向泷不在的指挥台,脸上浮现出意味深长的笑容。 

    「……久美子,你这次的测验没问题吧?」 

    「什么?」 

    不明白她的意思,久美子下意识反问。明日香正经八百地重复同一个问题。 

    「我是说,不只社团活动,你的功课也没问题吧?」 

    「哦,呃,那个,这个,数学不太妙……」 

    「既然如此,要不要我教你?」 

    「咦?」 

    出乎意料的发展令久美子忍不住惊呼出声,只见躲在人群后面的秀一噗哧一笑。 

    「不愿意?」明日香直勾勾地盯着久美子,娇艳欲滴的唇畔轻泄的气息,不知怎地令久美子一阵脸红心跳。明日香撩起长发,散发出甜甜的香味。 

    「倒、倒也不是不愿意……」 

    结果久美子还是屈服了。她没有足以对抗明日香的强韧意志。或许是很满意她的顺从,明日香笑眯了双眼。 

    「那下次放假的时候来我家。」 

    「欸,要去学姐家吗?」 

    「怎么?不行吗?」 

    明日香的嗓音突然变得低沉,久美子急忙把头摇成一个波浪鼓。 

    「怎么可能不行,只是有点紧张。」 

    对久美子而言,即使是再容易亲近的学长姐,也是高不可攀的存在,不管再怎么热络,都不可能变成朋友。在久美子心中,学长姐与同学之间有一条明确的界线,绝不能跨越那条界线。 

    到对方家里等于是私下交流,如果是绿辉那种跟谁都好的性格,大概不会觉得有什么不妥。久美子望着低音大提琴的方向,或许是察觉到她的视线,绿辉朝她挥挥手。 

    「话先说在前面,只准你一个人来。」仿佛看穿她的心思,明日香撂下这句话。 

    学姐都这么说了,已经没有退路。久美子下定决心,直视学姐的脸。「那就请您多多指教了。」 

    明日香对鞠躬致意的久美子回以满意的微笑。 

    离开车站时,时针已指过九点。车辆行驶过宇治桥的灯光,照亮了暗夜的道路。马路两旁的茶坊皆已打烊,隔着玻璃可以看到店内静悄悄的。过了桥,背对宇治川的地方有一座紫式部\footnote{日本平安时代(约西元七九四至一一八五年)的女性文学家。}的雕像,雕像的轮廓十分光滑,给人柔和的印象。久美子瞇着眼睛,坐在雕像前方的石头长椅上,没人经过眼前的斑马线,红绿灯徒劳地反复从绿灯变成红灯。

    巨大的鸟居竖立在邻接平等院通的府道入口,府道通往县神社,县祭时人满为患,平常倒是没什么人经过,所以这个时间只有车子呼啸而过。久美子心不在焉地盯着一闪一闪的煞车灯,叹了一口大气。 

    「听说你要去田中学姐家玩?」 

    久美子抬起头,秀一就站在面前,以粗鲁的动作将体育用品袋往地上扔,在她身旁坐下。一股汗臭味袭来,久美子皱眉,秀一辩解:「刚才被其他人拖着一起跑步。」 

    「跑步?肯定是玩捉迷藏吧?」 

    约莫是说中了,秀一难为情地搔搔头。一年级的男生利用休息时间在中庭玩捉迷藏是很有名的事,他们的借口是可以借此锻炼肺活量。 

    久美子抱着头,发出宛若呻吟的哀号。 

    「啊!感觉一切的一切都好讨厌。」 

    「为什么?这么不想去田中学姐家吗?」秀一看着她。 

    「不是的。」久美子按住自己的额头。「总觉得太多事超出我的负荷。」 

    泷的事、明日香的事、还有整天关在家里的麻美子,那才是最令人头痛的事。一次发生太多事了,久美子的大脑根本处理不过来。距离比赛只剩不到一个月,久美子实在不想为这种事烦恼。 

    「对了,听说麻美子姐回来了?」 

    「谁告诉你的?」 

    「我听伯母说的。」 

    「我妈?」 

    老妈真多嘴。久美子忍不住叹气。秀一不解地歪着脖子。 

    「大学不是已经开学了吗?」 

    「我姐说她不念了。」 

    「欸,为什么?」 

    被秀一这么一问,久美子嘟着嘴,闹起别扭来。 

    「我怎么知道。她什么也没说,每天关在房间里。」 

    「大概是有什么苦衷吧,你不问问看吗?」 

    「要问吗?」 

    「当然要问啊!」 

    明明没有任何根据,秀一却斩钉截铁地这样说。久美子抱紧放在膝盖上的书包,把脸埋进去,拉链贴着脸颊,冷冰冰的。「唔……」久美子发出意味不明的声音,秀一轻拍她的背。 

    「你们是家人吧,家人有困难的时候不是该支持对方吗?」 

    这句话让久美子缓缓抬起头来。 

    「如果是你的话,你会问吗?」 

    「会吧。不过我没有兄弟姐妹,所以也不是很确定。」 

    秀一不知所措地搔着头回答。或许因为宇治川就在附近,掠过脸颊的风好冷。停在红绿灯前的车子隔着窗户传来嘈杂的音乐,重低音节奏吹散夜晚空气,瞬间打破寂静。 

    家人。久美子在嘴里咀嚼这个字眼。她有勇气跨过这条挖了好几年的鸿沟吗?久美子抓紧裙摆,再度趴回书包上。感觉身旁的秀一局促不安。久美子用力闭上双眼,静静等待那阵喧闹过去。 

    直到信号变成绿灯,车子都一动也不动地停在原地。 

    后来又聊了一会儿,两人就回家了。明天还要练习,不能一直待在这种地方。临别之际,久美子对秀一挥手道别,秀一也朝她大力挥手,还对她说:「加油喔!」久美子深深叹息。该对什么加油才好呢?这种事不用说她也知道。 

    久美子从走出电梯伸手放在玄关门把上的那一瞬间,就有股不祥的预感。 

    「麻美子,你到底在想什么?」 

    母亲的声音从客厅里传来。声线之低沉,充分显露出她是真的生气了。久美子尽可能不发出声响,蹑手蹑脚踩在地板上,悄悄从门缝偷看屋子里,只见爸妈和麻美子分别坐在桌子两边,木制餐桌上有个大信封,角落用斗大的黑体字印着经常可以在电视广告上看到的专科学校名称,是要让久美子去考吗? 

    「久美子,你在干什么?」 

    「呃。」 

    被父亲点到名,久美子僵在门前,看样子早就穿帮了。久美子为了掩饰尴尬,面无表情地走进客厅。屋子里的气氛异样沉重,秒针移动的声音听起来吵死人了。久美子假装没听到,把书包立在沙发旁边。客厅里充满了不寻常的紧张感,她想若无其事地回自己房间,但似乎无法如愿,又没有勇气坐在姐姐旁边,久美子只好坐在沙发上玩手机。 

    母亲狐疑地看了久美子一眼,视线随即转向桌上的信封。 

    「你是认真的?」 

    「我从一开始就是认真的。」 

    「就算你这么说,妈也不答应。」 

    久美子的指尖在手机萤幕上下滑动,意识却集中在家人身上,一个字也没看进去。 

    母亲按着太阳穴,大大叹了一口气。麻美子的眉毛心浮气躁地微微跳动。 

    「你还是面对现实吧!你已经二十一岁了,只要稍微想一下,就知道怎么做对你的将来比较好。」 

    「我想过了。这是我深思熟虑后的结果。」 

    「这么重要的事,怎么可以擅自决定……不想去学校的话,可以休学半年啊,没必要就这样不念吧?」 

    「我不是因为不想念大学,我是认真的。」 

    「就算你说你是认真的……」 

    麻美子涂成红色的指甲扫过桌面,房间里响起咯吱咯吱的呆板噪音。父亲始终不发一语,只是目不转睛盯着信封。麻美子闭上一度张开的嘴,若有所思地安静下来,视线有一瞬间射向久美子这边,后者连忙假装玩手机,避开她的视线。 

    「我想当美容师。」麻美子说道。 

    久美子大吃一惊,手机从手中滑落,发出咚的一声巨响,只有父亲瞥过来一眼。 

    「我已经办妥报考的手续了,明年就要去念这所专科学校。」 

    姐姐是说真的吗?久美子赶紧吞下差点脱口而出的问题。 

    母亲摇摇头说:「明年就要找工作了吧?别说傻话,给我专心读大学。这是为你好。」 

    麻美子冲动地站起来。受损的棕色发丝沐浴在日光灯下,闪烁生辉。麻美子握紧桌上的信封,瞪着母亲。母亲大概是被她吓到了,屏住气息。 

    「为我好?」麻美子的声线很低沉,丝毫不掩饰自己的愤怒。 

    「你说就这样念到大学毕业是为我好?」 

    「至少比退学去念专科学校好。」 

    「那是对你好吧!」 

    麻美子的手作势往下挥,用力拍在桌上,发出清脆的声响。麻美子直盯着母亲,咬牙切齿地释放出自己的情绪。 

    「我从国中就一直说我想当美容师,是你不答应。过去我一直照你说的话做,几乎放弃了一切。因为从小到大,你一直把『因为你是姐姐』挂在嘴边。」 

    姐姐居然有这种想法,我都不晓得。久美子连要捡起手机都忘了,只是茫然看着气到发抖的姐姐。 

    麻美子从小就比久美子懂事很多,从她嘴里说出来的每一句话都很有道理,也因此经常让久美子恨得牙痒痒。光靠社团活动是考不上大学的。你不念书要干么?每次听到这句话,对姐姐的反抗意识就会在久美子内心萌芽。姐姐说的没错,但她会硬把自己认为正确的观念套到别人头上,这点很讨厌。因为久美子是自愿花时间在社团活动上,轮不到姐姐说长道短。久美子一直是这么想的。还以为姐姐说的每句话都是真心话,从来不曾想象过姐姐心里的想法。 

    母亲长叹一声,轻轻把手放在麻美子手上。母亲的手背刻画着深深的皱纹,暗示她的年纪也大了。 

    「这是为了你的将来着想。」 

    「什么是将来?将来真的值得舍弃现在的价值吗?」 

    麻美子那双水水嫩嫩、不知疾苦、也尚未衰老的手,急躁地划过半空。 

    「不管是转学,还是准备考试,其实我全都讨厌死了。我也想象久美子那样继续社团活动,我也不想放弃长号!」 

    咬牙切齿的台词令久美子为之屏息,遥远的过去揪紧了久美子的心脏。忆起小学时,姐姐小巧的掌心。她圆润的手,曾几何时开始只握住铅笔。为了准备考试,姐姐无时无刻埋首案前,对着教科书。久美子一直以为这是理所当然的,一直以为姐姐是自愿选择这条路。 

    「为了考上好学校,我一直用功读书,结果怎样呢?我以为只要当个乖孩子就会有回报,像个傻瓜似地拼命忍耐,结果还不是一场空?早知道会变成这样,不如全部由我自己决定。早知道就不要装乖,至少不会这么后悔。」 

    「还不知道有没有回报吧?现在放弃的话,以前的努力就都白费了。」 

    「因为再不放弃就来不及了。」麻美子气急败坏地说。她那歇斯底里的声音让久美子联想到明日香的母亲。 

    「这是最后的岔路了,要回头只能趁现在。再继续往前走,我会后悔一辈子。」 

    沉默重若千金地坠落在地板上。压迫着肩膀的张力令久美子忍不住叹息。麻美子喘着粗气,恶狠狠瞪着母亲。这是久美子有生以来,第一次看到姐姐对父母表现出这么叛逆的态度。她长长的指甲刺进紧握的拳头,光鲜亮丽的红色指甲油看在久美子眼中,显得非常廉价。 

    麻美子一直是个优等生,笔直走在世人口中正确的道路上。如今她却说她错了,认为自己过去的人生是一场错误,这让久美子惊慌失措。 

    「……为什么是现在?」 

    始终保持沉默的父亲突然开口。被汗水濡湿的衬衫衣领软趴趴的。父亲卷起便宜的袖子,重重叹了一口气。 

    「要是你早点这么说,或许爸爸也能理解。但你已经念到大三了,才突然说不想念大学,你觉得做父母的能接受吗?爸爸只觉得你在逃避现实。」 

    「才不是那样!」 

    「那你为什么现在回来?不就是因为不想找工作吗?」 

    这句话堵得姐姐哑口无言。母亲忧心忡忡地轮流看着父女俩。父亲的语气很平静,充满说服力。久美子一路听他们吵下来,不由得产生只有父亲的意见才是正确的错觉。 

    「话说回来,你以为你从小到大的学费是谁付的?还有你住在外面花的钱。对这一切视而不见,一味指责你妈,会不会太过分?爸妈对你的要求或许真的太高,但最后决定要怎么做的还是你自己,不是吗?」 

    「我……」 

    「要是你真的打算退学,就给我搬出去,自己去申请奖学金来支付生活费和学费。」 

    「等一下,老公。」母亲阻止父亲,但父亲不听母亲的劝阻。 

    「别以为可以做自己想做的事,又不用承担风险。你的想法太自我中心了。如果你是认真的,就给我看到你的觉悟。」 

    平常沉默寡言的父亲很少说这么多话。或许他也很激动,只是没表现出来。母亲苦苦哀求地抓住父亲的手,对姐姐投以稍安勿躁的眼神。母亲软弱无力的视线里,夹杂着同情与困惑。 

    「麻美子,妈妈不会要你搬出去,只希望你能冷静下来想想。你在那边一定出了什么事,才会突然说这种话。可是,只要你冷静下来,一定能明白妈妈说的才是对的。」 

    对的。这两个字在久美子的舌尖上滚动。沙沙地,苦涩的味道在口中散开。没错,父母说的话永远都是对的。正因为如此,麻美子才会一直听父母说的话活到现在。久美子的视线悄悄落在自己的掌心,干燥的皮肤表面刻画着淡淡的纹路,长长的生命线从中间以不自然的形状岔开。 

    麻美子眼睛眨也不眨地凝视着母亲的脸,她撩起自己的头发。棕色的发丝被白皙的手指扯断,掉落在地板上。在她身上,已经完全看不到高中时期曾有过一头乌黑亮丽的秀发。 

    「我去吹吹风。」 

    姐姐丢下这句话,走出客厅。原本闹哄哄的空间一下子归于寂静。爸妈各自钻牛角尖地茫然对坐。皱成一团的信封还孤零零躺在桌上。久美子伸手,指尖轻抚信封表面,粗糙的触感缠绕在指腹上。 

    「我想姐姐是认真的喔!」 

    久美子好不容易才挤出这句话。 

    「我知道。」母亲低眉敛眼缓缓说道,唇畔流泄出深深的叹息。「难道我错了吗?」 

    久美子假装没听见母亲的颓然低语。 

    久美子打开节拍器,目不转睛地盯着乐谱。对她来说,从指定曲的C大调开始的部分是一道难题。从低音突然爬升到高音F的地方总是吹不顺,都会不由自主提醒自己高音要来了。可是好不容易才分配到副旋律的啊!久美子忍不住叹息。身旁的明日香一向演奏得很轻松,猛一看还以为乐谱很简单。厉害的人演奏起来为何总让人觉得那么轻巧呢?久美子在脑海中反复重播明日香的演奏,与乐谱大眼瞪小眼。 

    叶月在一旁胀红脸练习下次演奏会的曲子。〈冬季仙境〉(Winter Wonderland)是一九三四年发行的美国流行歌曲,由李察.史密斯(Richard B. Smith)作词、菲利克斯.贝尔纳(Felix W. Bernard)谱曲,也是日本冬季必听的曲子,走到哪里都可以听到,充满了冬天的欢乐气氛。 

    叶月从大大的号口吹出低沉的声音,久美子怔忡地看着她的低音号微微震颤。叶月的演奏还称不上高明,但是充满了热爱音乐的人才有的情感,有种吸引人的特质。 

    「你的精神也太涣散了吧!」 

    「欸。」 

    突如其来的指责令久美子悚然一惊,她抬起头,看到抱着上低音号的夏纪一脸严肃盯着自己。 

    「全国大赛前怎么可以这么不专心。」 

    「对、对不起。」 

    久美子赔罪的话下意识脱口而出。虽然没有自觉,但注意力的确比平常不集中也说不定。夏纪夸张地对垂头丧气的久美子叹气。 

    「你可能没有自觉,但你今天一直心不在焉。脸也好红,是不是发烧了?」 

    「欸?久美子没事吧?」 

    夏纪这句话立刻引来绿辉的大惊小怪,直接把低音大提琴放在地上,小跑步冲过来。久美子还以为她要做什么,结果额头已猝不及防吃了一记头捶,同时响起叩的闷响。 

    「哇!听起来好痛。」 

    叶月也一脸惊吓地皱眉。久美子一手撑着上低音号,另一只手摸着额头。绿辉垂头丧气垮着肩膀说:「抱歉,久美子。我只是想帮你量体温,不小心用力过猛。」 

    「也不必用额头量吧……」卓也目瞪口呆地喃喃自语。 

    因为撞到的关系,绿辉的脸也变得红通通。久美子按着还隐隐作痛的额头,不愠不火地笑着说:「没关系,别放在心上。」 

    梨子关切地观察久美子的脸色。久美子迎向她温柔的眼光,不知怎地有些心神不宁,或许是联想到自己的母亲。 

    「要是真的很不舒服,还是早点回去吧!这个季节的感冒如果不及早治疗,可能会拖很久才好喔!」 

    「小绿也觉得这样比较好!因为久美子的额头好热。」 

    「那是被你撞热的吧!」 

    绿辉握拳,叶月冷静指出她的盲点。夏纪傻眼地耸耸肩。 

    「万一明日香学姐不来,A部门就只剩下你一个上低音号了。虽说笨蛋不会感冒,但你还是多注意自己的身体比较好。」 

    夏纪溢于言表的辛辣台词大概是为了掩饰害羞。这段时日相处下来,久美子也算摸清她不是有话直说的性格。 

    绿辉看着久美子怀中的上低音号说:「你要早退的话,我帮你把杰克放回去吧!」 

    杰克是绿辉擅自为久美子的上低音号取的名字。 

    「没关系,我自己放回去就好了。」 

    「好吧!」 

    绿辉一脸遗憾地点头。说不定她只是想摸上低音号。久美子苦笑,轮流看着围过来的社员,从他们担心的表情充分得到被关怀的感觉,内心深处一阵悸动。不知道为什么,谢谢两个字就是卡在喉头说不出来,久美子只能微微颔首。 

    久美子已经很久没有提早结束社团活动回家了,天还没黑就走在通往车站的路上,感觉好不可思议。她独自一人走在回家路上,阳光颇刺眼。大概是因为每次回家都和朋友一路打打闹闹,明明是同一条路,今天走起来却异常漫长。平底鞋踩在柏油路上的脚步声、打身边驶过的车声听起来都很无趣,全世界仿佛都黯然失色。久美子稍微加快脚步,想快点回家睡觉。 

    「久美子。」 

    突然有人叫住她,久美子回头,定睛一看,手里拿着参考书的葵正朝她招手。葵扎成马尾的黑发似乎比最后一次见到她时长了些,她恬淡地微微一笑,走到久美子旁边。 

    「怎么会在这个时间看到你,社团活动呢?」 

    「啊……我好像感冒了,大家要我回家休息。」 

    「感冒啦?还好吗?」 

    葵微侧螓首。久美子点点头,视线落在遮住膝盖的藏青色裙子上。 

    「还好,是大家太紧张了。」 

    「话说回来,管乐社好厉害啊,居然真的打进全国了。」 

    葵的唇瓣扭曲,隐含寂寥的语气,令久美子悄悄垂下眼帘。挂在书包上的乐器形状钥匙圈,闪闪发亮的金色正一脸无邪地看着自己。 

    「全都是泷老师的功劳。」 

    「泷老师真的很有一套呢!」 

    「嗯,他真的是很厉害的老师。」 

    葵噤口不言,陷入沉思。久美子不知道该说什么才好,默默偷瞄她的侧脸。直到几个月前,葵的脖子上都还挂着用来固定萨克斯风的吊带。久美子回忆她在车站前红着脸解释着「忘了拿下来」。金色的次中音萨克斯风好适合苗条的葵。她不再玩乐器了吗?她放弃音乐了吗?久美子胸口闷闷的,悄然吐出一口气。葵的双手已经被红色参考书占满,没有空间给其他东西了。 

    「对了,我在教职员办公室看到明日香。」葵猛然想起似地抬头说道。 

    久美子急忙望向她的脸。因为身高的差距,自然变成自己居高临下地看着她。 

    「明日香学姐吗?」 

    「嗯。她母亲也在,好像在吵架。」 

    「啊……」 

    领悟到发生什么事,久美子回以不置可否的反应。怕是明日香参加社团活动的事让她母亲发现了。久美子想起前几天的骚动,脸上表情蒙上一层阴影。明日香的母亲不希望她继续参加管乐社,但又觉得不方便向葵这个局外人说明那对母女的纠葛,久美子避重就轻地说:「明日香学姐好像跟她母亲有些矛盾。」 

    「是噢,真想不到。」 

    葵并未继续追究下去,只是坦诚地陈述感想。想不到?久美子不明白她的意思。 

    「因为明日香好像没有任何烦恼。脑筋好,运动神经也不差,还会玩乐器……简直受到上天太多的眷顾,所以我还以为她不会有这方面的烦恼。」 

    「没有这回事喔!」 

    大概是受到小笠原前几天的演说影响,否定句没想太多便脱口而出。 

    「明日香并不特别。」当时小笠原从咬紧牙关的口中挤出这句话。那时她一定很后悔,后悔在明日香与自己之间画下一条以特别为名的界线。 

    「……那我就放心了。」 

    「什么?」 

    她出乎意料的反应令久美子瞠目结舌。葵的唇畔挂着自嘲的笑意,踩着轻快的脚步往前跨出一步。从她脚底延伸出来的影子,与久美子的双脚交缠。 

    葵说:「原来她也是普通人。」 

    久美子回过神来,自己正站在似曾相识的地方。「是小学。」她在内心深处喃喃自语。看样子是在作梦,久美子置身事外地想。 

    这个地方充满了夕阳的味道,对于现在的久美子来说太小了。正方形的音乐教室一隅密密麻麻摆放着黑色谱架,而且是整座立式的,不是折叠式的。墙上挂着头发花白的音乐家照片,久美子只认识贝多芬和莫札特。世上的音乐家多如繁星,但这个年纪的久美子几乎都不认识。她看着自己不认识的人写的乐谱,透过喇叭聆听自己不认识的人演奏的音乐,就算没有太详尽的知识,久美子还是喜欢听音乐。 

    小学的钟声响起。四年级的久美子紧张地站在音乐教室里,或许是太用力了,手中的入社申请捏得皱巴巴。只有寥寥数人的儿童一起望向久美子。不同于国高中,小学的铜管乐队人数不多。年幼的久美子红着脸,把入社申请交给个子最高的女生。对于当时的久美子而言,六年级几乎已经是大人了。 

    「你叫久美子吗?」貌似社长的学姐说,脸上绽放成熟的笑靥。 

    「是的。」久美子以细如蚊蚋的音量承认。 

    「我听老师说过了。」学姐向其他社员介绍久美子:「这位是黄前久美子,今天开始加入社团。」 

    「请、请多多指教。」 

    久美子低头行礼,其他人笑嘻嘻地欢迎她加入。 

    「再来要决定吹什么乐器,你有经验吗?」 

    「没、没有。」 

    社长牵起久美子的手,带她走到乐器室。隔着窗户可以看到暮色已笼罩校园,黑点般的乌鸦在上空盘旋。 

    「呵呵,四年级就加入的话,有希望成为明日之星呢!」 

    「明日之星?」 

    社长半开玩笑地说,久美子的脑中满是问号。明日之星是指明天的星星吧!她还在思考明日之星的意思,社长已打开乐器室的门,一阵灰尘的臭味扑鼻而来。 

    乐器室里陈列着许多乐器盒,好像一座宝山。 

    「有你想吹的乐器吗?」社长问她。 

    姐姐的身影浮现在久美子的脑海。那时候的姐姐好帅气啊!久美子不禁悲从中来。上了国中以后,麻美子满脑子只剩下学习,望着姐姐的背影,久美子领悟到,姐姐已经不再玩乐器了。 

    「那个,我姐姐是吹长号的,我也想吹长号。」久美子回答。 

    社长的表情有些困扰。 

    「呃,长号的人数已经够了。」 

    「这样啊……」 

    久美子大失所望地说。不忍心见学妹那么失望,社长啪地拍了一下手。 

    「上低音号呢?」 

    「幽浮?」\footnote{上低音号euphonium的发音听起来很像幽浮的英文「UFO」。}

    陌生的字眼令久美子感到莫名其妙。 

    「不是啦。」社长笑着说。貌似已经很习惯面对久美子这样的反应。 

    「我是指这个乐器。这个乐器的名字叫上低音号。」 

    社长打开乐器盒,拿出一把金色的乐器。好大啊!久美子看得双眼发直。跟直笛不一样,这种乐器只有三个按钮,要怎么吹呢? 

    「上低音号的音域跟长号差不多,吹嘴的大小也一样。管弦乐团没有上低音号,所以一般人不太知道这种乐器,但我觉得音色很美丽。」 

    「拿着。」社长递给久美子一个类似银色陀螺的零件。「这是吹嘴。」社长补充。 

    「我也很少听到专业的现场演奏,不过厉害的人真的很厉害喔!该怎么说呢,即使是简单的曲子也很动人。」 

    这时,社长好像想到什么,开始在乐器室的小柜子里翻找,那里面塞满了与铜管乐队有关的杂志及CD。 

    「就是这个,这是老师推荐的CD,你带回家听听看。还有,这本是给初学者用的导读。」 

    社长拿出一本旧旧、薄薄的书和一张CD,水蓝色的封面上,以黄色的粗体字写着《简单!上低音号&低音号入门指南》。久美子的视线落在封面上,淡然想说低音号和上低音号的形状长得好像,简直跟亲子一样。 

    「这个人是专业的上低音号演奏家,老师说他的演奏很值得学习。」 

    社长指着久美子手中的CD说道。进藤正和。没听过这个名字,是指封套上的叔叔吧,吹着上低音号的侧脸长相端正。久美子把CD翻到背面,上面密密麻麻写着收录的曲目,全都是没听过的曲子。久美子轻抚写在最上方的专辑名称:《默剧》。 

    「总之先练习发出声音。吹嘴可以让你带回去,在家也可以练习。」 

    「好!」 

    久美子对社长的交代用力点头。社长笑逐颜开地摸摸她的头,温柔的触感令久美子闭上眼睛,感觉社长有点像以前的姐姐。 

    久美子睁开双眼,坐起来的动作让某样东西从胸前滑落,定睛一看,是毯子。她深呼吸,调匀紊乱的气息。久美子满身大汗,衬衫紧紧贴在身上,扑通扑通的心跳声听起来好吵。好像睡着了,久美子在开着暖气的房间里怔忡思考。大概是从社团回来以后就直接睡着了。她不觉得是感冒,或许是累积了太多的疲劳,一觉醒来,感觉神清气爽。 

    环顾室内,爸妈好像不在,大概还在上班。理当在家的麻美子也不见人影。自从那天与爸妈大吵一架以后,麻美子就一直关在房间里。 

    电视机里传来搞笑艺人胡闹的笑声,望向萤幕,上头的人正在寻找东京的美食,偶像明星以高八度的嗓音说:「这里的意大利面很好吃喔!」告诉我东京的资讯又能怎样?久美子按下遥控器开关。随着啪叽一声,萤幕顿时变成黑屏。 

    话说回来,作了令人怀念的梦。为何事到如今还会想起小学的事呢?久美子站起来,慢条斯理地伸了个懒腰。脊椎骨一带发出吱吱嘎嘎的声响。房间里悄然无声,上低音号的柔美音色不期然在脑海中响起。 

    「……默剧。」 

    久美子很喜欢当时社长借给她的CD,还请爸妈买了同样的给她,反复聆听。最近太忙了,没什么时间听,不知怎地突然很想念那个旋律,久美子在CD架里翻找。 

    「找到了、找到了。」 

    久美子从盒子里拿出CD,放进播放器。CD在播放器里发出旋转的声音,不一会儿,演奏开始播放。 

    曲子始于上低音号悠扬的音色。圆润的旋律静静流淌,仿佛被风吹动的海面。厚重的音色回荡在号口内,接着传来木管与上低音号交缠的轻柔音色。每个音符都很清晰,而且水乳交融。 

    多么美好的音色啊!久美子静静闭上眼。〈默剧〉(Pantomime)在菲利浦.史巴克(Philip Sparke)为数众多的上低音号独奏曲中,也算是最有名的作品,整首曲子由优雅抒情的前半部与轻快且需要高度技巧的后半部构成。 

    进入后半部,音乐变得愈来愈活泼,尽管音符正以飞快的速度弹跳,美丽的音色依旧清澈无比。如果想要正确又不失高雅地吹出这么快的过渡性音节,需要非常高超的技术,然而演奏者毫不费力完成这个艰难的任务,让人差点就误以为这首歌应该很简单吧!令人心醉神迷的甜美高音与气贯丹田的低音,两者相辅相成,交织成动人的旋律。专业的乐手果然好厉害啊!久美子将脸埋进毯子里。想也知道,自己完全比不上。 

    一曲既罢,音乐进入下一首歌。久美子心不在焉地边听音乐边想。这么说来,这个人吹的上低音号音色与明日香的音色好像。不,应该说是明日香的音色很像这位演奏者的音色。说不定明日香是以此人的演奏为目标,努力练习到现在。 

    「……进藤正和先生吗?」久美子自言自语地说。要是自己也能吹成这样该有多好。久美子盖着毯子,回忆起刚才的柔美音色,银色的乐器在紧闭的眼皮内侧闪闪发光。 

    第二天,久美子的身体已经完全康复,她便跟平常一样去学校。因为是假日练习,虽说两点以前集合就好,但许多社员都已经在音乐教室集合了。 

    「你听说明日香学姐的事了吗?」绿辉问她。 

    久美子静静摇头。绿辉的双手无意识地在自己的肚子上交叉,指尖互搓。 

    「听说她母亲昨天又来了。学姐果然是瞒着母亲参加社团活动。」 

    「能瞒到现在也真不容易呢!」 

    「她母亲工作很忙,平常几乎都不在家,所以才能一直瞒到现在,可是这次或许真的瞒不下去了。」 

    「是噢!」 

    久美子低着头,躲避绿辉的视线。自己扭曲的脸倒映在怀中的上低音号表面。明日香的母亲是真的认为要女儿退出社团是为她着想吧!久美子脑海中浮现出明日香的脸,随即扭曲变形,变成麻美子的脸。 

    这是为了你的将来着想。 

    久美子在口中反刍母亲对麻美子说的话。将来是什么?真的比现在重要吗?久美子悄悄地垂下眼。一旁的夏纪正一脸凝重地与基础练习的乐谱大眼瞪小眼,用指尖翻开透明的文件夹。久美子的视线一隅被〈东海岸风情画〉的文字勾住。夏纪没发现倒抽了一口气的久美子。不,或许只是装作没发现。 

    「对了,听说桥本老师今天会来。」绿辉笑着告诉久美子。 

    「这样啊!」久美子回答,不知怎地觉得异常口渴。 

    桥本上次来社团指导已经是暑假的事。当他出现在音乐教室里,原本气氛紧绷的音乐教室感觉一下子放松下来。他一如既往穿着没品味的马球衫和短裤。明明已经是秋天了,他的打扮却一点季节感也没有。 

    「哎呀,抱歉啊,一直抽不出时间来。我也很惦记各位,只是我实在太优秀了,很多地方都找我去。」 

    桥本个子矮小,比泷矮了好几个头,皮肤被太阳晒得有点黑,外表与泷形成光谱两端的对照。泷望向久美子身旁的空位,叹了一大口气。光是这样,社员就打直背脊,音乐教室充满一股剑拔弩张的紧张感。 

    「事不宜迟,可以让我从头到尾听一遍吗?我想知道比起关西大赛的时候,你们进步了多少。」 

    「这个嘛……那就从指定曲开始,从头到尾吹一遍。」 

    「好。」 

    泷举起指挥棒,从微微掀起的袖口可以看见他的手腕。银色的手表慢吞吞地刻画出时间。久美子配合指挥棒勾勒出的优雅曲线吸气,呼吸的声音在号口幽幽作响。 

    指定曲〈娥眉月之舞〉与自选曲〈东海岸风情画〉已经在桥本面前演奏过好几次了。桥本是专业的打击乐手,也是泷的老朋友,暑假还陪他们一起去集训,帮北宇治高中管乐社打进全国大赛。相较于严格的泷,桥本的性格爽朗大方,深受打击乐组员的信赖。演奏过程中,桥本坐在为他准备的椅子上,专注地倾听演奏,唯有眼珠子转来转去,目光如炬地观察社员的表现。 

    演奏完两首曲子后,桥本意兴阑珊地拍手。「还可以。」话说得轻松,但表情可不是这么一回事。他拍拍泷的肩膀,凑到泷耳边不知说了些什么,大概是为了不让社员听见。泷一再摇头,两人交头接耳了好一会儿,这才重新面向社员。 

    「不觉得太没劲了吗?」 

    这是桥本的第一句话,然后宛如溃堤般滔滔不绝。 

    「哎呀,该怎么说呢,各位是不是比上次退步了?声音僵硬又死板,我也知道开学以后,练习时间变少了,要维持以前的水准并不容易,可是啊,该怎么说呢,听起来好难受,以前明明不会这样。」 

    被批评得体无完肤,社员全都无言以对。桥本一脸伤脑筋地抓抓头。 

    「难不成因为是全国比赛,所以会紧张?还是有其他原因?不管是什么原因,比起现在的演奏,暑假的演奏还好多了。各位根本一点都不开心嘛!合奏是酷刑吗?为何大家都跟泷一样,露出那么恐怖的表情。」 

    「我才没有露出恐怖的表情。」泷一脸平静反驳。 

    桥本语带揶揄地说:「才怪,明明就恐怖得要死,所以才说毫无自觉的人最伤脑筋。」桥本开玩笑的说词稍微缓和了社团内的气氛。他双手扠腰,装模作样地叹气。 

    「还有啊,我对很多学校的学生也说过,我其实不太喜欢比赛。说老实话,只要拼命努力过,金奖或银奖根本无所谓。可是听我这么说,管乐社的人都反驳:『非拿下金奖不可!』这也是无可厚非的反应。」 

    久美子偷偷望向双簧管的座位。讨厌比赛的学姐一如往常面无表情,聚精会神地注视着桥本。硬质的黑发沿着她的脸颊滑落,隐约可见的耳垂异常白皙。 

    「别人的评价固然也很重要,毕竟音乐是演奏给别人听的,流于自我满足的演奏当然不行,但也不用太在意别人的评价。再说了,音乐这两个字就是要享受声音、得到快乐才写成音乐,所以吹奏的人也要乐在其中才行。要是因为『耶!要比赛了,得好好吹才行!』而绑手绑脚,听的人也会觉得很无趣。必须以『好,我要在大舞台上表现我的音乐!』的士气来迎战才行。因为心情会表现在音乐上,所以要避免闷闷不乐的演奏。」 

    桥本露出洁白的牙齿笑着说。社员也回答:「是。」他往教室里看了一圈,绷紧表情。「接下来请听我严肃地说两句。」桥本稍微压低音量,视线突然射过来,久美子还以为自己的心脏要停了。 

    「完全听不到上低音号的声音,你真的有在吹吗?」 

    「有、有在吹。」 

    久美子被他盯着看,汗水从毛孔喷出来。她胀红脸,好不容易才挤出这句话。桥本摩挲着下巴说:「或许是只有一个人的关系吧,音量太小了,听不见。平常那个技巧高超的学姐上哪儿去了?那个戴红框眼镜的学姐。」 

    「明、明日香学姐,那个……」 

    「她今天请假。」坐在低音号座位的卓也替无言以对的久美子回答。 

    「请假?」桥本微微挑眉。「这么重要的时候?算了,如果有什么事情也没办法,以前多亏有那位学姐,所以感觉不太出来,但今日就很明显了。明明只是少了一个人,音量却只有平常的四分之一。请好好地发出声音来。」 

    「是。」 

    明日香深谙如何用少量的气息有效率地吹出声音来,她的演奏不只音量大,每个音符都从号口嘹亮地响起。如果今天请假的是久美子而非明日香,桥本肯定不会听不见上低音号的声音。自己与明日香的实力差距实在太明显了,感觉一切都被摊在阳光下,久美子用力咬紧下唇。 

    「另外,长号的音程一定要准确。还有一点要注意的,在这么小的教室里,滑管会不会撞到上低音号啊?看得我冷汗直流。」 

    「啊,不要紧的。」 

    长号的组长连忙回答。「那就好。」桥本点点头。 

    长号的形状比其他铜管乐器特殊,由两个长长的U型管相连而成,以滑管的伸缩来调整音程高低。因为可以用滑管调整音程,演奏者必须集中精神来调节伸缩的程度。要是太使劲拉,可能会直接拉出滑管,所以练习时偶尔会发生滑管掉落的意外。 

    桥本继续指导,社员专心聆听他宝贵的指导。他的纠正有些和以前泷说过的内容重复。他们果然都注意到相同的地方了。久美子在乐谱上写下注意事项,悄悄望了泷一眼。他始终若有所思凝视着明日香的座位,静静垂下视线,仿佛做出某项决定,静静握拳。平常给人柔和印象的双眸,隐含着凌厉的光芒。 

    「记得我现在说的话,再来一遍吧!各位也不是笨蛋,提醒过的地方一定要纠正过来。只要小心就不会犯的错,绝对不要犯第二遍,听清楚了吗?」 

    「是!」 

    社员答应桥本的要求。他的手刻画着皱纹,以有些粗鲁的动作拍拍泷的肩,害他藏青色的西装略微起皱。泷惊讶地面向桥本,后者的视线依旧停在社员身上。 

    「接下来才是关键。」桥本说道。 

    现在的久美子无法判断这句话是说给泷听,还是说给社员听。 

    合奏练习结束时,已经过了七点,窗外一片漆黑,夜色彻底取代了黄昏。或许是从一早就开始合奏,感觉好困。久美子边打哈欠边望向泷。今天要留到几点呢?她的大脑一隅无意识地想着这件事。 

    「那么,今天的合奏就到这里结束。」泷宣布。 

    社员同时起立。小笠原打直背脊,望向坐在椅子上的桥本。 

    「谢谢老师今天一天的指导。」 

    「谢谢老师。」社员异口同声随社长道谢。桥本朝他们挥挥手,手里还拿着总谱。小笠原正打算跟平常一样向泷道别,泷却以掌心制止她,对一脸诧异的社员清了清喉咙说:「我有件事要告诉大家。」 

    顾问的话让大家面面相觑。久美子下意识皱眉,总觉得有股不祥的预感。泷静静深呼吸,往室内看了一圈。一旁吹法国号的学姐不安地咽了口水。 

    「是关于田中同学的问题。」 

    众人闻言,脸上浮现出困惑的神色。久美子大概能想象泷接下来要说什么,紧紧握住自己的指尖,眼皮里闪过今早基础练习时,看到的夏纪侧脸。当时,久美子确实看到她有些粗糙的指尖翻过乐谱的瞬间,曲名隔着透明的资料夹清晰可辨。久美子马上就明白那是什么意思,只是不想承认而已。 

    泷轻声叹息。 

    「倘若田中同学无法在本周末以前,保证可以继续参加社团活动,全国大赛将由二年级的中川同学出场。」 

    音乐教室内一下子变得鸦雀无声,时间仿佛凝结了,没人有办法动。久美子无法承受重若千金的沉默,抖了一下。 

    「技术上或许是田中同学比较优秀没错,但我认为,抱着不知道她什么时候才能来练习的不确定要素参加比赛,绝非上策。当然,我并非认为田中同学本人不想参加比赛,只是这次的事已经超过可以通融的范围了。」 

    泷的脸上没有一丝迷惘,语气也跟平常没两样,不见丝毫动摇。社员都领悟到此事已成定局。桥本在泷身边静静低垂着眼,一脸于心不忍。 

    「我早就向田中同学和中川同学提过这个可能性,她们也同意了。我请中川同学从明天就加入合奏。我会在下周做出最后的判断,希望大家也能有心理准备。」 

    泷说到这里,朝社长投去一瞥。呆若木鸡的小笠原至此总算六神归位,这才猛然回神似地,像平常那样向泷道谢。 

    「那么,今天的练习到此结束,谢谢老师。」 

    「谢谢老师。」 

    久美子的嘴巴反射性地蠕动,尽管思绪已经不晓得飘到哪里去了,身体依旧违反自己的意志,遵循平常的习惯,弯下腰,低下头,头发随动作扎到脸,皮肤表面掠过一阵刺痛。她拨开发丝,不小心拔掉一根天然鬈的头发。久美子目不转睛盯着掉在地上的头发,微微鬈曲的发丝被灯光折射,看起来既是黑色,又像是棕色。 

    泷就这样走出音乐教室。平常会留下来跟社员谈天说地的桥本今天也一溜烟跑掉。剩下来的人全都不知所措地彼此互看。 

    「接下来会怎么样呢?」 

    绿辉跑过来,紧紧抓住久美子的水手服下摆,纤细的指尖捏皱了藏青色的布料。久美子不晓得该说什么才好,模棱两可地回答:「我也不知道。真的不知道。」 

    距离比赛还有二十一天!绿色的黑板上写着像是出自女生之手的圆形字体。只剩不到三周,所有问题能在那之前解决吗?久美子悄悄望向邻座,空荡荡的座位无法回答她的问题。 

    每次假日练习,久美子和丽奈都是早上六点就到学校。清晨的空气十分新鲜,弥漫着一股冷冰冰的凉意。明明这么早就来了,但最早出现在音乐教室里的还是霙。久美子每次都想着今天一定要比霙早到社团,和丽奈跳上第一班电车,即便如此,霙还是会抢在她们前面抵达音乐教室。她的理由是住得离学校比较近,但也太早了。 

    校舍里几乎没有半个人影,只有棒球社的吆喝声稀稀落落地从校园里传来。自窗外洒落的阳光十分柔和,不含一丝热度。久美子的手指轻轻在由窗櫺隔开的光与影的分隔线上滑过。旭日美则美矣,但一点也不暖和。 

    「明日香学姐和夏纪学姐,谁会参加比赛呢?」 

    久美子的声音回荡在走廊上。丽奈一脚踩着楼梯,回头看。 

    「这要看明日香学姐吧!」 

    「说的也是。」 

    久美子陷入沉默,丽奈傻眼叹气。 

    「唯有这件事,我们想再多也没用。明日香学姐肯定也想参加社团活动,可惜有些事不是自己能决定的。本人再有才华,周围的人不让她发挥也没有意义。只要她母亲不答应,学姐就无法参加比赛。与其那样,早点做决定对我们还比较好。我最讨厌大家因为不确定她能不能参加比赛而变成一盘散沙。」 

    「可是,明日香学姐真的愿意那样吗?」 

    「不愿意也没办法,谁叫子女不能选择父母。」 

    丽奈的声音在走廊上回荡。久美子被堵得说不出话来,她停下脚步,灼热的感情从腹部涌上来,好想大叫「才不会没办法!」好想说「这是不对的!」但嘴唇仿佛冻住了,无法张开。大脑角落有个冷静的自己,很清楚对丽奈说这些也没用。 

    丽奈下楼,走向呆站在原地不动的久美子。乌黑的长发随风摇曳,迎面而来的风微微吹乱了她的裙子。 

    「久美子为何要露出这种表情?」 

    丽奈伸手捧住久美子的脸颊,冰凉的触感令久美子垂下眼帘。 

    「好不甘心。」 

    明日香早在泷来任教以前,就一直很认真练习。当大家士气低落时,明日香也没随波逐流,仍旧继续努力练习。结果却是这样吗?她这三年要以这种方式画下句点吗?这未免太不合理了。明日香的努力应该要有所回报。 

    丽奈放松嘴角,手轻轻绕到久美子背后,胸部与胸部紧贴,隔着皮肤,可以感受到她的心跳。 

    「久美子总喜欢把一切都揽在自己身上。」 

    丽奈用力抱紧久美子。从她口中吐出的气息撩拨着久美子的耳朵。 

    「只要尽全力处理好自己能做的事就好了。要是别人的事都一一放在心上,会看不到自己真正该做的事。」 

    什么是自己真正该做的事?久美子静静开始思考。要改掉昨天合奏时被泷纠正的部分、改掉轮到自己的时候会紧张,导致音准有点偏高的坏习惯、比赛前要变得更强大,还有、还有…… 

    光是随便想想,课题就多得不得了。或许自己能做的只有这些微不足道的小事,但是五十五人份的小事加起来,才能完成一个完整的演奏。耿耿于怀的事多如天上繁星,但是如果一直耿耿于怀,绝对无法前进。 

    「……丽奈,谢谢你。」 

    久美子微笑道,丽奈不以为然地别开脸说:「不用谢。」耳垂微微泛着红晕。 

    这天的合奏练习时,完成基础练习的夏纪留在原地不动。明日香今天也请假,她的乐器盒不在乐器室里,恐怕是把上低音号带回家了。久美子怔忡看着自己倒映在窗户玻璃上的影子。 

    夏纪的上低音号和久美子一样,都是金色的,两个人站在一起,反而有种整体感。而明日香和久美子的上低音号分别是金色和银色,所以偶尔会有对乐器没兴趣的人误以为她们吹的乐器不一样。 

    明日香总是站在久美子的右侧,所以当自己左侧有个吹上低音号的学姐时,感觉好奇怪。久美子偷眼瞥向左手边,只见夏纪的乐谱上写满工整的文字,周围还贴了一堆粉红色的便条纸。仿佛老师用红笔订正的成熟笔迹,浓缩了这个夏天泷对久美子和明日香耳提面命的重点,这些字大概出于明日香之手。 

    「换成我会不安吗?」或许是察觉到久美子的视线,夏纪抬起头说。没想到她会这么说,久美子抖了好大一下。 

    「呃,不会,没那回事……」 

    「我倒是非常不安。」夏纪说道,缩了缩肩膀。 

    「我怎么可能代替得了明日香学姐。可是他们说上低音号只有一个人太少了,所以我也没办法。」 

    「学姐一直在练习比赛的曲子吗?」 

    「嗯,明日香学姐和泷老师很早就对我说:『不好意思,希望你能开始练习。』」 

    「原来如此。」 

    不知不觉间,久美子抱紧自己的乐器。或许是因为调音时拔出音管再插回去,指尖还留着刺鼻的金属臭味。 

    合奏练习如常进行。夏纪的演奏到底比不上明日香,但是从她吹出的声音,可以察觉她有多努力。 

    久美子往旁边一看,粉红色的便条纸被暖气的风吹动,轻飘飘地摇晃。明日香是以什么样的表情交出这些便条纸呢?是直接陪夏纪练习,写上需要注意的地方吗? 

    泷还在对小号组进行没完没了的指导,久美子左耳进、右耳出地听着,偷偷看了夏纪一眼。 

    万一明日香回来参加社团活动,夏纪的练习就白费了。可是如果夏纪就这样参加比赛,就表示明日香不能出场了。无论如何,都会有一位学姐无法站上比赛的舞台。话说回来,夏纪原本在暑假选拔A部门的成员时落选,所以这次的事对夏纪来说无疑是个机会。还是说,只会增加她的负担呢? 

    「老实说,我没有把比赛放在心上喔!A也好,B也好。」 

    久美子的脑海中不期然闪过夏纪说过的话。她虽然表现得不当一回事,但是对一直待在B部门的夏纪而言,比赛练习应该是相当沉重的负担,但夏纪还是撑过来了,明明一旦明日香回来,这一切可能就全都只能付诸流水。 

    久美子悄悄垂下眼帘。真希望大家都能得偿所愿,但现实没有这么容易。泷的白色指挥棒在视线一隅轻轻摆动,成为指挥对象的单簧管反复吹奏同一小节。久美子侧耳倾听单簧管的音色,悄然叹息,气息让金色的上低音号表面蒙上一层白雾。 

    「喂。」 

    结束合奏练习后,久美子在洗手台被叫住,她回头一看,夏纪正盯着自己看。平常眼神就很凶恶的夏纪,今天看起来更凶了。是什么惹她生气了,久美子不由得冷汗直流。夏纪貌似完全没注意到久美子的心情,抓住她的手,没好气地对她说:「过来一下。」 

    「我有话想跟你说,可以给我一点时间吗?」 

    都已经迈步往前走了,才一脸没事地问她有没有时间,久美子也只能点头如捣蒜地说好。 

    久美子被带到离音乐教室有一段距离的楼梯间,几乎没有其他学生会经过这个通往屋顶的地方。这么说来,之前决定由丽奈独奏时,和学姐闹得不愉快的丽奈也是在这里发泄压力。明明只是前阵子的事,感觉上却好像已经是很久以前的事了。 

    「坐吧!」 

    夏纪粗鲁地坐在楼梯上。还穿着裙子,脚会不会张得太开了,内裤都快要露出来了。久美子一面担心,一面默默在她旁边坐下。通往屋顶上的门照样锁着。 

    「我就单刀直入地说了,听说你和明日香学姐约好要一起温书?」 

    「咦?嗯,对啊!」 

    还以为她要说的是跟比赛有关的事,没想到会是这个问题,久美子一时反应不过来。夏纪抱着胳膊,发出陷入沉思的嘀咕声。 

    「嗯,看来机会就只有那么一次。」 

    「机会?请问你在说什么?」 

    久美子不解地侧头反问,夏纪一脸错愕,脸上写着「这家伙也太状况外了」。 

    「当然是『带回明日香学姐大作战』啊!」 

    「……是、是噢!」 

    作战名称听起来好蠢。或许是察觉到她在想什么,夏纪刹时面红耳赤。 

    「啊,千万别误会,这名字不是我取的。」 

    「那是谁?」 

    「香织学姐。」 

    一旦抬出这个名字,社团里应该没有人会表示意见。 

    「真、真是别出心裁的作战名称呀!」久美子言不由衷地挤出这句话。 

    「可不是嘛!」夏纪淡淡地回答。 

    诡异的沉默横亘在两人之间。为了打破沉默,夏纪清了清喉咙。 

    「作战名称先摆到一边,内容才是重点。」 

    夏纪不动声色从裙子口袋里掏出一张纸条。纸上绘有缎带的图案,设计得很秀气。 

    「照这样下去,再怎么想,明日香学姐都无法回社团。这时只能靠我们说服明日香学姐的母亲了。唯一的办法是趁你去学姐家的时候,巧妙地与她母亲接触。」 

    「欸!这太强人所难了啦!」 

    「才不会强人所难!俗话说得好,有志者事竟成。」 

    「怎么这样。」 

    久美子拼命将头摇成一个波浪鼓。夏纪没亲眼见过明日香的母亲,才能说得那么轻松,照久美子在教职员办公室里看到的情景,她母亲绝不是久美子应付得来的对手。久美子光是想象,汗水就从额头泉涌而出。 

    「别担心,香织学姐给了我秘密武器。」 

    夏纪递给久美子那张纸条。 

    「这是香织学姐写的纸条?」 

    「没错,学姐说只要照着做就没问题。」 

    久美子心不甘、情不愿地接过纸条。滑过掌心的纸片表面意外有些粗糙,香织以可爱的字体写着:车站前,幸富堂的栗子馒头是最佳选择! 

    「啥?」 

    久美子不明白这行字的意思,茫然看着夏纪的脸。夏纪眨眼,竖起大拇指。 

    久美子莫名其妙地问她:「这是什么意思?」 

    「听说是明日香学姐的母亲爱吃的食物。只要带上这个伴手礼,一定没问题。」 

    「不不不,想也知道不可能没问题!」 

    久美子不假思索地站起来。夏纪哈哈大笑。她该不会是在捉弄自己吧!夏纪擦拭眼角笑出来的泪痕,轻佻地说:「没问题,你一定能办到。」 

    「凭什么这么说。」 

    「因为……」夏纪站起来。因为台阶的高度,平常总是比久美子矮一截的脸,今天落在她的头上。夏纪的眼神十分坚定,久美子一时无言以对。 

    「截至目前,你不是已经摆平过很多疑难杂症吗?像是高坂同学的事,还有霙的事。」 

    「……我什么也没做。」 

    无论是香织和丽奈的独奏之争,还是希美和霙的心结浮上台面时,自己终究什么忙也帮不上,只能冷眼旁观。 

    夏纪的手伸向一声不吭的久美子,干燥的指尖粗鲁地抓住她的肩膀。 

    「你什么都不用做也没关系,只要陪在那个人身边,就能给对方力量。」 

    「……」 

    「你在她身边的话,明日香学姐的心情或许也会轻松一点,总比什么都不做来得好。」 

    久美子的视线落在纸条上,写着可爱文字的纸张角落,有一只毫无特征的猫咪,还用对话框以工整的字体写着「加油」。久美子以指腹按压其上,看了夏纪一眼。结实的小腿从藏青色的裙子底下探出来,白色的长筒袜与被太阳晒黑的肌肤相映成趣。 

    「你真的这么想吗?」 

    久美子反诘的语气异常平静。狭窄的走廊上没有半个人影,从柱子延伸出来的阴影吞噬了一切。夏纪有点脏的室内鞋轻踹着绿色的地板。 

    「真的啊!」她回答。「要是你能让明日香学姐回来的话就太棒了。」 

    「可是,明日香学姐一旦回来,夏纪学姐这阵子的练习就……」 

    「那也没关系。」 

    夏纪斩钉截铁地打断久美子。压低的声线带了点压迫感,夺走久美子反驳的余地。夏纪言辞之间犹有留恋,也有对比赛那个大舞台的执着,但她只字未提。聪明如她,深知此刻应藏起自己的真心话。 

    「因为由明日香学姐上场比赛,对社团才是最好的。」夏纪说道,眼神无比坦诚。 

    久美子不由自主避开她的视线。说的也是。才没有那回事。感觉这两句话都不适合现在说,久美子用力握紧纸条。 

    夏纪粗糙的手搭上久美子的肩膀。夏纪什么都知道。正因为知道那句话代表什么意思,才主动说出对自己而言无比残酷的台词:「拜托你,久美子,带明日香学姐回来。」 

    太阳已经完全消声匿迹,夜色笼罩大地。久美子望向窗外,路灯的光线宛如指标在黑暗中排成一排。宇治并非大都市,没有金碧辉煌的夜景。不过,她很喜欢繁星点点,闪着微光的景色。 

    「还没走?」 

    曾几何时,音乐教室只剩霙和久美子。每次留到晚上,最后通常都只剩她们两个。 

    「我想再练习一下,学姐呢?」 

    「我要回去了。」 

    霙拆开双簧管的上管和下管,塞在管内,附有纽带的布称为通条布。霙小心翼翼上下移动通条布。水分要是残留在管内,乐器的状态就会恶化,因此必须像这样仔细保养。久美子怔忡地看着一半泡在水里的簧片。不同于铜管的吹嘴,使用簧片的木管乐器真的很费神。 

    「那个……」 

    「什、什么事?」 

    霙冷不防开口唤她,久美子连忙转过头去。白天将桌椅排成合奏阵形的音乐教室已经恢复成平常上课的样子。跟暑假不一样,这个场所明天也要用来上课。 

    「你要去明日香学姐家吗?」 

    「约是约好了。」 

    「是吗?」 

    霙边说边把乐器放回盒子里。久美子对木管不是很了解,但很清楚霙深爱着自己的乐器。 

    「请转告她,我等她回来。」 

    「转告明日香学姐吗?」 

    「没错。」霙面无表情地点头。 

    久美子耳边响起阖上乐器盒的喀嚓一声。霙抓住窗帘,静静拉上。乳白色的窗帘布隔开窗外的景色,转瞬便什么都看不见了。 

    关掉音乐教室的灯、确实锁好门、拿钥匙去教职员办公室归还,以上一向是最后一个离开的学生的任务。小巧的钥匙上吊着写有「音乐室」的白色钥匙圈。 

    「报、报告。」 

    久美子走过漆黑的走廊,敲了敲办公室的门,提心吊胆地从门缝探头探脑,只见阴暗的室内只有一个角落还亮着幽微灯光。 

    「请问……」 

    久美子出声,没有人回应。无计可施下,久美子走进用隔板隔开的室内。泷坐在角落的座位。 

    「老师?」 

    定睛一看,泷趴在桌上睡着了。灰色西装挤出皱褶。直接把钥匙放桌上就回去好像不太好,既然如此,只能叫醒眼前睡着的顾问,但是看到他熟睡的模样,总觉得于心不忍。久美子拿着钥匙,在他背后走来走去,不经意看见顾问桌上有个木制相框,四个年轻男女正眉开眼笑地朝自己比出胜利手势。学生可以进入的办公室场所相当有限,这是第一次在老师桌上看见私人物品。久美子抵挡不了好奇心,忍不住打量照片。 

    从后面拍到的风景来看,大概是在哲学之道上拍的。正中央满面笑容的男人是桥本,看来比现在年轻,但气质跟现在差不多,穿衣服的品味也一模一样。笑着站在他旁边的人大概是暑假来社团指导木管的新山,乌黑的长发扎成一束,打扮有几分运动女孩的味道,跟现在的感觉相去甚远。站在角落露出苦笑的则是泷。黑发的他跟现在不太一样,还有些稚气未脱。他现在也很帅,但年轻时肯定更受欢迎。这是久美子与站在一旁的桥本比较之后诚实的感想。泷身旁有个身材娇小的女性,柔和的形象与香织有些雷同也说不定,不算是特别标致的美人,但是长得很可爱。她的手放在自己嘴边,正在跟泷讲悄悄话。 

    「啊,对不起。」 

    耳边传来移动的声响,久美子连忙从照片上移开视线。只见泷刚醒来,有些羞赧地搔搔自己的头,脸上清楚留下用来代替枕头的手臂印痕。平常总是梳得很整齐的头发,此时此刻有些凌乱。还很困吧,揉着眼睛的泷感觉比平常更稚嫩。 

    「我拿钥匙来还。」 

    「啊,好的。」 

    泷望向久美子的手心。久美子递出钥匙后,他的双眸勾勒出微笑的弧度。 

    「抱歉让你久等了。」 

    「没、没有,没关系。只是看老师睡得好熟,是不是很累?」 

    「这我倒不否认。」泷苦笑说道。大手从久美子的小手中拿起钥匙。这就是成年男子的手啊,久美子茫然地想。 

    「那个,请问老师是住在学校里吗?感觉你从早到晚都待在学校。」 

    「怎么可能,我还是会回家喔!」 

    「这、这样啊!」 

    想当然尔,久美子也知道老师不可能住在学校里,可是泷一直守护着管乐社,不由得让她有一瞬间产生这种愚不可及的错觉。早上比谁都早到学校,晚上又比久美子更晚回家。有必要这么投入吗?顾问的工作不见得能得到所有社员的认同。春天的教室里,经常可以看到学生将泷骂得一文不值的风景。明明可以更轻松,对社团活动不必那么用心的。过去,北宇治高中管乐社一向是在那样的风气下练习,若能承袭那种风气,肯定不会有人抱怨吧,反而会被当成温柔的顾问,受到众人爱戴也说不定。既然如此,他为何要做到这种地步? 

    「为何要对我们尽心尽力到这种地步?」 

    久美子心里的想法不知不觉脱口而出。泷惊讶地眨眨眼睛。久美子无地自容地捂住自己的嘴。脸好热。居然自言自语,真是太丢脸了。 

    或许是察觉到她的心情,泷噗哧一笑,不以为意地反问:「你在说我吗?」 

    「呃,那个,我只是有点好奇。」 

    「黄前同学经常会注意到别人不会注意到的地方呢!」 

    「是吗?」 

    「我的事不值一提。」 

    「才没有那回事。」 

    久美子紧咬不放地反驳,泷笑得更开怀了。既然如此,非问个水落石出不可!久美子不顾一切指着桌上的照片。 

    「这张照片是和桥本老师他们一起拍的吗?」 

    「哦,你说这个啊?」 

    泷用修长的手指抓住相框边缘,拿到久美子面前,让她看清楚里头的照片。 

    「这是我们学生时代的照片,桥本老师硬要放在这里。不过家里也没地方放,所以就一直放在学校了。」 

    泷的指尖从刚才就以十分温柔的动作轻抚相框,足见「那家伙真是有够鸡婆」的抱怨根本是违心之论。这张照片对他来说,想必非常重要。 

    「你们从大学就是朋友了,对吧?」 

    「你听谁说的?」 

    泷一脸诧异地看着她。久美子不知怎地,感觉好像受到责备,赶紧找借口似地忙不迭解释:「是桥本老师集训时告诉我的。」 

    「那家伙又随便拿别人的私事来乱讲……」泷大为傻眼地深深叹息。他平常不太会表现出自己的情绪,但只要扯上桥本,就会变得孩子气。 

    久美子深呼吸,鼓起勇气问他:「这个女生是泷老师的太太吗?」 

    这个问题让泷瞪大双眼,突出的喉结上下震动。原本平静的气氛瞬间冻结。只见他拿着照片的手一僵,久美子立刻低头赔罪:「对、对不起!问了不该问的问题。」 

    紧握的掌心充满汗水,不该多问的,久美子很后悔自己五秒钟前的莽撞行为。为何就是无法压抑好奇心呢?久美子在脑海中对自己开起批斗大会。满腔的自我厌恶,几乎要击溃她的内心世界。 

    「别这么说,是我不好意思,让你费心了。」大概是不忍心见久美子的脸色再继续铁青下去,泷一脸歉意地垮着肩膀说。 

    「不,是我不好。」久美子再次低头道歉。 

    泷摩挲着自己的下巴,探究地望向久美子。 

    「桥本老师是不是对你说了什么?」 

    「呃,这个,那个……」 

    「不用隐瞒喔,请老实说。」 

    尽管泷脸上浮现出爽朗的笑容,久美子的冷汗依旧从刚才就流个不停。屈服于他笑脸之下无言的压力,只好从实招来。 

    「集训时稍微听他提到过一点。」 

    桥本并没有叫她不准说出去,但总觉得背叛了桥本。久美子安抚着隐隐作痛的良心,观察泷的反应。只见他不以为忤地静静微笑:「别担心。我没生气。」 

    「是、是噢。」 

    久美子闻言松了一口气。泷的视线落在置于膝头的照片上,眉尾困扰地下垂,苦涩的情绪令他嘴角微微扭曲。 

    「你猜的没错,这个人的确是我太太。我们和桥本老师、新山老师是同一个管弦乐社团的人。」 

    「管弦乐的社团啊!」 

    「没错。是大学的社团,不过只是玩票性质就是了。」泷静静垂下眼。 

    「桥本老师告诉你多少?」 

    「呃,那个……他说泷老师的太太五年前过世了。」 

    「这样啊!」泷静静点头。桌上有个暖色系的马克杯。往里头一看,杯子里安静地装满黑色液体,弥漫着咖啡香气。泷的手指在把手上游移,开朗地说。 

    「我太太其实跟桥本老师一样,都是北宇治高中的学生。」 

    「真的吗?」 

    「真的。他们是我父亲还是顾问时的管乐社成员,相当于黄前同学的学长姐。」 

    「泷老师的父亲吗?」 

    泷的父亲泷透直到十年前都还是这个社团的顾问,也是为北宇治高中管乐社建立起辉煌时代的人物。挂在音乐教室里的奖状及奖杯,都是他担任顾问时得到的荣誉。自从十年前,他调到别的学校以后,北宇治的管乐社就开始一蹶不振。听丽奈说,他已经退休了,目前的兴趣是登山,同时也在某个乐团担任指挥。 

    「不瞒你说,我曾经很讨厌我父亲。」 

    「欸?」 

    泷突然降低音量,仿佛对她说悄悄话。久美子感到十分意外,目光闪烁。 

    「我父亲是很热血的指导者,所以假日几乎都花在社团活动上,完全不顾子女。我小时候基于反抗心理,打定主意绝不加入管乐社,还故意跑去参加校外的乐团。」 

    「老师没接受过令尊的指导吗?」 

    「父亲想教我,但我拒绝了。我想大概只是在闹别扭吧!」 

    或许是想起过去的自己,泷莞尔一笑。虽是理所当然的事,但久美子忍不住陷入沉思,原来老师也有小时候啊,真是难以想象。 

    「我太太很尊敬我父亲,逮到机会就和桥本老师一起大聊高中时期的回忆。我原本没打算当老师,但光靠演奏长号无法养家活口,所以才成了音乐老师。我太太倒是打从一开始就想当老师。」 

    「泷老师的太太和您一样,都是管乐社的顾问吗?」 

    「没错。不过我对管乐并没有太深刻的感情,在第一所任教的学校也是兼任合唱团的顾问和围棋社的副顾问。我太太和我相反,她非常想成为管乐社顾问,就算只是很小的社团也无所谓。」 

    「那老师为什么会变成管乐社的顾问?来这所学校以前,老师当过管乐社的副顾问对吧?」久美子问道。 

    泷微眯起眼。室内的日光灯只照亮了两人所在的角落。泷身下那张带有轮子的椅子发出尖锐的嘎吱声响。 

    「我是受人所托才当上副顾问的。我想既然要当,就得全力以赴,才对得起学生,所以花了很多精神学习……虽然人数不多,还是得到相当不错的名次。和我太太不同,我好像挺适合指导别人。」 

    「尊夫人不适合指导别人吗?」 

    「嗯,以顾问来说,她太温柔了。」 

    久美子又看了刚才的照片一眼,看起来的确是很温柔的女性,不太能想象她像美知惠那样严肃、生气的样子。 

    「倘若真想在比赛中获胜,就有很多地方必须严格要求才行。但是这对我太太来说太为难了。我认为那样的社团活动也没什么不好,但她似乎有很多烦恼。」 

    泷的妻子跟北宇治高中去年的顾问或许是同一种人。直到半年前,这个社团的口号还是「大家一起开心地演奏吧」。 

    泷在手心里把玩着相框。隔着透明的玻璃,可以看见泷年轻时的模样。他身旁的妻子当时对他说了些什么呢?脸颊微微泛红的女性,含羞带怯地笑眯了双眼。 

    「我太太是个很有活力的人。」泷说道。 

    「热爱运动,自认身体很强壮。所以当医生宣布她来日无多的时候,我脑中真的一片空白,只想尽可能守在她身边。那是我唯一能做的事。于是我想尽办法陪在她身边,工作一结束就立刻赶到医院……」 

    「真的只是转眼之间……」泷说得十分平静,反而紧紧揪住了久美子的心脏。不知怎地,她的喉咙好痛。涌上心头的感情,强烈撼动着久美子的胸口。 

    泷重新打起精神,恢复开朗的语气说:「我太太的梦想是带领母校的管乐社进军全国,在比赛中拿下金奖。父亲还在校时,北宇治确实是全国大赛的常胜军,但也只在全国大赛拿过一次金奖。那一届刚好是我太太入学的前一届。她努力了三年,终究还是没能拿到金奖,因此才会希望自己当上老师后,能带领学生在全国大赛拿下金奖,想帮学生实现自己没能实现的梦想。」 

    「因此泷老师才会接下管乐社的顾问一职吗?因为在全国大赛拿下金奖是尊夫人的梦想。」 

    泷露出自嘲的苦笑,视线抚过照片的表面。 

    「我太太去世以后,我就和管乐保持距离,无论如何都没有力气再拿起乐器。可是当我决定到北宇治任教时,父亲来找过我。」 

    「老师的父亲吗?」 

    「没错,他问我愿不愿意当管乐社的顾问,好像是校长拜托父亲的。我想了很多,要是一直钻牛角尖,我太太一定会生气,所以就接下顾问的工作。」 

    久美子边听泷说明,脑海中浮现出平常的音乐教室。挂在墙上的照片的确都加上了全国大赛这个璀璨生辉的字眼,但是只有一张照片出现过金奖二字。就连那个黄金时代,也只有一次征服过全国大赛金奖这座高不可攀的门槛,今年首次出战的久美子等人,真的有办法达成那个目标吗?周围的人都说「光是能参加全国大赛就已经很了不起了」,但是既然要参加,大家还是希望能拿下金奖。都走到这一步了,绝不能随便吹吹。要做就要全力以赴,这是全体社员的真心话。 

    或许是讲述自己的私事很不好意思,泷难为情地撩起刘海。大大的手没有戴任何饰品,不经意地刺激着久美子的记忆。 

    「对了,台风那天,老师戴了戒指。」 

    那天,泷的无名指的确戴着一枚银色的戒指。泷有些惊讶地眨了眨眼。 

    「你当时也问过我这个问题呢!女生果然比较容易注意到细节。」 

    女生这个字眼让久美子不由得羞红了脸,仿佛被当成一个大人看待,真令人害臊。不知不觉间,久美子握住自己的指尖,或许太用力了,指甲底下隐隐作痛。 

    「那天是我太太的忌日,所以才特别戴上。平常我不戴首饰,因为吹奏乐器时,可能会刮伤乐器。」 

    「那束白色向日葵……」 

    「哦,你是指意大利白向日葵吗?那是我太太喜欢的花。说来真不好意思,那是我求婚时送给她的花。」 

    泷细说从头的音调中,有着缅怀过去的声响。他现在肯定也还深爱着妻子。久美子静静躲开他落在照片上的柔和视线,丽奈的脸庞浮现在脑海,旋即又消失。 

    「意大利白向日葵的花语是——」 

    绿辉天真无邪的声音回荡在久美子耳畔。她想起泷那天买的白色花瓣,小小的花,煞是可爱。泷买那束花的时候在想什么呢?冒着风雨,西装都淋湿了,仍然执意要买花,明明收花的人已经不在了。 

    「老师很爱你太太呢!」 

    久美子无意识地脱口而出。在寂静填满的办公室里,她的声音跌落在地板上。泷回以苦笑,尴尬地低垂着眼。那张脸上的表情早已说明了一切。 

    \section{吹响吧!上低音号}
    放学后的分组练习教室里,夏纪正专心练习指定曲,〈娥眉月之舞〉具有特色的旋律在狭小的教室里穿梭来去。绿辉认真弹奏比自己身体还高大的低音大提琴,梨子和卓也则在合奏自选曲需要改进的部分,叶月一脸正经进行基础练习,看起来跟稍早之前并没有太大的差别,只是不见明日香的身影,也听不见她那美丽的音色。 

    久美子为了掩饰失望,正在为活塞阀上油。透明的液体流经金属部分的银色表面,只要上下滑动,就能让液体遍布全体。确定活塞的动作变得顺畅,再转回固定用的零件部分。沾满油的手散发出刺鼻的金属臭味。久美子将上低音号立在地上,小心不打扰到其他人,安静走向洗手台。 

    久美子翻看过行事历,时间正一分一秒地流逝。她回过神来,明天就是与明日香一起温书的日子。学校因为有说明会,明天放学后禁止所有社团活动,可以提早放学回家固然高兴,但不能进行社团活动很糟糕。久美子指尖浸泡在从水龙头奔流而出的水中,悄然叹息。 

    「啊,好想吃起司汉堡排!」 

    耳边突然传来一阵声音,把久美子吓了一跳,她转向声音的来处一看,只见秀一腋下夹着毛巾,正弯腰驼背转开水龙头。看样子是刚上完厕所。他按了几下泡沫式洗手乳的压头,搞得洗手台都是泡沫。久美子目瞪口呆地说:「去吃不就好了。」 

    「现在没钱,吃不了。」 

    「又买了什么无聊的东西吧!」 

    「才不无聊,只是有点迷上手机游戏。」 

    秀一找借口似地辩解,仔细洗干净自己的手。纯白的泡沫从他手中溢出,滑落在洗手台表面,无声无息地被排水口吸入。 

    「花了多少钱?」 

    「……两千圆。」 

    「哇!」 

    秀一感受到来自久美子的轻蔑视线,连珠炮地反驳:「哇什么哇,那是我自己的零用钱,爱怎么用是我的自由吧!」 

    「可是,你也因此变得口袋空空不是吗?」 

    「嗯,这倒也是。」 

    他让水冲走泡沫,用毛巾擦干手上的水分。大概也没什么特别想说的事,秀一打算就这么结束话题。久美子随口对他说出突然浮现在脑海中的话。 

    「秀一,你听过意大利白向日葵吗?」 

    「啥?那是什么?颜料的名称?」 

    果然是预料中的反应,久美子装模作样地耸肩。 

    「秀一没办法变成泷老师那样呢!」 

    「什么?那个什么向日葵的很重要吗?」 

    「并没有。」 

    久美子用手帕擦干双手,直接转身走向分组练习室。「等一下啦!」秀一说道,慌慌张张追上来。那模样不知怎地让她联想到大型犬,久美子努力忍住不要笑。 

    久美子一回到家就扑到床上。大概是因为最近突然变得好冷,身体比平常还要用力,肩膀一带硬邦邦的。久美子决定明天要穿大衣去明日香家,就在不敌睡意的诱惑,闭上眼的瞬间,房门被粗鲁地敲得咚咚作响。 

    「久美子,你在吗?」 

    声音从门外传来,不由分说的态度摆明知道她在家。家里只有一个人会发出这种有气无力的声音。久美子压下乌云密布的心情,默不作声地开门,果不其然,穿着汗衫的麻美子就站在门口。 

    「……什么事?」 

    久美子不知该有何反应才好,语气不受控制地不是很好。姐姐的瞳孔轻轻上下移动,仿佛在打量还穿着制服的久美子有几分斤两。她已经很久没看过脂粉不施的姐姐了,双眼没有假睫毛助威,看起来比平常软弱。 

    「CD借我。」 

    「哪张CD?」 

    「你吹的CD。」 

    意想不到的要求令久美子大吃一惊,她将录有关西大赛演奏的CD递给麻美子。这张CD是秀一的母亲在比赛会场买来送给久美子母亲的。秀一母亲具有热心收集所有与自家儿子相关事物的习惯,从他们国中时候开始,每次有比赛就会买好几张DVD或CD作纪念。据秀一透露,她现在非常中意泷顾问。 

    「嗯哼。」 

    麻美子看了写在盒子上的文字「关西管乐大赛」一眼,百无聊赖地哼了一声。没兴趣就不要借啊!久美子心想,但死也不会说出口。因为姐姐生起气来很可怕。 

    「那我借走了。」 

    麻美子丢下这句,便走出久美子的房间。久美子还以为她一定会说些讨人厌的话,害她有些愣住。母亲状况外的叫声从客厅冲进呆站在原地的久美子耳中。 

    「久美子,吃饭了!」 

    「好。」 

    久美子顺从地回答,关掉房间里的灯。早知道就先换掉制服了。她低头看着已经皱巴巴的百褶裙,忍不住叹了一口气。 

    终于来到与明日香一起温书的当天,久美子凝视着冉冉上升的旭日,手脚俐落地开始准备。 

    她们约好放学后直接去明日香家,万一准备得不够周到,可是会出大事的。为了不要忘记东西,久美子检查过书包好几次。铅笔盒,带了。笔记本,带了。钥匙,带了。确定没有漏掉任何东西后,久美子重新面向镜子,反复压平有点自然鬈的头发,将水手服的领结调整到完美的形状。比起藏青色的袜子,白色的比较好吧!裙子太短可能会很失礼。 

    因为对方是明日香,久美子才会对穿着打扮讲究到这种程度。不知充满谜团的学姐家是什么样子,万一是豪宅该怎么办?久美子冷汗直流,她可以轻易想象学姐笑得云淡风轻地说「这里是茶室」。 

    久美子一整天都很不安,无法专心上课。她与明日香约在楼梯口,想到再过不久就是只有两个人独处的读书会,心情愈来愈低落。她不讨厌明日香,只是很紧张,感觉胃那一带好像被谁紧紧捏住了。久美子撑着下巴,叹了一口大气。 

    「黄前,你有在听吗?」 

    「有,我有在听!」 

    迫力十足的声音响遍教室,瞬间将久美子的意识拉出思考之海。她抬起头来,美知惠正傻眼地看着自己,手里拿着一张薄薄的纸。 

    「赶快来拿你的出路调查表。」 

    「对、对不起。」 

    叶月和绿辉偷偷看了她一眼。久美子连忙站起来,冲向美知惠。满是皱纹的手指递出调查表给她。第一志愿、第二志愿,文字旁边的栏位一片空白,什么都还没写,唯有截止日期清楚印在纸上。 

    「下次的双方面谈除了调查表以外,还会讨论暑假结束时的模拟考成绩,请仔细考虑清楚,写下未来的目标。本周末就会公布模拟考的成绩。」 

    「老师。」叶月笔直地举起右手。美知惠继续发下调查表,要她接着说。 

    「什么事?」 

    「如果还没决定要念哪所大学该怎么办?」 

    美知惠抱着胳膊,稍微想了一下。过程中,学生依旧七嘴八舌地谈天说地。「说的也是。」美知惠往教室里看了一圈,室内突然变得鸦雀无声。 

    「才一年级秋天,这种学生还不少。只要写下大概的范围就行了,国立大学、私立大学、专科学校、就业……选择要多少有多少,先写下自己的目标即可。重点在于将来想成为什么样的人。」 

    将来啊……久美子低头看着一片空白的调查表,怔忡地想。明明才刚考上北宇治高中,就得思考三年级以后的事吗?明明自己还跟以前一样,只有时间追过还在原地伫立的她往前跑。好想保持这样,难道只是孩子气的任性吗? 

    往隔壁一看,绿辉正用自动铅笔在空格内写下自己也听过的女子大学名称。 

    「小绿已经决定好学校啦?」 

    「嗯,这里可以学习服装设计。」 

    「这、这样啊!」 

    「我一点头绪也没有。」 

    叶月坐在对面的座位搔头。听到还有人跟自己一样,久美子稍微松了口气。 

    「现在开始想就好了,船到桥头自然直!」 

    叶月强而有力地断言道,绿辉噘起嘴。 

    「船到桥头是会自然直没错,可是,小绿不喜欢用船到桥头自然直的心态来决定出路,因为这关系到自己的一生。」 

    「哇!不要用大道理攻击我啦!」 

    叶月以非常夸张的动作捂住耳朵,手里还捏着一片空白的出路调查表。丽奈和秀一大概也已经决定好自己的出路了。望向窗外,厚厚的鼠灰色云层优雅地悠游于空中。从云层落下的水滴将地面染成黑色。 

    宣告放学的钟声响起,学生一起移动。久美子走到楼梯口,明日香已经等在那里。只见她正倚在墙上看文库本。阳光穿过玻璃窗,温柔地照亮明日香的侧脸。久美子好想一直看着那个画面,她悄悄闭上正打算开口说话的嘴巴。明日香静静扬起脸,低垂的瞳孔轻轻颤动,捕捉到久美子的身影,抿成一条线的唇瓣微微绽放出笑意。 

    「啊,久美子。」 

    啪嗒一声,明日香阖上书本。那本书好厚。久美子在好奇心的驱使下,偷看了封面,没看过的书名。 

    「好奇?」 

    「啊,是的,好看吗?」 

    「嗯……还可以。」 

    明日香回答,把书收进书包。黑色的书包没有任何装饰,设计得十分简单,仿佛只为了追求实用性。 

    「学姐家在哪一带?」 

    「嗯?就在这附近,离学校很近。」 

    明日香将室内鞋换成球鞋,指着西口的后门。久美子没看过的白底球鞋上,有着藏青色的线条。这才想起,明日香参加比赛时总穿平底鞋。 

    明日香或许是察觉到久美子的视线,露齿一笑。 

    「这是我最近新买的球鞋,很可爱吧?」 

    「的确很可爱。」 

    下次我也来买双上学用的鞋吧!久美子边想,不经意盯着两人同行的脚,轮流观察自己的鞋和明日香的鞋,两者的差异历历在目。 

    「学姐的脚好大啊!」 

    明日香苦笑回答:「我从以前就为脚太大伤透脑筋,淑女鞋很难找到我的尺寸。」 

    「听说个子高的人脚都很大。」 

    「欸,真的吗?有关系吗?」 

    两人走在通往西口的走道上,一路闲聊着。因为腿长,明日香的一步比久美子的一步大得多,久美子必须加快脚步才能与明日香并肩而行。 

    「明日香,等一下!」 

    久美子听见背后突如其来的叫声,惊讶地停下脚步,她回头一看,眼前是上气不接下气的香织。她大概是一路跑过来,脸颊泛着淡淡的红晕。 

    「怎么了?」 

    明日香不解地歪着脖子。香织摇摇头,愁眉苦脸,有些不知所措。 

    「没什么特别的事,只是看到你的背影,想跟你一起回家。」 

    「所以你刻意追上来吗?」 

    明日香愉快地咯咯笑。香织背着书包,望向久美子。 

    「黄前同学,我可以和你们一起回家吗?」 

    「当然可以。」久美子忙不迭点头。「香织学姐家也在同一个方向吗?」 

    「嗯,还满近的喔!」 

    香织点头,不动声色卡进明日香和久美子之间,简直像是死守那个位置。久美子悄悄窥探香织,只见后者的神情跟平常一样。 

    「明日香,社团活动没问题吧?」 

    「嗯?总会有办法的。」 

    「泷老师说要夏纪代替你上场比赛。」 

    「目前的确是那个可能性高一点,毕竟比赛时不能只有一把上低音号。」 

    「可是,我希望跟明日香一起比赛。」 

    「我也想上场啊!」 

    明日香与香织都和平常没两样,但两人之间弥漫着一股剑拔弩张的紧张感。久美子不敢插嘴,默默往前走。香织的语气明明夹杂着苛责的情绪,明日香却不当一回事,净回些四两拨千金的话。 

    「明日香这样也无所谓吗?」 

    即使被香织逼问,明日香依旧一脸镇定。穿过西口的后门,三人走在杳无人烟的后巷。久美子平常绝不会经过这条路,她瞥向稍远处的平交道,四下张望。穿着北宇治制服的学生正拼命踩着脚踏车,奔驰在马路的对向。嘎啦、嘎啦、嘎啦,车轮转动的声响空虚地回荡着。 

    「当然不是无所谓,只是不得不先做好最坏的打算。」 

    「还是无法说服令堂吗?」 

    「我是没办法了。」 

    明日香将黑发塞到耳后。香织眯着眼,咬紧下唇,毫不掩饰自己的不满。她的脸颊勾勒出柔美的曲线,有些激动地染上了红晕。 

    「因为那个人讨厌管乐。」 

    明日香苦笑,静静垂下视线,打算以透着死心,与大人无异的表情来接受这一切。久美子沉默地凝视她的红框眼镜,内心突然涌起一股狂暴的情绪,好想用力扭断架在耳朵上的细致金属部分。她总觉得火冒三丈,觉得明日香的母亲太过分了。 

    「黄前同学,」不期然被点到名,久美子抬起头,与一脸严肃的香织对上眼。香织从书包里拿出一个纸袋。「这个给你。」 

    香织递给她的纸袋沉甸甸的,白色的表面以龙飞凤舞的笔迹写着幸富堂。 

    「啊,这不是那个栗子馒头吗?」 

    久美子还来不及看内容物,明日香已经先发难了。 

    「没错。」香织点点头。「读书会需要茶点吧?」 

    「我可以收下吗?」 

    「因为我一直受到明日香的关照嘛!」 

    香织嫣然一笑。纸袋里确实有个用包装纸包得很漂亮的盒子。「栗子馒头」的文字上描绘着兔子傻里傻气的脸,敢情是这家日式糕饼店的吉祥物。 

    「黄前同学,我往这边。」 

    香织站在十字路口,指着反方向。设置于街角的转角镜正兴味盎然俯视着她们。倒映在镜子里的世界被拉长,扭曲变形。 

    「明日香就拜托你了。」 

    「好,好的。」 

    她指的大概是那个作战行动。带回明日香学姐大作战。想起夏纪之前的交代,久美子绷紧脸上的肌肉。作战名称虽然很蠢,但夏纪无疑是对自己有所期待,才把这个重责大任交给自己。久美子用力握紧纸袋,香织静静微笑。 

    「拜托她?照你这种说法,好像是她要照顾我。」 

    明日香故作姿态地说,顶了顶久美子的肩膀,笑咪咪低头看着拼命解释「呃,不是那个意思啦」的学妹。 

    「啊!」 

    冷不防,香织突然蹲下。久美子的视线因此又回到她身上。 

    「明日香,你的鞋带松了。」 

    香织伸手探向明日香的鞋带。她宛如白鱼的手指抓住明日香的雪白鞋带,藏青色的裙子底下,隐约可见香织柔嫩的大腿,她的膝盖跪在地上,面向明日香的鞋子,重心往前倾,藏青色的制服挤出圆弧形的皱褶。从明日香脚下延伸出来的影子,覆盖着香织的身体。明日香只是居高临下,眼睛眨也不眨地看着香织帮自己绑鞋带的身影,眼神里没有感情。 

    香织的手指灵巧地将松开的鞋带绑成蝴蝶结。「好了。」香织笑着扬起脸。明日香将长发塞到耳后,喉咙咕嘟一声。 

    「谢啦!」 

    「不客气。」 

    香织站起来,脸上挂着区区小事不足挂齿的表情。久美子感觉好像看到什么不该看的画面,视线局促无措地飘来飘去。 

    「那么黄前同学,温书要加油喔!」 

    香织拍拍久美子的肩。擦身而过的瞬间,甜甜的香气掠过久美子鼻尖。 

    「谢、谢谢学姐。」 

    「掰掰。」 

    明日香语气轻松地挥挥手,与毕恭毕敬鞠躬道别的久美子互为对照。香织只转过头来,轻轻挥手。明日香目送香织的背影渐行渐远后,终于面向久美子,嘴角微微上扬。 

    「香织很可爱吧?」 

    这句话到底是什么意思呢?久美子只能不置可否地点头。 

    「这就是我家。」 

    明日香的指尖前方是极为普通的独栋房子,而非久美子想象中的豪宅。看起来已经有些年代,在今时今日算是比较罕见的日式建筑。门口有一排显然受到悉心照顾的小巧盆栽,盛开着久美子不知其名的黄色花朵。 

    「请进。」 

    「打扰了。」 

    久美子听从明日香的指示,蹑手蹑脚踏进玄关,将鞋子整齐摆在角落,学姐已率先进屋。木头发出干燥的气味。她们踩在木头走廊上,老旧的地板嘎吱作响,有点恐怖。 

    「我房间在这边。」 

    明日香边上楼,边对久美子招手。屋子里静悄悄的,没有其他人在。壁钟的声响让人感到不寒而栗。 

    「到了。」 

    明日香打开拉门,催久美子进房。她的房间整洁得不像是高中女生的房间,干净的榻榻米散发出蔺草的味道。久美子依言正襟危坐在房间正中央,最先映入眼帘的是书架上密密麻麻的参考书。房里有张与矮桌相仿的小桌,上头井然有序堆满铅笔盒和教科书,除此之外没有任何私人物品。和室当然没有床铺,但是就连电脑或玩偶或CD播放器都没有,总觉得非常寂寥。 

    「我去倒茶。」 

    明日香走出房间。久美子东张西望,发现上低音号的乐器盒躲在书架后面。堆满在桌上的红色题库印有超难考大学的名称。房里只有学习用的工具和乐器,与久美子的房间相去甚远。 

    「不自在?」 

    明日香打开拉门走进来。久美子伸直背脊,连忙摇头。 

    「还、还好。」 

    「不用那么紧张。」 

    明日香在桌上放下两只玻璃杯,杯子上印有猫咪的图案,设计得相当可爱。明日香边往杯子里倒入麦茶,莞尔一笑。 

    「这是我生日的时候,香织送给我的。」 

    「这样啊!」 

    「对呀。听说是组对杯,香织家也有。」 

    液体发出咕嘟咕嘟的声音。久美子心不在焉盯着缓缓上升的气泡。明日香把麦茶放回托盘,慢条斯理翻开数学参考书,空白部分巨细靡遗写满了补充说明。这就是聪明人的参考书啊……久美子看了看自己与全新没两样的参考书。 

    「你哪里不懂?」 

    明日香突然切入正题,久美子一时之间答不上来。 

    「呃,二次函数……」 

    「的确有很多人都会死在二次函数上。配方法应该没问题吧?」 

    「配方法?」 

    这是什么咒语?见久美子满头问号,明日香目瞪口呆地仰天长叹。 

    「原来如此。看样子问题很严重呢!」 

    「先从基础做起吧!」明日香说道,开始地狱般的特训。 

    「好,休息一下吧!」 

    三个小时后,还以为会永远持续下去的复习总算暂时告一段落。久美子握着自动铅笔的手好痛,中指都长茧了,又红又肿,她筋疲力尽地趴在桌上,明日香哈哈大笑。 

    「你这种程度居然能混到现在。」 

    「我对二次函数真的很不在行。」 

    「不过,解了这么多道练习题以后,已经变得比较习惯吧?」 

    明日香是很优秀的老师,听她解说后,过去怎么也搞不懂的地方仿佛也能理解了。原来判别式是这样用的啊……居然反过来佩服起来了。 

    「只要背熟公式,搞懂对应的方法,就能解开这些问题。要是因为不喜欢而逃避,以后要用到的时候就会不知所措。」 

    「说、说得也是。」 

    久美子捧着玻璃杯,喝下一口里头的麦茶。明日香家的麦茶煮得比久美子家的浓一点。为什么会觉得别人家的茶味道不一样呢?久美子看明日香咕嘟咕嘟地喝光麦茶,思考这个问题。 

    「明日香学姐,你平常在家都做些什么?」 

    「嗯?看了这个房间就知道吧?」明日香耸耸肩。 

    「读书吗?」 

    「没错。反正除此之外也没别的事可做。」 

    明日香脱掉袜子,伸直脚丫。只消一眼,就能看出裸露的小腿肚十分紧实,与香织柔若无骨的双腿截然不同。久美子视线落在学姐大胆地从裙子底下露出来见人的大腿上,下意识吞了口水,眼神不晓得该往哪里去。 

    「学姐,你为什么主动说要帮我复习功课?」 

    为了转换气氛,久美子从矮桌上探出身子问道。明日香不看她,一脸无趣把玩着参考书的边边角角。 

    「我听到小道消息,说久美子的成绩很危险。」 

    「欸,谁告诉你的?」 

    该不会是叶月或绿辉打的小报告吧?明日香对不自觉往前倾的久美子投以调侃的视线。久美子隔着眼镜和学姐四目相交,不知怎地,心脏跳得好大声。明日香纤长睫毛下的双眼静静看着久美子。 

    「呵呵,开玩笑的。」明日香说道,往玻璃杯里倒入麦茶。 

    久美子不知道该怎么回嘴才好,求救地凝视着眼前的学姐。明日香不耐烦地拨开自己的头发,将几乎满出来的杯子推到久美子面前。玻璃杯每晃动一下,表面也跟着微微震颤。 

    「其实是我想跟你聊聊。」 

    不晓得是对明日香的声音做出反应,还是久美子呼吸得太用力,麦茶轻易从玻璃杯里溢出,棕色的水珠滴落在白色的桌上。明日香连忙拿面纸擦拭,液体马上被白色面纸吸收。 

    明日香眯起眼,拉过放在自己手边的玻璃杯。久美子问她:「跟我聊聊?」 

    「没错。我认为告诉久美子也无妨。」明日香说道,从架上抽出一本书。 

    久美子也见过那个水蓝色的封面。《简单!上低音号&低音号入门指南》,就是小学时学姐借给她的书。 

    「你看过这本书吗?」 

    「嗯,小学的时候看过。」 

    久美子老实地点头。那本书写得很好,从乐器的保养方法到基础练习的方法,一应俱全。 

    「这本书的作者叫进藤正和,你知道吗?」 

    「当然知道,他是专业的上低音号演奏者。」 

    不如说凡是吹上低音号的人,没有人不知道他。他就是这么有名的人。 

    久美子的反应令明日香大感无趣地将嘴唇抿成一条线,靠在桌上,指腹用力按住作者名称的地方。长长的黑发在久美子的视野边缘形成一道飞瀑。明日香以无所谓的语气说:「这个人是我以前的爸爸。」 

    「什么?」 

    久美子听不懂这句话的意思,足足傻了三秒。明日香不理会反复眨着眼睛的久美子,自顾自地将茶点送入口中。久美子目送两片加入巧克力脆片的饼干消失在她口中后,终于开口。 

    「呃,那个,我有好多问题想问。」 

    「请说、请说。」 

    相较于不知所措的久美子,明日香跟平常一样泰然自若,打开一盒新的饼干,催久美子问下去。 

    「首先,以前的爸爸是什么意思?」 

    「我爸妈在我小时候就离婚了。」 

    原来如此,所以才是「以前的」爸爸啊。久美子紧握着明日香给她的饼干,内心了然。 

    「不过,虽说是小时候,但我当时才两岁,几乎什么都不记得了。」 

    「好、好沉重的话题啊……」 

    没想到除了丽奈以外,还能遇到父亲是专业演奏者的人。久美子忍不住叹息。明日香露出自嘲的笑容,视线始终落在饼干盒上。 

    「久美子上次在教职员办公室见到我妈了,对吧?」 

    「啊,对。」 

    突然转移话题,久美子愣了一下。明日香用指尖撕开饼干的个别包装,将撕下来的包装纸扔进垃圾桶。黑色塑胶垃圾桶里装满了揉成一团的纸屑。 

    「老实说,你觉得怎样?」 

    「呃,那个……」 

    久美子想不到好听的形容词,只能打马虎眼。当时,明日香的母亲对泷穷追猛打,模样怎么看都不正常,那个画面让人感觉好不舒服,至今仍像逐渐融化的冰淇淋般,黏在久美子的脑子里。 

    「那个人的脑子不太正常。」 

    明日香仿佛猜到久美子的心情而苦笑着说。表示赞成也很失礼,久美子咬了一口明日香递给她的饼干,巧克力脆片的苦涩顿时在舌尖上化开。 

    「我的监护权在我妈手上,那个人死都不愿意我跟父亲扯上任何关系,所以从小就不让我和父亲见面,我也没有这位进藤先生就是我父亲的感觉。」 

    「令堂为何那么讨厌令尊呢?」 

    「因为那个人的占有欲很强。」明日香低眉敛眼地说。 

    「动不动就歇斯底里,我爸大概是受不了这点才离家出走吧!要是我有这种老婆,我也受不了。那个人认为自己就是因为这样才被抛弃的,结果还真的变成那样了。」 

    明日香的口吻十分平静,简直像是在谈论别人的事。久美子对她的冷静感到毛骨悚然。对久美子而言,同时有一对名为父母的生物是理所当然的。单亲家庭在现在的社会的确已经不稀奇了,久美子的朋友当中也有几个单亲的小孩。但是至少在久美子身边,没有人会像明日香这样坦承自己的家务事。 

    明日香是非常冷静的人,总是细心观察四周,充分了解自己的立场。久美子曾经隐隐认为纯粹是明日香的聪明造就了这样的个性,但或许是她的过去迫使她非这样不可也说不定。 

    「别误会,我并不讨厌那个人。」 

    那个人。明日香反复说着这三个字。从她的声线可以听出,她对母亲的轻蔑与少许的怜悯。 

    「再怎么说,毕竟是那个人把我养到这么大。要把我养到这么大,肯定花了不少钱吧!照顾我也花了不少心力。我欠那个人很多,一定得偿还才行。」 

    钱、心力……接连从明日香口中冒出来的字眼,令久美子听得头昏眼花。她的每一句话都像是例行公事,说好听是客观,说难听是非常冷漠客套。想当然,久美子也曾经不只一次对自己的父母感到抱歉。都去补习了,成绩还是不见起色的话,等于是浪费补习费,感到很过意不去;对每天晚上回家都有热腾腾的饭菜可吃心存感激。然而,明日香口中的「心力」和「钱」听起来远比久美子想到的那些歉意与感激更为冰冷。她或许打从一开始就当父母是外人,所以不认为父母给她的一切是理所当然,所以才会不以为意地说出「偿还」二字。 

    「……学姐其实很讨厌令堂吧?」久美子问道。 

    明日香嘴角挂着一抹苦笑。眼镜镜片反射着日光灯的光线,难以读取她的表情。明日香撑着腮帮子,置身事外地说:「或许吧!事到如今,已非喜欢或讨厌的问题了。」 

    「是这样的吗?」 

    久美子不晓得该怎么回答才好,只能再咬一口手中的饼干,酥酥脆脆的轻快声音响彻了狭小的和室,显得非常突兀。明日香冷不防伸出手来,拿起丢在地上的参考书,红色的考古题封面印有超难考大学的名称。 

    「那天,那个人不是说了吗?『社团活动只会扯我女儿的后腿。』」 

    她指的是泷与她母亲起争执那天吧!久美子已经不记得对话的内容了,但仍旧默默点头,不想打断她的谈兴。明日香一脸忧郁,怔忡地抚摸参考书的封面。 

    「对我来说,比起社团活动,那个人才是我的枷锁,而且是一生都无法挣脱的。」 

    「枷锁……吗?」 

    「没错。只不过啊,那个人其实无意让我受苦,她是真的以为那么做是为我好。那个人心里已经画好一个幸福的蓝图,努力想把我塞进去,所以绝不允许我超出那个框架一公分。之所以反对我继续参加社团,就是基于这个原因。」 

    只一瞬间,姐姐的脸闪过久美子脑海。幸福的蓝图。久美子的父母大概也跟明日香的母亲一样,依照幸福的蓝图把她们养到这么大。话说回来,真的有人在养儿育女时完全没有任何期待吗?自己还是小孩,不是很清楚大人在想些什么,或许长大成人,变成母亲以后,自然就会了解。 

    「不过,她之所以痛恨管乐,不只这个原因。」 

    明日香说道,望向放在书架后面的乐器盒。一尘不染的盒身保养得很周到,与学校提供的明显不同,这是明日香私人的乐器。 

    「刚才也说过了,那位进藤先生是我父亲。小学一年级的暑假,我突然收到这个和一本破破烂烂的笔记本。」 

    「这个指的是上低音号吗?」 

    「没错。」 

    「欸,寄到家里?」 

    「事前没有任何预兆,害我吓了一大跳喔!那个人去上班不在,只有我在家,所以是我直接从送快递的手上签收。进藤先生大概也是故意锁定这个时机。」 

    万一以父亲名义寄出的礼物是母亲在家时寄到,可能会被母亲没收。为了避免这种事发生,他所采取的行动确实很正确。 

    「盒子里有一封信,写了很多对我说的话,像是一直很惦记我之类的,但那一点都不重要就是了。信上写着进藤先生从小学一年级就开始吹上低音号,所以当我长到相同的年纪,也想送我相同的乐器。说是这么说,寄来的上低音号并不是全新品,而是进藤先生以前基于玩票性质买的上低音号。」 

    明日香或许是想到过去的事,而苦笑着说。 

    「突然收到只听过名字的父亲寄来莫名其妙的东西,一定会有点在意吧?想知道那是什么。从此以后,我就完全迷上上低音号了。以前这一带有家乐器行,现在已经倒了,那里的店员以前吹过上低音号,告诉我很多知识。我从国中才开始正式加入社团吹奏乐器,在那之前,一直是在店员的指导下自行摸索。」 

    「听起来好像漫画情节。」 

    「就是说啊!」 

    明日香修长的手指抚摸黑色的乐器盒。年幼的明日香对父亲突然寄来的礼物有什么感想呢? 

    「可是,那个人好像非常痛恨女儿吹奏父亲寄来的乐器。我说要加入管乐社时,我们大吵了一架。尽管如此,她还是在我承诺会一直保持好成绩的条件下同意了。」 

    「学姐拼命学习就是基于这个原因吗?」 

    房里只有满坑满谷的参考书,令人呼吸困难。久美子四下看了一圈,淡淡问道。 

    「或许是吧!」明日香模棱两可地微笑,轻声细语回答。「为了继续做我想做的事,唯有用功一途。」 

    听到这句话,久美子顿时面红耳赤,突然觉得自己好丢脸。为了掩饰泛红的双颊,久美子将脸埋进膝头。 

    不同于随波逐流进社团的自己,明日香凭自己的意志选择了这个地方。她是自己决定要加入管乐社、吹奏上低音号。 

    「可是,或许是遭天谴了。」 

    「天谴?」 

    明日香中气不足的软弱台词,令久美子蓦地扬起脸。明日香不甚在意地撩起自己的刘海,叹了一口大气,手里握着一把光艳照人的黑发。 

    「老实说,我以前从不把比赛当回事,就算其他人吹得很烂,只要我能吹上低音号就好。」 

    「可是,」明日香说到这里,眯起双眼,黑色乐器盒正无精打采躺在她视线前方,「泷老师来了以后,害我变得贪心了,开始真的想进军全国。」 

    「那才不是贪心,表示明日香学姐对社团活动是认真的不是吗?」 

    「不是。」 

    明日香不假思索地回答,否认的词汇冷淡得令人心惊。久美子无言以对。明日香露出与平常没两样的冷笑,笔直地面向久美子。或许是激动到出汗,几缕黑发黏在她白皙的颈项上。 

    「你知道今年全国大赛的评审有谁吗?」 

    「咦,评审吗?」 

    突然转移话题,久美子老实表示不知。久美子平常在比赛时会注意到的,充其量只有上场顺序。明日香或许是预料到她的反应,故作姿态地频频点头。 

    「不知道是当然的,因为知道也不能怎样。」 

    明日香站起来,从书架后面拿出一本薄薄的简章。那是去年全国管乐大赛的简章。她是从哪里弄到的呢?只见她以粗鲁的动作翻页,指着满是文字的页面,催促久美子阅读并排在那一页的「评审简介」文字。 

    「进藤先生是去年国中部全国大赛的评审。」 

    光滑的表面的确印有进藤正和的名字。 

    「评审不会连续两年担任同一部的评审,通常会改当其他部门的评审。比方说,这一年担任国中部评审的人,隔年就会变成高中部评审。我只是半信半疑,没想到好像真的被我猜中了。」 

    原来如此。久美子明白学姐的言下之意了,她组合搜集到的讯息片段,对明日香提出自己的推测。 

    「因为令尊是评审,所以明日香学姐想打进全国。」 

    明日香既没肯定、也没否认久美子的推测。手中的简章貌似已经翻过无数次,皱得不能再皱。 

    「京都大赛时,我还不觉得真的能进军全国,所以不管是香织独奏,还是高坂同学独奏,我真的觉得无所谓。可是面对关西大赛,我开始觉得搞不好真能进军全国……霙和希美闹不愉快的时候,我其实只想到自己。」 

    「只想到自己吗?」 

    「没错。不瞒你说,希美不在的话,比较不会惹事生非,打进全国的可能性也比较大不是吗?所以我才故意冷处理。其实早点让她们和好才是上策也说不定,但是考虑到霙在关西大赛前崩溃的风险,不能让希美回社团。因为我无论如何都想打进全国,无论如何都想让进藤先生听到我的演奏。 

    「所以我利用了社团。」明日香滔滔不绝的告白令久美子咕嘟地咽了一口口水。 

    明日香见久美子整个人僵住,不由得叹息,丰满的胸部随她的呼吸上下震荡。摊在地上的脚趾前端尴尬地微微颤动,她平常都藏在袜子里的脚趾,指甲剪得整齐又漂亮。 

    「或许因为我满脑子只想到自己,才会变成这样。不过,我已经将夏纪调教到一定的水准了,就算没有我,她也会代替我好好表现。」 

    「学姐打算就这么放弃吗?我问你会不会退出社团的时候,你不是说你会想办法吗?难道你想的办法就是让夏纪学姐代替你出场吗?」 

    明日香莞尔一笑。很讨厌的笑法。笑容里透露出死心的味道。她的反应非常成熟,久美子感觉内心涌上一股不祥的预感。 

    「我对社团的人觉得很过意不去喔!」 

    这就是明日香的答案。她伸手制止正要开口的久美子,静静站起来。久美子拿起桌上的杯子,盯着杯子里看。棕色的液体表面静静掀起涟漪。自己倒映在那上头的德性实在很没出息,久美子悄然叹息。 

    「集训的时候,我不是吹过一首莫名其妙的曲子吗?」明日香说道。 

    久美子搜寻脑内的记忆。集训第三天早晨,明日香独自一人在广场上练习。久美子直到现在都还记得当时优美的旋律,就连听的人也感到愉快的曲风。询问曲名,当时的明日香不肯告诉她。 

    「那首曲子就写在跟上低音号一起寄来的笔记本上。」 

    明日香站了起来,从书架上抽出笔记本,翻到某一页。笔记本已经很老旧,页面都泛黄了。不是乐谱用的笔记本,而是普通的笔记本。占满整张纸的五线谱歪七扭八,大概是自己徒手画的线。 

    「这首曲子该不会是令尊写的吧?」 

    「好像是。信上说这是他高中时写的曲子,还说想托付给我。老实说,我一点都不需要。」 

    「好厉害喔,还会作曲。」 

    明日香露出复杂的表情,静静把笔记本放回原位。仿佛是把父亲给她的笔记本藏放在书架后面。 

    「凭良心说,我不太喜欢这首曲子。」 

    「欸,可是你一个人的时候不是常吹吗?」 

    「这个嘛……就只是吹吹而已。」 

    「不喜欢的话才不会想吹呢!」 

    这句话让明日香微微皱眉。久美子交叉着手指,抬头仰望明日香的脸,戒慎恐惧地问始终站着不动的学姐:「那个,学姐,上次你为什么要让我听这首曲子?」 

    明日香抱着胳膊,静静垂下眼,陷入沉思,表情难得如此专注。她抓住两条手臂的指头心浮气躁地上下移动,说不定那是明日香的习惯。只见她慢条斯理地开口,试探性地说:「或许我希望你能对这首曲子嫌弃得一无是处。希望给某个人听,从那个人口中听到批评的字眼。」 

    「可是我好喜欢这首曲子,既温暖,又明亮。」 

    「真的吗?」 

    明日香的脸一下子近在眼前,久美子诚实地点头如捣蒜。 

    「我还想多听一点,现在就想听。」 

    「欸,现在?」 

    「没错,现在。」 

    明日香没料到久美子会提出这种要求,一脸错愕,视线一度在空中飘来飘去,然后慢慢降落在放置于房间角落的乐器盒上。 

    「你是说真的?不是开玩笑?」 

    「我想听学姐吹的曲子。」 

    「吃错什么药了,今天居然这么积极。」 

    「因为学姐也跟平常不一样嘛!」 

    「是吗?我倒不觉得。」 

    明日香半开玩笑地说,手伸向乐器盒,手指滑过黑色的皮制提把,勾起嘴角。 

    「去河边吧!我突然想吹了。」 

    「不能在家里吹吗?」 

    「在没有隔音设备的地方吹奏铜管乐器是想吵死邻居吗?啊,你该不会都在家里练习吧?」 

    久美子连忙否认:「才没有!我平常都在堤防上练习。啊,不过……」久美子说到一半,猛然想起一件事。 

    「丽奈家好像有隔音设备,类似录音室那种,她每天都在那里练习。」 

    「欸,专业的家果然不一样。」 

    明日香附和久美子说的话,没穿袜子就走出房间。久美子跟在她后面,拿打赤脚的学姐没办法。 

    明日香穿上凉鞋,就这么大步走出家门,鞋跟明明有点高度,脚步却四平八稳。水手服、光脚、凉鞋……真是奇妙的组合。 

    「我平常都在这里练习。」 

    明日香指着设置于堤防上的老旧长椅。周围杂草丛生,到处盛开着不知名的花,几乎与久美子的身高差不多高的芦苇在河对岸看似很舒服地迎风摇曳。明日香熟门熟路坐在长椅上,轻拍身旁的空位,看样子是要她也坐下来。 

    「夏天好像会被蚊子叮。」 

    「不要紧,大概是我的血不好喝,蚊子不太叮我。」 

    「有好喝的血吗?」 

    「我也不晓得,但香织的血好像很好喝。」 

    「咦,是吗?」 

    「你不觉得看到她的脖子就很想咬下去吗?」 

    「不,我从没这么想过。」 

    「欸,骗人的吧,久美子好怪。」 

    「才怪,是学姐比较怪。」 

    明日香边扯一些有的没有的,动作俐落地准备演奏。她的银色上低音号不同于学校那些已经很老旧的乐器,亮晶晶跟新的一样。这把上低音号要多少钱啊?看明日香修长的手指轻盈在活塞上移动,久美子心不在焉地想。 

    管乐社的成员中,自备乐器的人并不少。有些人数太多的社团,如果不自备乐器就无法挤进比较抢手的声部。体积轻薄短小的乐器通常是请父母买,但是像低音号或低音大提琴这种庞然大物,很少会有学生自掏腰包购买。 

    「准备好了。」 

    明日香朝吹嘴进气,从号口滑出圆润的中低音。光是单纯的音阶就美极了,厉害的人演奏起来真的很厉害。明日香演奏了几首基础练习用的曲子,带起乐器的温度。久美子默默倾听。上低音号不像其他乐器那么花稍,柔美的音色让听的人感到身心安顿,圆润的悠扬音色流淌在黄昏的天空里。 

    「学姐,吹那首曲子啦!」 

    「……你真的想听?」 

    「真的想听。」 

    明日香垂下视线,思索了半晌。她放开银色的吹嘴,若有所思地叹息。 

    「拜托你。」 

    「好吧,既然久美子都这么说了。」 

    明日香缓缓开始演奏。曲风极为单纯,并未特别强调技巧。即使曲式并不复杂,还是能明确表现出上低音号美好的音色。曲风缓慢平稳。明明是不熟悉的曲子,却让人觉得很怀念。 

    明日香的父亲是抱着什么样的心情写下这首曲子呢?抱着什么样的心情把这首曲子寄给女儿呢?久美子倾听余音绕梁的旋律,悄然叹息。明日香一路走来,又是抱着什么样的心情吹奏上低音号呢?此时此刻的久美子还没有勇气追问。 

    结果那天久美子并没有见到明日香的母亲。她母亲总是工作到很晚,从以前就是由明日香张罗晚饭。明日香对她说:「今天谢谢你。」久美子也向学姐道谢。香织带来的栗子馒头最后都进了明日香和久美子的肚子里,带回明日香学姐大作战显然失败了。 

    电车上几乎没人,久美子得以有位子坐。蓝色天空隔着四角形车窗由左向右流逝。车身一摇晃,垂在头上的吊环就不安地晃来晃去。脚边吹出来的热风染红久美子的脚。 

    明日香刚才的演奏至今仍萦绕在久美子耳边,她吹奏的上低音号音色总是那么美,可是又夹杂了少许令人窒息的情绪。 

    大概是因为明日香太成熟了,摆出什么都懂的样子,借此对自己的想法视若无睹。对她来说,尊重母亲的意思才是最重要的。但久美子不以为然,因为人生是自己的,只要做自己想做的事就好了。久美子会这么想,难道是因为自己生活的环境太养尊处优吗?闭上双眼,脑子里闪过明日香苦涩的表情。 

    所以我利用了社团。 

    每次忆起这句苦不堪言的话,久美子都感觉自己的心脏充满了不安的悸动。 

    第二天,明日香连学校的课都缺席。约莫是很在意吧,分组练习一到休息时间,低音组的成员就不约而同围到久美子身边,动作之快,令她有点不知所措。 

    「明日香学姐还好吗?」叶月问道。 

    久美子沉吟了半晌才回答。 

    「看起来很有精神,但是会不会回来社团,还很难说。」 

    「栗子馒头呢?」夏纪从背后发问。 

    栗子馒头?四周出现不明所以的反应,久美子不理他们的反应,径自回答:「她母亲不在家,所以都被我们吃掉了。」 

    「唉,人算不如天算呢!」 

    夏纪抱着胳膊皱眉。卓也和梨子面面相觑,不安叹息。绿辉大失所望地垮下肩膀。 

    「学姐还是不能参加比赛吗?期限是本周末的合奏吧?」 

    「泷老师是这么说的。」 

    「再这样下去,明日香学姐就回不来了。啊,我绝对没有夏纪学姐不能参加A部门的意思喔!」 

    绿辉连忙开始解释,夏纪苦笑着说:「别紧张,我明白。」 

    梨子在一旁撑着下巴,喃喃自语:「果然还是不能违抗她母亲的意思吗?」 

    绿辉闻言,噘着嘴抱怨。只见她像个孩子似闹别扭,火冒三丈地滔滔不绝:「要是连明日香学姐的成绩都不满意,小绿不是要去自杀了吗?学姐明明对课业和社团都全力以赴,还要反对的话岂不是太过分了。那样的妈妈,小绿才不要!」 

    「明日香学姐的母亲也有自己的考量吧!或许等你长大以后就能明白了。」 

    「可是、可是!小绿现在还是高中生,才不想知道大人在想什么。社团活动对考试是没有帮助没错,但人生又不是只有考试!」 

    「问题是明日香学姐的母亲不这么想啊!或许已经没有我们帮得上忙的地方了。」夏纪咬牙切齿地说道。 

    卓也点点头,表示同意。 

    「这是明日香学姐自己的问题,我们只能静观其变。」 

    「可是我好想帮助她。」梨子烦恼地说道,学妹都无法反驳。大家都想着同一件事,想为明日香学姐做点什么,可是又不知道该怎么做。 

    「很遗憾,」梨子有气无力接着说。「考虑到将来的事,就这么退出社团,专心学习,才真的是为她好也说不定。」 

    久美子等人听到这里都沉默了。教室里充满令人窒息的沉默。梨子说的或许没错,他们所做的一切或许只是为了自我满足,或许明日香的母亲才是正确的。 

    「可是,」绿辉不依地噘着嘴。「即便如此,小绿也想跟明日香学姐一起玩管乐。」 

    夏纪默不作声揉乱了绿辉的头发。 

    久美子回到家,大门难得没上锁。大概是因为姐姐在家,父亲嫌麻烦懒得锁门吧!好危险啊!久美子转动门把。 

    久美子推开客厅门,一阵强烈臭味扑鼻而来,她下意识皱眉,是锅子烧焦的臭味。 

    「好臭!」久美子犯嘀咕。 

    厨房传来心浮气躁的嘟嚷声。久美子定睛一看,麻美子绷着一张脸,凝视着冒黑烟的锅子,久美子赶紧冲上前去关掉瓦斯。姐姐这时才注意到她的存在,束手无策瞥了她一眼。以前是姐姐比较高,曾几何时,久美子的身高已经追过姐姐了。久美子若想看姐姐,即使不是故意的,也会变成居高临下的角度。这个发现让久美子受到冲击。 

    「……你在做什么?」 

    「我想煮味噌汤。」 

    「为什么煮个味噌汤会把锅子烧成焦黑呢?」 

    「呃,不是要先煮熟蔬菜吗?」 

    「就算要煮熟,这也煮过头了吧!煮滚就行了。」 

    久美子的纠正让麻美子露出不好意思的表情。没想到这个人居然是料理白痴。她平常是怎么料理一日三餐的?久美子忍不住叹气。 

    「所以呢,你怎么突然想做菜?」 

    「因为妈妈说他们今天会晚点回来,我想帮忙做饭。」 

    「你们和好了?」久美子问道。 

    麻美子无言摇头,染成棕色的头发随之轻柔摇晃。 

    「接下来才要和好。」 

    「……这样啊!」 

    久美子望了烧焦的锅子一眼,又看了看放在桌上的食材,量多到几乎放不进冰箱。一想到姐姐是为了与母亲和好才买那么多菜,不免有点同情眼前的姐姐。久美子收拾起不甘的心情对姐姐说:「结果只是增加要洗的碗盘。料理就算了,你想办法处理一下烧焦的锅子吧,饭我来做。」 

    这句话让麻美子瞪大了双眼。 

    「……你会做饭?」 

    「比你会一点。」 

    「哼。」 

    麻美子自讨没趣地哼了一声,戴上橡胶手套,拿起鬃刷,用力刷着锅子表面。铁氟龙加工的涂料只怕都要刷掉了。久美子想归想,什么也没说。 

    「你今天也去社团活动了?」 

    「嗯。」 

    「是噢。」 

    「……」 

    「……」 

    水从水龙头奔流而下,形成一道透明的瀑布,延伸到麻美子手中。她的双手沾满泡沫,忙着刷锅子。久美子从冰箱里拿出红萝卜和洋葱,用菜刀唰唰唰地切碎。菜刀在砧板上敲出轻快的节奏。 

    「我啊……」 

    麻美子开口,打破沉默。久美子开火,默不作声看了姐姐一眼。 

    「一直在逃避自己做决定。不管是考试,还是其他的事。即使抱怨,也还是照爸妈说的话做,这辈子都在随波逐流。」 

    这或许是姐姐第一次对久美子提起自己的事。为了不打断她的谈兴,久美子安静地附和。 

    「我怕负责任,只要照爸妈说的话做,就算错了也能怪到爸妈头上不是吗?错不在我,都是妈说要怎样怎样。为了能有这样的借口,我一直走在爸妈规画好的路上。」 

    久美子用柴鱼片和昆布熬汤,从橱柜里拿出另一个锅子,把汤倒进去,利用高汤煮沸的空档切剩下的菜。 

    「可是啊……」麻美子接着说。「回顾过去为了找工作做的努力,不免怀疑自己这辈子到底做了些什么。没有任何想做的事,就只是一个口令、一个动作地活到现在。所以看你无忧无虑参加社团活动,我真的非常火大。为什么只有你可以想做什么就做什么。我这么努力,为什么得不到回报……很白痴吧?明明是我自己决定要这么做的。」 

    久美子感觉姐姐简直是在说给自己听,心脏跳了好大一下。 

    久美子将蔬菜倒进沸腾的锅子里,盖上锅盖。再从冰箱拿出猪肉,浸泡在用酱油和酒混合拌匀的腌料里,丢入磨碎的生姜泥,然后放回冰箱,冷藏备用。没事做了,久美子有些手足无措。这段过程中,麻美子始终用力刷着锅子。烧焦的部分早就已经刷掉了吧!久美子就连这句话也不敢说,继续装忙。 

    「我啊……」 

    麻美子边动手边喃喃自语。久美子闻言望向姐姐,毛躁的棕色头发,怎么看都不适合她。 

    「我一直很羡慕你。」麻美子若无其事地说。 

    久美子不知道该有何反应。她左思右想、思前想后,最后说出口的却是不痛不痒的「是噢!」 

    「因为你跟我不一样,看起来非常自由自在。我一直在忍耐的时候,你依旧做着自己想做的事。爸妈都只容忍你的任性。」 

    「才怪,我才觉得爸妈都偏心姐姐呢!爸妈只是懒得理我罢了,因为我成绩不好。」 

    「才没有这回事。」 

    「就是有。姐姐是妈引以为傲的女儿,妈只会称赞姐姐,害我嫉妒得不得了。」 

    叩、叩、叩、叩。明明不需要,久美子却把小黄瓜切成圆片,薄薄的绿色切面紧贴在菜刀上。 

    「我承认自己是妈引以为傲的女儿。」麻美子耸肩说道。 

    久美子掀开锅盖,白色的水蒸气立刻从缝隙窜出来。材料已经煮熟了。她取出味噌,放进大汤匙里,一点一点溶入高汤中。透明的液体没两下就染成味噌的颜色。袅袅上升的香味刺激着久美子的食欲。 

    「可是,我决定不再扮演妈引以为傲的女儿了。」麻美子说得笃定,表情豁然开朗。 

    久美子剥下黏在菜刀上的小黄瓜,移到碗里,加一小撮盐巴,以指尖搓揉,再用力握紧,拧干水分,加入调味料。久美子的视线捕捉到麻美子停下手边的动作。 

    「过去我一直假装自己是大人,假装什么都懂,有想做的事也不敢说。但仔细想想,这样不是很奇怪吗?因为再不情愿,也会变成大人,根本不需要从小就装大人。我不应该在还是高中生的时候就装大人样。就算后悔,就算失败,也要自己承担一切,自己选择自己要走的路。我应该这样告诉爸妈。就算他们反对,也该自己决定。」 

    久美子关掉炉火,熊熊燃烧的蓝色火焰瞬间熄灭,消失无踪。麻美子摘下橡胶手套,扔在桌上,搞得水滴四溅。 

    「所以这次我不想再犯错了,我要自己决定自己的将来。」麻美子斩钉截铁地说道。 

    烧焦的锅子暂且恢复原本的面貌,但如果仔细看,上头满是鬃刷的刮痕,久美子决定当没看见。沾着水滴的锅子反射着光线,闪闪发亮。 

    「你要搬出去吗?」久美子问道。 

    见麻美子点头。久美子静静低垂视线说:「这样啊!」 

    麻美子窥探她的脸色,半开玩笑反问:「舍不得?」 

    「才没有。」 

    「虽然只有一点点,我倒是有点舍不得。」 

    经姐姐这么一说,还真有点舍不得。她一直以为自己和姐姐的关系不亲密。麻美子对一声不吭的久美子爽朗微笑。 

    「对了,我听过你的CD了,吹得很好。」 

    「啊,嗯。」 

    「要去全国大赛吧?我也会去听,加油喔!」 

    「……咦?」 

    姐姐意料之外的宣言令久美子目瞪口呆。麻美子不知怎地,心情很好,哼着歌,试了一口味噌汤的味道。 

    「欸,你要来听是什么意思?」 

    「就是字面上的意思。我在学校里有个管乐社的朋友,想尽办法弄到入场券,约我一起去。」 

    「你、你要来看吗?」 

    「我不是这么说了吗?」麻美子轻拍久美子的肩。 

    「那我去睡一下,妈回来再叫我。」 

    「啊,好。」 

    思考追不上状况的变化,久美子只好点头。麻美子将自己的头发扎成马尾,伸了一个大大的懒腰。戴着假睫毛的眼皮看起来好重,眨巴眨巴地上下颤动。 

    「你也不要后悔喔!」 

    姐姐丢下这句话后就走出厨房,留下久美子茫然望着姐姐逐渐消失的背影。 

    姐姐说她这辈子都在随波逐流。她总是表现得自信满满,原来内心深处跟自己有一样的烦恼,这点令久美子大吃一惊。久美子把伤痕累累的锅子放进烘碗机,轻轻叹了一口气。 

    「不要后悔吗?」 

    久美子舀起锅子里的味噌汤,放进嘴巴里。作法明明跟平常一样,为何感觉比平常咸了些。 

    隔天午休,久美子将分毫未动的便当收回书包,慢条斯理站起来。正在吃面包的叶月一脸诧异看过来。 

    「你不吃午饭吗?」 

    一旁的绿辉也表示不解。 

    「该不会是肚子痛吧?要不要紧?」 

    「没事,不要紧。我有点事,你们先吃。」 

    「这倒是无妨……」 

    久美子丢下面面相觑的叶月和绿辉,离开了教室。由于是中午用餐时间,走廊上几乎没人。久美子的心跳愈来愈快,她反复深呼吸,在空无一人的走廊上前进,直接踏进三年级的校舍。或许是很少看到一年级,周围不断射来好奇的视线。尽管有些却步,久美子总算走到目的地的教室,深深吸了一口气,伸手开门。 

    「咦,这不是吹上低音号的一年级吗?有什么事?」 

    在门口吃午餐的三年级打击乐器学长吃惊地看着久美子,她的手局促不安地动来动去,鼓起勇气问他:「请问明日香学姐在吗?」 

    「你找明日香有事?等一下喔!」 

    学长大喊明日香的名字。明日香左手还拿着三明治,大步走过来。 

    「这个时间找我有什么事?」 

    「那、那个,我有话跟你说。」 

    明日香疑惑地微侧螓首,黑色长发随动作滑落在肩上。 

    「该不会是要向我示爱吧?」 

    「才、才不是!」 

    见久美子慌张否认,明日香哈哈大笑。看来自己又被她捉弄了。 

    「没问题,我们去没人的地方聊吧!」明日香把剩下的面包送入口中。 

    明日香的目的地是体育馆后面。那里的确没人,但也用不着在这种地方说话吧!她不理会傻眼的久美子,以轻盈的动作爬上逃生梯。 

    「这里、这里。」 

    明日香朝她招手,久美子跟上去。混凝土制的楼梯上方有一扇铁制的小门。从位置来看,大概是通往广播室。明日香坐在最顶端的台阶,从口袋里掏出一盒牛奶糖。 

    「要不要吃?」 

    「好,谢谢。」 

    明日香递出市售的巧克力牛奶糖。久美子放进嘴里,有点被硬度吓到,大概是要含着才会慢慢融化吧! 

    「学姐,你喜欢牛奶糖啊?」 

    「不,只是刚好收到。」 

    咬了两下,变软的牛奶糖黏在臼齿上。久美子喜欢牛奶糖,但不太喜欢这种黏牙的地方。久美子边咬牛奶糖,边望向扶手的另一边。为了避免有人掉下去,安装在室外的逃生梯于楼梯口设置了围墙。从围墙的缝隙清楚看见长在校舍后面的老樱花树。到了春天,肯定可以从这里看到美丽的景色吧! 

    「所以呢,有什么事?一个人跑来三年级的教室,想必是非常重要的事。」 

    明日香只转过视线,久美子硬生生吞下牛奶糖,深呼吸。久美子的肺部配合呼吸动作隆起,凭着吐出那口气的势头,她对明日香说:「学姐,请你上场比赛!」 

    明日香惊讶地瞪大眼。 

    「咦,你来找我就是想说这件事?」 

    「对。」久美子老实承认。 

    明日香重新扶好眼镜,装模作样地叹气,双眸犀利眯成一条线。「老实说,我不参加比赛,对社团比较好。」 

    「不,才没有那回事。」 

    「怎么会没有这回事?一直缺席,也不知道能不能上场比赛的家伙,对大家都是一种困扰。换成我是其他人,肯定恨死了。」 

    「可是,学姐有苦衷。」 

    「有苦衷的人可多了,只是大家都没说,你不知道而已。」明日香跷着二郎腿说道。 

    阳光从围墙与天花板的缝隙间洒落进来,这丝丝缕缕的光线太刺眼了,久美子忍不住眯起眼。明日香延伸的影子温柔勾勒出她的轮廓,只见她轻声叹息,不以为然地看着久美子。 

    「像是夏纪,为了代替我非常努力。明明是我要她努力练习,结果这次又要她别练习吗?那她岂不是太可怜了。」 

    「可是,夏纪学姐也认为比起自己,应该由学姐出赛,说这才是为社团好……」 

    「我比她更了解这个社团喔!」 

    明日香不容置疑地打断久美子的话,语气里充满过去整合社团的自信。即使一时被她的迫力吓阻,久美子仍然握紧拳头撑下去。 

    「再说了,你真的觉得关键时刻让大家感到不安的家伙可以大摇大摆上场比赛吗?你不觉得这样很奇怪吗?」 

    「那是因为……」 

    「我明明基于对社团没好处的理由阻止希美回社团,轮到自己的时候却当成个案处理,说不通吧?」 

    「可是,大家都在等学姐回来……」 

    久美子深怕自己被她说服,大声地说。明日香闭上嘴,傻眼叹气,朝久美子投以冷淡的视线,眼神中没有任何情绪,语气平板问道:「谁是大家?」 

    强大的压迫感令久美子一时语塞。 

    「久美子口中的大家是指谁?身边的朋友?交情好的学长姐?你怎么知道他们说的是真心话?」 

    「是真心话喔,一定是。因为一直以来都是学姐带领大家往前冲不是吗?」 

    「带领大家往前冲嘛!」明日香意在言外地说道。 

    「我并没有要带领大家往前冲的意思,只是单纯采取对自己有利的行动。」 

    久美子不知该如何反驳,拼命想搜集足以说服明日香的理由,但还没来得及组织成形,明日香继续口若悬河接着说:「刚才你说夏纪认为比起自己,应该由我出赛。」 

    「对,对呀!」 

    「可是夏纪真的那么想吗?假设大家都跟你说的一样,希望我回去,那夏纪可能只是为了配合周围气氛才这么说,这跟只是善于察言观色不同,你为什么就是不懂呢?」 

    「才不是……」 

    「才不是这样?你真能说得如此笃定吗?你和夏纪的感情好到足以让你做出这样的判断吗?」 

    明日香说出口的每一个字都化为利刃,刺向久美子的心脏。明日香是对的,她说的话永远都是正确的。也因此,久美子无法反驳,她心中的「正确」在说出口以前就先被明日香击溃了。 

    「香织和高坂同学的时候也是,霙和希美的时候也是,你总是身先士卒想办法解决。为了维系大家的感情,你总是拼了命努力。可是你也绝对不会跨过最后那条线。你怕受到伤害,也怕伤害别人,所以只敢唯唯诺诺在一旁守护。既然如此,你怎么会一厢情愿以为对方告诉你的都是真心话?」 

    久美子停止呼吸,脑子里一片空白,什么都无法思考,她无意识吁出一口气。过于尖锐的大道理,有时会重伤对方的感情。久美子的心脏揪成一团,黏腻的汗水从额头一口气喷出来,她无言以对,只能紧盯着眼前的学姐。明日香以冰冷的视线攫住她,美丽的黑色瞳孔仿佛浓缩了世上所有的幽暗,映照出久美子随时都要哭出来的脸庞。 

    「我就这样淡出社团是最好的结果。起初或许会有点不适应,但大家很快就会忘记。由包括夏纪在内的五十五个人上场比赛就好,我会为你们加油。话说回来,比赛结束以后,我就要退休了,如今只是稍微提早一点。我就这么消失,对这个社团才是最好的。 

    「你明白了吗?」明日香慢条斯理地问她,简直是在说给三岁小孩听。她的理论毫无破绽,久美子找不到反驳的理由,只想认输回答:「好的,我明白了,学姐说的没错。」学姐说的没错,是久美子错了。明日香永远都是正确的,听她的话准没错。一直以来都是这样,未来一定也是这样。 

    久美子脑筋转得飞快,毫无滞碍列出一大堆借口,试图替快要一蹶不振的心情找理由。明日香本人都说这样就好了,轮不到别人来下指导棋。没错。明日香不可能犯错,因为明日香—— 

    「明日香并不特别。」 

    小笠原说过的话唐突地在脑海中响起,打断她的思绪。久美子倒抽了一口气,热腾腾的脑浆急速冷却。那时,小笠原说了,我们都认为明日香是特别的。自己又要重复相同的错误吗?久美子握紧拳头,用力吸气,避开明日香的视线,不顾一切说:「学姐是对的,也比我更为社团着想,可是……」 

    久美子的嘴巴自顾自动起来,尽管思绪还是一盘散沙,她依旧冲动站起来,低头看着明日香的脸。久美子突如其来的动作令明日香惊讶地瞪大了双眼。 

    「问题是正不正确根本一点都不重要。什么才是对社团最好的?那种困难的事我完全不知道!我只想和学姐一起比赛,为什么不行呢?」 

    明日香闻言,不胜其扰地苦着一张脸。 

    「别说那么孩子气的话……」 

    「孩子气有什么不好!高中生本来就还是孩子。反倒是学姐,为什么要勉强自己表现出大人的成熟?一副什么都懂的样子,以为只有自己与众不同。明明就只是普通的高中生!」 

    久美子的话让明日香一口气噎住,她正要踏进危险的领域。明日香坚决与其他人拉出明确的界线,而且从小到大死守着这条线,如今久美子就要跨过那条线。明日香的眼神闪烁不定,似乎有些不知所措。 

    「这种态度哪里好了?学姐想演奏给令尊听吧?比谁都想进军全国吧?为什么要抹煞这一切呢?以为自己忍耐就能让一切圆满收场,这只不过是学姐的自以为是吧,至少我就希望学姐回来!」 

    久美子的视线变得模糊,她用手背拭去夺眶而出的泪水,咬紧唇瓣。不甘心。不甘心眼前的学姐接收不到自己的讯息。 

    或许如明日香所说,夏纪说的不见得是真心话。但就算是那样,也不会改变夏纪拜托久美子带回明日香的事实。既然夏纪宁愿压下内心的矛盾纠葛,把希望寄托在学妹身上,那么当然要努力完成她的期待。 

    「请不要放弃,不要做出会让自己后悔的选择。绝对不要假装成熟,表现出没有受伤的样子。就算要放弃,也请努力到最后再放弃。或许是我的任性要求,尽管如此,我还是希望学姐能站在比赛的舞台上,我想在那个音乐厅里,听学姐演奏上低音号!」 

    久美子说完所有想讲的话,终于闭上嘴巴。她大概是太兴奋了,上气不接下气。久美子用制服的袖子抹了抹嘴角,这才总算喘了口气。她深呼吸,让冷冰冰的空气流进肺部,然后再反复深呼吸,久美子开始恢复冷静,过热的思考终于回复正常。 

    久美子直到刚才都还激动到浑然忘我的地步,所以没有意识到,自己该不会对学姐说了相当失礼的话吧?她意识到这一点的瞬间,顿时惊慌失措、如坐针毡,明日香面无表情地问她:「难得看你如此慷慨激昂,这么快就没电了?」 

    「呃,不是啦,那个……」要是惹学姐生气可怎么办才好。久美子太过紧张,嘴巴像只金鱼似地张开又闭上。她只是说出心里的想法,为此道歉肯定很诡异。久美子在内心不甘示弱地自我申辩,看到明日香的脸,却一句话也说不出来。久美子避开她的视线,扭扭捏捏开始玩起自己的手指。久美子太害怕了,无法长时间直视明日香的脸。 

    明日香始终一言不发,尴尬的沉默充满在两人之间。怎么办?久美子冷汗直流。学姐果然生气了吧!久美子绞尽脑汁思考该怎么打破眼前的僵局,冷不防听见一阵止不住的笑声。她提心吊胆地抬头,明日香不知吃错什么药,正哈哈哈抚掌大笑。 

    「既然那么害怕,不如一开始就别说。」明日香以指尖拭去眼角笑出来的泪水。 

    「因为,」久美子为自己找借口。「因为我无论如何都想让学姐知道,所以……」 

    「所以你才鼓起勇气,一个人闯入三年级的教室。」 

    明日香语带调侃地接下久美子的话头。久美子胀红了脸,完全无法理解这有什么好笑的。 

    明日香站起来,大大的掌心落在久美子头上。久美子想抬头,但学姐用力按住她的头。久美子视线范围内只有明日香从藏青色裙子底下露出来的脚。 

    「说老实话,我很高兴。」 

    「咦?」 

    被按着头的久美子为之屏息。明日香修长的双腿微微颤抖,盖住大腿的裙子被风吹得掀了起来。 

    「……学姐,我可以看你的脸吗?」 

    「不行。」 

    「为什么?」 

    「因为我现在很害臊。」 

    明日香说得落落大方,久美子反射性地想抬头,但明日香的力气实在太大了,按住她的手文风不动。 

    「明日香!」 

    楼梯下方传来似曾相识的声音。明日香这才放开久美子,从扶手探出身子。久美子也赶紧有样学样,视线落在体育馆后方。定睛一看,应该正在吃午餐的葵朝她们冲过来。春天为了准备考试而退出社团的葵为何会出现在这里?满头问号的不只久美子,明日香也一脸匪夷所思地侧着头。葵逐渐缩短与她们之间的距离,耳边传来她慌不择路地冲上逃生梯的脚步声。 

    「真是的,让我找了好一阵子,没想到你躲在这种鬼地方。」 

    葵跑到她们跟前,目瞪口呆地叹气。她到底跑了多远?原本扎成一束的头发变得乱七八糟。葵以指尖拨开贴在脸颊上的发丝,双手扠腰。 

    明日香问她:「你怎么知道我在这里?」 

    「香织告诉我的。她说你有重要的事要谈时,通常都会来体育馆后面。」葵调整紊乱的气息。 

    「所以呢,你这么急有什么事?」 

    「其实是级任老师在找你。」 

    「池田老师吗?找我什么事?」 

    「好像是关于今天公布的模拟考成绩……」 

    她指的是暑假结束时举行的模拟考吧!这么说来,差不多该收到成绩单了。想到自己无可救药的数学,久美子打了个冷颤。 

    「是噢!」 

    仿佛想到什么,明日香杏眼圆睁,唇畔勾勒出上扬的弧线,露出睥睨一切的神秘笑容,揉乱了久美子的头发。 

    「久美子,我说不定能说服那个人。」 

    「那个人是指学姐的母亲吗?」 

    「没错!」 

    明日香难得喜形于色地回答,转身往楼下跑。突如其来的反应令久美子看得瞠目结舌,明日香朝她挥挥手。 

    「我去领模拟考的成绩单了,你们先回去吧!」 

    明日香丢下这句话,头也不回地冲向校舍,乌黑的长发随风翻飞。久美子还搞不清楚发生什么事,看着葵,希望她能解释一下,但葵也被同学不按牌理出牌的举动给搞迷糊了。 

    「她怎么那么高兴?」 

    「天、天晓得。」 

    葵微侧螓首,久美子也一头雾水。明日香到底想到什么,当时的久美子还不知道。 

    「发回模拟考卷!」 

    美知惠的声音在狭小的教室里回荡。正准备去社团活动的学生在朝会的教室里,迫不及待等待解散的指示。美知惠完全忽略大家的期待,举起一叠纸。由补习班主办的模拟考明明在假日实施,却规定所有学生都要参加。 

    「唔,这真是太惨了。」 

    绿辉接过成绩单,脸颜色铁青发抖起来。叶月在一旁尚可接受地猛点头:「还可以。」 

    「黄前,快点来拿。」 

    「啊,好的。」 

    在美知惠的催促下,久美子连忙站起来。美知惠交还考卷给她的时候,有些傻眼地告诉她:「数学惨不忍睹喔!」 

    「是、是的,对不起。」 

    「我听说你请田中同学教你功课,期待你下次的模拟考成绩。」 

    曾几何时,就连老师也听闻久美子去明日香家温书的事。久美子实在没把握能不能回应老师的期待,她想是这么想,但还是点头应允:「我会尽力而为。」 

    模拟考的结果不好也不坏,几乎所有科目的成绩都落在平均分数上,只有数学的部分在雷达图上不自然地凹进去。 

    「丽奈的模拟考成绩想必也很好吧?」绿辉与自己的成绩大眼瞪小眼,语重心长地喃喃自语。 

    开学典礼上代表新生致辞的丽奈是一年级成绩数一数二优秀的优等生,模拟考的成绩跟自己肯定是天壤之别。久美子心想,走回自己的座位。美知惠发完模拟考的成绩单,往教室里看了一圈,清了清喉咙。 

    「我说过好几次了,面谈时将以这次模拟考的成绩为基准,讨论各位的出路。还有,选择比较特殊的人,双方面谈前会先被出路科叫去,请做好心理准备。」 

    听完美知惠的叮咛,叶月不解地反问:「选择比较特殊?」 

    「像是出国留学或就业,不参加一般升学考试的人。这所学校好像很少这样的人。」绿辉小声回答叶月的疑问。 

    「出国?欸,是我无法想象的世界呢!」叶月缩着脖子说。 

    久美子听到出国这个单字,又看了一下自己的模拟考成绩。英听的分数不太好看,这种成绩要出国,简直是痴人说梦。久美子把成绩单塞进书包里。 

    明日香并没有出席那天放学后的练习。随着比赛的日子愈来愈逼近,低音组的气氛也愈来愈沉重。久美子正在练习比赛用的曲子,一旁的夏纪也在拼命练习同一首曲子。 

    「这里你平常都是怎么吹的?C的地方。」夏纪转向久美子问道。 

    久美子放开吹嘴,端详她指的谱面。 

    「哦,要跨八度音的地方很吃力呢!」 

    「我每次吹到这里,音准都会快要跑掉,有什么诀窍吗?」 

    「这个嘛,可以一开始先意识到高音,再一口气吹到这里。」 

    「原来如此。」夏纪佩服地低喃。这位二年级的学姐并不忌讳向学妹请教。久美子很欣赏她这种实事求是的性格。 

    「学姐,如果你不介意的话,要不要一起……」 

    合奏呢?久美子还来不及说完,就被突然出现的访客打断了。 

    「大消息!」 

    小号组的优子边喊边推开教室的门。低音组的视线全都集中在优子身上。 

    「你吵什么,我们正在练习耶!」 

    夏纪大力皱眉,但优子完全不顾她的阻止,直直向卓也走去。组长明日香不在,身为副组长的卓也现在就是低音组实质上的领队。他把低音号放在地上,转向优子。 

    「后藤,你听说明日香学姐的事了吗?」 

    「没,什么也没听说。」卓也难掩困惑地推了推眼镜。 

    原本在一旁练习的梨子问优子:「明日香学姐怎么了?」 

    「听说学姐这次的模拟考挤进了全国前三十名!」优子兴奋得一口气说完。 

    其他人则愈来愈困惑。那的确是很好的成绩没错,但是有必要在练习时特地跑来报告吗?或许是感受到周围诧异的视线,优子一掌拍在桌子上。 

    「还不明白吗?明日香学姐打算拿模拟考的成绩去和她母亲谈判,打算说服母亲她会继续用功读书,请求母亲能让她再从事社团活动一阵子。」 

    「所以明日香学姐能上场比赛了吗?」 

    绿辉顿时充满了希望。优子一脸严肃地摇头。 

    「还不清楚,不确定她能不能说服她母亲,只能说是有一线希望了。我光是看到香织学姐知道这件事,高兴得不得了就满足了。」 

    优子如释重负地笑着说。自从明日香与母亲的争执浮上台面,香织虽然佯装平静,但常会不经意露出郁郁寡欢的表情。 

    「希望学姐能回来。」 

    夏纪静静放下上低音号。她望向这边的眼神,看上去没有丝毫虚假。久美子瞥了夏纪写满注意事项的乐谱一眼,点点头。 

    「嗯,就是说啊!」 

    距离比赛的日子,已经没有多少时间了。 

    久美子下了京阪电车,独自穿过车站上到地面出口,四周已经一片漆黑。其他学校的学生都筋疲力尽坐在巴士回转道的长椅上打瞌睡。久美子看了时钟一眼,吞下哈欠。 

    「麻美子姐的事解决了吗?」 

    久美子冷不防被问到这个问题,转头望向声音的来处。果不其然,是秀一。久美子耸肩,简短回答:「暂时解决了。」 

    「她有什么打算?」 

    「打算退学,去大阪念美容师的专科学校。」 

    「欸,真的吗?」 

    秀一大吃一惊往后仰。久美子重新围好脖子上的围巾,老实地「嗯」了一声。 

    「我妈原本很反对,但是好好谈过以后,总算是接受了。」 

    「是噢,不过和好就好。」 

    秀一轻拍久美子的背。 

    「还好啦!」 

    「家里有个美容师,以后就不用花钱剪头发了,真幸运。」 

    「我才不要让姐姐剪头发呢,感觉会被剪得很花稍。」 

    「欸?麻美子姐以前都是走保守路线耶!」 

    「现在变得非常花稍喔!下次你来我家玩的时候不妨见识一下,包准会吓到。」 

    「那……等我有空的时候再过去吧!」 

    秀一伸了个懒腰,望向宇治川对岸。已经是枫叶季了,山上充满红色及黄色的暖色系。现在是晚上,所以看不太清楚,但是白天可以一览无遗美丽的风景。尤其是清晨,景色特别美。早起在塔之岛上悠闲散步是久美子最近的乐趣。 

    「那个……高坂今天被出路科叫去了。当时我刚好在教职员办公室,看到她走进后面的房间。」 

    「丽奈吗?为什么?」 

    「我哪知道。模拟考成绩公布了,大概跟那个有关。」 

    秀一一脸茫然歪着脖子回答。说的也是,丽奈的成绩那么优秀,不可能是因为考不好被骂。久美子判断应该没什么大事,便转移话题。 

    「说到出路科,你知道吗?卓也学长将来要去东京的学校,听说是为了成为乐器修理师。」 

    「欸,那长濑学姐怎么办?他们不是在交往吗?」 

    「梨子学姐好像要念这边的大学,等卓也学长回来。」 

    「所以是远距离恋爱吗?」 

    「正是。」 

    久美子点头。秀一一脸佩服,喃喃自语:「东京啊……」 

    「秀一决定好出路了吗?」 

    「还没。你呢?」 

    「我也还没。」 

    毕竟我们才一年级,久美子在内心咕哝着。她望向宇治川,一片漆黑的水面缓缓流动,小石头从水面探出脸来,好似要拦住水流,但那只是无谓的抵抗,黑黝黝的水纹避开小石头,一起往同一个方向流去。 

    「大家都已经决定好出路了吗?」 

    「天晓得。」秀一回答。 

    久美子垂下眼帘。竖起耳朵,可以听见潺湲水声里夹杂着从河边呼啸而过的汽车引擎声。总是太过于理所当然,久美子几乎都忘了那些声音的存在。 

    「毕业以后,和你见面的机会或许也会减少呢!」秀一说道。 

    久美子盯着他的侧脸,无计可施地轻声叹息。 

    「谁叫我们只是普通朋友呢!」 

    这句话让秀一停下脚步。车灯的光线从背后追过他们,瞬间照亮了他的脸。秀一眯起眼,欲言又止地看着久美子。毫无防备露出制服领口的喉结上下震动,好似为了掩饰动摇,散落在脚边的影子死命攀住他颤抖的身体。躲藏在寂静里的涓涓水声比刚才更大声,撼动着久美子的耳膜。秀一又开口,但依旧一句话也说不出来。久美子耐着性子,等秀一把话说完。打从心底期待他想对自己说的话。然而眼前的男生只是像平常一样傻呼呼地笑,简直像是为了隐瞒什么。 

    「说的也是,我们只是朋友嘛!」秀一说道。久美子无言以对。 

    直到周末的练习,明日香始终没来社团,只有好消息跑得比谁都快,害社员全都失望至极。小笠原看着挂在墙上的时钟一分一秒往前走,难以启齿地说:「就快要开始合奏了,明日香真的不来吗。」 

    「可是,明日香每次请假的时候都会提早说。」香织的视线落在乐谱上,活像是说给自己听。 

    久美子瞥了旁边的空位一眼,悄然叹息。要说服明日香的母亲果然是不可能的任务吗?音乐室里充满了沉重的气氛。大家都发自内心在等同一个人出现,那个人却迟迟不出现。 

    「会来的。」 

    「欸?」 

    突然响起的声音令久美子抬起头来。夏纪正真挚地看着自己,眼神里没有不安。 

    「明日香学姐一定会来的。我相信她。」 

    「学姐……」 

    久美子一时说不出话。她很清楚,原本对社团活动不是很热中的夏纪这段时间多么努力练习比赛的曲目。为了填补明日香不在的缺口,夏纪拼命反复吹着同一套乐谱。明日香一旦回来,她的练习等于全部白费。尽管如此,她依旧一脸坚定说她相信明日香会回来。久美子觉得她好坚强。夏纪好坚强,比明日香以为的,还要坚强。 

    「……开始基础练习吧!」 

    小笠原说道,站了起来。基础练习的指导原本是明日香的任务。小笠原走到社员面前,轻轻拍了一下手,就像明日香平常做的那样。众人的视线一起集中在她身上。小笠原用电子琴弹出调音用的音阶,望向霙。 

    「那么,先从调音开始。双簧管……」 

    「抱歉,我迟到了!」 

    音乐室后方响起大家翘首以盼的人的声音,盖过小笠原的指示。站在前面的社长睁大双眼,无声凝视着现身的人物,喉咙紧张地微微震动。一旁的夏纪倒抽了一口气。久美子胆战心惊地回头望向来人。香织以沙哑的声音轻声低喃:「明日香。」 

    明日香提着乐器盒站在那里,脸颊微微肿起,大家一看就知道发生过什么事。咯噔。椅子被香织的大腿弹开。她站起来,摇摇晃晃走向明日香。叽——叽——香织每走一步,地板就发出声音,她颤抖的指尖牢牢抓住明日香的藏青色制服。 

    「抱歉,让你久等了。」明日香说道。 

    香织几乎哽咽,只能无言摇头。 

    「没关系,你来了就好。」香织说道,抱住明日香,把脸埋进她的肩头。明日香吓了一跳,撑住她的身体。站在前面的小笠原泪中带笑。 

    「真是的,你也让我们等太久了。」 

    「抱歉。」 

    明日香温柔拍抚香织的背。父亲送给她的黑色乐器盒依偎在她脚边。 

    社长的话打开了开关,所有人一起围到明日香身边,其中也有不少学生泪流满面。久美子目睹这一切,也悄悄拭去泪痕。 

    原本在一旁冷眼旁观的夏纪静静走向明日香。 

    「学姐。」 

    明日香笔直地望向夏纪,眼神里有一小簇罪恶感。隔着透明的镜片,明日香纤长的睫毛缓缓上下搧动,一脸苦涩咬紧下唇,眉头打了个死结。 

    「夏纪,我对你……」 

    「别这样!」 

    或许是猜到她要说什么,夏纪尖锐地打断她的话。声音充满迫力,令明日香咕嘟一声,咽了口口水。众人的视线都集中在夏纪身上,只见她摇摇头,又说了一次同样的话:「别这样。」 

    「我一直在等学姐回来,所以不要向我道歉。」 

    「夏纪……」 

    明日香一字一句喊她的名字。在离她们有段距离的地方,卓也正摘下眼镜,粗鲁地抹着自己的眼角。梨子、叶月、绿辉……低音组的成员全都祝福明日香的归队。夏纪慢慢跨出一步,与明日香面对面,唇畔勾勒出上扬的曲线。 

    「学姐,欢迎回来。」 

    这句话让明日香狼狈得目光闪烁,眼里滑出一丝情绪,顺着脸颊,在藏青色的制服上烙下黑色的小小印痕。 

    明日香说:「我回来了!」 

    感动的重聚告一段落,大家开始迅速准备合奏练习。完成与平常无异的基础练习后,音乐教室里此起彼落地响起乐器的音色。久美子翻开乐谱,悄悄吐出一口气,借此安抚紧张的心情。明日香就跟平常一样,在旁边吹响银色的上低音号。这个事实让现在的久美子高兴得不得了。 

    「大家早。」 

    「早啊!」 

    当大家都在各自练习时,泷和桥本走进音乐教室。众人停止演奏,望向顾问。泷看了坐在上低音号座位区的明日香一眼,嘴角浮现微笑。 

    「我想各位已经知道了,田中同学会照原订计划上场比赛。」 

    明日香尴尬地垂着眼。 

    「距离正式比赛没多少时间了,一股作气进行最后冲刺吧!」 

    「是!」 

    大家中气十足地回答,桥本满意地笑了,卷起长袖,双手拍打自己的脸颊,为自己加油打气。 

    「很好,那就抬头挺胸往前冲吧!先从合奏开始。」 

    众人闻言,同时拿好乐器。 

    这天的练习比平常还要充实,或许是拜大家都很专注所赐,犯错少了,演奏的音色也充满朝气。 

    「嗯,比上次好多了。」 

    桥本这句话让大家眉开眼笑看着彼此。为了让欢天喜地的社员绷紧精神,泷用指挥棒叩叩叩地敲打乐谱。 

    「可是,千万不能因此就轻忽大意。我不是说过吗?练习时要认为自己是最蹩脚的演奏者,上台时再认为自己是最高明的演奏者。骄傲会导致技术退步,绝不要以为自己很厉害。」 

    「是!」 

    「很好,那么休息十分钟以后,再合奏一次,别忘了补充水分,听清楚了吗?」 

    「听清楚了。」 

    「那就休息到四点半。」 

    进入休息时间,久美子俨然被掏空似地一屁股坐下,伸直双腿,再打直背脊,伸了个懒腰。一旁的明日香正仰头灌下瓶装水。 

    「久美子。」 

    声音从背后传来,久美子回头一看,只见丽奈一脸苦恼站在正后方。 

    「怎么啦?」 

    「我有话跟你说,今天晚上可以一起回去吗?」 

    「可以啊!」 

    「那就好。」 

    丽奈丢下这句话,摇摇晃晃走回自己的座位,她好像没什么精神,发生什么事了?久美子转身窥探她的反应,但丽奈一脸凝重地低着头,动也不动,好像真的发生了什么事。久美子正要站起来,泷又回到音乐教室。 

    「重新开始练习。」 

    久美子只好再坐回椅子上。 

    当天回家的路上,丽奈一句话也不说,一声不吭的侧脸宛如精巧的洋娃娃,没有半点人味。久美子也保持沉默,走在她身边,棕色的平底鞋慢条斯理踹着石板路。两人始终一言不发往前走,终于走到要分开的岔路。久美子望了绿灯一眼,对丽奈说:「那……我走这边。」 

    久美子说完就要过马路,丽奈却用力抓住她的衣摆。红绿灯闪烁,无声地从绿灯变成红灯。 

    「怎么了?」 

    久美子停下脚步,转身面对丽奈。天色太暗,看不清她的表情。从刘海筛落的阴影,在丽奈脸上烙下忧郁的痕迹。丽奈的视线若有所思地飘来飘去,以气若游丝的音量说:「要不要去大吉山?」 

    「欸,好啊!」 

    这意料之外的发展,让久美子不禁有些傻住,但还是点头答应了。丽奈从书包里拿出手电筒,递给久美子。她该不会一开始就打算去爬山,所以特地带着这玩意儿参加社团活动吧!久美子按开手电筒开关,看了丽奈一眼,她依旧无精打采。 

    走在蕨类才刚长出嫩芽的路上,久美子和丽奈前往大吉山的登山口。当地人口中的大吉山正式名称为佛德山,海拔一百三十一公尺,从总角古迹附近的登山口延伸出来的山路没什么阶梯,路也够宽,只要二十分钟左右就能抵达瞭望台,但路灯稀少,所以晚上不拿手电筒去爬的话很危险。 

    久美子和丽奈相对无言地往前走,总算走到瞭望台。久美子把书包往设置在瞭望台的长椅上一放,隔着扶手看到的夜景美不胜收,她悄悄吁出一口气。这么说来,县祭时也和丽奈一起来爬过大吉山,她就是在这里向久美子坦承自己喜欢泷。 

    丽奈没在久美子的身边坐下,而是直接躺在长椅上。她的裙子掀起,雪白大腿裸露在昏暗的路灯下。 

    「你今天到底怎么了?」久美子问道。 

    丽奈欲言又止地努了努嘴巴,眼皮慢慢地垂下。久美子的手指在丽奈的黑发上滑动,低头看她的侧脸。 

    「久美子,你早就知道了吧?」 

    轻声细语的问话让久美子的心脏漏跳一拍。丽奈的声音听起来与平常无异,但隐含着怪罪的意味。久美子放开她的头发,不解地反问:「知道什么?」 

    「泷老师有太太的事。」 

    久美子的心脏发出嘈杂声音,突突跳着,她屏住呼吸,凝望着丽奈的脸。丽奈凌厉的视线射向久美子,细致的手指紧紧抓住久美子的手臂。 

    「为什么不告诉我?」 

    丽奈的语气异常冰冷,久美子吓得往后弹开。动作太大,平底鞋踢在沙地上,鞋底与地面摩擦出刺耳的噪音。久美子逃也似地避开丽奈的视线。秋夜凉如水,呼啸而来的风用力打在久美子脸上。 

    「你是怎么知道的?」久美子问道。 

    丽奈悄然叹息。 

    「昨天去出路科的时候,刚好在办公室遇到泷老师,也看到那张照片,然后就听他说了他太太的事。老师说他也告诉过你。」 

    丽奈说到这里,拉扯着久美子的手臂。 

    「为什么不告诉我?」 

    久美子的视线在空中彷徨。丽奈的手指隔着衣服陷入久美子的皮肤,粉红色的指甲反射着路灯的光线,闪闪亮亮。久美子总觉得呼吸好困难,忍不住叹气,自己的声音不听使唤地夹杂在吐出的气息里。 

    「因为我不希望你受到伤害。」 

    久美子的回答让丽奈一时半刻无言以对,描绘出平滑曲线的喉咙不知所措地微微震颤,指尖描摩久美子的手,心烦意乱地叹气。丽奈冰冷的指尖令久美子打了个冷颤。 

    「我知道。」丽奈低垂着眼说道。 

    「可是就算是这样,我也希望你能告诉我。」 

    「……抱歉。」久美子道歉。 

    丽奈轻轻摇头,说了声:「没关系。」她坐起来,几缕黑发黏在脸颊上。久美子静静伸出手,温柔地为她拨开发丝。脸颊被久美子碰到时,丽奈静静垂下眼,睫毛如蝶翼般颤动。 

    「我被自己的软弱吓到了。听说他有太太时,我简直六神无主。」 

    「会六神无主是很正常的喔!我也吓了一大跳。」 

    「可是,我还以为自己是更坚强的人。」 

    丽奈说完,咬紧唇瓣。久美子悄悄拥住她沉默不语的身体。暴露在夜风中的制服冰冷极了,唯有从她口中吐出的气息还带有一丝热度。 

    「全国大赛的舞台对泷老师而言也有别有意义呢!」 

    「就是说啊!」 

    「我想拿下金奖。」 

    「嗯。」 

    「我想实现老师的梦想。」 

    枫叶乘着风飘到久美子脚边。红色黄色的叶片铺满整条路,色彩缤纷的步道宛如铺了地毯。 

    「我们一定要拿下金奖。」 

    丽奈默不作声地点头。接下来有好一会儿,她整个人靠在久美子身上,一动也不动,无疑是在想些什么。久美子边用手指梳理她的头发,边怔忡地凝望夜景。眼前繁星点点般的灯光有点像乐器的颜色。河对岸的万家灯火散发出金色的光线,不断地明明灭灭。 

    「我不会放弃。」丽奈说道,她用力握紧久美子的手。 

    路灯的光线有如坠落的星辰,落在染上夜色的凝眸深处,耀眼夺目。久美子觉得丽奈没有一丝迟疑的眼神好美。她永远都这么美。 

    「我支持你。」久美子也握紧丽奈的手,回应她的想法。 

    「谢谢。」丽奈的唇畔浮现出温柔的微笑。 

    转眼间就来到比赛前一天。所有人一大早就到学校集合,把大型乐器搬上卡车。话说回来,定音鼓为何这么重呢?三人合力将庞大的乐器塞进卡车,久美子忍不住叹气。搬运乐器的作业是人数不多的男社员表现的机会,久美子佩服地看着卓也在楼梯上跑来跑去,对梨子说:「卓也学长就连这种作业也不会偷懒,好认真。」 

    「对呀。」梨子含羞带怯笑着回答。「我就喜欢他这种地方。」 

    久美子听到她爱的告白,不由得苦笑。一旁的绿辉兴奋得双眼发出光芒。 

    「小绿觉得你们两个应该要结婚!」 

    响彻整条走廊的叫声令梨子羞红了脸。楼梯下方传来吹口哨的声音,看来是卓也的朋友跟着起哄。久美子心想这两个人真的很般配,自然而然脱口而出:「我也希望你们结婚。」 

    「真是的,怎么连久美子也这么说。」梨子说道,脸色愈发红润。 

    绿辉非常满意地欣赏学姐可爱的反应。 

    从京都到名古屋搭巴士花费的时间意外的长,上午从学校出发的巴士,下午三点过后才抵达爱知县的旅馆。一抵达旅馆,社员就自动自发将行李搬进房间。外宿的时候,女生会依学年分配到通铺,男生则不分学年,全部挤在同一个房间。 

    久美子前脚才踏进室内,就立刻放下肩上的背包,发出咚的一声巨响。她脱掉袜子,双脚在榻榻米上伸直,凉凉的感觉很舒服。真想就这么睡去,不要再动了。榻榻米上摆放着好几个五颜六色的背包,吹风机及整发器之类的用品散落一地。女生的房间总是充满这些用来打扮的道具,所以每次集宿都会上演插头争夺战。 

    「喂,是谁把牙刷放在这里?很碍事耶!」 

    「哇!糟透了,竟然拿到我妹的衬衫。」 

    「窗外的景色也太差了,只能看到隔壁建筑物的墙壁。」 

    大家七嘴八舌地发话,所以通铺总是从早吵到晚。在没有学长姐,只有一年级的空间里,大家的言行举止都比平常更随兴。 

    丽奈的下巴搁在桌上,慢吞吞提起眼皮。 

    「惨了,觉得好困。」 

    「不要紧吧?接下来才要开始练习。」 

    「不要紧,不过晚上就很难说了。」丽奈说道,打了个大大的哈欠。 

    久美子也跟着打哈欠。昨晚明明睡得很饱,是被丽奈的哈欠传染吗?这时,走廊上传来二年级吵吵闹闹的说话声。 

    「差不多该走了。」 

    瞥了挂在墙上的钟一眼,丽奈迅速站起来。刚才还说很困的她,貌似已经切换成练习模式。 

    「今天是可以练习的最后一天了。」 

    「对呀!」 

    「啊……已经是前一天啦,时间怎么过得这么快。」 

    久美子也同意丽奈所说。升上高中以后,每天都在转眼之间就变成昨天。照这样下去,长大以后该不会眨个眼就过了一年吧! 

    「对了。」 

    「嗯?」 

    久美子突然开口,丽奈侧着头等她说下去。 

    「出路科找你做什么?」 

    她的确说过,自己是在被出路科叫去的时候听泷谈起他的过去。话说回来,她去出路科做什么。丽奈害羞笑着回答久美子:「哦,因为我想念国外的音乐大学,所以才被找去。」 

    「国外?」 

    这个意料之外的回答,令久美子不由得睁大双眼。丽奈或许是察觉到久美子内心的疑惑而苦笑。 

    「就算要出国,也是毕业以后的事,还早得很。」 

    「可是这么一来就得离开日本吧?」 

    「我是有这个打算,先在日本学半年英语,再出国留学。虽然老师都劝我去念四年制的大学。」丽奈说完,看了久美子一眼。 

    「留学啊……」久美子感慨良深地喃喃自语。 

    「丽奈已经决定好出路啦!」 

    「久美子呢?」 

    「我嘛……还没。」 

    「这样啊!」 

    出路啊……久美子静静低眉敛眼。或许自己也差不多该思考将来的事了。这时突然从后方传来啪嗒啪嗒的喧闹脚步声,打断了她的思绪。久美子回头一看,优子和夏纪正争先恐后朝这边快步走来。 

    「喂,不是叫你等一下吗?」 

    「啥?为何我非得等你不可?」 

    两人一如既往跳针着没营养的斗嘴抬杠,打久美子她们身边匆匆经过。 

    「那两个人就连在这种地方也能吵架吗?」丽奈说道,目瞪口呆地耸了耸肩膀。 

    外聘的指导者新山和桥本也出现在泷为比赛商借的音乐厅中。上次接受新山的指导已经是暑假的事了。社员分成A部门和B部门,在各自的舞台开始准备。排好椅子、放好谱架后,打击乐成员正与桥本热烈讨论着什么,好像是在仔细研究乐器的摆放位置。 

    准备完毕,众人排成平常的合奏队形,在泷的指导下进行基础练习。明日香理所当然坐在久美子身边,这个事实让久美子太高兴了,下意识翻开乐谱又阖上。 

    「基础练习到此结束,从头到尾合奏十次。」 

    「好。」 

    顾名思义,从头到尾合奏十次就是依指定曲和自选曲的顺序,从头到尾合奏十次的练习方法。不仅非常耗费体力,每次演奏都要接受顾问的纠正,所以精神上也很疲劳。借用音乐厅练习期间,一定会从头到尾合奏十次,已经成了北宇治高中的惯例。不只是社员,就连顾问也挥汗如雨地努力练习。 

    泷卷起衬衫的袖子,举起指挥棒。桥本和新山坐在观众席,正在观察他们的演奏。 

    「那么,第一次合奏。」 

    泷一声令下,在音乐厅的练习开始了。 

    比赛前一天的练习是以细节的调整为主要目的,例如每种乐器的音量比例和乐器的位置、音程有没有跑掉、节奏跟不跟得上,巨细靡遗地修正每个不够完美的地方,逐渐提高演奏的完成度。万一练习到筋疲力尽,可能会影响明天的比赛,所以练习的结束时间设定得比平常还早。 

    「十次从头到尾的练习到此结束,辛苦了。」 

    泷说出这句话的时候,大部分社员都已经累得不成人形。泷似乎也很累,成串的汗珠从额头滑落。已经十月了,但是持续练习下来,还是觉得很热。久美子拔出上低音号的号管,朝水桶倒入凝结的水滴。撑住乐器的手臂好沉重,因为一直维持同样的姿势,身体的各个部位都痛得要死。 

    「请独奏的人留在音乐厅接受指导,其他人原地解散。要继续练习也无妨,但是千万不要太勉强,以免影响到明天的比赛。」 

    「好。」 

    社员异口同声回应泷的交代。或许是练习终于结束,大家都很开朗。久美子朝天花板伸直手臂,伸了个大大的懒腰。她回头看,丽奈正拿着乐谱夹站起来。一对上眼,手里还拿着短号的丽奈,满脸歉意地耸耸肩。 

    「抱歉,我要留下来接受指导,你先去吃饭吧!」 

    「嗯,好的。练习要加油喔!」 

    「包在我身上。」 

    丽奈嫣然一笑,志得意满的表情倒映在金色的短号上。 

    「久美子,吃饭前先去洗澡吧!」 

    绿辉抱着低音大提琴,蹦蹦跳跳走过来。是乐器太大,还是绿辉太矮呢?她的身体几乎都被乐器挡住,只能看到从旁探出的小脸。 

    「啊,还有,小绿带了扑克牌,晚上大家一起玩吧!」 

    「小心明天爬不起来喔!」 

    丽奈的叮咛让绿辉伤脑筋地挑着眉毛。 

    丽奈的叮咛让绿辉伤脑筋地挑着眉毛。
    
    「说的也是,那不要玩扑克牌,改玩UNO\footnote{UNO:起源自美国的纸牌游戏,玩家打出倒数第二张牌时要喊出UNO,因此得名。}吧!

    「等等,这两种游戏差不多吧!」久美子说道,绿辉闹别扭地鼓起脸。丽奈看到她的反应,愉快笑出声音来。 

    结果绿辉带来的扑克牌在睡前派上了用场,盛况空前地玩了五局大富豪\footnote{大富豪:日本的扑克牌游戏,玩法类似大老二。},叶月五局都是最后一名。

    「叶月太容易表现在脸上了。」绿辉咯咯咯地笑着说。 

    叶月连忙反击:「呃,这只是运气的问题吧!谁知道会两次被革命\footnote{玩家同时打出四张点数相同的牌时,点数的大小会反过来,2变成点数最低的牌,3变成点数最高的牌。}翻盘。

    「话说回来,小绿也赢太多次了吧,拿到的牌都好强。」 

    「因为小绿就只有运气比别人好。」 

    绿辉自信满满地拍着胸脯说,丽奈对她投以既佩服又傻眼的目光……这种游戏也会反映出性格吗?久美子边想边收拾成堆的扑克牌。 

    顺带一提,久美子玩了五次都是平民\footnote{游戏结果由最赢到最输分成大富豪、富豪、平民、贫民、大贫民。}。这种游戏也会反映出性格吗?久美子边想边收拾成堆的扑克牌。

    「低音组,要关灯喽!」 

    「好……」 

    吹萨克斯风的一年级说,久美子等人异口同声地回答,整理好扑克牌,放回盒子。榻榻米上一个挨着一个铺好所有人的被子,有些疲惫的社员早就睡着了。久美子滑进被窝,被单表面异样冰冷,只好先乖乖躺着,直到被窝变暖,她缩成一团躺在被窝里,室内的电灯突然熄灭。 

    「晚安。」 

    耳边传来绿辉的声音,久美子也回以「晚安」。或许还残留着在音乐厅练习的疲劳,一躺下来,疲惫的感觉就渗入四肢百骸。她感觉丽奈在一旁辗转反侧。睡不着吗?久美子心想,静静闭上眼。眼皮内侧漂浮着晶晶亮亮的光点。久美子整个人埋进被窝里,努力想要赶快睡着。 

    众人的鼾声回荡在通铺的房间,就连翻来覆去的丽奈也逐渐入睡,安静下来。喀嚓、喀嚓、喀嚓……唯有秒针前进的声音听起来异常清晰。 

    「……睡不着。」 

    久美子忽地坐起来。其他成员都睡着了,没人听到久美子的低喃。她看了时钟一眼,才十一点。 

    再这样下去也睡不着,去自动贩卖机买瓶热饮吧!久美子蹑手蹑脚从被窝站起来。 

    「等一下!」 

    突然被人叫住,久美子吓得全身僵硬。她望向声音来处,只见叶月正蠕动着嘴巴,念念有词地翻了个身。看样子刚才是她的梦话。久美子隔着衬衫按住扑通扑通跳得飞快的心脏,悄悄松了口气。 

    久美子不太习惯在自己家以外的地方睡觉,暑假集训也因为这样几乎没睡,但也多亏当时的对话所赐,才能和霙以及优子建立好关系。现在回想起来,自己伤脑筋的习惯也不见得只有坏处。 

    她下楼走向设有自动贩卖机的地方。泷帮他们订的旅馆非常老旧,到处设置着陈旧的桌球台和按摩椅,贴在墙上的海报是上一个时代的偶像泳装照。既然都来到名古屋了,今天的晚餐居然是普通的烤鱼定食,机会难得,真想尝尝味噌猪排,但泷可能担心吃太油腻会搞坏肚子。久美子回想厚实的白肉鱼,觉得自己的肚子好像有点饿了。这个时间回想晚餐菜色简直是自寻死路。不行、不行,她拼命摇头,试图把食物赶出脑海。 

    「你怎么了?干么做出那么恶心的动作。」 

    久美子还以为没有别人,冷不防居然有人对她说话,不由得面红耳赤。 

    「呃,那个,不是的。」 

    不知道是什么不是,总之久美子先否认再说。她回头一看,秀一正目瞪口呆看着自己。久美子发现眼前的人是青梅竹马后,松了一口气。 

    「什么嘛,是秀一啊!」 

    「这话说得也太没礼貌了!」 

    秀一抱着胳膊抗议。明明已经是秋天了,他还穿着T恤和学校规定的运动裤,这种衣服在这时候光看就很冷,从袖口隐约可见他肌肉结实的手臂。秀一从还是小学生时,就对季节更迭没什么概念。这么说来,不管是夏天还冬天,他从小都穿短裤上学。 

    「你穿成那样不冷吗?」 

    「不会啊,是你太怕冷了。」 

    「欸,才不是这样。」 

    久美子说道,突然冒出一个问题。 

    「对了,你这种时间来这里做什么?」 

    「你干么抢我的台词。」 

    「我睡不着,来买果汁。」 

    「什么嘛,跟我一样。」 

    秀一举起钱包给她看。他从国中用到现在的钱包上有龙的刺绣。男生为什么会以为那种东西很酷呢?久美子完全不能理解。 

    「自动贩卖机在哪边?」 

    「从这里直走到底。快点去买吧!」 

    「啊,等一下啦!」 

    久美子连忙追上迈步往前走的秀一。他一步的距离非常大,要小跑步才能追上。如秀一所说,自动贩卖机设置在走廊尽头,旁边是硬质的沙发和生锈的金属烟灰缸,烟灰缸里还有几根只剩下一小截的烟蒂,从褪色的程度来看,显然不是最近才丢弃的,现在抽烟的人口比以前少多了。 

    久美子在自动贩卖机买了可可亚后,二话不说地坐下。刚买的可可亚还很烫,她拿罐子贴着脸颊,传导到皮肤的热度很舒服。久美子享受地眯起眼,秀一傻眼地耸肩。 

    「不喝吗?」 

    「等一下就喝。」 

    「再拖下去会冷掉喔!」 

    秀一边说边一屁股坐在久美子旁,两人间有段不自然的空隙。不坐近一点吗?久美子赶紧吞回差点脱口而出的话。这种说法仿佛是希望他坐近一点,久美子才不要让他这么觉得。她默默打开可可亚的拉环,甘甜香味从开口散发出来。久美子喝下一口,甜味黏住舌头。 

    秀一完全不把久美子当回事,一口气灌下刚买的汽水,他从T恤露出来的喉结咕嘟咕嘟上下震动。像这种时候,久美子都会不经意深深感受到他是个男人了。 

    「我有点不安。」 

    喃喃自语的声音令久美子不由自主抬起头来。秀一缩着肩膀,喝完的空罐放在桌上,发出一声清脆的声音,响遍被寂静填满的走廊。 

    「明天的比赛?」久美子问道。 

    秀一苦笑着点点头。 

    「没错。虽然一直说想进军全国,一旦明天真的就要比赛,突然觉得好害怕。躺在被窝里也完全睡不着。」 

    「不是兴奋到睡不着吗?像是迫不及待之类的。」 

    「要是那样就帅气了。」 

    秀一自嘲地微微一笑。自动贩卖机透出的光线柔和照亮他的侧脸。他重重叹了一口气,抓乱自己的头发,这是他不知所措时的习惯。 

    「我也很不安。」 

    久美子无意识地伸直脚尖,拖鞋从脚上滑落,掉在地上。 

    「可是跟过去的不安有点不太一样。」 

    「不太一样?」 

    「过去我担心自己在比赛时犯错,但现在不是那样。现在我怕的是比赛结束的现实。该怎么说呢?我害怕走到终点。」 

    久美子又喝了一口可可亚。秀一跷起二郎腿,仰天长叹。 

    「怕自己燃烧殆尽吗?」 

    「或许是吧!我也不是很清楚。」 

    久美子低头看着自己手中的可可亚,白烟冉冉上升,无声无息消散在空气里。 

    「不过,会这么想不就是一种进步吗?久美子国中时期有点漫不经心,感觉像是随波逐流从事社团活动。如果是那个时候的你,应该不会为社团活动燃烧殆尽吧!」 

    「这么说倒也没错。」 

    久美子静静垂下眼皮。国中时的确很少像这样为比赛投入一切,只是依照写在行事历上的行程按表操课,几乎没有主动想过什么、做过什么。 

    现在的自己又是如何呢?比起当时的自己,是否有所改变呢? 

    久美子陷入沉思,秀一也噤口不言。无话可说,寂静盈满在狭小的空间里。久美子把变少的可可亚往旁边一放,不经意伸长的指尖稍微碰到秀一的手。那一瞬间,秀一大惊失色往后仰,久美子也赶紧收回手。碰到的地方好热。久美子为了驱散皮肤的灼热,连忙抓住沙发一角,皮制表面冰冷得令人咋舌。 

    「啊,抱歉。」 

    秀一的音调明显比平常稍高一点,可惜久美子并没有余力指出这一点。心跳的声音吵死人了。为了掩饰狂乱的心跳,久美子不住摇头,长长的发丝轻柔拂过自己脸颊。 

    或许是受不了尴尬的沉默横亘在两人之间,秀一「啊!」地发出一声故作开朗的不自然叫声,好像这才突然想起来似地,在口袋里摸了半天。 

    「对了,差点忘了给你。」 

    秀一从钱包里拿出包装很精美的袋子,蓝色的包装纸上印有当地百货公司的名称,边缘贴着明显是给小朋友的小熊贴纸。该不会一直放在钱包里吧? 

    「喏。」 

    秀一将袋子递给她。 

    「这是什么?」 

    「上次说过的生日礼物。」 

    「你一直带在身上啊?」 

    「因为一直找不到时机给你。」 

    秀一不好意思地背过脸去。久美子接过,小心翼翼用指尖撕开封口,以免破掉。 

    「欸,现在就要打开吗?回房间再看啦!」 

    秀一不知怎地,手忙脚乱要阻止她。久美子对他的制止视而不见,拿出里头的东西。是个小巧的向日葵发饰,娇小的向日葵有着白色的花瓣。久美子用指尖拈起发饰,观察秀一的反应。 

    「好可爱,这真的是秀一选的?」 

    「这句话是什么意思?」 

    「因为以秀一的选择来说,太有品味了。」 

    「你真没礼貌耶!」秀一对久美子老实的反应耸耸肩。「我也不是很清楚,你喜欢这种花吧?」 

    「咦,我说过这种话吗?」 

    「你之前不是说过吗,什么白色向日葵的。」 

    这句话让久美子记起自己一个月前问他的问题。 

    「难不成是意大利白向日葵?」 

    秀一恍然想起似地击掌。 

    「没错,就是那个。听你提起后,我一直很想知道那是什么花,就问了川岛。她告诉我你很喜欢那种花。」 

    「小绿吗?」 

    「当我说要送你礼物时,她简直乐坏了,真伤脑筋。川岛是那种个性的人吗?」 

    「小绿对八卦最感兴趣了。」 

    绿辉兴奋到双眼发直的模样太容易想象,久美子忍不住莞尔一笑。 

    「话说回来,没想到有那么特殊的向日葵。我还是第一次看到白色的向日葵。设计成那种花的商品超级少见,害我找了半天。」 

    久美子举起手中的发饰,拿到日光灯底下端详。白色的花瓣微微透光,就连细致的加工部分都看得很清楚。 

    「这是泷老师向师母求婚时送给她的花。」 

    「什么!」 

    久美子不经意脱口而出的秘密令秀一不知所措瞪大了双眼,脸庞不知不觉染上红晕。久美子看到他的反应,不知怎地,就连自己都觉得好害羞,她感觉自己的脸一口气胀红,怎么会这么热。为了掩饰,久美子连忙将可可亚送到嘴边。 

    秀一站起来,拼了老命似地滔滔不绝:「呃,我不是那个意思!只是刚好选中这个当礼物。」 

    「这种事,不用你说我也知道。」 

    久美子不满地回嘴,秀一脸上蒙上一层阴影。他到底希望自己有何反应?再说了,有必要那么拼命否认吗?久美子一口气喝光可可亚,同时也吞下涌到喉咙口的话。 

    秀一着急地开口:「呃,我的意思是说……」 

    「啊,看到不该看的画面了。」 

    第三者的声音响彻整条走廊,盖过秀一的辩解。久美子和秀一都吓了一大跳,望向声音的来处。只见明日香正抱着一堆空宝特瓶站在那里。她穿着学校规定的运动服,而不是睡衣,学生都嫌那套运动服土气,但是穿在明日香身上,看起来就很时尚,真不可思议。这就是所谓的阶级吗?久美子偷看站在一旁的秀一。 

    「明日香学姐,你这个时间在做什么?」久美子问道。 

    明日香露出意味深长的笑容说:「这是我的台词吧!再怎么相亲相爱,也不至于在集训时私会吧,你们还真是大胆。」 

    「不不不,我们并没有相亲相爱!」 

    「学姐,你在胡说什么!」 

    秀一和久美子不约而同地矢口否认。 

    「是吗?」明日香把宝特瓶塞进自动贩卖机旁边的垃圾桶,表现出不甚在意的反应:「我拿房间的垃圾来丢,没想到会听到情侣吵架。」 

    「才不是情侣吵架!」 

    「就是说啊,请学姐不要拿我们寻开心!」 

    明日香笑嘻嘻看着胀红脸否认的学弟妹。这个人已经完全玩上瘾了。 

    「今天就到此为止吧,二位。明天就要比赛了,感情再好,也不能不去睡觉。」 

    明日香丢下一句「晚安」,与出现时同样神出鬼没地离开。她那英姿飒爽的背影确实充满了存在感。 

    「我想我一辈子都赢不了那个人。」 

    站在一旁的秀一无精打采地喃喃自语。我也是。久美子在心里附和。 

    旅馆一早就热闹非凡。 

    「久美子!起床!」 

    气势惊人的声音让久美子倏地睁开双眼,使尽全力撑起沉重的眼皮,从被窝里爬出来,往四周看,还在贪睡的社员全被绿辉毫不留情地挖起来。 

    「小绿好有精神啊!」叶月打着哈欠说。 

    插头已经被吹风机及整发器占领了,大家争先恐后挤向洗脸台。女生打扮起来是很花时间的。 

    「早安,久美子。」睡在身旁的丽奈从被窝里探出头来向她打招呼,凌乱的黑发披散在被单上。 

    「丽奈早安,昨晚睡得好吗?」 

    「还可以。」 

    丽奈掀开被子,畏光地眯着眼。瓷器般光滑的肌肤清晰地留下被子的痕迹。 

    「留下印子喽!」 

    「欸,讨厌,糟透了……」 

    丽奈边抱怨边倒向久美子,柔软的触感隔着T恤压在久美子身上,她不经意说了声「好重」,只见丽奈露出满脸笑意,眼皮已经完全闭上了。丽奈仿佛为了寻求温暖,脸滑向久美子的腹部,久美子不由得仰天长叹。 

    「哎呀呀,丽奈睡迷糊了。」 

    已经准备就绪的绿辉一蹦一跳靠近她们,叶月笑着帮忙打圆场。 

    「独奏的人肯定有很多压力吧!昨天也只有独奏的人留下来练习。」 

    「丽奈很努力呢!」 

    久美子轻手轻脚抚摸丽奈还没睡醒的脑袋,或许是很舒服吧,只见她愈来愈把重心往久美子的身上倾。真的好重。 

    「Help me……」 

    久美子忍不住求救,绿辉开始手忙脚乱起来,叶月则一脸好笑地袖手旁观。 

    「丽奈起床,再这样下去,久美子要被压扁了!」绿辉轻拍丽奈的脸说道。 

    丽奈纤长的睫毛微微颤动,静静提起眼皮,睡眼惺忪看着绿辉,再望向久美子。冷不防四目相交,丽奈原本还半睁半闭的双眸顿时睁得雪亮,怔忡的表情开始恢复平常的冷静,讶异地又看了绿辉一眼,终于搞清楚自己目前正处于什么状况。 

    「你真的很喜欢久美子耶!」叶月没头没脑地喃喃自语。 

    丽奈的脸颊顿时胀红得像煮熟的章鱼,她连忙起身,用力拍打久美子的背,试图掩饰羞赧。 

    「真是的!怎么不早点叫我起床。」 

    「我叫了啊!」 

    也不知丽奈有没有听到久美子的辩白,她红着脸,逃也似地冲向洗脸台。绿辉望着她的背影,调侃说:「呵呵,丽奈真是不老实。」 

    「看的人倒是不会腻就是了。」叶月边伸懒腰边笑着说。 

    「别再笑她了,小心丽奈又要生气了。」 

    久美子说完,也站了起来。天气非常好,秋高气爽的蓝天从窗外映入眼帘。但愿今天能有好事发生。久美子专注望向洒进室内的晨曦,慢条斯理伸了个懒腰。 

    为了吃早餐,全体社员开始往餐厅集中。几乎所有人都还穿着昨晚的睡衣,但是明日香及香织、小笠原等三年级已经换上制服了。 

    「各位,请注意。」 

    明日香坐在餐点前,对众人做出指示。 

    「吃完饭后,八点开始练习一开始的地方,八点半再稍微合奏一下,十一点离开这里。因为要直接上台,请打扮得恰如其分。忘记带绑头发的橡皮筋的人,可以来三年级的房间借,报告完毕。」 

    「是!」 

    明日香似乎很满意大家精神抖擞的反应。 

    「那么请双手合十,我要开动了。」 

    「我要开动了。」 

    社员一起双手合十,开始用餐。热气蒸腾的味噌汤非常好喝,感觉可以配好几碗饭。久美子用筷子剥散鲑鱼,配着白饭送入口中。 

    「马上就要上台了,别吃太饱喔!」 

    「要是吃坏肚子就糟了。」 

    梨子和卓也正经八百地提醒学妹。久美子和绿辉吓得停下筷子,叶月却正把第二碗饭扒进嘴里。 

    「话说回来,这天终于到了。」 

    夏纪喝着红味噌汤,感慨万千地说。「就是说啊!」梨子附和。 

    「总觉得结束后会燃烧殆尽,好可怕。」 

    「请不要燃烧殆尽,还有明年呢!」 

    夏纪哼了一声。卓也擦拭起雾的眼镜,点头说道:「明年你也要来A部门喔!」 

    夏纪大概是作梦也没想到他会这么说,一口气噎住,眼里闪过一道光。梨子或许是体会到她的心情,温柔微笑。 

    「对呀,明年再和大家一起进军全国。」 

    夏纪开口,结果什么也没说,连同酱菜一起吞下白饭。表情虽然不是很开心,但发丝间隐约可见的耳朵却异样地红。 

    「叶月明年也一起代表A部门出赛吧!」绿辉笑咪咪地说。 

    叶月大啖水煮蛋,自信十足地点头:「那当然。」 

    大家回到一年级的房间,开始换衣服。只有女生的房间根本没有害羞的概念,所有人都大剌剌脱光衣服,久美子忍不住取笑只穿内衣昂首阔步的同学。 

    「啊,这件内衣好可爱!」 

    「对吧!今天可是战斗内衣。」 

    「那种分不出正面还是背面的胸部是要战斗什么?」 

    「少啰唆!这是要讨个吉利。但愿学长姐能拿下金奖,我穿了关西大赛时的内衣。」 

    「比赛成绩跟你的内衣一点关系也没有吧!」 

    「闭嘴!重点是心意好吗?」 

    B部门的成员讲着毫无营养的蠢话。丽奈站在稍远处,正在调整裙子的长度。正式比赛时的标准长度是刚好盖住膝盖。她藏青色的裙子底下露出细细的双脚,小腿肚与小腿肚之间形成空隙。久美子瞥了优美的曲线一眼,轻拍丽奈的肩膀。 

    「状况如何?」 

    「普普通通……」 

    丽奈以一如往常的淡漠表情回答,视线停在久美子的肩头。 

    「久美子,我帮你绑头发吧!」 

    「欸,为何?」 

    「别问了,让我绑就是了。」 

    丽奈硬是让久美子坐在榻榻米上,哼着歌,用梳子梳开久美子的头发。比赛时,长发的学生都必须把头发扎成马尾,以免让看的人感觉烦躁。 

    「我啊,今天的比赛会全力以赴。」 

    久美子的耳朵被丽奈的手指碰到,感觉好痒。自然鬈的头发被绑紧,散落的发丝则用黑色发夹加以固定。 

    「你要独奏嘛,责任重大。」 

    「可是好像有点紧张。」 

    丽奈停下为她绑头发的手。久美子正要回头,丽奈却厉声说:「不要动。」 

    「一想到这次的比赛对泷老师也很重要,就开始担心万一失误怎么办。」 

    「别担心。」 

    久美子打断丽奈的丧气话,不容反驳地打包票。 

    「丽奈一定没问题。因为是丽奈嘛!」 

    丽奈没料到久美子会这么说,沉默了几秒,然后目瞪口呆叹了一口气,柔若无骨的手包住久美子的双颊。 

    「这算什么理由。」 

    「这就是最好的理由。」 

    「才不是。」 

    相对于噘着嘴反驳的久美子,丽奈的喉咙咕嘟一声,愉悦地笑了。 

    「接下来换久美子帮我绑头发。」 

    「欸,我也要吗?」 

    「拜托你了。」 

    丽奈一屁股坐在久美子跟前。伸直一双长腿,孩子气地将头顶到她面前。丽奈的头发跟久美子的不一样,她滑不留丢的黑发直通通的。当梳子滑过没有任何鬈度的直发时,丽奈享受地闭上眼。久美子觉得她好像猫咪,手指伸进光艳照人的发丝。 

    「这一天终于来了。」泷看着所有社员的脸说道。 

    结束比赛前的最后一次练习,搬运完乐器后,A部门成员全都坐上巴士。丽奈坐在久美子旁边的座位,从刚才就莫名开心地用手指绕着马尾前端。久美子不由自主凝视她的侧脸,美少女果然绑什么发型都好可爱,再看看倒映在车窗上的自己,忍不住叹气。 

    「昨晚睡得好吗?」泷问道。 

    社员一口一声地回答这个问题。至少一年级的人都很早就睡了,学长姐呢?久美子偷眼望向明日香,只见她不知怎地正调笑着顶了顶香织的侧腹部。 

    「春天立下要参加全国大赛的目标,我们终于走到这一步。我自己也是初次参加全国大赛,什么都不懂,或许大家会觉得我很靠不住。但是在这样的情况下,还能留下这么丰硕的成果,都是拜各位的努力所赐,感谢各位一路跟我走到这里。」泷说道。 

    巴士上鸦雀无声。每个人都有自己的想法。回想春天的社团,表情不禁变得苦涩。尽管很多学生都对泷颇有微词,但大家还是相信他的能力,才能得到现在的成果。久美子静静垂下眼,丽奈在一旁聚精会神听泷说话。 

    泷继续接着说:「未来不晓得还有没有机会在这么多人面前演奏。我不会要各位别在意成绩,但是既然都走到这一步了,比起别人的评价,不妨尽全力演奏出不要让自己后悔的音乐。尤其对三年级而言,今天真的是最后一场比赛了,请在这个众所瞩目的舞台上让大家见识我们的演奏。」 

    「是!」 

    全体社员一起中气十足回应泷为大家加油打气的话,巴士上充满了热烈的气氛。 

    「我很不甘心。」 

    久美子闭上眼睛,深不见底的黑暗中,浮现出国中最后一场比赛的画面。那时候,丽奈哭着对久美子说她很不甘心。明明只过了一年,感觉却已经是好久以前的事。丽奈哭泣的脸始终鲜明地烙印在她眼底。 

    「我不想再说出不甘心这三个字。」 

    「什么?」 

    喃喃低语从耳边传来,久美子下意识转向隔壁。丽奈眼睛直直注视着泷,手抓紧裙子边缘,仿佛下定决心,重复着同一句话。 

    「我再也不想后悔了。」 

    这句话让久美子一口气哽在喉咙。记忆中,久美子没为哭泣的丽奈做任何事,只是怔怔看着始终站在原地不动的她。 

    可是,现在不一样了。 

    久美子用力握紧拳头说:「一定要拿下金奖。」 

    丽奈惊讶地睁大双眼,抿成一条线的唇瓣慢慢上扬。「嗯,一定。」丽奈说道,她抓住久美子的手,热情隔着皮肤缓缓传了过来,好温暖。久美子悄悄露出笑容。 

    名古屋比赛场地的音乐厅,是久美子至今造访过的会场中最大的,光是巴士的数量就超级多,停车场都快挤爆了,穿着制服的参加者各自排成一队,鱼贯前进。 

    入口处有一座巨大的纪念碑,许多观众都兴高采烈地在纪念碑前拍照留念。纯白的巨大骑马像全长将近九公尺。 

    「哇,好大的马!」 

    一旁的绿辉大声欢呼,久美子已经没有余力注意这一点。从下巴士的瞬间,四周的空气就带着刺痛皮肤的热度,穿着制服的学生抱着乐器,手忙脚乱地来来去去。工作人员的人数也不少,看似大学生的年轻人穿着西装,兵荒马乱地忙得不可开交。上半场比赛已经结束,参赛学生正在兴奋交谈。 

    「听说明工拿下金奖了,果然好强啊!」 

    「坂江居然是银奖,真是莫名其妙。」 

    「西条女中是铜奖啊,不过以第一次出场来说,算是很努力了。」 

    「大阪来的都拿到金奖吗?哇,完全制霸耶!」 

    大音乐厅已经开始进行下半场的演奏,北宇治的出场序是下半场近尾声时,当他们演奏完毕,也会像这样被评头论足吧!久美子下意识摸着胸口的缎带,吐出一口大气。要说不紧张绝对是骗人的,可是比起紧张,更多的是期待。不同于必须考虑到还要挺进下一场比赛的关西大赛,这里就是全国大赛的终点。不管这次演奏的结果是好是坏,久美子他们花了一整个夏天的挑战,到这里都将告一段落。 

    「既然是最后一次,不乐在其中就亏大了!」 

    绿辉抓住久美子的手,意气风发地用力上下摇晃。久美子被她的气势感染,也点了点头。 

    「比赛加油吧!」 

    「那当然!」 

    绿辉展颜而笑。 

    搬完乐器后,打击乐器和其他声部分头开始行动。久美子从乐器盒拿出上低音号,仔细检查有没有问题。霙在一旁组装双簧管,有些得意地对她说:「今天用了最棒的簧片。」 

    霙的脸颊微微泛起红晕,尽管依旧面无表情,还是看得出她比平常兴奋。 

    「这样啊!」 

    「嗯,是我珍藏的簧片。」 

    对木管演奏者而言,簧片至关重要。市面上的簧片有好有坏,演奏者要从中仔细判别,选出品质精良的簧片,小心翼翼地使用。总是使用同一片簧片,想也知道很快就会耗损,所以手边要囤积几片好簧片备用,轮流吹到习惯为止。铜管的吹嘴不是消耗品,所以久美子对这方面的事不是很懂,但是考虑到这个小东西带来的影响,不由得深深感受到乐器的构造真的非常复杂。 

    「心跳得好快,独奏能吹好吗?」 

    「没问题的,学姐从未在比赛犯错不是吗?」 

    「可是,这是我第一次参加全国比赛。」 

    就连霙学姐也会紧张啊!久美子思考着这件理所当然的事。因为霙无论在哪个舞台上都能面无表情、落落大方吹奏乐器,还以为她完全不会紧张。 

    「我的心脏也跳得好快,但或许是太高兴也说不定。」 

    「太高兴?」 

    「没错。能在梦寐以求的舞台上演奏,真是太期待了。」 

    「我也是。」 

    霙露出腼腆的微笑。粉红色的唇瓣间隐约可见一口白牙。 

    「要是拿下金奖,希美也会以我为荣,所以我要全力以赴。」 

    「是、是嘛!」 

    「嗯。」 

    她的世界还是老样子,总是围绕着希美运转。不过,只要能因此孕育出那么美妙的音色,久美子觉得吹奏乐器的动机是什么都无所谓。 

    「一起加油吧!」 

    霙面无表情朝她伸出拳头,她会有这种动作还真难得。久美子一时愣住,霙云淡风轻地告诉她:「优子说比赛前要这样。」 

    这么说久美子就完全理解了。优子的确有可能会喜欢这种动作。一直让学姐保持同样的姿势也很失礼,久美子把自己的拳头轻轻撞向霙的。 

    「呵呵。」 

    不知是哪里好笑了,霙看起来很满意。久美子看着看着,原本绷紧的力量也逐渐放松。相较于噗哧一笑的久美子,霙依旧面无表情。 

    调完音,在小音乐厅进行最后一次发出声音的练习,北宇治高中的出场顺序一分一秒逼近。为了不让乐器降温,久美子隔着吹嘴朝上低音号进气。 

    「跟平常一样,检查一开始的地方。」 

    依泷的指示,社员反复吹奏曲子的开头部分。指定曲一开始是小号的主旋律,一旦走音会很明显,因此无论是京都大赛,还是关西大赛,泷都会花很多时间练习进入曲子的小节。要是一开始演奏就出错,将会酿成无法挽回的悲剧。 

    练习时间结束,社长与副社长一如往常站在众人面前,小笠原的表情与平常略有不同,感觉神清气爽。 

    「终于要正式上台了。」社长缓缓开口。 

    「凭良心说,当上社长以后,我有好几次都想退出社团。可是,光是今天能站在这里,过去那些不愉快的事都能全部一笔勾销。此时此刻的我,就是这么开心。唯有今天,我不想再说泄气话。全力以赴吧!拿下金奖,为今天画上完美的句点。」 

    「是!」 

    小笠原掷地有声的演说让大家也回答得铿锵有力。挂在脖子上的上低音萨克斯风今天看起来也比平常更加自豪且闪亮。明日香站在一旁,露出调侃的笑容。 

    「怎么突然变得这么像社长了。」 

    「人家本来就是社长。」 

    两人一搭一唱,让紧绷的气氛稍微放松了些。明日香环视整座音乐厅,默默开口:「全国大赛以前,给大家添了许多麻烦。多亏各位的帮忙,我现在才能站在这里,真的非常感谢大家。」 

    不必明说,大家也知道明日香指的是什么。明日香站上全国大赛的舞台之前,的确经历过各式各样的难关。但那些已经不重要了。因为大家都愿意伸出援手,明日香现在才能站在这里。这样就够了。 

    「对我来说,全国大赛是很特别的舞台。啊,对大家当然也很特别。我今天一定要表现得无懈可击。」 

    明日香没说要表现给谁看,这里恐怕只有久美子知道她这句话真正的意思。今天的舞台不只是明日香高中生活最后一次比赛,同时也是唯一一次演奏给父亲听的机会。父亲送给她的银色上低音号形影不离倚在明日香身边。 

    明日香看了泷一眼,绷紧脸部的肌肉。 

    「如果是网球或篮球或其他的社团活动,指导者不会跟选手一起参加比赛,但管乐社不同,包括泷老师在内,我们五十六个人要站在同一个舞台上,我认为这是非常不简单的事。只有管乐社,才有机会由顾问和社员同时上场比赛。」 

    明日香说到这里,停顿了一下。红色镜框熠熠生辉地散发出充满企图心的光芒。 

    「所以,今天就由我们五十六个人演奏出最完美的音乐吧!包含老师在内,大家一起。然后笑着结束。」 

    「是!」 

    众人的回答让明日香笑得煞是满意,柔和的眼神望向小笠原,再自然不过地捶了社长的肩膀一记。小笠原似乎是察觉到她的意图,用力点头,顺势转向泷说:「今天也请老师加入。」 

    「我也要吗?」 

    「那当然。」 

    小笠原说得斩钉截铁,泷有些害臊地搔头。已经想到接下来会发生什么事的社员热切看着小笠原。小笠原依序看了一轮音乐厅内的所有人,深呼吸,胸脯隔着藏青色的水手服高高隆起。 

    「那么,请随我一同高呼……北宇治加油!」 

    「加油!」 

    泷沉稳的声线夹杂在社员的声音里,回荡在音乐厅内。久美子捧着上低音号,深深吐出一口气。是因为紧张吗?总觉得呼吸困难,心脏扑通扑通跳得飞快,紧紧贴住耳膜,显得更加嘈杂。 

    「北宇治高中的同学,时间到了。」女性工作人员推开门说。 

    接下来严禁发出声音。泷率先往外走,后面跟着低音号的成员。 

    「别太紧张。」 

    秀一从久美子身边走过时,压低声音提醒她。久美子默不作声地点头,跟在明日香背后往前走。通往大音乐厅的走道安静得连一根针掉在地上都听得见。泷推开厚重的门扉,其他高中的演奏声顿时涌了出来。一旦比到全国大赛,没有哪所学校吹得蹩脚,前后左右都是名闻遐迩的强校。可是,不可以心虚,因为北宇治也已经跻身强校之林。久美子抱着上音低号,咽了口水。 

    从布幕隔开的后台,可以看到一点点其他学校正在比赛的样子。舞台上充满耀眼的强光,与乌漆抹黑的后台互为对照。经历再多待在这里的时间也不会习惯。为避免紧张到神志不清,久美子在手心里写个人字,吞下三次。想必不会有特殊效果,但总比什么都不做来得好。丽奈正站在稍远处望着观众席。金色的小号在她手中闪闪发光。 

    「葵说她会来看。」 

    久美子耳边传来低喃,她蓦地扬起头,定睛一看,站在身旁的明日香小声对她说。 

    「葵学姐吗?」 

    「嗯。听说票超难买的,她好像想尽办法才买到票。」 

    管乐的全国大赛入场券非常抢手,理由十分单纯,因为想买的人太多了。 

    「好厉害啊!特地为此一个人大老远跑来名古屋吗?」 

    「她说要搭新干线来。」 

    明日香说道,微微笑眯了眼。舞台上的演奏来到尾声,钹的华丽音色响起。音乐厅的空气在悠扬的旋律下,余音绕梁地震动。等待上台的泷活动了一下身体。 

    久美子望向漆黑的观众席,从声带里挤出声音来告诉明日香:「不瞒你说,我姐今天也来了。」 

    其实没必要说的,但久美子无论如何都想告诉某个人这件事。对其他人来说,这根本不值一提吧,但是对自己而言,姐姐来听她演奏的事实跟奇迹没两样。 

    「来看久美子演奏吗?」 

    「嗯。」 

    「哼……」 

    明日香冷哼一声,吐了吐舌头,视线望向观众席搜寻。 

    「那可不能漏气呢!」 

    明日香轻拍久美子的肩膀,后者静静点头说:「嗯。」这个舞台对明日香和久美子都是特别的,她们绝对不想失败。 

    布幕对面传来观众如雷贯耳的掌声。前一所学校演奏完了。 

    「以上是东关东代表,成合市立成合高中管乐社带来的演奏。」 

    司仪口齿清晰地介绍。久美子硬生生从肺部挤出气体,借此吐出紧张感。 

    「接下来,由编号第十号,关西代表京都府立北宇治高中管乐社的同学为大家演奏。」 

    北宇治高中的成员一起移动到台上。确定明日香开步走,久美子赶紧跟上去。站在阴暗的后台,刚才那所高中的学生鱼贯离开舞台的背影映入眼帘。隐约可见有个少女珍而重之地捧着小号,脸颊挂着明艳的水光。大概是比赛时出错了,在朋友的安慰下,一下子就从台上消失了踪影。久美子看着有如泄气皮球般的小小背影,心跳快得令人生厌。那个女生肯定是自责地哭了。比赛是由所有人共同完成一首音乐作品,自己犯错就会扯其他人后腿,不是自己丢脸就可以了事。 

    久美子就座,把手里的上低音号置于膝上。后台的空气明明那么冰凉,舞台上却还残留着刚才演奏的激情。聚光灯的光线集中在舞台上。好热。被寂静填满的空间只有社员行动的声音。众人同一时间沙沙作响地翻开乐谱,「娥眉月之舞」几个字隔着透明文件夹浮现出来,空白处写满了密密麻麻的字。没问题,照平常那样演奏就好了。久美子告诉自己,深吸一口气,肺部隆起,呼吸没那么困难了。 

    「指定曲是第五首曲目,自选曲为尼格尔.赫斯作曲的〈东海岸风情画〉,指挥为泷升。」 

    灼热的白光照得泷异常耀眼,灰色西装上浮现出削瘦的身形。泷低头行礼,观众报以掌声。啪、啪、啪响个不停的掌声宛如一种生物,在音乐厅里穿梭来去。从浴沐在明晃晃灯光下的舞台看出去,看不清阴暗观众席上的反应。姐姐就在那群人里面。久美子一思及此,背脊不知不觉打得笔直。耳边传来明日香在旁边深呼吸的声音。关西大赛时紧张到头脑一片空白,现在却没有那种感觉,或许是神经已经兴奋到变得迟钝了。舞台上没有的东西全都隔着一层透明的薄膜出现在另一边,不管是评审委员的席次,还是接下来的评价,全都无所谓了,只剩下自己怀中的乐器展现出鲜明的重量,久美子沉浸在几乎使脑浆融化的兴奋里。 

    泷轮番看着社员的脸,表情十分柔和,仿佛要让社员放心,以唇语无声地说:「没问题的。」大家确实接收到这个无言的讯息。没问题的。都已经练习成这样了。久美子反复对自己说同一句话,凝视着泷。 

    泷静静提起手臂。社员慢慢拿好乐器,配合指挥棒轻轻颤动,众人同时吸气。指挥棒往下挥的那一刹那,小号华丽的音色一起从号口倾泻而出。是刚才在音乐厅反复练习过无数次的地方。行云流水的高音主旋律,与空气水乳交融。长号仿佛要与其比翼双飞似地演奏出中音,长笛与单簧管的音色如彩蝶般在四周翩翩飞舞。从低音号的巨大号口响起的低音,颤巍巍地震动着空气。确定明日香的手微微一动,久美子慢慢拿好放倒的乐器。休止符之后,与法国号一起注入上低音号的温和音色。只要对吹嘴进气,乐器就会给予回应。柔美的音色溶解在音乐里。钹的声响破空而起,音乐逐渐变得激烈起来,速度愈来愈快,铜管缤纷多彩的主旋律响彻整座音乐厅。 

    久美子自己的上低音号发出来的声音盖过了其他乐器的音色。此时此刻,自己听到的音乐跟观众听到的音乐大概完全不同,会觉得近在咫尺的乐器比其他乐器大声是再自然不过的事。从久美子的位置其实听不太到木管的声音,反而是号口朝向前方的小号及长号的音色听得一清二楚。小号的音色从后方窜出,一直线地奔向观众席。上低音号柔和的音色传到比较远的地方时,马上就会被其他声音盖过。 

    久美子的视线追逐着乐谱,手指拼命按压活塞阀。演奏到浑然忘我时,曲子马不停蹄地前进,热闹非凡的曲调逐渐归于平静,接着是木管在指定曲的重头戏。铜管的喧嚣沉静下来,由长笛和单簧管、萨克斯风的音色交织出如泣如诉的主旋律。低音管的低音静谧地震动空气,音乐来到双簧管的独奏。泷的视线望向霙,她会意地点点头,含住簧片,从双簧管里吹出圆润饱满的音色。久美子每次听到霙演奏的音乐,都觉得脑浆快融化了。甜美温柔又湿润的声响无疑是她拼命苦练的成果。霙细致的手指在银色的音键上行云流水地滑动,被寂静填满的音乐厅内只剩下她的独奏。久美子专心聆听双簧管的音色,发现自己的心脏快要跳出来了。心跳的声音回荡在耳边,声响震天。安静好可怕。因为万一犯错,马上就会穿帮。然而,霙本人似乎完全没注意到四周。她的双眸一瞬也不瞬地盯着泷。独奏最后是一个强而有力的延长音。只见她以几乎与平常无异的表情完成了独奏。 

    轻快的主旋律开始狂奔,一口气撕开归于寂静的空间。主旋律由充满存在感的法国号担任,再由长号和上低音号接下主角的棒子。已经用无数次的反复练习克服了吹不好的部分,这个曾经令久美子伤透脑筋的小节,如今也成了她的拿手好戏。从低音到高音的急速爬升。明日香轻轻松松就能搞定的旋律,久美子也能完美重现了。丹田用力,集中注意力,不要让高音跑掉。逐渐变得沉重的低音,融入了小号的高音。从舞台上奔腾流出的声音粒子整合成扎实的音色。负责打击乐器的一年级社员满脸通红地用鼓棒敲响吊钹,小鼓连绵不绝的节奏为乐曲增添了色彩。随着曲风来到高潮,指挥棒激烈地上下挥舞。泷冒出额头的汗水在灯光的反射下,闪闪发光。音量提高到极限,然后猝不及防地画下休止符。 

    演奏还残留着指定曲的余韵,直接进入自选曲。 

    上低音号、低音号、长号。自选曲从中低音的合声开始,长笛的旋律再加入温润饱满的音色里。短号承接先前的曲风,维持庄严的气氛,开始独奏。低音交棒给木管乐器,久美子等人暂时放下乐器。丽奈吹奏的短号真诚无伪,传遍了整座音乐厅。她所孕育出来的音色永远都那么充满自信,耀眼夺目。扎实的音色极富弹性地在音乐厅内无限延伸,逐渐融入炽热的空气里。令人心荡神驰的高音破空而起,其他小号加入演奏。维持轻柔舒缓的曲风,音乐开始一点一点进入高潮,化作一团的大量音符朝着同一个方向编织成音乐。铜管气势磅礡的主旋律荡气回肠,木管则在一旁吹奏出剧烈的连音。钹用力敲响,让高潮继续往上推升。泷停止指挥的动作,来到最高潮的音乐至此唐突地告一段落,空气还充满了足以刺痛肌肤的张力,顿时被静谧包围。 

    泷以慎重的动作划下指挥棒,曲子进入第三乐章。第三乐章〈纽约〉从铜管热情奔放的主旋律开始,号口吐出的音色分毫不差,这也是练习了一整个夏天的成果。整齐画一的层层乐音给人厚重的印象。木管的过渡性音节快得连听的人都不禁为之咋舌。中音鼓敲出节奏,让音乐变得更加热闹非凡。 

    身体好热,喉咙好干。久美子吞了口水,把掌心的汗水抹在裙子上,为乐谱翻页。自选曲即将进入尾声,这个事实让她觉得好舍不得、好想继续演奏、好想这样继续吹下去。涌上心头的冲动撩拨着久美子的意识。 

    木鱼轻快的声响沉潜在音乐里,随着交织而成的旋律,三角铁敲出响亮声音。配合主旋律奔驰的时机,响板发出挥鞭般的声音,啪啪啪的轻快音色听起来总是令人心旷神怡。 

    法国号、小号、单簧管。主旋律的变幻令人目不暇给,音乐逐渐进入高潮。充满弹性的音色清澈动人,没有一丝杂质。加入尖锐的哨声之后,音乐愈来愈慷慨激昂,转瞬间又归于寂静。鸦雀无声的空间里,缓缓响起长笛的旋律。游刃有余的音乐填满了音乐厅的每一个角落,这时,裂帛般的警铃声突然响彻云霄。震天价响的警铃让人联想到火车的声音。方才的宁静直转直下,音乐再度加快。随着节奏愈来愈快,泷挥舞指挥棒的动作也愈来愈剧烈。演奏维持热度,冲向最后的高潮。配合指挥强而有力的动作,大家反复吹响同一个音。尽管体力几乎已经耗尽,久美子还是奋力挤出最后一个音,使尽吃奶的力气加强乐音。当乐音到达极限的瞬间,泷的指挥棒戛然而止。 

    隔了几拍以后,掌声如潮水般涌来。泷如释重负地吁出一口气,朝社员微笑。确定他的手上下一动,久美子赶紧站起来。全体社员动作整齐画一转向观众席。这么一来,掌声更热烈了。好像一场梦。比赛时的记忆十分模糊,只有类似真实感受的余温,似有若无地残留在掌心里。 

    「以上是关西代表北宇治高中管乐社带来的演奏。」 

    司仪冷静地介绍,社员开始准备下台。久美子还在恍神,明日香拍了拍她的肩膀提醒她。久美子猛然回神,连忙抓住乐谱和上低音号。 

    「辛苦了。」 

    明日香说道,微微一笑。久美子发现她的指尖微微颤抖,不由得悚然一惊。明日香小心翼翼抱起银色的上低音号,头也不回地走向出口,久美子也连忙跟上。 

    「辛苦了。」 

    久美子好不容易才挤出这句话,明日香有没有听见呢?她没有回头,久美子也没再重复一次。 

    社员手里拿着乐器,在广场拍照留念。每次比赛都会像这样拍照留念,先拍所有人的合照,再分组拍照。由专业摄影师拍的照片稍后会寄发样本到学校,由社员挑选、购买想要的照片。 

    拍完照片再搬运乐器,然后由泷召集所有社员,美知惠一如往常站在旁边,以锐利的眼神看着众人。她的眼角微微泛红,透露她刚刚哭过的事实。 

    「各位,比赛辛苦了。」泷说道。 

    「老师也辛苦了。」社员异口同声地回答。 

    其他学校的学生从旁边经过时,只要看到泷,皆会发出尖叫声。泷清了清喉咙说:「比赛的演奏很完美,彻底发挥了过去的练习成果。我想你们每个人都有不同的想法,或许也有人会觉得自己可以演奏得更好,但我认为今天的演奏就是各位现在的实力。这次演奏是我们今年夏天的成果,无论得到什么评价,各位都可以抬头挺胸地回家。截至今天,大家真的非常努力。」 

    每个人对这句话都有不同的感慨。久美子把乐器放在地上,隔着衣服,按住自己的心脏。心脏还是跳得好快,完全没有现实感,也不觉得比赛已经结束了,甚至怀疑站在舞台上的十二分钟是不是在作梦。 

    泷四下看了一圈,绷紧脸上的肌肉说:「以下是接下来的指示。一楼的观众席几乎已经坐满人了,所以公布成绩的时候请在二楼观众席的南口附近集合,听完结果以后就要马上离开会场上巴士。听清楚了吗?」 

    「听清楚了。」 

    「那么就先原地解散。总之一句话,今天辛苦大家了。」 

    听完泷的指示,社员各自展开行动。绿辉冲上前来,摊开不晓得从哪里弄到的小册子。不知为何,她的皮包里居然还有另一本小册子。 

    「小绿,你拿那么多小册子做什么?」 

    「当然是珍藏用啊!只有一本的话,多翻几次就会变得皱巴巴的。」 

    「珍、珍藏用?」 

    久美子佩服不已。丽奈和叶月走向她。 

    「快点去欣赏演奏吧!」叶月提议。 

    丽奈有些困扰地抱着手臂说:「可是人好多噢!接下来是清良女中的演奏。」 

    「欸,已经是最后一所学校了?只能看到一所学校的演奏也太可惜了。」 

    绿辉失望到肩膀都垮下来了。清良女中是最后一所比赛的学校,她们的演奏一旦结束,就要公布成绩了。为了按捺跳得愈来愈快的心脏,久美子悄然叹息。 

    「抱歉,我可以去外面等吗?」久美子说。 

    「去外面?」 

    丽奈眨了眨眼。久美子回以暧昧的微笑,点头称是。 

    「嗯,有点不太舒服。」 

    绿辉闻言,开始手忙脚乱,不知所措地东张西望,指着美知惠说:「很不舒服吗?没事吧?要不要跟老师说?」 

    「呃,不是生病啦,所以不要紧,只是有点紧张。」 

    久美子的回答让绿辉放下心中大石。丽奈从背后轻拍绿辉的肩膀。 

    「久美子就交给我,小绿你们先去看比赛吧,公布成绩的时候再会合。」 

    可是……绿辉还放心不下地看着久美子,叶月不由分说推着绿辉。 

    「这时就应该接受别人的好意。小绿很想看清良女中的演奏吧!」 

    「是这样没错啦……」 

    「既然如此,那就去看吧!二位,我们先走一步。」 

    叶月挥手,久美子和丽奈也朝她挥手。绿辉还进退两难地皱着眉头,但一往音乐厅走,马上变得神采奕奕,还说:「不快点去就没位子了」,眉飞色舞地冲向二楼。久美子和丽奈默默目送她的背影渐行渐远。丽奈以指尖把玩着发尾,看着久美子问道:「找个地方坐吧!」 

    「嗯。」 

    广场上已经没有人影了,大概只剩下久美子和丽奈。拿着乐器盒的女学生正匆匆忙忙绕过她们狂奔而去。穿过入口走进音乐厅,只见有一台电视正拨放着贩卖的DVD画面,其他学校的社员都围在那里看。 

    「啊,拍得好清楚。」 

    「哇!拍到青春痘了,好丢脸。」 

    「没有人会在意啦!啊,是佐佐木学姐。」 

    「这里的独奏真的好酷噢!唉,我明年也想独奏。」 

    每次拍到谁的特写,其他人就发出近似欢呼的叫声。久美子和丽奈坐在设置于会场门口的沙发上,心不在焉地看着这一切。隔着厚厚的大门,隐约可以听到清良女中的演奏。好美的音色。回想车站大楼音乐会的演奏,久美子悄悄垂下眼帘。 

    「好像还不觉得已经结束了。」 

    丽奈说道,伸直双腿。略长的裙子和袜子之间露出一截白皙的肌肤。 

    「我的记忆也很模糊。因为太投入了。」 

    「真的只是转眼之间呢!」 

    丽奈静静低垂眉眼。久美子也点点头,竖起耳朵倾听依稀可闻的音乐。工作人员在视线一隅走来走去。印着「比赛会场限量商品贩售中!」的文字四周几乎没有半个人影。 

    「比赛好开心啊!」久美子说道。 

    丽奈也简短附和了一句:「对呀!」接下来还会有这种经验吗?从上高中到今天为止,久美子一路埋头苦干实干过来,今天终于达成目标,站上梦寐以求的舞台演奏。想到这里,身体开始没力,好想就这样沉沉睡去。结果是什么都无所谓了。什么都不想知道,只想直接回家。之所以有这样的想法,是因为自己太软弱吗? 

    高中毕业也同时放弃音乐的人意外的多。即使是强校的学生,比赛结束同时燃烧殆尽的例子也所在多有。久美子过去一直以为这种事与自己无关,如今也能充分体会到那种人的心情了。付出一切后,的确什么都不会剩下。 

    「打进全国的每一所学校果然都好厉害。」 

    丽奈拄着脸颊说。她把手肘放在自己的大腿上,愁眉深锁地轻声叹息。 

    「老实说,我好害怕听到结果,心脏直到现在都还扑通乱跳。」 

    「我也很害怕。总觉得呼吸困难。」 

    「小绿在这方面真的好了不起。」 

    「因为她的心脏很大颗嘛!」 

    久美子大概一辈子都不可能变成她那样。绿辉和丽奈各有不同的坚强。绿辉的心灵柔软又晓得变通,无论发生什么事,都能处之泰然。 

    丽奈目不转睛盯着电视萤幕里其他学校的演奏看,突然站了起来。久美子抬头往上看,丽奈告诉她:「演奏结束了,我们该进去了。」 

    语声未落,音乐厅的门同时打开。密闭的空间顿时开展,盈满在里头的热气向外涌出。震耳欲聋的掌声让久美子和丽奈顿时身体僵硬。欢声雷动到耳膜都快震破了。 

    「Bravo!」 

    赞赏的话语此起彼落地响起。那是自己演奏时绝不会听到的赞美。清良女中真不愧是全国大赛的金奖常胜军。 

    久美子呆站在原地不动,丽奈轻轻拉起她的手,指尖热得仿佛要烧起来。长长的裙子扬起,纤细的脚往前跨出一步。 

    「走吧!」丽奈说道。 

    久美子回握她的手,对她始终如一的强大默默点头。 

    为了知道比赛结果,所有参赛者都聚集在音乐厅。观众席自然是座无虚席,就连走道也挤满了人。从二楼的观众席往下看,不难发现每个学校都自成一个个集团等待结果。每个学校的服装都各有巧妙不同,会场内看起来就像分成好几个色块。久美子和丽奈总算与北宇治的集团会合,已经有许多社员迫不及待聚集在那里等待成绩公布。当然已经没有空位了,久美子等人只好坐在最靠近扶手前的走道。 

    「啊,来了来了。」 

    叶月对她们招手,一旁的绿辉探出身子偷看舞台。 

    「你们看那边!有好多奖杯和奖牌。」绿辉指着一整排金色的奖杯说道。看到那些奖杯,充分感受到自己现在真的在比赛。 

    「啊,代表进场了。」叶月说道。 

    久美子连忙将视线转回舞台上。代表各校的社员与顾问一起上台。绿辉满心欢喜地拿出望远镜,望向明日香。少女沐浴在聚光灯下,被托烘得明艳照人。像这样与其他学校的顾问站在一起,泷看起来更年轻了。 

    有位大有来头的大人物正在台上致辞,但社员几乎都没在听。心跳逐渐加速,仿佛随时都要爆裂开来。久美子抓紧裙摆,丽奈则紧握住她的手。 

    「首先致赠指挥奖给带领各位征战到这里的各校顾问。」 

    隔着麦克风传来的声音令北宇治的学生面面相觑。 

    「指挥奖?」 

    「欸,那是什么奖?」 

    不理会身后窃窃私语的学长姐,参加过全国大赛的绿辉压低音量说明:「那是颁发给所有带领乐团进军全国大赛的指挥的奖,之后才会发表金银铜的结果……」 

    「最喜欢阿山了!」 

    会场的一楼响起尖叫声,打断绿辉的说明。接过奖牌的顾问害臊地搔搔头,静静行了一礼。接下来每位顾问获颁奖牌之际,都会有某个团体送上欢呼。看样子,趁着颁发指挥奖时对顾问献上平日的感谢,似乎已成惯例。 

    「怎么办,我没听说有这个桥段。」 

    「该说什么才好?快点决定。」 

    学长姐在后面交头接耳,迟迟决定不了要对泷说什么。 

    「东关东代表,成合市立成合高中管乐社,吉崎修介先生。」 

    排在北宇治前面的学校顾问正接下奖牌。 

    「谢谢老师!」 

    一楼前方的集团以整齐画一的动作鞠躬,该校顾问难为情地对社员挥挥手。再来就轮到泷了,后面的学长姐还没做出结论。这时要是明日香在就好了,无奈她现在人在舞台上。 

    「关西代表京都府立北宇治高中管乐社,泷升先生。」 

    泷闻言往前跨出一步,不知从哪个方向传来女学生夹杂着叹息的尖叫声:「哇!好帅。」学长姐还在你一言、我一语地没有定论。 

    你说啦!欸,说什么?什么都可以啦!就算你这么说…… 

    推来推去的过程中,泷已经接过奖牌。久美子提心吊胆地窥探学长姐的反应,丽奈冷不防站起,所有人的视线都集中在她身上。丽奈牢牢抓紧扶手,深深吸进一口气,腹部隆起,呼吸的声音传入耳中。只见她白皙的手指用力,张开樱桃小口。 

    「老师,我喜欢你!」 

    丽奈的声音响彻了整座音乐厅。绿辉掩住自己的嘴巴,双眼放光,叶月大吃一惊地拼命眨眼,泷也貌似吓了一跳,但随即换成平常的笑脸,往二楼的观众席挥手,行了一礼,走进后台。 

    「东京代表……」 

    继续颁奖给下一所学校的指挥,丽奈方才的告白仿佛没发生过,每个学校又开始各喊各的。 

    「丽奈?」 

    丽奈动弹不得,久美子扯了扯她的裙摆。丽奈的视线还定在舞台上,虚脱似地一屁股坐下,白皙的脸颊布满红晕,眼里浮现泪光。平常的冷静自持不知消失到哪里去了,唇畔流泄叹息,就连旁人也能察觉她的惊慌失措。 

    「怎、怎么办?久美子。我干了蠢事,居然在这种场合告白。」 

    丽奈说道,伸手就要遮住自己的脸。久美子赶紧抓住她的手腕。怎么回事,突然好想多看一会儿她现在的反应。 

    「完全不用担心。」 

    「可是……」 

    「再说,大家好像也不认为那是告白。」 

    「咦?」 

    久美子的安慰让丽奈双眼发直,身体也僵直。优子从背后慢条斯理地靠近。 

    「高坂,你刚才表现得太好了,谢谢你。」 

    「欸?呃,那是……」 

    「要是没有高坂同学,恐怕没有人敢对泷老师说任何话吧!真是谢谢你了。」 

    香织在背后笑咪咪地说。社员大概只当丽奈的告白是单纯的声援。丽奈目瞪口呆凝视着学姐的脸,又是放心、又是失望地摇摇头说:「不客气。」泷老师想必也没意识到丽奈这句话的真意。久美子望着舞台想。 

    「总觉得又是失望,又是松了一口气。」丽奈说道,羞赧地抓抓脸颊。 

    「丽奈喜欢老师啊?」 

    绿辉小声对久美子咬耳朵,叶月轻轻顶了一下她的侧腹说:「现在不是讨论这件事的时候吧!」绿辉这才心不甘、情不愿地缩回去。 

    「接下来开始公布成绩。」 

    这句话让七嘴八舌的会场顿时归于寂静。丽奈绷紧脸上的肌肉,再次从扶手的空隙盯着舞台看。 

    「终于要开始了。」绿辉说道。久美子点头附议。 

    站在舞台上的男士隔着麦克风宣布:「一号,北陆代表义雁商业高中,铜奖。」 

    会场左手边的团体不约而同发出失望的叹息。公布成绩时,通常会在金奖前面加上GOLD的英文单字,以免与银奖搞混。 

    久美子不经意想起关西大赛时,如同秀大附中那么强的学校有时也会马失前蹄,像北宇治这种没没无名的学校也有可能突然脱颖而出,不知道会发生什么事正是比赛有趣之处,但绝对没有人喜欢这种等待宣判的时间。清良女中的学生站在北宇治旁边,闭上双眼,双手交握祷告。实力坚强的学校也有他们的烦恼吧,活在胜利被当成理所当然的世界,肯定让人喘不过气来。清良的社长泫然欲泣地凝望舞台,久美子看了她一眼,悄然叹息。 

    「五号,东京代表东京都立片敷高中,GOLD金奖!」 

    学生的欢呼声自一楼的观众席响起,身穿黑色燕尾服的学生你推我挤地抱成一团。「耶!」「太好了!」就连二楼的观众席也能听见他们的欢呼声,还在等待结果的社员脸色益发铁青。 

    成绩逐渐公布。久美子等人只能目不转睛地紧盯着舞台。站在台上的明日香也比平常紧张,反复把手贴在胸口深呼吸。感觉站在她前面的小笠原根本已经发抖了起来。 

    「九号,东关东代表,成合市立成合高中管乐社,银奖。」 

    舞台前方的学生反应很平静,对站在舞台上的代表报以掌声。穿着正式服装的学生恭敬地接下奖牌。终于轮到久美子他们了。久美子就连看着舞台的勇气也没有,双眼紧闭,抓住丽奈的手,白皙的喉头上下震荡。丽奈紧张地望向久美子。表情已不复见平日的从容。 

    「十号,关西代表,京都府立北宇治高中管乐社,铜奖。」 

    司仪的声音隔着麦克风四平八稳地传来。久美子的力量一口气从抓紧着丽奈的手中消失殆尽,无力下滑的掌心撞向冷冰冰的地板。丽奈缓缓吐出一口气,手指按着自己的额头。 

    「铜奖啊……真遗憾啊!」绿辉垂头丧气地撇着八字眉说。 

    叶月不知所措地看着她们,嘴巴一下子张开、一下子闭上。 

    「对呀!」久美子回答的声音沙哑到令她大吃一惊,紧紧闭上双眼,借此压下涌到喉咙口的灼热,只怕再多说一句话,眼泪就要掉下来了。 

    「十四号,九州代表,清良女子高中,GOLD金奖!」 

    清良女中的学生在一旁欢声雷动,欢天喜地的声浪狠狠撞击着久美子他们的脑袋。某联盟的人还在台上发言,但欢呼声太大,什么都听不见。已经有社员蹲下来哭了,周围的学生皆一脸复杂凝视哭泣的社员。久美子也想祝福取得金奖的学校,但始终无法完全排除嫉妒的情绪。 

    久美子偷看明日香站在舞台上的表情,只见她的态度十分坦然,抬头挺胸、神态自若地伫立在台上。 

    「铜奖啊!」 

    优子在背后失望地叹息。与回头张望的久美子互为对照,丽奈一动也不动。没想到是香织伸出手,细致的手指温柔地抚摸丽奈的头。 

    「这段日子辛苦你了。」 

    丽奈把头埋进自己的膝盖里,唇瓣滚出哽咽的啜泣声,透明的泪水夺眶而出,顺着脸颊滑落。 

    「对不起,学姐。」 

    香织也擦拭自己的眼角,又哭又笑地说:「没有什么好道歉的,我们能走到这一步,都是托丽奈的福。」 

    某联盟的男士还在台上进行闭幕的致辞。周围响起零零落落的掌声,大概是他的演说终于结束了。音乐厅内有满脸笑意的学生,也有哭花脸的学生。这就是现实。久美子用力握紧拳头。 

    「好不甘心呐。」 

    她的自言自语消散在观众的掌声里。不甘心。不甘心。强烈的情绪咕嘟咕嘟地冒着泡泡从胃底升起,久美子有生以来第一次感受到这种情绪。 

    「北宇治的同学,请过来这边集合!」 

    明日香干练的声音在广场上响起。公布完成绩后,大家迅速在刚才的广场上集合。还有些三年级在哭,其他人则拼命安慰她们。 

    「大家辛苦了!」 

    美知惠抱着好几个女学生大声说。她被学生称为军曹老师,避之唯恐不及,但是像这样跟社员在一起时,看起来就跟普通的母亲没两样。明日香轻轻拍了一下手,大家的视线都集中在她身上。小笠原与明日香一如往常站在北宇治的成员面前,泷静静在一旁伫立。丽奈粗鲁地揉了揉自己的眼角,抬起豁然开朗的脸,扎成马尾的黑发迎风摇曳。 

    小笠原往前跨出一步,一群正走向巴士的学生兴味盎然地看着他们。小笠原双手高举刚才拿到的奖杯,吸气,慢慢说:「我想大家也都看到了,我们拿到了铜奖,也得到评审的讲评,稍后再由泷老师在巴士上发表。」 

    比赛时,各评审委员都会在纸上写下评语交给各个团体,以作为面对下一场比赛的参考。 

    小笠原抱紧奖杯,清了清喉咙,眼眶又红又肿,可见她一直哭到刚才。 

    「凭良心说,我还以为或许会出现奇迹,因为过去一路顺利地过关斩将,说不定真能拿下金奖。可是想也知道,事情不可能这么顺利。」 

    大家听完小笠原这句话,各自浮现若有所思的表情。久美子偷瞥了丽奈一眼,只见她的背打得笔直,直勾勾地凝视前方。 

    「今天的演奏完全发挥了我们的实力,大家都全力以赴,没有人打混摸鱼。已经全力以赴了,只是其他人更厉害。我们三年级就要退休了,一、二年级还有明年。明年再来一次,届时要是能拿下金奖就好了。请各位继续加油。」 

    「是!」 

    「感谢各位一直跟着我这个不中用的社长走到现在,这一年虽然发生很多不开心的事,但有更多更开心的事。我其实满心惶恐,无数次怀疑自己真的没问题吗?可是……」 

    社长坚毅的声音说到一半开始哽咽。受到她的传染,学姐们吸鼻子的声音此起彼落。明日香将小笠原的脑袋拥入怀中,轻轻拍了拍。小笠原说:「抱歉,我说不下去了。」明日香回以苦笑。 

    「真拿你没办法,以下就由我代替这个爱哭鬼说完吧!」 

    半开玩笑的调侃让小笠原破涕为笑。明日香用指尖推推眼镜的红框,跟平常没两样,睥睨一切地环顾众人。 

    「简单一句话,大家辛苦了。」 

    其他人也回以同样的话:「你也辛苦了。」 

    「老实说,该说的话早在比赛前几乎都已经说完了,我没有特别想说的话。三年级即将退休,此后是二年级的天下。我现在只想知道下一任社长是谁,只有这点让我放心不下。」 

    这句话引来哄堂大笑。的确,明日香和小笠原的组合太令人印象深刻,完全无法想象明年将由什么人接任社长。 

    明日香回头看了泷一眼,视线落在他手中的奖牌上。 

    「泷老师来了以后,这个社团整个脱胎换骨。大家都变得好认真,不知不觉就连北宇治也跻身强校之林。不过辛苦的还在后面。我想从明年开始,各位就得面对各式各样的要求。因为一旦跨过某个门槛,周围就会习以为常提出更高的要求。或许会发生比现在更痛苦的事,也或许得不到相应的收获。 

    「可是,」明日香说到这里,停顿了半晌,双眼轮流扫过每一张社员的脸。「我不像晴香那么温柔,不会说『要是能拿下金奖就好了』这种话。别看我现在一副不当回事地嘻皮笑脸,实不相瞒,我超不甘心的。一年级和二年级想必不想再尝一次这种滋味吧?所以明年一定要拿下金奖!这是副社长最后的命令,听见了吗?」 

    「听见了!」 

    「很好,那我就拭目以待了。」 

    明日香提起嘴角,心满意足地咧嘴一笑。最后。这两个字眼在久美子的舌尖上滚动。不知怎地,除了砂砾般的触感以外,还有股凄怆的味道。 

    「大家往巴士移动。泷老师会在车上告诉各位明天以后要做什么,所以不要睡迷糊了,请仔细听。那么没事的人可以先上车了。」 

    明日香一声令下,众人开始移动。 

    「久美子,一起坐吧!」 

    丽奈朝自己走来,久美子二话不说地应允。 

    「太累了,可能会在车上睡着。」 

    「我也是,今天真的好累。」 

    两人边讨论边慢吞吞往前走。绿辉和叶月在离她们有一大段距离的前方正冲向巴士。梨子叮咛:「小心别摔跤了。」 

    「高坂同学。」 

    意外响起的声音,久美子和丽奈回头。定睛一看,泷正朝她们微笑。明日香不知怎地也站在他身边,脸上挂着乐不可支的笑容。 

    「啊,是的。」 

    认出是谁叫住她的瞬间,丽奈的脸染上一层红晕。泷拿着大会颁给他的奖牌,难为情地搔搔脸。 

    「颁发指挥奖的时候,谢谢你的声援。」 

    「不、不客气,这种小事完全不值得道谢。」 

    丽奈猛地把头摇成一个波浪鼓,几乎可以听见咚咚的鼓声。长长的发尾随着她的动作不停甩在久美子脸上。太开心了吧,丽奈完全没留意到这点。明日香看在眼里,拼命忍住更加放肆的笑意。久美子心想,那个人绝对在看戏,视线转回泷身上。 

    「老实说,我有点没自信,所以很高兴。」 

    「自信?什么自信?」久美子问道。 

    泷苦笑回答:「刚来这所学校的时候,我知道大家并不喜欢我,也担心自己的指导是不是太自以为是,会不会只是强迫学生接受自己想做的事……」 

    「才没有那回事!」 

    丽奈大声打断泷的话。泷一时惊讶地张大双眼,然后静静微笑。 

    「能听到学生这么说,我真的很幸福。」 

    「不只是我,大家肯定都很尊敬泷老师喔!或许起初的确有人忍不住抱怨,但现在大家都很感谢老师。对吧?久美子。」 

    「嗯,对。」 

    突然被问到,久美子连忙点头。丽奈依旧胀红着脸,缩小一步与泷的距离,杏眼圆睁到眼珠子随时都要掉下来,紧握住自己的水手服下摆,滔滔不绝地说:「我喜欢老师。之所以来念北宇治,也是因为老师在这里任教。所以,那个,我真的很喜欢老师。」 

    「能听到你这么说,身为老师真是死而无憾了,谢谢你。」泷爽朗地微笑。 

    他的反应让在一旁守护两人谈话的明日香与久美子互看一眼。老师显然没听懂。 

    然而,得到他的道谢,告白的当事人一脸幸福洋溢。算了,只要丽奈高兴就好了。久美子隔着玻璃望向音乐厅内,刚才还人声鼎沸的会场此刻已安静下来,方才的热烈好像是骗人的,让人怀疑刚才发生的一切该不会只是梦一场。 

    久美子的意识从他们的对话飘远,直到袖口被丽奈扯了一下,这才连忙拉回视线。丽奈凑近久美子的耳边,小声地说:「久美子,我去一下洗手间,你先上车。」 

    「啊,好,我知道了。」 

    曾几何时,他们已经聊完了。丽奈跑向音乐厅,剩下久美子不晓得该做什么好,总之先望向明日香再说。接收到她的视线,明日香看着泷。 

    「对了,有句话一定要转告两位。」 

    泷恍然想起似地击掌说道。往外套的口袋摸了半天,掏出一张纸。折得方方正正的纸没有一丝皱褶,从这种细节就能看出他一丝不苟的个性。 

    「两位知道进藤正和先生吗?他是今天的评审委员。」 

    呃,当然知道,还知道那个人是明日香学姐的父亲。久美子连忙咽下反射性冲到喉咙口的话。从泷的语气听来,明日香应该没告诉他,既然如此,就不该由久美子的口中透露。明日香瞥了默不作声的久美子一眼,若无其事地说:「进藤先生是上低音号的演奏者,对吧?」 

    「没错,你很清楚嘛!」 

    泷摊开那张纸给她们看。似曾相识的笔迹在纸上写了一个英文字。明日香皱眉。 

    「……C吗?」 

    C是铜奖的意思。明日香学姐的父亲显然是个公私分明的人,给自己女儿的学校的评价啊!这是久美子直率的想法,但也因此愉愉地对公归公、私归私的进藤产生好感。 

    泷低头看着那张纸,悄然叹息。 

    「只有C评价确实很遗憾,但我刚才在走廊上偶然遇到进藤先生了。」 

    「欸,是吗?」 

    久美子太过惊讶,声音都分岔了。不知泷如何看待她的反应,只见他莞尔一笑。 

    「这么惊讶,难道黄前同学是进藤先生的粉丝吗?」 

    「啊,对,从小学生的时候就是。」 

    「他是很杰出的演奏家嘛,最好现场听一次专业的演奏喔,对自己的演奏会很有帮助。啊,这么说来,听说市民中心这次邀请到欧洲的乐团……」 

    「老师,话题扯远了。」 

    明日香的提醒让泷回过神,难为情地抓着头说:「不好意思,刚刚讲到哪里?」 

    「讲到老师在走廊上遇到进藤先生。」久美子帮忙补充。 

    「对对对。」泷猛然想起似地点头附和。「他要我传话给吹上低音号的学生。」 

    「传话?」 

    明日香微微挑眉,平常的扑克脸有些松动,脸上闪过一丝失措。不知泷是否留意到这点,语气四平八稳接着说:「他说,『难为你一直努力到现在,很美的音色。』能让评审委员赞赏到这种地步可是很不容易喔,要心存感激。」 

    明日香倒抽一口气,双眸在透明的镜片后面闪闪烁烁。 

    「是嘛!」 

    明日香一字一句地喃喃低语,静静垂下眼帘,长长的睫毛撞击在镜片上。这句话能让她的努力有所回报吗?久美子抬头仰望明日香的脸,企图探索她的心情。或许是察觉到久美子的视线,明日香苦笑着摸摸久美子的头。 

    「太好了,上低音号受到夸奖了。」 

    「啊,对呀!好开心。」 

    「我也很开心。」 

    明日香开怀大笑,露出洁白的牙齿。久美子从未在她脸上看到过这么单纯的笑容。是太高兴了吗?明日香一直摸她的头。难不成她不好意思表现出发自内心的喜悦,想借这种方式蒙混过去吗?久美子的脑袋被粗鲁地摇来摇去,却没阻止学姐。泷满脸笑意看着她们。 

    「抱歉,让你久等了。」 

    丽奈挥着手从音乐厅的入口朝他们跑过来。久美子也朝她挥手,明日香倏地放开她的头。久美子不经意回头看,只见明日香始终低头看着自己。视线交会,久美子仿佛被下了定身咒,当场呆住。于是明日香突然缩短彼此间的距离。她那姣好的五官一下子近在眼前,久美子下意识闭上双眼。明日香的气息撩拨着久美子的耳朵,好痒。明日香在久美子的耳边轻声说道:「久美子喜欢上低音号吗?」 

    莫名其妙的问题,久美子几乎是不假思索地回答:「喜欢。」 

    或许是回答得太快,明日香顿时惊讶得双眼圆睁,毫无防备地露在制服外面的喉咙愉快地上下震动。明日香乐不可支眯起眼,一拳捶在久美子的肩膀上。路灯的光线在没有涂指甲油的粉红色指甲上跳动。 

    「呵呵,我也是。」 

    明日香笑得有如年幼的孩子。 

    \setcounter{secnumdepth}{-2}
    \section{尾声}
    大批学生聚集在体育馆里,墙上挂着红白布幕,后方摆满了折叠椅,打扮得花枝招展的家长正面带微笑,望着舞台。 

    毕业典礼 

    舞台上的字让久美子悄悄垂下眼,怀中的上低音号百无聊赖地看着自己,曲线光滑的表面倒映出学生站成一排的身影。略短的裙摆随风轻飘,露出白皙的小腿。鲜明的蓝白对比象征着青春无敌。她们今天就要告别这身熟悉的制服了。 

    「校歌齐唱。」 

    训导主任的声音隔着麦克风从台上传来。管乐社的成员静静在体育馆后方等待自己上场的时机。三年级退休后,管乐社的社员人数一口气少了很多。冬去春来后,新生就会填满空缺吧!明日香已经不在久美子旁边,再也不会有人在她旁边吹奏那把银色的上低音号了。 

    「毕业生起立。」 

    三年级一起站起来。明日香就在那群学生当中,香织和小笠原也在。一想到这里,不知怎地,喉咙好热。 

    泷举起指挥棒,社员拿好乐器,看了熟悉的乐谱一眼。北宇治高中的校歌。开学典礼上,第一次听到管乐社演奏就是这首歌,当时的演奏真是糟透了。事到如今已经可以当成笑话来讲,但是这对当时的久美子可是大问题。没想到能变得这么厉害。看着其他社员的脸,不由得感慨良多。 

    指挥棒往下挥。吸气的声音通过号口溶解在空气里。乐器交织出带点古风、有点死板的旋律。毕业生的歌声乘着音乐而来。明明早已经历过几次毕业典礼,视野依旧不听使唤地起雾。以后就算来学校也见不到明日香了,这让久美子觉得好寂寞。 

    毕业典礼结束后,学生各怀心思地找人一起消磨时间。美式足球社与篮球社的送别非常夸张,依序举起毕业生往上抛。在校生与毕业生在中庭人来人往,闹哄哄的。久美子避开人潮,绕到体育馆后面。其他社员都去送三年级了吧!明知自己也得向明日香道别,却迟迟迈不开脚步。大概是害怕。害怕说再见。 

    「你在这里做什么?」 

    冷不防从头上传来一阵声音,久美子了吓一跳,抬起头来。她定睛一看,明日香正从逃生梯上走下来,胸口抱着装有毕业证书的黑色纸筒和陈旧的笔记本。 

    「学姐才是,来这种地方做什么?」 

    貌似亲卫队的女学生在中庭到处寻找明日香,还以为她早被抓住了。明日香苦笑着说:「呃,因为吱吱喳喳地实在太吵,我就逃出来了,发现久美子一个人在这里。」 

    「哦……」 

    找不到借口,久美子模棱两可地打哈哈。眼前的樱花树已经开出粉红色的花,从树上延伸出来的影子充满暖意,密不透风地覆盖住她们的身影。 

    「恭喜学姐毕业。」久美子好不容易挤出这句话。 

    「谢谢。」明日香不以为意地漫应一声,视线不经意停在久美子头上。 

    「你那个好可爱。」 

    「什么?」 

    明日香的手指轻轻滑过久美子的头发。她是指久美子别在耳朵上方的向日葵发夹。小小朵的向日葵,雪白的花瓣令人印象深刻。 

    「男朋友送的?」 

    「欸,呃,那个……对。」 

    久美子有点害臊,急忙按住发夹。明日香显然很享受她的反应,笑得一肚子坏水。 

    「别遮了、别遮了,感觉很青春。」 

    「不要取笑我啦!」 

    「哈哈,抱歉抱歉。」 

    久美子不依地噘起嘴,明日香随便摇手。「学姐好过分。」久美子正要抗议,却把下半句话吞回肚子里。这也是最后一次像这样拌嘴了。察觉到这一点,不禁黯然得说不出话。涌上心头的情绪哽在喉咙里。粉红色的樱花瓣散落在脚边。久美子凝视花瓣,静静等待情绪的浪头消退。 

    「傻孩子!」 

    久美子的脑袋轻挨了一拳,她吸口气,缓缓抬头,视线前方蒙上一层水气,明日香的脸看起来扭曲模糊。明日香傻眼地叹了一口气,递给她一本笔记本。那是一本有些历史的老旧笔记本。久美子还记得褪色的蓝色封面。 

    「这给你。」 

    明日香笑着说。久美子大吃一惊,眨了眨眼睛。 

    「可以吗?」 

    这不是令尊写的乐谱吗?明日香对她的反问不置可否。她的反应非常成熟,跟水手服格格不入。 

    「因为我已经不需要了。」 

    明日香不知所措地垂下眉尾。久美子硬生生吞下「你要放弃上低音号吗?」的问题。倘若那是她的决定,久美子无权阻止。 

    明日香给她的笔记本微微泛黄,刻画着岁月流逝的痕迹。明日香的父亲亲手写下乐谱,托付给女儿,如今又来到久美子手中。久美子颤抖着指尖接过。 

    翻开笔记本,页面上画满歪七扭八的五线谱,大概是明日香的父亲一笔一画亲手画的线。无数的小蝌蚪在用尺画的歪斜线条上快乐地游来游去。明日香父亲年轻时,是以什么心情写下这首歌呢?久美子的指头抚过粗糙的页面,挤出沙哑的声音:「我舍不得和学姐分开,好想多听几次学姐吹这首曲子。」 

    这句话让明日香有些动情,白皙的大手轻轻抚摸久美子的头。动作之温柔,害久美子险些哭出来。不要当我是小孩子啦!明明平常轻易就能说出口的玩笑话,唯有这天说不出来。 

    「下次由久美子吹给学弟妹听。」 

    自己是否能好好回答了呢? 

    「一直以来谢谢你了。」 

    预测到她下一句会说什么,久美子忙不迭摇头。 

    「我不想说再见。」 

    自己肯定是在闹孩子脾气吧!久美子这句话让明日香有些难为情地苦笑了。 

    「那就不说。」 

    她的手慢慢放开久美子。久美子扬起脸,粗鲁地揉揉眼睛。明日香往前走了几步,回头望向这边,黑发随她的动作轻晃。夹带着樱花瓣的春风轻柔抚过久美子的脸颊。 

    「改天见。」明日香说。 

    久美子点头如捣蒜地回答:「改天见!」 

    明日香离去后,周围突然变得好安静。久美子几乎在满出来的寂静里灭顶,当场滑坐下来。刚才的对话一点真实感也没有,久美子从肺部吐出所有空气,视线落在明日香交给她的笔记本上。 

    久美子在很多地方听过明日香吹这首曲子,也从她的演奏中得到许多勇气。下次轮到自己了。三年级毕业后,接下来会有一年级新生来报到,自己也将成为学姐。当学弟妹心情低落的时候—— 

    久美子慢慢地翻到笔记本的第一页,至此终于知道这首曲子的曲名。 

    「总算找到你了!中午开始合奏练习。」 

    声音从头上传来,久美子愕然抬头,丽奈正一脸傻眼朝她跑来,大概是特地来找她。久美子阖上笔记本,急忙站起来,拍拍裙子,另一只手对丽奈用力挥手。 

    「抱歉,我马上回去!」 

    久美子手中的笔记本里,有一行字规规矩矩罗列在五线谱上的空隙。明日香的父亲想必对自己吹奏的乐器引以为傲,而明日香应该也继承了父亲的心情,接下来轮到自己了。涌上心头的情绪该怎么形容才好?闪闪发光,却又让人心痒难耐。现在的她愿意付出一切努力。久美子压抑满腔激动,冲向好友身边。 

    嘴里珍而重之地默念那首曲子的曲名。 

    〈吹响吧!上低音号〉 

    久美子一辈子也忘不了那柔美的音色。   
\end{document}