\documentclass[UTF8]{ctexart}
\usepackage[perpage]{footmisc}
\usepackage{geometry}
\usepackage[bookmarksnumbered=true]{hyperref}
\CTEXoptions[today=old]
\CTEXsetup[name={第,章},number={\chinese{section}}]{section}
\geometry{a4paper,left=1.27cm,right=1.27cm,top=1.27cm,bottom=1.27cm}
\hypersetup{
colorlinks=true,
linkcolor=black
}
\pagestyle{plain}
\title{吹响吧!上低音号\\ \Large{外传\,欢迎来到立华高校舞奏队\,前篇}}

\date{August 28, 2017}
\author{武田綾乃}
\begin{document}
    \maketitle
    \tableofcontents
    \setcounter{secnumdepth}{-2}
    \section{序章}
    \setcounter{secnumdepth}{3}
    夕阳如火,那炽热的红光染红了梓的视野。抬起头,便见绀色的天空中溶入了月亮皎洁的白光。宇治川的潺潺流水,打破了寂静。面前的友人,带着一副些许困扰的表情看着梓,她那茶色的头发轻轻摇曳。

    「久美子,你为什么没去南宇治高中?」

    面对梓的提问,黄前久美子的眼中闪过一丝动摇。梓默默地看着久美子手中握着的抹茶味雪糕融化滴落到河岸上。

    「我倒是想问,为什么梓会觉得我会去南宇治呢?」

    久美子的语气相比往常重了点。到了五月依旧稍带寒气的风从两人之间吹过。梓没理会被吹起来的裙子,一步步往前走着。

    「因为北中的人大部分都是去南宇治的嘛,所以我以为久美子你也去了。」

    停下脚步,梓看向久美子。一直以来,久美子都有个癖好,那就是当她想糊弄过去的时候,都会做出少许困扰般眉头低下,口角微微放松,露出一副笑容的表情。

    「选学校也有很多因素啦,成绩啊离家距离啊之类的。」

    「不过南宇治离你家更近吧,而且成绩也差不多。」

    「嘛,或许吧」

    仅此一句后,久美子便闭上了嘴。

    是不是说得太多了呢?梓把视线往下移到胸部。鲜明的水色外套,沉着稳重的百褶裙,然后是从外套中隐约看到的白色衬衫,在领子附近还系着刚买的黑色缎带。这套制服梓从很久以前就幻想着自己能穿上。

    「……没特别的理由啦。」

    久美子轻轻说道。尔后她貌似害羞似的挠了挠头,接着用宛如对待易碎工艺品般慎重的口吻,从那粉色的舌头中吐出真心。

    「是因为想从头开始啦。」

    「想从头开始?」

    听到这不符合久美子性格的话,梓不禁歪头。

    久美子深吸一口气,她有那么一瞬间看向了夜晚的宇治川。樱色的嘴唇紧闭,然后她做出了决意,一下子抬起头,一口气说道:

    「只是想像一张白纸一样从新开始,所以才去了没什么熟人的北宇治啦。就仅此而已。」

    她如释重负一般呼了一口气。是觉得自己的话不好意思吗,久美子不断闪烁着目光。看到她这一如平常慌张的样子,梓露出了笑容。

    「嗯,这样啊。」

    自己的声音相比刚才多了一份轻快。久美子绀色的水手服进入了梓视野的一端,自不言喻,大家都不是中学生了。虽然身穿不同的校服,走着不同的道路,不过唯独音乐,将彼此联系。不知为何对此事实有一种麻麻的感觉,梓不禁微闭眼睛。

    「原来久美子也有好好想过了啊。」

    「想过是想过什么啦?」

    看着嘟起嘴的久美子,梓笑着说:

    「抱歉抱歉,我还以为大家去哪你就跟着去哪。」

    「久美子也终于有自己的想法了啊」,梓把这句话马上就要说出口的话,随着雪糕一并吞下。

    冰凉的感觉让梓抖了一下,而后为了赶走身体的颤抖,梓咻地把手举向空中,拉直的身体在地面投下歪曲的影子。

    「我们下次见面,是在大赛吧。」

    「是啊。」

    「虽然中学那时去不了……不过上了高中希望能进军全国啊。」

    自己的话语带着岩浆般的炽热。每走一步,通过鞋子传来地面坚实的触感。现在,自己便走在了通往全国的路上。从中学的桎梏中解放,自己也和久美子一样迎来了新的开始。

    ——对梓来说我是什么?

    在耳边苏醒的冰冷絮语,还有那批判自己的带着热度的视线,记忆深处那名少女的面庞,赫然浮现于眼前。为了拂去眼前的这张脸,梓用力往前迈出一步。久美子以温柔的微笑说:

    「梓的话能去全国啦,毕竟是立华的嘛。」

    这率直的语气,让梓抬起了嘴角。
    \section{暴走的行奏赛}

    80个人,光是以同样的角度,时间,方向踏出一步这种简单的动作,如果配合得当,也具有相当的迫力。梓往前探出身体,微微离开了椅子,还有一年,还有一年自己就能成为其中的一员了。紧握手中的册子,梓大吸一口气,力聚丹田,晃荡着悬空的双脚。即便如此也按捺不住自己的感情,所以梓紧闭嘴唇以免自己不禁笑出声。

    连衣裙的服装舞动着,那鲜明的水色充满了视野。到处都有女生发出热烈的欢呼。金色的乐器不约而同朝向同一方向,手持水色旗子的少女踏着欢快的步伐。乐鼓低鸣,奏出熟悉的旋律。大家的期待一同爆发,因兴奋而流露的叹息,一瞬间席卷全场。

    然后那故意制造出间隔的声音,从喇叭中迸发,奏出的旋律是再熟悉不过的乐曲——《sing·sing·sing》,立华高校的拿手好戏。少女们面露微笑欢跳舞动,那激烈的舞步并没有剥削她们的笑容。从翻动的裙子下露出的洁白大腿和黑色的及膝袜形成鲜明的对比,她们每一次动作都带来视觉的冲击。

    一丝不乱的动作不愧是强校的高难度表演,她们以游刃有余的样子轻松做出高难度的动作。

    「果然是水色的恶魔!」

    究竟是谁发出的感慨呢?

    「HI!」

    以此声音为终止符,她们的表演结束了。随之的是观众的热烈掌声。梓不惜疼痛大力拍手,看着她们再一次确信道——果然立华是最强的。

    ——————

    妈妈做的蛋卷比平时的要咸。在3、4节课间的十分钟,在别人都在谈东谈西时,梓就以习惯的动作把便当的东西塞入口中。带汁的肉丸和只用微波炉叮了一下的意面梓一下子就搞定了,接着把沾上鲑鱼片粉色的饭扒入口中,梓终于长舒一口气。

    刚上高中那时还不习惯这么早吃午饭,不过两个月后便习惯这种节奏了。

    「呼,饱了。」

    旁边和自己一样一心不乱咬着面包的少女如此自语。这名长相天真的女生叫名濑雨未华,同是吹奏乐部一年级生,乐器同是长号。

    注意到了自己的目光,她用水灵灵的眼睛看过来,短刘海随之摆动。

    「小梓小梓~」

    「嗯?」

    「今天我该练什么好呢?」

    雨未华微微侧头,松软的长发滑过她的脸,瞥了眼手边的便当盒。

    「我午练时想想等下再告诉你。」

    「好,一直都麻烦你了。」

    她露出笑容说道。

    她略显成熟的左脸有一颗小小的黑痣。把手指的面包屑舔干净,雨未华说:

    「最近终于觉得练习有趣了,长号的件位也总算记住了。」

    雨未华的包上用蓝色绳子系着长号的挂件。长号是以两根U型管组合在一起为特征的金属乐器,以滑动伸拉管改变音程。长号又分为几种。

    梓的是高低音长号,其他的还有专攻高音低音的高音长号和低音长号。部员们使用的类型,则是根据各自担任的职位决定的。眼前这位女生同样是高低音长号。

    「快上课了,赶快吃。」

    铃声响起,同时数学老师也走进了课室。吹奏乐部的人赶紧收拾掉桌子上的食物,梓也把空便当盒用布袋装好。红色的布袋上印着几只开心地跳起来的兔子。

    把东西放入包时,雨未华依旧一副笑容看着老师呆呆地坐着。

    「上课准备呢?」

    「啊,忘记了!」

    经梓的提醒,她才慌忙反应过来。对此,梓既是惊讶又是好笑。

    梓和雨未华的第一次见面得回到两个月前。梓因吹奏乐部的推荐升上了立华,对梓来说,不,对属吹奏乐部的中学生来说,立华是一间特别的学校。

    立华是京都市的私立学校,作为排球和足球强校为人所知,不过作为吹奏强校的名气也不低。因此许多学生为了加入这间学校的吹奏乐部特意搬来京都。

    100多名部员里夹杂着非关西口音,而雨未华便是从东京搬来的其中一人。

    「我以前没接触过乐器。」

    在迎新会上梓恰好和雨未华坐在了一起,因是同班的,所以梓向她搭话,然后她如此说告诉自己是新手。

    「这样啊,那为什么你想加入吹奏乐部?」

    「在电视看到吹奏练习的样子有点吃惊,就想看下我也能不能做到呢?所以就来了。」

    「是星期天的那档节目吗?」

    「对对!超级帅的!」

    雨未华合掌开心地说。垂落到胸口的松软长发拉成弧线,更一步衬托出她的少女味。不自觉地梓用手指穿过自己的高位马尾,皮肤滑过硬硬的触感。

    「那个我也看了,舞奏赛真的很好看!」

    舞奏赛,全称全日本舞奏大赛。为纪念全日本吹奏乐联盟成立50周年,1988年在神户Port Island会场举办了「第一届全日本舞奏节」,以此为前身,在17届后改名为「全日本舞奏大赛」。规则和全日本演奏大赛一样,通过地区大会和都道府县大会,再经支部大会,最后才能进军全国大赛。

    「我不是很清楚舞奏是什么,和坐着有什么不同吗?」

    梓抱起手对这个问题开始了思考。音乐室的墙上贴满了奖状,全日本吹奏大赛的,全日本舞奏大赛的,京都大会的,关西大会的,全国大会的。看着这些以理所应当的样子贴在墙上的奖状,梓的胸口一阵发热。

    这就是强校,全国这样的字眼并非只会出现在梦中。

    「舞奏简单来说就是边跳边演奏。交响乐上大家都是坐着吹吧,而舞奏就是站起来,一边做动作一边吹。」

    「这样啊,那没有舞奏的学校就不像电视里的立华那样咯?」

    「没错,况且就算是舞奏,在全国也找不出能像立华这样夸张的了。说实话立华就是个异类,动作编排太个性了。」

    「原来是这样,我都不知道。」

    她眨着大大的双眼皮眼睛。听到没接触过乐器的人这么直率自白,梓不禁苦笑。

    往后靠在椅背上,椅子发出了金属声。阳光从春风吹拂淡绿色窗帘形成的缝隙中射进来,温柔照在了雨未华的脸上。

    「小梓真是万事通。虽然我是新人什么都不懂,不过有小梓感觉可以努力一下。」

    从害羞笑着的嘴唇中可以窥见结洁白的牙齿,雨未华轻柔的声音逗弄着梓的自尊心,看着她那天真的笑脸,梓也露出了笑容。脸上那一阵发热,就是自己害羞的证明了吧。

    「谢谢你的夸奖。」

    就在梓挠着头的时候,门被拉开了。缓缓流入室内的空气,顿时化作了紧张。30名新人的目光齐刷刷看向前方。而在那里的是被称为干部的三个人。

    「大家好。」

    最先开口的是中间的女生。听到她的招呼,大家也一同回应。

    她挺直的背板是比梓高了几厘米呢?长长的黑发绑成团子盘在了脑后,一眼看上去以为是运动部的她身穿着印有「一音入魂」字样的衣服。她挽起衣袖,环视了一眼室内后,露出了亲切的微笑。

    「我叫森冈翔子,立华吹奏乐部第36任部长。和我一起练习的一年级生可能已经有人认识我了,不过有很多人是第一次见面。不过今后你们有很多机会可以见到我,所以请记清楚我的脸。我的要求只有一点,要确实向前辈报告联络和商量。说实话,立华的训练是很严格的,不过只要一步步熬过去,就肯定能坚持到最后。去年的新人没有一个掉队的,我希望今年也一样」

    说完后翔子行了一礼,大家鼓掌。

    旁边的雨未华松了口气。

    「部长好像很好人。」

    对此梓只是「嗯」地敷衍了过去。梓是推荐入学的,所以在入学式前就已经参加练习了。翔子就是那时候认识的。翔子出身于京都吹奏强校圣女中学,乐器是圆号,和梓一样是推荐入学的。

    性格潇洒同时不失对他人的关心,是一个极富人格魅力的人。所以用「好人」这个普通的形容词来形容她是否合适,梓自己也说不清。

    拍手停后,翔子左边的人往前一步。把发量多的头发高高扎在两旁,形象和翔子相比显得比较时尚。

    「我是副部长小山桃花,乐器是大管,同时负责舞奏的挥旗手训练。我就首先说了,我桃花最讨厌的就是干的不好还不努力的人,所以今后大家哭的时候也不会少,也许会后悔来吹奏乐部。不过只要坚持道最后,我能保证你们能体会到至今都没有过的经验。所以在尝到甜头之前,请给我咬牙坚持」

    拍着手,梓舔了舔干燥的嘴唇。小山桃花,是为了来立华而特意从关东和她母亲一起搬过来的三年级生。因为好胜的性格和冲人言行而为人所知。因为她是挥旗手的指导,所以可以预想训练也是斯巴达式的。对雨未华小声嘀咕「好可怕」,梓举双手赞成。

    最后的是部长右侧的女生。她刚走到前方,教室里就被刺人寂静包围了。细长的眼眸中倒映着新人们不安的神情,把头发别到耳后,她面无表情地说:

    「我是神田南,三年级,担任领队,鼓手。」

    嘟起的嘴唇吐出冷淡的声音,那震动空气的女低音让新人抖了三抖。

    「舞奏的指导主要是我和二年级的领队负责。现在立华吹奏乐部有130名部员,能出场舞奏比赛的是80人,就是说有人会从编成成员中剔除。」

    她若无其事说出的话让梓吞了口口水。才刚入部严酷的事实就摆在了面前。

    「我就先说了,我们学校的编成成员不是固定的,就算京都大会时能出场,如果之后候补的人表现得更好,那全国大会时人员就会变更。我们的目标是全国舞奏大赛的金奖,所以出场的人必须是最好的。绝不允许无聊的争端和私人感情浪费时间,这点请大家牢牢记住。」

    「是!」

    大家不约而同发自内心的回应,一瞬间席卷了教室。对此满足般,南无言点了点头。在大家都身穿有图案的T恤时只有她穿的衣服是纯黑的。透过薄薄的布料,可以看到南苗条的身材。小腹呈现出S曲线,从衣襟处可以窥见锁骨。

    她就是神田南吗?梓眯着眼看着她。

    在私底下大家都叫她魔鬼领队的同时她也是吹奏乐部的核心,从她没有温度的眼中,窥探不出任何感情。

    不愧为强校,三名干部齐聚一堂就有如此气势。虽然梓的中学北中也是强校,不过同年的女生们一次都没有表现出如此的压迫感。是该说人生经验的差距呢?还是说个人的气质呢?不管怎样,她们都不是容易亲近的人。

    部长翔子走出一步,在穿旧的鞋子上有用圆圆的字体印的她的名字,旁边的「do」音符,才终于带有这个年纪少女应有的感觉。

    「接下来,开始部内会议。」

    听到午休的铃声响起,梓才回过神来。其他学生才把便当拿上来,吹奏乐部的人就一同起身匆匆离开了课室。梓把笔盒放进抽屉后,也急忙站了起来。梓拉了一把旁边眼困似的揉了揉眼睛雨未华懒懒的手,赶忙追上那些已经不在教室的部员。

    「哇,刚才完全没听课。」

    「我也是,刚才一直在发呆。」

    「小梓看起来很困呢,啊,难道说你刚才睡着了?上课睡觉可不行哦。」

    「我才不想被刚才还打呼噜的你说。」

    「骗人!你说真的?」

    「假的。」

    看见梓轻描淡写带过,雨未华鼓起了脸。看到走廊的墙上已经贴不牢的海报上那几个用黄色粗体写着的「严禁奔跑」的字,梓和雨未华以走和跑中间的速度走向音乐室。

    「早啊同志们,还是这么精神呢。」

    「不如说梓和雨未华你们两个太精神了,在6班都听见你们的声音了。」

    听到后方传来的活泼声音,梓马上回头。在那里的是最近变成吹奏乐部吉祥物的双胞胎西条姐妹。她们都留着黑色短发,而且还特意穿着同样的开襟毛衣,甚至还穿着同样的鞋。一开始根本无法区分这对同卵双胞胎,不过熟了之后一眼就能分出来了。

    「早啊,花音美音。」

    中午都过了,雨未华还是说着早晨的问候。

    「Good morning,雨未华。」

    语气活泼的是姐姐花音。

    「都中午了还morning……」

    冷静吐槽的是妹妹美音。

    区别的方法就是看她们的眼睛。眼角微微上翘的是花音,下垂的是美音。

    她们以前的中学是琦玉的强校,为了来立华而特意从琦玉搬到京都。花音是长笛,美音是双簧管,在舞奏的时候她们是旗手。

    旗手指的是在舞奏时利用旗子和其他道具进行视觉表演的人。立华里有人担任多种乐器,同样的有人也同时担任旗手和吹奏,而角色是根据舞台要求决定。

    「啊,是翔子部长」

    随着美音的嘀咕,四人都呆住了。看到穿着制服单手拿着乐谱的翔子从走廊对面走过来,四人都60度低头问好。

    「部长好。」

    「嗯,你们好,午练加油。」

    听到部长轻轻的声音从头上传来,在听到脚步声远离后,梓她们才抬起头来。雨未华呼了一口气。

    「还是不习惯向前辈问好啊。」

    「现在不是说这种话的时候。不赶快午休都要结束了。」

    梓一如往常用力拉了拉呆站着的雨未华的手。仿佛这很好笑似的,花音美音相视而笑。

    「梓真的很喜欢照顾人呢。」

    「好像妈妈哦。」

    「不是我爱照顾人,是雨未华爱遭人照顾。经常发呆放着她总感觉不放心。」

    梓不自觉用食指抵着自己的下巴说道。雨未华听后,双手遮住口呼呼地笑道:

    「梓妈妈,一直都让你担心了。」

    「你也别跟着演戏。」

    「嗷。」

    轻轻地敲了她的头,雨未华就夸张地后仰身体。梓的手上,留下了她头发软软的触感,感觉痒痒的。

    ————————

    开学两个月以来,梓就没几次在自己的桌子上吃过饭,而理由也很单纯。

    「动作配合好。」

    「是。」

    「那里的动作乱了,要按五米八步来。」

    「是。」

    「手低了,没拿乐器就这样,拿了乐器时你要怎么办。」

    「是。」

    「再来一次。」

    「是。」

    对杏奈前辈的每一次严厉的批评,一年级的新人每次都重复同一个回答。在校舍后面的地面上,拉着上面有等间距标志的线。这是为了让大家习惯5米8步,简称5米8的动作。

    对于立华的吹奏乐部的人来说,午休是社团训练的一部分。为了充分利用短短的30分钟,大家都在第四节课开始前就吃完午饭了。然后在下课铃响的同时就冲出课室,赶紧换上体育服赶向训练的地方

    「一、二、三、四。」

    大家配合着杏奈的声音迈出脚步,双手模仿拿乐器的样子放在嘴边。挺直背,面向正面,大家以每步相同的间隔前进。

    对于行奏,步幅是关键所在。因为人各有异,身形也不尽相同。比如梓旁边的两名同是长号的一年级生。户川志保和的场太一。志保出身西中,中学是吹奏乐部,身高165cm。然后同是西中同样是经验者的太一,身高158cm,然后他本人也是对自己的身高颇有说辞。

    157cm的梓如果和152cm的雨未华站在一起,那就一高一低了。如果大家都按自己的节奏走,动作肯定是不齐的,所以便制定了五米八步的规则。

    五米八步,一步62.5cm,在行奏开始前,要好好把这个节奏刻进脑中。脚后跟先着地,接着脚尖要像旋转似(role)的往地面踢。这种优美的走法可以让上半身保持不动。

    对于新人来说,一边看着别人的动作一边注意自己的细节是相当难的事。如此这般精神的消耗也不少,不自不觉间动作多了份迟钝,而这部分前辈们当然不会看漏,都听到好几次责骂了。

    「好,还有五分钟,午练结束。大家辛苦了。」

    「谢谢前辈的指导。」

    大家一同向指导的杏奈前辈鞠躬。梓用手背擦去汗滴,明明离夏天还有一段时间,天气就这么热了。

    「刚才吃的猪排饭都反胃了,肚子好痛。」

    按着右边的肚子,太一蹲了下去。他身穿的红色运动服貌似是毕业前辈的。瞥了一眼,梓用指尖抓住了自己的绿色衣服。立华的运动服是一年级绿色,二年级蓝色,三年级红色。然后各自的衣服都绣有自己的名字,不然一借都搞乱了

    「哇,你别吐在这里哦,要吐去厕所。」

    听到梓的话,太一更是一脸扭曲沉下脸。从他满是汗的领口处,可以看到少年向青年转变期特有的苗条颈部曲线。再加上他这样的身材,更加显有少女特有的纤细感,他本人倒是很在意这点啦

    「佐佐木你好像只对我特别刻薄啊。」

    旁边的志保听到这个,一副无奈的样子叹了口气

    「都是你整天被前辈训,所以梓对你的态度也差不多啦。」

    「哈?我做了什么了吗?」

    「练习的时候你都没认真。我就说了,我可看的一清二楚。」

    「因为练习很辛苦啊,放下水又不怎样,我这种人不偷下懒就会死。」

    「哇,你还破罐破摔了。」

    看到这两人又开始了,梓和雨未华互相看了看。因为这两人中学一样,所以入部的时候就认识了。

    「再说了又不是只有你自己一个人辛苦。」

    志保说得太快喘着气,用手指推了推无框眼镜。她细长的黑发,从正中间分开扎在两边,她有着一张优等生的认真脸庞。

    「反正我都会了有什么关系,我天赋好,完全没问题。」

    雨未华对太一笑嘻嘻说的这话点了点头。

    「的场君确实学得很快。」

    「就是这点你才讨厌,如果你干得不好我还能说说你。」

    志保嘟起嘴说。然后二年级的杏奈拍了拍太一的肩。

    「你别得意忘形,你这种小把戏别的前辈一下子就看穿了。」

    「啊,对不起。」

    「知道就好。」

    就算对同级生振振有理的太一,面对前辈也不得不低头。因为在吹奏乐部,那种像体育部一样的风气已经深深根植于其中了,所以在这种氛围中,太一也自然受到了影响

    长号声部有十三人。三年级的四人,二年级五人,一年级四人。桥本杏奈是二年级的,同时也是长号声部的指导员。在齐刘海的黑发上,别着就像真的鸡蛋烧一样的发饰。虽她喜好食物造型的发饰可以回溯到她中学时代,可当时食物类的发饰也就只有带着糖果饰品的橡皮筋,还没有她现在戴的这么精美的出现。

    梓偷偷看着面前这个人。杏奈同样和梓一样是北中的,中学在吹奏乐部的长号声部里也多次受到她的照顾。在梓的记忆中,她有着一张仍留有天真的脸,从她侧面脸的线条,可以看出她温柔的性格。

    杏奈毕业不过一年,然而她脸上紧致的线条和稍微能看出自我的双眸,已然没有了当初的天真胆小模样了。在立华的一年里她到底经历了什么才变成这样了呢?梓不得而知。

    「啊,时间快没了。」

    看了看手表,志保慌张地说。墙上的钟,秒针无声地向前走。看到第五节课快开始了,部员为了换掉满身汗的体育服,赶紧跑了起来。

    —————————

    「哦,今天是去音乐室啊,lucky。」

    梓开心地说着,旁边雨未华正整理着文件夹里的基础练习乐谱。

    放学后的第一件事就是确认贴在黑板上的地点分配表,在音乐室,中庭,二楼走廊这些地点上标着各声部的名字。只要看这个,就可以知道每天都变化的练习地点。

    在立华,吹奏乐部能练习的地方不多。操场是足球和田径的,体育馆是排球和篮球的,教室是补习学生的,所以为了不打扰别人的学习,吹奏乐部只好选择不会扰民的地点。

    「小梓,今天我该练什么。」

    听到雨未华每天例行的问题,梓从自己的包里拿出笔记本。一翻开,就被眼前满满的日程吓了一跳。

    在立华吹奏乐部几乎每周都有活动,在中学时想都不敢想。前天本地一年一度的sunlight festival才结束,马上的明天又发下了新活动的乐谱。

    「恩~,三周后是中学生来参观的日子,曲子是sing呢。雨未华你位置方面还不稳定,今天你就先练长音,然后再练一下位置,最后是跟着sing一起来这样。」

    「好,我知道了。」

    看到雨未华对梓的安排百般顺从,旁边的杏奈佩服地笑着说:

    「有梓在,雨未华就以一顶百。」

    「没错!」

    看到雨未华毫不犹豫,梓有点脸红。

    「没有啦。」

    「不过有梓你在真的帮大忙了。毕竟雨未华在我们部里是罕见的新手,作为前辈的我也必须要好好指导她,何奈我能力还不够。不过梓你从中学开始就很强没问题,当雨未华老师可以说是刚刚好。把雨未华交给你就放心了。」

    「没这回事,还差的远。」

    「又谦虚了。其他前辈也说咯,要是大意A组的名额可要被梓你给拿去了。」

    在杏奈开玩笑的语气里,可以隐约感觉到那刺人的紧张。梓露出亲切的微笑,大力摇手。

    「整天被前辈这么夸怪不好意思的。我是知道自己能力不够,所以才为了能超过前辈努力练习而已。」

    「说的也是……你的目标是超过未来前辈吗。」

    听到她的口中迸出未来前辈的字眼,梓感觉脊髓有一股电流窜过。她那挺直的背影,在眼前浮现而后褪去。

    梓吸了一口气,一边希望对方不要察觉到接下来的话就是自己的真心,一边以开玩笑的语气微笑着说:

    「我是有这种打算啦。」

    「哦,小姑娘不错喔,那我也不能输给你了。」

    轻轻拍了拍梓的后背,杏奈从梓旁边走过去了乐器室。

    梓无法判断刚才自己的话杏奈是到底怎么想的,是觉得梓是认真的呢,还是纯粹在开玩笑。三年级的还在上课,所以现在没三年级的在。

    「小梓好厉害,前辈们都说小梓好厉害哦,我也觉得小梓仅次于未来前辈。在我们声部里,小梓绝对是第二位。」

    雨未华紧紧抓住了梓的手。和梓的手比起来,雨未华的手圆圆的软软的。那像婴儿般QQ的手指,滑过梓干燥的皮肤。

    「哈哈,听你这么说很高兴,不过记得别在其他前辈面前这样说哦。」

    「为什么。」

    「不为什么。知道了吗。」

    以别有意味的眼神看着雨未华的眼睛后,她抿紧嘴唇点了点头。在她衬衣的领口处,绑着学校指定的黑色缎带,看着那随着雨未华摇晃的带子,梓不意想到了颈圈。

    ——————

    一走进连在音乐室旁边的乐器室,就微微闻到了灰尘的气味。乐器的箱子就紧密放在奶油色的乐器架上。梓从最靠近自己的架子里抽出自己的乐器箱。

    透露着黑光的箱子是依照着长号形状的长方形。银色的卡扣反射着灯光,用手指一碰,便留下了新月形状的指纹。

    梓旁边的位置是空的,看来志保和太一已经在练习了

    「梓你的好帅啊,这就是私人的?」

    雨未华兴趣满满看过来,在圆弧的喇叭表面映出了雨未华被拉伸的脸。

    「嗯,中学时买的。雨未华你什么时候买。」

    「下周,爸妈终于答应买给我了。未来前辈也帮我选了。不过长号好贵呢,妈妈看了吓一跳。」

    「是很贵,乐器都不便宜,不过也物应所值。」

    把盖子完全打开后,就看到了那金色的乐器。拿起后,把拉管和本体用金属扣扣住,之后把银色的吹嘴插入,组装就完成了。

    回到音乐室,梓往吹嘴吹入一口气,从嘴唇喷出的气流快速地通过乐器管,发出了轰鸣的声音。伸展右手,慢慢滑动拉管,声音也随之拉长,透过空气传到了手上震动,梓很是熟悉。

    挺直背,盯着一点,然后想象把声音投向那一点,梓用力吹气。从像牵牛花形状的喇叭中传出了宏朗的声音,乐器也因为声音微微振动。梓不断吹着长音,一遍遍地从低音过渡到高音。

    旁边的雨未华在组装乐谱架,把粉色的罩子拿开,就可以看到架子黑色的本体。雨未华用笨拙的动作把支架扭松,然后放到木质地板上。把乐谱放好后,才开始发声练习。

    瞥了眼她认真的样子,梓看着自己全新的乐谱,突然想起了志保一个月前问自己的问题。

    「梓你对新人是怎么想的?」

    那是万里无云的一天。樱花树上还残留些许春天的气息,树下,已经干枯变色的樱花瓣堆了薄薄的一层。四月已经来到了尾声,距离Sunfes(sunrise festival简称)的日子也渐渐近了。

    在空教室的一端,志保靠在窗框上。那天练习结束后,梓要做的就是和等雨未华收拾好乐器,然后和她一起回家而已。

    志保透过厚厚的镜片看着自己。她及肩的黑发绑在了两边,额头上渗出了汗珠。她没有修剪过的眉毛吊起,她现在的表情就是平常她有重要事要说时的样子。

    「你是说雨未华?」

    梓这么一问,志保显得有些尴尬。她以僵硬的动作用手指搓着深绿色的运动服。梓像是要擦干手汗一样抓住了自己衣服的下摆,不断用衣服扇着风。

    「就是雨未华。」

    她轻轻点头。她的下巴比起其他女生显得要小些。志保舔了舔干燥的嘴唇。

    「我先说了,我没有多余的精力去指导她,光是自己就管不过来了。虽然这样的说法听起来很自私,不过我真的没有余力去照顾她。」

    「才不会认为你自私啦,毕竟立华的练习不轻松。」

    听到自己否认了她自己自虐的说辞,志保明显松了一口气。

    「我读西中那时没有行奏,所以来了立华后每天都有一堆不懂的东西。所以雨未华来请教的时候,我都不知怎么面对她。因为其他声部都没有新人,所以听到其他人在讨论一些我不懂的东西之后雨未华来问我一些基本的问题,怎么说,就觉得很着急,因为我自己都跟不上别人了,哪还有时间一一回答这些简单的事情。说实话,我觉得很烦。」

    立华是吹奏的强校,所以入部的人都是有经验的。当然也有一些新人,但最多就一个一年纪一个。对于做事认真的志保来说,辅导新人的雨未华负担确实不小。

    梓闭上眼吸了一口气,为了不让志保在意,于是笑着说:

    「我知道了,之后雨未华的辅导就全部由我来吧,我不讨厌教人。所以没问题。」

    「真的?」

    「嗯,所以志保你就别在意专心自己的练习就好,雨未华的事情你就不用担心了,我会把她培养成合格的部员的。」

    梓看着志保,眼神里说着「之后的事你就别担心了」。志保听着梓故意装作轻松的语气,很是烦恼地陷入了思考。有那么一瞬间,她的表情因自己过高的责任感而扭曲,她的眼中闪过了因为太过有责任感,所以在将雨未华辅导的责任转嫁给他人时对自身的厌恶。

    想必她相当讨厌抛弃雨未华的自己吧。对于她这份可以说是到精神洁癖程度的近乎于愚蠢的诚实,梓倒是挺欣赏的。

    「……抱歉,麻烦你了。」

    从志保干燥的唇间发出的声音,听得出她挣扎的感情。梓心想我并非想让你觉得对不起我才这样说的,同时也痛恨自己表达能力低下,导致对方不明白自己真正的想法。

    「前辈好。」

    听到周围吵闹起来,梓从过去的记忆中回到现在。抬起头,就见下课的三年级生们齐刷刷走了进来。看到那身穿水色制服的波浪涌来,梓停下了演奏的手。

    「大家好。」

    前辈们机械性回应后辈的招呼。看到其中有自己认识的人,梓不自觉挺直背。

    从长度不到后颈的黑发中可以看到白皙的耳朵,在分刘海之下的是杏仁形状的眼睛,其中带着锐气。

    在梓认识到她就是濑崎未来之前,就有人向她问好了。那光泽的薄嘴唇露出害羞的笑容,修长的手指无意识摆弄着自己的头发,头发缠上了她洁白无瑕食指。梓只是看着眼前这无声场景。

    「有好好在练习吗?」

    从头上传来三年级生的沉着声音。她就是立华引以为豪的长号声部的组长。

    梓小小声地嘟嚷了句「未来前辈」,可她的名字都要到舌头上了,却突然觉得有股莫名的羞耻感。

    「嗯,是,有好好在练习。」

    隔了一阵后梓才挤出这样的话来。未来不知有什么好笑地说:

    「雨未华最近吹得不错了,是梓老师教导有方?」

    「没有,雨未华她也很努力。所以最近都快忘了雨未华是新手这件事了。」

    「梓还是一如既往谦虚呢,别人夸你接受就好了嘛。」

    「是,抱歉。」

    一不小心就提高了音量。平时口齿伶俐的自己,在面对未来是不知为何说不出话来。闻到未来身上清爽的香气,不自觉吞了吞口水时,梓的喉咙抽动了一下。

    「不过梓吹得还是一如既往的好。」

    从未来后露出脸的是副组长的高木栞(拼音kan)。她用手遮住嘴,以高雅容姿发出笑声。她的长刘海绑在了额头后(ポンパドール,看google图片就可以了)及胸的头发发梢带着淡淡的茶色。

    「没前辈说得那么厉害。」

    一边摸索不让对方不愉悦的底线,梓一边否定。长号声部组长副组长的她们两个经常一起行动。

    「又来了,别不好意思啦。」

    栞一副「你就得了吧」的样子前后晃了晃手。看到像大妈一样,梓感觉有几分亲切。从她温柔的细眼可以看出她的性格。突然,未来开口:

    「有事要通知一年级生。」

    看到收起笑脸的未来,其他一年级生屏息聆听。放下的拉管和地面接触,然后那里的黑色垫块,就是保护拉管前端用的。

    「虽然sunfes刚结束,也知道你们很辛苦,不过在下个月和北中联合举行的happy concert上,一年级将首次演出sing。从明天开始sing的步法练习,记得带换的衣服。」

    「是。」

    一年级一听见sing这个单词,就忍不住露出激动的神情。她说要步法练习,就是说我们要在舞台上舞奏吗?(stage drill,虽然字不同,但意思相近)梓压不住自己的笑意,呼呼地笑出声

    《sing sing sing》是1930到1940这一期间筑起一时代人记忆的benny goodman乐团的热曲名字。因为是代表swing jazz的名曲,所以也有在电影《swing girls》中出现。

    《sing sing sing》是立华每年在舞奏大赛上的定曲。激烈的动作所带来的悦动感决定了立华和其他学校的根本性不同,同时也让立华的名字响遍全国。很多人就是为了《sing sing sing》才来立华的。

    「happy concert什么?」

    雨未华为了不打扰前辈,小小声问。梓凑了过去,在她耳边说:

    「就是立华定期举行的音乐会,会邀请本地中学的学生一起上台演奏。对那些学生来说,既能在现场看立华的演奏,又能和立华的人一起表演,可以学到不少东西哦。而且对立华而言,也可以借此机会吸引那些有能力的人。可以说是双赢的活动。」

    梓在读北中的时候也参观过立华的音乐会,那个时候体育馆内摆满椅子,到场的有家长还有学生,梓看了就在眼前的舞奏,兴奋得晚上睡不着觉。

    未来抱起手,她的视线有一瞬间落在了雨未华的乐谱上,然后就像估价一样眯起眼睛,以业务的口吻说:

    「我想大家都拿到了sing的乐谱了,看着乐谱吹自不必说,但上场的时候可是要边动边吹的,所以为了了解大家的能力,到下周的星期四为止,要去声部长那里测试,清楚了吗?」

    「清楚了!」

    听到大家的回答后,未来呼了一口气。然后她的视线停在了梓的身上,嘴角像恶作剧般翘起,然后毫无顾虑地说:

    「期待梓你的表现哦。」

    感觉血一下子涌了上来,兴奋和害羞的热度一下子冲到脸上。未来前辈说了「期待自己的表现」,而且还只是对自己一个人说。

    「谢谢夸奖。」

    自己的声音有没有颤抖呢?心脏咚咚咚地剧烈跳着,撑着乐器的手指也不自觉用力。

    「那我走了,你们练习加油。」

    未来白皙的手轻轻拍了拍梓的肩膀,然后在旁边,看到了栞在轻轻挥手。

    看到三年级的走后,坐在旁边的志保松了口气。太一也像是为了让自己的安静下来不停甩着手。

    「每次和她们两个说话都紧张得不得了。」

    「我懂!」

    志保听太一的话后不停点头。太一那比起女生还要稍微纤细些的手,轻抚般架住了乐器。

    「会吗?我倒觉得和其他前辈一样。」

    雨未华说着把脚伸了出去。她平时都是怎么坐的,裙子上的折痕都坐得乱糟糟的。

    志保哼地说:

    「我看雨未华你很喜欢那些前辈嘛。」

    「没有啦。」

    「反正你是新手,平时受不少照顾吧,羡慕死人了。」

    对于志保的讽刺,雨未华的眼睛不知所措地游离着。在不可见的人际关系下,存在着互相争夺高低的地位之争,根据对方的地位是否在自己之上,还有不惹对方生气的底线到底在哪里,综合种种因素从而得出所谓的人际关系。然后志保的答案,就是雨未华的地位在自己之下。

    「这和新手没关系啦,雨未华惹人担心的是天然的部分,什么时候把拉管整根拉出来也说不准。」

    梓看不下去,所以帮忙搭了下嘴。太一想起了那时候雨未华的平地摔,嘻嘻地笑道:

    「刚入部那时雨未华说把乐器弄坏的时候,真的吓了一跳。」

    「只不过是把拉管拉出来了就慌得不得了,明明是手伸太长而已就吓得快哭了。」

    「谁知道一下子就拉出来了嘛。」

    看到雨未华嘟起嘴,志保和太一又笑了起来。看到气氛缓和下来,梓静静地垂下眼睛。

    雨未华中学那时是怎么过来的呢,她没有一点所谓的社交技巧,面对无关紧要的玩笑也好,含有恶意的讽刺也好她都全盘接受。当周围的人都把真正的自己藏起来的时候,雨未华就这样坦然裸露,一点防备都没有就闯入了名为学校的战场。看到这样的雨未华,梓不禁想拉她一把。

    ——————

    立华的门前有一条长长的缓坡,走下铺有沥青的斜坡,就来到了大马路。刚买的鞋有点大,梓后跟的位置空空的。

    「不知道sing的步法是怎样的呢。」

    走在旁边的雨未华比梓矮了5cm,为了配合梓的速度她小小的腿正卖力走着,然而她却没有出声说要梓等等她。看到就算喘着气也拼命跟上自己的她,梓也为了自我满足而放慢脚步。每当稍大的鞋踩在路面上,就发出塔塔塔这样有点蠢的声音。

    「不知道呢,练习到底是怎么样的呢。」

    「不知道我能不能跟得上大家。」

    梓一下子看向雨未华,她貌似是不自觉就说了出来,以僵硬的表情慌忙用手捂住自己的嘴。

    「抱歉,刚才的就当没听见。」

    透过手,雨未华小声地说。平常练习结束的时候已经是七点了,看到天空中的月亮发出淡淡的光芒,梓觉得有些寒颤。

    「没事的啦。」

    包包的角檫过了已有年代的护栏,梓那低喃被吸入了夜空中。空气中已经有少许的热度,感觉夏天快来了。

    「比起一开始雨未华已经进步不小了,要是练习跟不上,到时我再陪你练吧。」

    「……小梓真能干。」

    「嗯~,突然间怎么了,就算夸我也没有任何好处哦。」

    梓摸了摸口袋后拿出了两枚巧克力,然后给了雨未华一颗,雨未华开心地说:

    「哇,是黄豆味。」

    「最近我很喜欢这个,黄豆饼也很好吃。」

    「嗯,我也喜欢黄豆饼。」

    剥开包装,雨未华把巧克力放进嘴中,看到她吃得津津有味,梓感觉有点不安。

    「雨未华你啊,陌生人给你糖你可别乱吃哦。」

    「才不会吃,小梓你以为我都多大了。」

    「但是看你这样有点担心啊。」

    「巧克力是小梓你的我才吃。」

    「这样的话没问题。」

    找不到反驳的话,梓闭上了嘴。等间距路灯发出的灯光照亮两人的脚下,从自己脚下延伸出去的影子,看起来意外地大。

    「呵呵,小梓也真是瞎担心。」

    背后照来的车灯只抹去雨未华的影子,那刺眼的白光只在护栏上闪了一下后,车就从旁边通过了。梓夸张地耸了耸肩。

    「才不是瞎担心,是雨未华你我才担心的,总觉得你会吃陌生人给你的水果之类的然后食物中毒。」

    「水果这个绝对不可能的啦。」

    「天知道,谁叫是雨未华你。」

    「在小梓的心目中我到底是什么形象啊。」

    看到雨未华鼓起脸,梓笑了出来。握在手中有点融化了的巧克力,仿佛让夜风中带了一丝的甘甜。

    ——————

    梓的家就在京阪线黄檗(bo)站旁边再走一点点的地方,附近有黄檗山万福寺,梓在小时候和家人去过那里吃普茶料理。因为是著名的文化财产,所以有很多游客。而万福寺中又特别有名的是斋堂前面挂着的长鱼形的开梆(木头做的鱼)。开梆是通知用的法具,原型就是鱼。在小时候的梓看来,那口中含着珠子的木鱼让自己大吃一惊。所以就算是很久之前的事了,但唯独那木鱼还记得清清楚楚。

    在通往万福寺的路上转一个弯,就到了梓家所在的住宅区。这个住宅区是新建的,所以每家每户的房子看起来都很新。看到外墙涂成粉红色的家,梓有一个身处童话故事中的错觉。

    「我回来了。」

    打开门,里面没有回应自己的声音。通往客厅的走廊静悄悄的,前面也是漆黑一片。手扶着墙,按下开关,树枝形的灯就亮了,然后养的狗‘鳗鱼’就冲了过来。鳗鱼是马尔济斯犬,是祖父他们寄养自己家的。顺带一说,它的名字由来是因为祖母喜欢吃鳗鱼。

    把家里的灯都打开后,梓才在客厅坐了下来。冰箱里面有母亲已经做好的晚饭,上面还有纸条。

    「今天是汉堡排和土豆色拉,米已经洗好了,时间到了用电饭煲煮就行。」

    看了眼内容,梓默默地打开了电视。里面在播放综艺节目,听到人工作出来的笑声,梓的心是放松了些。一眼都没看向电视,梓按下了电饭煲的开关。就算这种节目一点兴趣都没有,做做BGM还是挺好的。

    梓的母亲是卖劲的职业女性,从早到晚,她人生中大部分时间都给了事业。对于每天都涂上口红一身职业装工作的母亲,梓很是尊重。

    梓没有父亲,原本应该处于父亲位置的某个男人,在梓还在幼儿园的时候就和母亲离婚了,因此梓对那人没有一点概念。

    虽然是单亲家庭,但在经济上却从没担心过。这个家是母亲买的,乐器也是母亲买的,生活上没有一点欠缺。所以梓没有向母亲表达过任何不满,而且也不会有不满。

    「看起来好好吃。」

    脱下制服挂在衣架上后,梓马上就换上了居家服。微波炉加热了的汉堡排发出啾啾的肉汁声,闻到香气,梓感觉肚子饿了。

    用托盘把饭,汉堡排,清汤,色拉搬到客厅的饭桌上,就看到电视里穿着裙子的女明星说着自己的恋爱观。她来这种综艺节目,大概是为了电影的宣传吧。梓一边吃着色拉一边看电视,鳗鱼就静静地在梓的身边卷起身体。

    「还是这么漂亮呢。」

    很久以前梓就有对着电视自言自语的癖好,当然这不是和电视交流之类的,只是无意识会发表自己的看法而已。

    梓从小就喜欢说话,一刻也不见得消停,就算是自己一个人,也喜欢故意发出一点声音。虽然自己也觉得这样很害羞,但从小的习惯不是一早一夕就能改的。

    抬起头,就看到窗帘轨上有许多的集体照。小学时候铜管乐器团的,中学时候吹奏乐部的,还有很多比赛时候的纪念照,母亲每次都会把照片放进白色的相框中摆好。

    最左边的是梓小学四年级的照片。红色的西装白色的裙子。脸上带着小学生天真笑容的梓的手上,抱着和现在一样没有变化的金色长号。

    「……都过了这么久了吗。」

    梓摊开手, 就躺在了沙发上。立华从小就是梓的梦想,自从在舞奏大会上看了他们表演的那一天起,梓就一直幻想着自己能加入立华的吹奏乐部。

    至今都无法忘记第一天参加立华吹奏乐部练习时,顾问对着紧张不已的一年级生说的话——「就由我带你们见识一下非立华无法去到的地方吧。」

    ——————

    立华休息日的练习集合时间是8点,梓出门的时间是7点,比其他学生都早。在中途的站和雨未华碰面,然后一起去学校。到校的时候足球部的人已经在练习了,到处都能听到男生特有的粗犷声音。

    「早~。」

    从音乐室露出脸来的是花音,她背后的妹妹美音则是专心吹着双簧管。

    梓放下包,就马上把脸凑近花音。

    「早。你们都知道今天开始就要练sing的步法了吧,我啊昨天开始就兴奋得不得了一点都睡不着。死啦,感觉太兴奋了,要死要死。」

    「梓你今天好吵耶。」

    美音放下乐器说道:

    「因为我就是因为sing才来立华的嘛有点兴奋又不怎样~。」

    「就是就是兴奋点又怎样嘛。」

    花音为了和自己同调举起了手,而且花音最后的话完全压过梓的话了。雨未华侧头。

    「花音和美音也是为了sing来立华的吧。」

    「没错。」

    「花音那时候吵死人了,看着电视就突然叫着说要来立华。」

    「你们不是住在东京的嘛?为了来立华从那么远的地方搬来啊。」

    听了雨未华的话,花音一副不以为然的样子。

    「因为立华的演奏就只有立华能做的出来,所以也只好来咯。」

    「就是花音你这种性格搞得爸妈很伤脑筋。」

    「美音你不是也说了想来立华吗。」

    「说是说了,不过我可没想到竟然全家都搬过来了。」

    「幸好爸爸的通勤点能改到这里,不然就只好留他自己一个人孤零零在东京咯。」

    虽然花音哈哈大笑,不过要是事情不顺利,现在肯定笑不出来。呆呆看着她乐天派的姐姐,美音叹了口气,接着用细长的手指摆弄着自己的头发。

    「不过幸好我们搬到的地方离学校近上学比较方便。栞前辈好像住在兵库,想想都觉得辛苦。」

    长号声部副部长的栞前辈,中学是兵库的名门学校。而未来前辈的家是徒步就能到学校的距离,所以她们休息日一起玩的时候,都是在中间点大阪汇合。

    美音听到栞前辈的情况后,睁大了眼。

    「啊,这么远,换我的话绝对不行。」

    「好像还有人是滋贺和奈良的,都不知道他们平时几点起床,想一下都受不了。」

    「不过他们也是有心里准备才来立华的吧。」

    美音往自己手指吹了口气,她仔细修剪的手指甲虽然没有涂任何东西却还是富有光泽。花音和美音的手形都一样是适合弹钢琴的修长形。

    「花音和美音的中学在哪。」

    听到梓唐突的提问,美音眨了眨眼。

    「怎么突然问这个?」

    「就是好奇。」

    旁边的花音则是刷地举起手。

    「我来我来,我们以前的学校也算得上是强校。虽然去不了全国,不过在附近还是很有名的。」

    「这样啊,那和我以前的一样。」

    「哦,那我们是同志咯……不过现在我想去全国,所以才选了立华,因为立华的表演太棒了。」

    花音一副无畏的笑脸。

    「所以今天我等了很久了,sing的练习可以说一半可怕一半兴奋,啊,我无法控制身体的颤抖了。」

    「你犯什么中二……」

    美音一副冷漠看着做作故意撩起头发的花音,不过只有梓注意到美音的嘴角微微上翘了。

    ——————

    立华休息日的练习首先是从部内会议开始的。在黑板前站着的是部长翔子,副部长桃花,领队南。身为干部的她们三人和各声部的部长商量之后,便决定好当天的安排。

    「今天的安排是这个。」

    翔子拿起粉笔,在黑板上写上字。

    8:15—9:00 个人练习(场地自定)

    9:00—10:00 基础练习(音乐室)

    10:00—12:00 一年级步法练习(第二视听室) 其他人部内练习

    12:00—13:00 午休

    13:00—18:15 一年级步法练习(第二视听室) 舞奏编排会议(13:00 书道室) 干事and各声部长会议(14:00美术室) 二年级三年级部内练习

    18:15—18:30 总结

    时间安排是各声部长根据自己声部的人的情况制定的。因为顾问很少管这些事,所以新人的培养到时间安排大部分都是由前辈们负责。

    翔子写完后,拍了拍手弄掉粉笔灰。

    「你们一年级今天开始就要练习sing的步法了,第一天总之先记住动作,记得做笔记。还有要听南的指导。」

    「是。」

    「中午之后有职务的人请在规定时间参加会议,不用练习步法的高年级生请自主练习下次比赛的曲子。特别要练习只有二三年级要吹的‘玫瑰肉’。明天中午之后熊田老师会过来,那时候二三年级需要进行合奏练习。一年级明天也是步法练习,记得带换的衣服。还有一些琐碎的事情总结再说。」

    「是。」

    「解散。」

    听到翔子的指示,大家都动了起来,在梓准备回自己的位置时,衣服的下摆被人抓住了。回过头就见一脸疑惑的雨未华抬头看这边。

    「‘玫瑰肉’是什么,吃的?」

    虽然梓知道雨未华是真心不知道才问的,但因为她的表情太认真了,所以梓是尝试着冷静回答,结果还是笑了出来。雨未华见后则是嘟起嘴。

    「人家问你为什么笑啦。」

    「没,但真的不是吃的。」

    「但一般听到‘玫瑰肉’都以为是吃的啊。」

    「嗯,正常是这样没错,不过你的表情太认真啦不小心就戳中笑点。」

    梓把手放在胸口调整呼吸,在此期间雨未华还是一副不服的样子等着梓的回答。梓这下也有点不好意思了,故意清了清喉咙后说:

    「玫瑰肉是《序曲‘玫瑰的谢肉祭’》的统称,是意大利作家奥利弗·多蒂为学校乐队所作的序曲,是吹奏的课题曲。」

    「这样啊,下次happy concert前辈吹的就是这首啊。」

    「嗯。」

    雨未华貌似理解了点了点头。在她穿着的绿色体育裤中露出的脚裸上绑着粉色和黄色的脚绳。雨未华似乎很喜欢做这样的手绳和脚绳,经常见她收到朋友在精品店买的绳子。

    「事到如今我可以问个问题吗。」

    「还有什么不懂的吗?」

    「我以前就在想,‘领队’到底是什么?」

    「这个还真是事到如今。」

    看到侧着头的雨未华梓苦笑道。把像快滑下的T恤的袖子卷到肩膀的位置后,梓说:

    「‘领队’简单说就是舞奏队的指挥者。行奏的时候南前辈不是走在最前面吗,普通来说那个人就是领队,虽然我们学校和其他的有些不同。」

    「记得二年级也有领队对吧。」

    「嗯,立华是秋天的时候一年级的被前辈推荐,到了二年级就辅助三年级的,然后升上三年级才是作为真正的领队在舞台上演出。而且我们立华的领队同时也要负责舞奏的指导。」

    「就是说在训练时要整天被南前辈训咯?」

    「大概吧,因为大家都叫她恶魔DM(领队)。」

    「哇,这下得好好要加油了。」

    雨未华发着抖,视线不小心放到了墙上的两张奖状上。全日本舞奏大赛金奖,京都府吹奏大赛金奖。这两张奖状都是去年获得的。一个是全国,一个是京都府。

    「立华的舞奏和吹奏实力上差距不小呢,我记得吹奏去年连关西都去不了。」

    「这也没办法,毕竟我们学校专注的舞奏。」

    「舞奏和吹奏果然是两回事,有没有学校两方面都很强的?」

    「——有哦。」

    后方突然传来声音,吓得回头一看,就见部长翔子站在自己的后面。三年级的运动服应该是红的的,不过翔子穿的裤子则是和梓她们一样是绿色的,而且大腿根位置的名牌上还用白色的线绣着铃木这样的字。因为翔子的姓是森冈,所以这条裤子是别人的。

    「啊哇,是翔子部长!」

    雨未华用手捂住嘴,梓也紧张得一瞬间屏住了气,不过还是努力挺直腰不让对方发现自己的紧张,就这样60度鞠躬。

    「辛苦了。」

    雨未华也赶紧和梓一样低下头。虽然翔子也若无其事回了‘辛苦了’,可雨未华和梓还是紧张的不行。翔子把手叉在腰上,身体微微前倾,和雨未华对上眼。

    「怎么样,雨未华能跟得上吗?」

    「是,跟得上,我会加油的。」

    雨未华捣蒜般点头。

    「这样就好,毕竟没有事情比新人中途退出更让人痛心了。希望你能体会到音乐的乐趣。」

    翔子微微一笑,可以看见她洁白整齐的牙齿。

    「话说你们刚才在谈的事情挺有趣的,雨未华对强校的事情有兴趣吗。」

    「也说不上有……就是好奇吹奏和舞奏都很强的学校究竟是怎样的。」

    「嗯,现在也没吹奏的练习难怪你不知道,不过知道的事情多一些也没坏处。」

    然后翔子招呼她们两个去黑板,之后她拿起粉笔,用习惯的语调说明:

    「吹奏大赛就是坐着演奏。晋级顺序是都道府县大会,支部大会,全国大会。参加队伍比较多的都道府县(东京都,北海道,京都府,其他县级地区),在县大会前还有地区大会,不过我们京都的起点则是府大会。关西有比较强的学校的地方是大阪,另外还有奈良,兵库。不过还是大阪的强校最多。其中称为三强的是明静工科高校,大阪东照高校,秀塔大学附属高校。这三间去年全部都拿了全国的金奖,普通的学校面对他们根本没戏。今年关西的代表应该还是这三间吧。」

    翔子在黑板上画出了金字塔,参加吹奏的有3000多间学校,而全国的名额只有29个,就像千军万马过独木桥。

    「然后我们专注的是舞奏,比赛时间是夏天,大概是吹奏的关西大会刚结束就到舞奏的京都大会了。舞奏的晋级顺序基本和吹奏一样,而且舞奏参加的学校也没那么多,因为舞奏还涉及到衣服和练习场地的问题,拿个金牌比吹奏费事很多。」

    翔子接着补了一句「不过舞奏的观赏性真不赖吧。」

    「大阪东照和明工的舞奏也很强,特别是明工,简直是大boss。他们每年无论吹奏还是舞奏都是拿金牌的。而且大阪的强校还有很多,都想叫my god了。而且兵库强校也不少,特别是泷上三中就强的不像话。总之现在关西的势力图是这样的。都懂了吗?」

    「是,很清楚了。」

    「这就好。」

    听到雨未华的回答,翔子的眼睛弯弯地笑着。大概雨未华是新人吧,自从她入部开始翔子就特别关注她。

    「六月之后我们就不会再把雨未华当新人了,会用对其他人的标准对你,如果做不到责备是正常的,而且也不会因为你是新人睁一只眼闭一只眼了,请做好心理准备。虽然有跟不上其他人的痛苦期,不过只要能迈过去,那就能真正体会音乐的乐趣了。加油吧。」

    「是!」

    「虽然有点为难梓你了,但还请你费心雨未华。」

    「好的。」

    翔子微微一笑,拍了拍梓的肩,从她身上闻到了柠檬的香味。翔子看了眼手表,之后就匆匆转过身。

    「抱歉把你们留下来了,那我先走了。」

    梓和雨未华小小低下头送别离去的部长。部长这个职务看来挺忙的,翔子没走几步,就被其他部员叫住了。

    「翔子部长真厉害。」

    雨未华双手放在脸两侧无意地说:

    「不管哪个前辈都厉害。」

    从口中流露的话,毫无疑问是梓的真心。

    ————

    第二视听室在音乐室旁一点点的位置。立华校舍中间是开放式的,所以从二楼可以看到出入口和不大的广场。走廊上排球部的人组队在做拉伸,为了不打扰他们部员从他们的缝隙中穿过走去指定的教室

    走过连接西校舍和东校舍的通道后,就到了功能教室的走道。和科学室与家政室在一起的便是第二视听室。一进课室部员们就赶紧把桌子推到墙边确保等一下的活动空间。梓换上了室内使用的运动鞋,然后拉伸了一下身体。梓现在穿的衣服是中学文化祭那时做的,黑色底,上面印有七彩变色的‘一身青春’,看起来有点傻。

    「训练开始。」

    南站在排好队的部员前面,和整天挂着笑容的部长翔子不同,她的表情不怎么丰富。在以前她还有冰美人的绰号,不过到了三年级当上领队后,就完完全全变成恶魔领队了。

    「sing的动作是以前的前辈部员留下来的,所以步法的形式也很多。接下来的舞台表演和夏天的舞奏的动作,主要都是这里所教的步法和到时全体的动作编排结合起来构成的。为了将来的自己考虑,这里就不要偷懒,请认真对待。」

    「是。」

    「那就从基本的开始。」

    最先的两个小时首先是南示范,之后是大家跟着做,大家都拼命往笔记本上记下动作的要领。

    立华之所以称作水色的恶魔,是因为激烈编排动作,如果没有接触过乐器,是无法想象此中的困难的。

    「注意体干和上半身,上半身晃动的话是无法好好吹乐器的。」

    南的这句话便是舞奏的精髓。初眼见动作虽然激烈,但实际上动的是下半身,上半身是保持不动的。就算脚的动作再怎么激烈,上半身也决不允许有半丝的晃动。

    「一、二、三、四。」

    南按照节拍不断重复动作,从后面看就可以很清楚她上半身的位置是固定的。与大家喘着气不同,南的气息丝毫不见紊乱。

    「动作慢了,再来一次。」

    「是。」

    两手做出吹奏乐器时的动作放到嘴边,一边意识拿乐器的动作把两肘伸开,一边想象着一根从天花板垂下的线连着自己的头顶把自己的身体拉直,最后再下意识保持这个姿势。

    大家都跟着南的动作,踢腿的时候听到衣服摩擦的声音,踏步的时候听到空气的撕裂声。这些声音都合为一体宛如某一生物的低鸣。

    梓一直想的就是这个。感觉着空气的流动和听着大家同步的呼吸,一股一体感油然而生。室内染上热度的空气,让汗珠流过梓的下巴。

    「上午的训练就到这里。」

    「谢谢指导。」

    随着南的一声宣布,大家一同低头道谢。南拿起脚边的瓶子,一口气把水喝完,用手背擦了擦嘴后,用清爽地说:

    「下午的训练会更加辛苦请做好准备。记得午饭不要吃太多免得受罪。那解散。」

    说完,黑发翻舞,留下飒爽的背影她就离开了。

    「搞得像电影一样。」

    旁边的花音开着玩笑,她的T恤吸满了汗贴在了皮肤上。

    「辛苦了,好累啊。」

    「那个要一边拿着乐器一边做吗,啊~,感觉做不到。」

    「南前辈看起来倒一身轻松,果然习惯了就不一样。死了,平时不用的肌肉用太多,感觉明天会很酸爽。」

    「这就是立华的洗礼啊小姑娘,嚯嚯嚯。」

    不知花音为何像局外人一样,说着事不关己的话愉快地摇着身体(いったい何から目線なのか、花音が愉快げに身を揺らす 不是很懂这句,如果直译是:到底是哪来的目光呢?花音愉快地晃着身体)。在她的身后,雨未华和美音正背对背拉着筋。

    「好痛,美音快放我下来。」

    「嗯?雨未华你的柔软性太差了吧。」

    「才不差,是美音你太好了。」

    虽然美音一脸没事,不过她的动作确实不得了。

    「那是杂技吗。」

    明明是心中嘟嚷,可不知为何却听到了声音,往旁边一看,就见满身汗的太一惊讶地看着她们。只见太一粗鲁用衣服的下摆擦起汗来,从中露出了有些晒黑了的肌肤,然后腰细的像女生一样,甚至还有S线。梓暗自不妙,无意识用手摸了摸自己的肚子。

    听到太一的声音,其他一年级生都看向了美音她们。

    「哇,美音好厉害。」

    「像水母一样。」

    听到周围的感想,美音自豪地说:

    「我和美音小时候是练体操的,柔软性是相当自信,拜次所赐也不怎么会受伤。」

    「原来西条你们运动神经这么好。」

    太一小小声说。在大家都看着美音的时候,志保在教室的一角静静做着前屈。虽然她拼了命,但不管怎么伸手都够不到脚,看来她身体柔软性不怎么好。

    「小梓柔软性怎么样。」

    雨未华不知什么时候就站到了自己身边。平时到肩波浪发也为了不干扰训练而用皮筋绑在了两边。

    「就普普通通。」

    「四月那时不是要体测吗,我体前屈就一直不行,够都够不到。」

    「确实有很多人够不到。」

    「不过每天洗澡之后我都会拉一拉,慢慢地就能够到脚尖了。」

    然后她示范地做了一下,淡粉色的手指稍稍碰到了地面。

    「以前只能到小腿,不过坚持下来就慢慢能行了。」

    雨未华呼呼笑着,察觉到雨未华想让人夸奖的言外之意,梓摸了摸她的头。

    ————

    吹奏乐部的习惯是在休息日练习的时候吃午饭按年级坐。三年级在音乐室,二年级在美术室,一年级在书道室,大家就围在一起吃。书道室离第二视听室有点距离,在走过去的时候,大家都和自己关系好的人一边聊着天,一边走到西校舍。

    「是熊田老师。」

    听到这大家都停下说话,看到对面走来的人,大家都动作整齐打招呼。听到这对方停下脚步。

    「大家好,都是一年级的吧,有好好加油吗?」

    大家都慢慢抬起头来,眼前的人带着爽朗的笑容等着大家的回答。她的名字是熊田祥江,立华吹奏乐部有名的顾问。

    吹奏乐部的顾问有三名,负责主要指导的熊田老师,负责B部门的城谷老师,然后是新来的山田老师。山田老师平时会辅助吹奏乐部事务之类的工作。

    熊田老师是专业的音乐老师,今年52岁,有三个儿子,他们都已经出来社会工作了。熊田老师平时的口癖就是「我的儿子都长得像我,可帅了」。

    她发福的体型有的说法是因为她喜欢吃黄檗站那家老字号和果子店幸富堂栗子馒头的缘故。而且她一天吃三个,再加上岁月的缘故,她肚子那里的轮胎是越来越多了。

    「今天练习什么,看大家都穿着体育服,是舞奏吗?」

    老师一个个看着大家的眼,刚想会不会对上眼,老师就恰好看向了自己,梓吞了吞口水,耐不住沉默,便开口道:

    「从今天开始是sing的步法练习,南前辈负责教我们。」

    「啊,这样啊,那怎么样,很辛苦对吧,在习惯之前都很不得了哦。哈哈,虽然这轮不到我说啦。」

    熊田老师豪爽的笑声听起来很舒服。

    「不过试着认真来一次sing,你们就知道其中的激动了。」

    老师干干的手轻轻拍了拍梓背,梓便低头。

    「谢谢老师说明。」

    梓来到立华后觉得惊讶的一件事便是顾问不怎么过问社团活动。熊田老师就是除合奏训练和正式比赛之前都不怎么露脸的。她平时就在办公室,通过负责消息的部员传达指示。中学时候顾问会安排细微的工作,不过高中顾问就不怎么直接安排训练了。

    立华吹奏乐部的事务都有各自负责的人,所以运转上不成问题。不是学生等待老师的指示,而是学生一边不断修正一边不断改进,从而使活动自发进行。在一些问题无法由学生自己独自解决的时候,老师就得帮忙推一把。而熊田老师的主要任务,就是辅助这个过程。

    「熊田老师好像又肥了。」

    一年级生吃饭时都围成一个圈。听到美音打开便当盒时说的话,周围也传来附和声。

    在书道室摆在了一起的不锈钢桌子上,堆着文件还有纸箱。一开始是因为近音乐室所以把这里当乐器室的一部分,而现在则完全将这里当成社团的教室了。

    「啊~好像休息啊。」

    坐在右边的志保吐出了真心话。在她蓝色的桌布巾前面,放着已经消毒的白色毛巾。看了一眼她的便当,颜色很丰富,而母亲做给自己的便当,则是茶色一片。

    「怎么了。」

    吃着姜汁烤肉梓侧头问,可以感觉到自己绑成马尾的头发在晃动。志保用筷子前端插入鸡蛋烧后,叹气地说:

    「因为入部之后一次都没有,每天都是舞奏的练习,感觉坚持不下去了。」

    「强校都是这样的吧。」

    「这我知道,不过还是想说说啦。」

    苦笑着她把头发别到耳后。志保右手的中指上有红红的茧,指甲剪得秃秃的,看着就觉得痛,一想到长长(zhang·chang)的指甲慢慢陷入肉里,梓就一阵发麻。

    「雨未华能跟得上吗。」

    左边的雨未华抱着手,她前面放着便利店买的点心包。奶油包,巧克力丹麦包,草莓蒸馒头,喜欢吃甜食的她饮食习惯挺杂的。

    「我记性不是很好,动作之类的都不怎么记得住,今天好难熬啊。」

    「的确,对有经验的人都不简单。」

    对梓的话,志保是大大地点头。

    「就连有练着舞奏的人都说不轻松,果然立华的动作太难了。」

    「不说还以为在跳舞呢。」

    「我都没自信到时自己能不能拿着乐器做那种动作。」

    志保大大叹了口气。看来步法练习对志保来说真的有点难,她今天光是在叹气。

    「我倒觉得不错,做到的时候很有成就感。」

    「梓你的话是这样啦。」

    「志保不是吗。」

    梓看到志保的喉咙抽动了一下。她微微眯起的眼睛一下子看向了津津有味吃着奶油包的雨未华,然后仿佛按捺着什么似的,她咬紧唇,手上的筷子也在小小地抖动。接着她放松了唇,眼睛带上几丝皱纹,笑着说:

    「梓你真的很喜欢吹奏呢。」

    「对啊,很喜欢。」

    看着没有犹豫回答的梓,志保又眯起了眼。梓不小心看到了志保那剪得光秃秃,一点多余都不留下的,诉说着志保神经质的指尖。

    「有点羡慕你这点。」

    她的声音带着几分憧憬。梓不知该如何反应,只好挤出笑脸道:

    「原来我还有这样的点哦。」

    听到自己开玩笑的口吻,志保的表情缓和了下来,就像突然脱力一样,志保小小吐了口气。

    「有,而且有很多。」

    ————

    下午的练习如南的忠告,吃太多的人都按着肚子度过了地狱的时光。运动太过激烈,侧腹刺痛,感觉有东西压着胃,梓皱起了眉头。

    「户川慢了。脚乱了。」

    「是。」

    确认着大家的动作,南不断对大家进行纠正。虽然志保被点名了,但她还是跟不上,大概已经到她能力上限了。跟不上的人不止一个两个,在南看到有跟不上的人时,就会一个个指出。

    「你们以为比赛时就光拿着乐器不用吹吗。」

    「挺直背,上半身不要晃。」

    「好好从肚子里发出声来,别跟我说现在就累了。」

    「名濑你慢了,脚动起来。」

    听到南一刻不停叱责,大家都抖几抖。一边喊着1 2 3 4,一边不停重复动作。雨未华是新人,被点名的次数最多,虽然有些泪目了,但还是咬牙坚持。

    「不会知道是谁吹错音,但如果动作错了观众一眼就能看出来,而且还会拖其他人的后腿,请牢牢记住这点。」

    如南所说,视觉上的失误会非常明显,而且一个人的失误会连带造成其他人的失误。因此个人吹得好就不用说了,动作也要做对。而且这还是前提,舞奏的至高目标是团队的动作统一。

    「接下来拿起乐器。」

    「是。」

    听到领队的指示,大家手脚麻利动起来,拿起放在一角的乐器后,再排回刚才的队形。负责上上低音号或者苏萨号这样大型乐器的部员明显要更加辛苦。

    「一、二、三、四。」

    大家跟着南的节拍,再次以刚才一样的节奏做出动作。不过因为乐器导致重心偏离,脚无法好好跟着节拍动起来。每当身体上下动的时候架在左肩的长号便陷入肉里,连长号都这般辛苦,很难想象那些负责大型乐器的人到底是怎样的感受。

    在南的结束喊节拍后,大家都马上坐了下来,脚快受不了了。不过南却说:

    「谁说可以坐下来的?」

    听到周围有人的悲鸣,坐下的人也赶紧站了起来。梓用挂在脖子上的毛巾擦去汗,汗从刚才就一直流个不停。T恤就像淋过雨一样湿透了。

    「光这样就倒下怎么行。比赛的时候还要加上吹呢。」

    「对不起。」

    「那些不能边跳边吹的人,happy concert上台的时候还请你不用吹奏了。」

    南用平常语气说出的这话让梓窜过一阵寒颤。没法跟上动作的人自然也没有闲心去吹奏,比起吹出不像话的音乐,还不如让他老老实实不吹做好动作更好。

    吹得不好就别吹了,这是理所当然的事,梓也一直都是这样认为的。可现在,自己却对理所当然的话语充满了恐惧。

    「那再来一次。」

    「是。」

    南的口吻绝非粗暴,她只是语气严厉而已。梓用粗鲁的动作擦去额头的汗,然后用丹田发出声。

    「一、二、三、四。」

    随着节拍,梓拼命让身体跟上。嘴抵住吹嘴,脑中预想自己能否边跳边吹。吹嘴的冰冷若即若离,非常不稳,根本不是可以吹奏的状态。

    「——啊、」

    突然自身的疑问有了回答。梓压住脚的抖动大大吸了一口气。原来自己只是担心不能在吹奏的同时确保动作吗。只是无意识地在害怕自己能不能打破上高中以来第一次拦在自己面前的高墙而已吗?

    梓看到了南为了作节拍而一直长时间打鼓的手,其中的动作没有一丝混乱和犹豫。

    「再来一次。」

    接着南如此说道。

    ————

    「谢谢指导。」

    在总结结束后大家低头致谢。终于从一天长长的练习中解放出来,一年级的人都一副焉了的样子,就可知练习的辛苦了。

    「哈哈哈,你们啊光步法就这样了,夏天的时候还有你们好受的。」

    「小朋友们接下来才是地狱。」

    长号的前辈开心地看着自己说,不知她们说的是真是假,梓之后微妙地点了点头。

    「一直期待的sing怎样?」

    声部长未来挽住了梓的手,不过梓担心自己身上的汗臭所以把未来推开了。

    「前辈我现在满身是汗。」

    「事到如今你这小姑娘说什么啊。」

    未来可笑地耸了耸肩。未来T恤下面的衣服是深青色的,是二年级的衣服,而且看来也有些时间了,还见绣着山田。

    「这是毕业了的前辈的。」

    注意到梓的目光,未来抓起那件衣服的下摆说:

    「我们吹奏乐部有个传统,就是前辈会把自己的衣服送给后辈。你看不是还有一些人穿着不是自己本年级颜色的衣服吗?那就是从前辈那里拿到的。给我这件衣服的前辈已经毕业两年了,衣服都旧旧的了。」

    「原来如此,我就好奇为什么其他前辈穿着不是自己名字的衣服。」

    「嗯,这个传统有点麻烦吧。」

    未来的手指摸着凸起来的刺绣。那张中性的侧脸让梓心动了一下。她从头发中露出来的耳朵有点点晒黑。

    「不过嘛。既然是传统,也不能在我们这届就断了。」

    未来笑了一笑。她这张天真的表情,让人联想到少年的形象。

    ————

    时间已经8点多一点点了,在四周安静的第二视听室里梓吹着长号。大概今天的步法练习很辛苦,一年级的人在练习结束后就三三两两回家了。

    雨未华也叫了自己,不过还没有回家的打算,梓就自己留在了音乐室练习。还有几个人和梓一样留了下来,时不时能听见各处传来的乐器声。

    「……嗯,记得是这样的。」

    回想白天南的话,梓练习着步法。梓不喜欢自己达不到别人的要求,所以梓架着乐器,嘴里念着节拍,用力抬起脚,当动作熟悉之后就开始慢慢加上吹奏。在自己的身体记住之前,无数次重复相同的动作。

    梓憧憬的是身为立华的人演奏sing,这是自己的选择,所以不允许一丝的妥协。梓垂下眼,回想起了入学前的那次会议,那天就是梓以自身意思做下决定的日子。

    经吹奏乐部推荐决定入学立华的学生,会在其他学生开学前就参加社团的练习。三月中旬,气温还有些低,樱花也还没开放,穿着学校制定的衣服,梓掩饰不住自己的紧张敲响了集合地点书道室的门。里面的全是推荐入学的人,花音和美音也在其中。穿着同样衣服,鞋子,留着相同发型的同卵双胞胎的她们在不下十人的空间里很是显眼。

    「大家好。」

    挥着手打招呼进来的是顾问熊田老师。看着眼前这名数次在比赛会场和电视中出现的人,梓的感觉有点奇妙。

    「怎么了,大家都很紧张吗。」

    花音用力点头。从窗缝隙中吹入的春风吹起了后面黑板的粉笔灰,可以看到灰尘在空中舞动。

    「嘛嘛嘛,每年看到新生都觉得挺激动的。因为大家都是自己决定才来这间学校的吧,所以看到这些有自己的志向的孩子们就很期待今年会交出怎样的答卷。」

    —嘿咻,她坐在了椅子上。她穿着的运动服装的logo也随着她的动作起了皱。

    「我之前就想问新入生一个问题,老实说,你们是怎样想sing的?」

    面对老师的这个问题,大家都议论纷纷,不清楚老师的意图,都互相看着脸不知所措。老师就像附近的大妈一样晃着对女性来说有些大的手。

    「呀~,其实每年我们学校都是sing,我就想每年都同一首曲子不会腻吗,但问前辈的人她们又好像有所顾忌,所以我就想问问你们这些小鲜肉是怎么想的。那边的,你怎么想。」

    突然被点名,梓睁大了眼睛。

    「sing怎么样?想吹这个曲子吗?」

    老师那什么特别意味的话让梓的脑子掀起一阵波澜。面对「想吹吗?」这个单纯的问题,梓马上有了自己的答案。梓无意识就站了起来,椅子发出了声响。

    「想吹!我就是为了sing才来立华的。」

    看到梓这样,其他人也纷纷同意。看到大家这样,老师发出愉悦的笑声。

    「既然大家都这样想,那就只好sing咯。」

    嘿咻,熊田老师站了起来。

    「你要站到什么时候。」

    美音从旁边扯了扯自己的衣服,梓才发觉自己还呆呆站着。脸上一阵发热,赶紧坐下后听到周围有小小的笑声。

    「谢谢。」

    「不客气。」

    美音露出微笑。她有着长长睫毛的眼看向了前方,梓也跟着看向了前面。老师就挺直腰站在黑板前,大家都等着老师的话。

    「我想大家都是抱着一颗努力的心来的。不过我首先要说的是对于比赛获奖,我并没有太多兴趣。而且说老实话,我们是强校,去年舞奏也拿全国金奖了。可是对于奖项,我认为那只是附赠品,可是说是努力的周边产物吧。我让大家参加舞奏的目的不是拿奖。」

    老师有皱纹的手滑过空中,她的两脚坚实踩在地面上,挺直的背板炯炯有神的眼睛,即便过了50依旧不见衰老的痕迹。梓不自觉吞了口水。

    「大家都是要把三年的时间交给我的,所以我想给大家献上一份特别的体验当做礼物。我想让大家尝到非我们无法达成事情的滋味。因此立华的目标是追求特别,从而动作才会如此个性,结果什么的,只是附赠品而已。」

    老师自信满满说完后,笑道:

    「就由我带你们去非立华无法企及的地方吧。」

    ————

    那时顾问的话语在梓的胸中回荡,不断前进,不断攀登,高扬感不断推着梓的后背不容许脚步的一丝停滞。

    「这么晚了还没走吗。」

    听到声音,梓立即停了下来,回过头就见抱着有些大的茶色信封的栞以惊讶的表情看着自己。钟已经超过九点了。

    「啊,都这么晚了。」

    「看来你挺集中的啊,汗也出了这么多。」

    「啊—,换好的衣服又湿了。」

    拉起胸口处的衣服,就闻到了汗味。看到皱起眉头的梓,栞小小声笑了。

    「前辈也这么晚?」

    「我?我要排舞奏构成。」

    栞选了一张椅子,礼仪规范地双脚并拢坐下。看来除了长号的副部长之外她还有其他的职务。

    「舞奏构成?」

    「嗯~就如名字所示啦。」

    「什么?」

    看梓满脸问号,栞从信封中抽出一沓纸。白色的复印纸上面画着大大的正方形,其中又有网格。这大概就是在谈论舞奏中整天听到的所谓动作编排表吧。

    「我记得梓你和杏奈一样的北中出身的?」

    「嗯,杏奈前辈比我高一个年级,中学时受过她很多照顾。」

    「那你之前舞奏过咯?那你懂这个编排表吧。」

    「嗯。就是写编排位置的表吧,格子交叉的地方是实际会场打点的地方,然后这些白色圆圈代表部员。」

    「嗯嗯,我刚才就是在想这个怎么填。」

    「呃?」

    梓不小心就发出了声,没有理会梓的动摇,栞自顾自接着说明:

    「每年立华都是由学生决定动作和编排的。这些真的很棘手,在合宿的时候和其他人一起讨论到半夜都讨论不出个所以然来——」

    「不不不,请等一下,编排是部员决定的?」

    梓实在太过惊讶,打断了栞的话。对方貌似不知道自己惊讶的点,栞眼睛一眨一眨的,头也侧了侧。

    「对啊,不然我说这个干嘛。」

    「难以置信。」

    通常位置编排都是交由专门的人负责的,而且舞奏的指导者也有很多,所以这些事情交给他们做也不少见。梓读北中的时候舞奏编排就是拜托顾问的熟人做的。

    对舞奏来说,编排是非常重要的,因为就算部员多有能力都好,如果编排不合理无法发挥个人的优势那也是没用的。既要考虑每个人的能力,还要编排得符合曲风,对于外行人难于上天。

    栞不好意思地挠了挠脸。

    「虽然有参考前辈各自的能力做出了这份表,不过之后还要交给舞奏的老师检查,就是外援的三川老师。」

    「即便这样原型也是自己做的吧,这样就很厉害了,吓了一跳。」

    「我当初第一次听的时候反应也和你一样,而且想不出来的时候很伤脑筋。」

    栞手边的纸上标满了红色的标记。例如‘再加一点冲击性’‘苏萨号的移动可能有点难’,关于细微的修改有一大堆,看来她手上这份还不是最终版。

    「在舞奏练习开始的时候就能做好了,敬请期待。刚才梓练的sing动作当然也有加。」

    「啊,练习看到了吗。」

    「嗯,不经意看到了。不过还没有人能像梓一样第一天就到这种程度,梓你真的很努力。」

    对于栞笑着对自己这样说,梓垂下眼。

    「如果真的这样就好了,不过不是我努力,只是我不能允许自己松懈而已。对于别人的要求我想做到最好。」

    「那梓是完美主义者咯。」

    「谁知道呢,我自己也不清楚。」

    只不过,自己讨厌不能回应他人期待的自己,讨厌输给别人的自己,所以梓只好不断努力。在自己松懈的那一个瞬间,感觉有无数的东西向自己压来把自己逼入死境。

    「未来也像你一样。」

    栞翻着桌子上的纸说。把刘海梳到额头后的她,额头比别人要窄一些。微微下垂的眉毛也修得很漂亮,给人温柔感觉的眼睛也有点点向下弯。

    「未来前辈和我一样吗?」

    「嗯。」

    栞的手指划过纸面,像樱贝一样指甲上有一点点的凹坑。

    「我们的声部人数还是挺多的。三年级有4个, 二年级有5个,一年级也有4个,加起来一共13个,如果问里面谁最强,吹奏乐部的人都会一致回答是未来。」

    「确实,未来前辈给人的感觉就是稳居第一的。」

    梓刚入部的时候,最先向自己搭话的也是未来。虽然紧张得记不清细节了,不过记得是无关紧要的闲话。未来是立华长号的绝对王牌,从来没有一个人质疑过她第一的地位。

    「未来她啊。」

    刚说,栞又闭上了嘴。如同挖出尘封的记忆一样,她的眼睛向上看,她修剪得整齐的指甲大力抓过纸的表面。

    「原本她是新手,上了高中才接触乐器,那时候她选了长号。」

    「额,是这样吗。」

    「嗯。」

    「我都不知道。」

    梓说出自己的感想,然后视线落到了自己手上的长号上,长号在自己手上看起来很憋屈。

    栞轻轻叹了口气,嘴唇歪曲。

    「一开始当然是吹得不怎么样,连音都吹不出来,所以就让她自己一个人练了。不过她是很不服输的人,一回神就发现她比其他人吹得更好了,可能也有天赋的缘故吧,总之不知什么时候她就成为第一了。」

    听栞的话看似轻松,不过梓知道其中的不容易,特别是在立华,在其他人都有经验的情况下,一个新手是怎样通过努力成为第一的,不用想都知道背后包含了多大的努力。

    「同年级的人都知道未来很努力,所以指名未来当声部长的时候没有人异议,而且solo大概也是她的。」

    栞翻过了几十张纸,现在的纸是长号solo时的位置编排。大家都以solo的人为中心围成弧形。

    ——不过呢。栞带着自嘲呢喃道。

    「如果要问我是不是100\%赞成,答案是否定的,因为我也想solo啊。我明白新手不可能一直是新手,不过我没偷懒也同样在努力,为什么我就不能吹得像未来一样好呢。如果当初我更加努力,是不是我也能更上一层楼呢……嘛,也有这样后悔过。」

    栞不好意思抓了抓脸。她的脸有点红,看来向后辈说这些还是觉得不好意思。梓紧紧抓着乐器盯着栞,总感觉脚有点站不稳。

    「新手不可能一直是新手」栞的这句在梓的脑海中回荡,不停刺激着梓的神经。

    「我不是很会说话,不过我觉得立华每个前辈都是自己的榜样。我觉得有一位值得尊敬的前辈是一件很幸运的事,而且栞前辈吹得也很好,所以那个……怎么说呢,会为你加油的。」

    自己说的话如预料一样,在得出结论前就结束了。梓对于接无关紧要的话是很擅长,不过对于这种严肃的话题,就不怎么能说出话来。特别是在面对前辈的时候,为了顾及不触及对方的痛点,更是劳费心机。

    「抱歉抱歉,让你费心了。」

    栞苦笑着站了起来,见此梓也松了一口气。

    「没有这回事,不如说能听前辈的话还得感谢呢。」

    「梓还是这样一板一眼的。」

    「完全没有这回事,我还差的远。」

    看着这样的梓,栞的眼睛看向了地面。虽然她保持笑容,不过梓无法推测她的内心。栞把纸整理好小心放进信封。

    「你还要练?」

    「嗯,再练一会。」

    「努力是很好,不过别勉强自己,明天还有练习。」

    「是,谢谢提醒。」

    梓低头,看到了自己白色的练习鞋,鞋带绑得很紧,就算动作再激烈也不会松开。
    \section{追忆的回望}
    一年级的大家之后都基本把时间花在了步法的练习上,放学后的训练也好,休息日的训练也好,大家都盯着自己的笔记本无数次重复相同的动作。

    当动作终于刻进了脑中,不用思考身体也能自己动起来的时候,梓慢慢就能边演奏边做动作了。

    「哇小梓好强已经能做到了吗。」

    下一个周一的放学后,长号的一年级生训练的地方是中庭的楼下空间。雨未华看着梓,不停发出感叹,梓对雨未华由衷的称赞感觉怪不好意思的。

    今天练习的方式是轮番一个人负责检查其余人的动作,让大家检查自己的动作。未来给我们的课题是一年级的四个人跳的时候动作能整齐。

    「到时候不用吹,总之先把动作做整齐了。感觉可以了就来找我。」

    立华的习惯是前辈给后辈出课题,有针对个人的,也有像这次一样要求团体的。如果能一个个攻克前辈所给的课题,到比赛的时候便具备相应的实力了。

    梓架好长号,用力往里面吹一口气,就发出了如同大象一样具有魄力的骤变调音。随意吹了一下下次比赛曲子的一节,雨未华就拍起了手。

    长号通过伸拉管改变音调,而能无极变调的骤变调音便是长号的魅力之一。

    「哈—,连休息的时候还得吹,太耗体力了啦。」

    志保一脸疲倦坐在了地面上。旁边的太一则是呼呼地用衣服扇风。

    「热死了——」

    太一闭着眼睛说:

    台阶上放着冰冷的饮料,步法练习以来梓喝的茶多了不少。暑假的时候会正式进入舞奏的练习,那时候喝水的量会更多吧。

    happy concert在六月中,到那时为此还有的几个假期都因为有其他比赛所以大家连一次休息的机会都没有。

    「之后不是有县祭典吗。」

    志保扇着衣服下摆说道。雨未华听到后不解地侧了侧头,梓放下乐器小小地缩了缩脖子。

    「就是当地举行的祭典,时间是每年六月五日,另外还有‘暗夜的奇祭’这个别名。因为古时候会在半夜跑神轿,那时家家户户都会熄灯。本地人经常会去看哦。」

    「听起来挺有趣的。」

    雨未华眼中闪耀着期待的目光,不过她对面的太一则大力摇头。

    「你别这么期待,祭典和跑神轿都去不了的,那天还有练习,而且隔天也有早练,别想着熬夜。」

    「这样哦,好可惜,毕竟社团为重。」

    「好想吃人形烧好想放假啊。」

    志保打从心底觉得可惜地说道,从她无神的眼睛可以看出平时的练习让她积累了不少的疲惫。坐在地上看了眼中庭的钟,粗鲁擦干了额头的汗后说:

    「时间差不多了,开始吧。」

    「好。」

    志保和雨未华也拿起了放在箱子里的乐器。因为长号特别的形状,如果直接放在地上伸拉管便会承担所有重力,如果有垫块便不用担心会损坏的问题了,只是没有那多么预算给每人都配一个,于是大家只好把装乐器的箱子拿来,只好在地上铺上垫子然后把箱子放上去。

    「梓你做的最好,这次梓你来看我们。」

    「好。」

    梓把乐器放进箱子,此间太一他们已经等间距站好了,这样一看太一果然比其他男生要低。只有160cm的他和梓与雨未华相差不了多少。

    「先不用吹,跳的时候哼歌就行了。我先看看你们跳得怎么样。」

    「OK。」

    和面对前辈时不同,同年级的说起话来也不用那么拘谨。梓是不讨厌平时练习时那种刺痛的紧张感,不过整天处于那种环境中果然精神上还是有点吃不消。于此相对的是同年级之间没有那种紧张,相应也会经常出现松懈的情况。

    「那来咯,一、二、三、四。」

    拍着手梓喊出口号,雨未华她们哼着sing动了起来。太一原本运动神经就好,动作已经基本掌握了,而且他本人也说自己悟性好。看到他连快节奏的部分也差不多了,梓觉得他很快就能边跳边吹了。

    和太一相对比,雨未华和志保的动作还是不流畅,那两人运动神经本来就不怎么发达,在能不能吹之前,她们要解决的是让身体跟上节拍。

    「志保和雨未华慢了。」

    从抬腿的部分开始她们两个就慢了。在节奏不快的部分她们抬腿的幅度也不够,她们还有一大堆问题。

    「啊—,这样可难办了。」

    太一抓着头,这是他感觉为难时候的习惯。梓也抱着手在脑中回忆她们两个刚才的动作。

    「根本上来说你们两个是动作跟不上,还有腿抬得不够高,可能是腿肌肉力量不够。如果你们还不能控制身体跟上,接下来很难说。」

    「你们之后还是这样南前辈会杀了我们吧。」

    「在南前辈干掉我们之前首先会被未来前辈干掉。」

    想了想前辈责备自己的场景,太一和梓都瑟瑟发抖。团队强调一体性,意味着有连带责任,一人做不好,全体受罚。

    「总之你们先看我和佐佐木做,佐佐木你就边吹边跳吧。」

    梓拿起乐器,为了不被动作让吹嘴偏离,梓用更大的力气让吹嘴压住嘴。志保拍手喊口号,梓和太一就分毫不差按着节奏跳了起来。抬腿的时候衣服的摩擦声重合了,从喇叭发出的声音传满了楼下的空间。

    抬腿的时机,下腰的时机,如果这些细微的时机都掌握得当,那表演时的一体感会更高。舞奏比赛时大家动作的整齐度便是评分标准中的一项。

    「大概就像这样。」

    吹完指示的小节,梓呼了一口气。连续激烈的动作也让太一喘着气。梓把乐器放回去后梓面对她们,志保认真看了自己的动作,不过她有点疑惑地说:

    「太一就不说了,梓你也学的太快了吧。」

    「为什么无视我啊。」

    「反正问你也白问,再说了你跳得也没有梓好。」

    「不过比你学的快。」

    「就动作比你差而已,论吹还是我吹得好。」

    雨未华微笑看着他们在争论,她是觉得他们两个关系很好吧。

    梓一番思考后,才说出自己感觉

    「我没觉得自己学得快,就普通在练习而已。」

    「那么是天才?」

    「怎么可能是天才,论天才应该是的场,我就真的普通在练而已。」

    在梓说出真心话后,太一就露出自傲的笑容。

    「你们看,佐佐木都说我是天才了。」

    「看你这张骄傲的脸就不爽,刚才的当我没说。」

    「买定离手,取消无效,我要向其他人晒佐佐木夸我了。」

    「有什么好晒的......再说被我夸是值得晒的事?」

    「这还用说嘛。」

    志保像补充太一的话说道:

    「因为吹得好其他人都很看得起梓你。」

    「而且前辈也说小梓很强哦,经常能听到今年的solo不是未来前辈就是小梓。」

    雨未华大力点着头说。梓听到这样的称赞,有种自己缩小了的感觉。自己的努力得到别人的评价当然很高兴,可一旦称赞超出了某一程度便会树大招风。要是处理得不好,还会招人反感

    梓开玩笑的说:

    「别这么夸我啦,我会蹭鼻子上脸的。」

    看到大家都笑了成功把话题岔开,梓才松了一口气。

    「比起那个,你们两个真的不妙啊,这样下去会被未来前辈骂的。」

    看到矛头突然指向自己,志保和雨未华都垂下了肩。沉湎在失落中只是单纯的浪费时间,梓叉腰,对太一说:

    「我们轮着来教她们吧,的场在训练结束后也尽量留下来。」

    「不带这样的。」

    「那早上早点来?」

    「不是这个问题。本来练习就练一天了,一想到还要加练就觉得受不了。」

    「有什么所谓嘛,练得多也没坏处。」

    一瞬间太一的眼睛睁得大大的,中途像是放弃一样叹了一口气。

    「佐佐木你这点说实话让人听着挺难受的。」

    「我说什么奇怪的话了?」

    「也不是说你不好,毕竟对你来说是挺正常的。」

    太一拉了拉身体,维持了一个动作太久,听到他关节发出嘎嘎嘎的响声。他穿着的体育服起了皱,他面对校庭,影子笔直在地上延伸开去。

    「没办法我也来帮忙吧,不过早上太辛苦了,我选择留下来。」

    「那特训从今天开始咯,姑且先练到未来前辈说OK的程度。」

    梓握紧拳头看向两人,雨未华很高兴似地抱住了梓。

    「谢谢啦小梓,我会好好加油的。」

    后面的志保则是有点苦闷地点了点头,带着一副不怎么愉悦的表情小声说了句谢谢。

    ————

    之后的几天在练习结束后四个人都留下来练习步法,而其他声部的人课题貌似也是和自己一样,到处都可以看到练步法的一年级生。

    梓坐在地上把水瓶拿过来,可以看到开放式的小会场到了晚上就完全变成了吹奏乐部的练习场。除了自己的长号声部,还有其他声部的人,在稍远处的地方,看到了颜色鲜艳舞动着的旗子,而挥动着银色旗杆的则是穿着同样衣服的西条姐妹。

    「花音她们在干嘛。」

    雨未华拉了拉梓的手。花音和美音背对墙并排转动着旗子。

    「那个是happy concert的练习,她们在上场的时候是旗手(guard)。」(注:接下来的名词可能与标注的英文有出入,不过意思尽量符合原文意思)

    「旗手(guard)?防御 ?」

    看到雨未华做出防御的动作,梓笑着说:

    「不是啦,是仪仗队(colour guard)的意思,舞奏的时候会有人负责舞动旗子还有帽子之类作用类似于仪仗剑的道具制造视觉效果。花音和美音她们同时负责木管和旗手。你还记得sun fes的时候她们也是挥旗吧。」

    「Sun fes 的时候太紧张了都不怎么记得了,原来她们还做这个啊。」

    「听前辈说舞奏的时候挥旗的人还要加几个,大概舞奏练习的时候就会招人了吧。」

    「还能做出那样的动作啊......花音和美音都还厉害。」

    雨未华完全把注意力放到那边了,她大大的眼睛倒影着水色的旗子,脸也因为兴奋带上了红晕,自己刚才的话她没听进去,看来她对那个挺好奇的。

    「雨未华,鞋带。」

    把瓶子放到地上时梓看到她的鞋带松了,叫了她却一点反应都没有,她的心已经不在这里了。没办法,梓只好帮她绑上。

    「啊,都没发现,谢谢了小梓。」

    注意到帮自己绑鞋带的梓,雨未华笑着说谢谢。梓自觉招架不住她的笑容,但还是严肃地说:

    「雨未华你也不是小孩子了,这种事情要好好注意。」

    「抱歉抱歉。」

    雨未华说着谢谢抓住了梓的手。她的手小小的,梓有那一瞬间,沉浸在她手软软的余韵中。

    ————

    「未来前辈现在有空吗?」

    在拼命练习后,梓她们终于决定让未来检查自己的动作。在走去中庭的途中,恰好前辈在练习。二年级和三年级平时练习的时候是架着乐器然后确认自己乐谱,为下次的上场做准备。就算互相有时会吹出声,不过很快声音便又消下去。

    梓大大吸了一口气,鼓起勇气向那边走去。

    「想让前辈看一下动作,现在有时间吗。」

    未来瞥了眼自己,嘴唇离开了乐器,放下乐器她为了转换心态用力甩了下头,之后把挡在眼前的头发拨开,直接问道:

    「已经行了吗。」

    「是。」

    梓用力点了点头。未来确认了梓的表情后说:

    「那走吧。」

    在第二视听室外的走廊,他们已经列好队了等着了。他们的表情比平时更加僵硬。来检查的前辈也收起平时的亲切绷着脸。梓的手往裤子上蹭了蹭,抹去紧张的手汗,裤子也吸收了汗水变成了深绿色。

    未来用力拍了下手。

    「抓紧时间直接从指定的地方开始。」

    「是。」

    「口号我来喊。你们谁能边跳边吹的。」

    只有梓举起了手。未来往自己看了一眼,眼睛眯了起来‘嗯’地回了一句。

    「先不用吹来一次。」

    「是。」

    「那准备,要好好动起来!」

    「是。」

    四人站成一列。把吹嘴拔了出来,这是为了跳的时候不撞到嘴。梓大大睁了下眼睛,慢慢吸了一大口气,好让空气充满肺部。

    「一、二、三、四。」

    未来喊着节拍,大家一边哼着sing的旋律一边按练习的动作跳着。左脚,右脚,然后以各自为支点交替踢起。接着以半蹲的姿势牢牢落地,再次以各自为支点把重心转移过去。动作眼花缭乱,观众可以大饱眼福。

    「到这里就可以了。」

    四人放下了乐器。雨未华拿着乐器的手有些发抖,太一则是呼了一口气,志保则是咳了一下。未来与肩同宽地站在面前,脚纤细的同时也紧致有力。从穿着半长裤下露出的小腿从小肚腿到脚裸有着优美的曲线,同时可以从她穿着的鞋缝隙中看到黑色的踩脚袜。

    未来转了转腿,然后说:

    「不合格。」

    她薄薄的嘴唇说出的话简短而清楚,一年级的四人听后都吞了口口水。

    「雨未华和志保完全跟不上,你们光是这点已经费尽全力了细节完全没有,腿也伸不直,看起来一点也不好看。根本达不到要求。」

    未来说的和练习时梓她们所担心的地方完全一致。志保和雨未华听后都垂下了肩,太一和梓也缩了一下。

    「太一和梓没问题,这样练下去就行。而且太一你应该是可以边跳边吹的吧。」

    「啊,不,还不行。」

    「不是不行,是你不想做而已,你一直都只挑简单的做,这点你得注意一下才行。我承认你天赋很好,但你一直这样得过且过追求轻松,是以为比赛的时候可以这样应付过去吗?」

    「对,对不起。」

    「既然有能力就做好,不然看着糟心。」

    「......是。」

    他的回答有点怄气。

    「那梓你试着一起来。」

    「是。」

    「节拍我来,从指定的地方开始。」

    梓赶紧把吹嘴从手帕中拿出来,之后插进去,吹了一口气。

    「准备好了吗。」

    梓深呼一口气,因为练得太多,鞋底有点磨光了。梓在心中给自己鼓一把劲后,抬起脸。

    「准备好了。」

    未来满意地点了点头,嘴也露出笑意。

    「那开始咯,一、二、三、四。」

    与口号一起梓大吸一口气,长号发出的声音震动着空气。可以感觉到其他人的视线,因为紧张,未来的声音仿佛裹着一层薄膜,手指的触觉也不那么明确了,头脑一片空白,即便如此身体还是按照练习时那样自己动着,操作伸拉管的手也没有一丝停滞。

    「这里就行了,辛苦了。」

    未来拍了一下手。梓停下了动作,乐器也从嘴边拿开。感觉氧气不足,头有点晕晕的。兴奋还没褪去脑子热热的,大口大口呼着气,梓一阵子后才正眼看向未来。

    「做得很好。」

    未来说的仅此一言。她紧闭的嘴唇欲言又止,平常坚定的目光此时也游曳不定。虽然未来在思考什么似的手抵住了脸,最后还是像丢掉烦恼似的甩了甩头。

    「额,没什么事。嗯,梓这样练就行了。」

    肩被轻拍了下。这样可以认为是未来前辈夸自己了吗?梓用手背擦了擦嘴,拿着乐器低下头。

    「谢谢指导。」

    可以感觉高马尾随动作一起动了。

    「动作还行,就是吹的还不够稳。正式比赛时吹的不稳可不行。特别要注意一下变换(swing)的部分。」

    「是。」

    「还有就是光自己练得好是不够的,舞奏讲求的是团体,梓有时间也要多辅导一下志保和雨未华,太一也要帮下她们。不光要自己练得好,也要帮练得不顺利的,不然整体的水平提高不了。如果觉得有难度,也不要自己钻牛角尖,必要的时候可以找前辈。懂了吗。」

    「是。」

    未来皱着眉从口袋中拿出一张叠好的纸,上面写着时间安排。

    「嗯,那下次测试是下周一。志保和雨未华你们两个不要拖后腿了。太一下次也要两项一起来,梓你就辅导一下他们。」

    「是。」

    「那我回去练习了,你们好好根据课题来做。」

    未来留下话便离开了。在确认她的身影完全消失的瞬间,大家都立刻坐在了地上。太一大大吐了一口气直接躺在了地上,志保则抱怨道「脏死了」。

    「还要来哦,都没心情了。」

    太一撒娇一样甩着脚。梓懒得理他把乐器放回盒子。手往上伸拉了拉身体,感觉头脑清晰了几分,保持手姿势不变向右弯了弯,可以感觉腰的肌肉有点绷紧了。

    「的场君对不起哦。」

    雨未华垂下眉,眼中泛着泪光。她那纤细的手指仿佛难受似地抵住胸口,那吐出的话语也染上了泪的味道。

    「我又拖大家后腿了,净给大家添麻烦了。」

    太一看到雨未华的样子慌忙站起来,他那黑色的衣服沾上了灰尘的白色特别显眼。窗外已经暗了下来,夕阳早已落到了地平线的彼方。半月藏于云后,外面只有只有人工的光照亮着地面。

    「没有啦,雨未华你是新人也办法。」

    太一赶紧安慰。看不见雨未华的表情,她只是轻轻说了句对不起。

    「而且在这之前,户川的问题更大,明明自己做不好平时还一脸威风地说我。」

    梓一听便感觉不妙,因为太一的话直接刺到了志保她那高贵的自尊心。梓中途挽救了一下,可太一没理解到这边的意思,还是将话像平时开玩笑一样说了下去:

    「我说你啊,这样下去新手的雨未华可是要追过你咯。」

    可以听到志保她急促吸了一口气,眼镜后的眼睛也睁得大大的,脸色也一片煞白,肩也在颤抖。她抓住衣服下摆的手不自然地放松了,志保什么都没说,只是紧咬着唇

    「的场你太过火了。」

    「我只是实话实说而已,而且户川的性格死认真却什么都做不成,只能说她没实力了吧。」

    「被人训偷懒的是你,别拿别人出气。」

    「哈?怪我咯?」

    梓若无其事提了一下,可太一只光顾着雨未华没有发现志保的变化。梓啧了下嘴,准备带志保离开。

    「志保,我们两个练一次,可能一对一的效果会好点。的场你就在这里看雨未华。」

    梓拍了拍志保的肩,志保没说话,点了点头。就在梓她们准备离开时,背后传来了急切的声音。

    「等等。」

    回过头,只见雨未华往这边伸长了手,她脸上带着如临大敌的表情抓住了梓的衣服,而梓也因衣服被抓住而停下了脚步。

    「怎么了?」

    雨未华没有说话,她抓住衣服的手越来越用力了,她低下头,头发挡住了她的表情。

    「......雨未华?」

    叫她名字后,就听到了吸鼻子的声音,梓以为她哭了想转过身去,雨未华的头便抵上了梓的背。雨未华像靠上来一样手抓紧了梓的衣服,小小声说:

    「等等,不要丢下我。」

    「你说什么傻话啊。」

    听到雨未华这么夸张,梓苦笑道。然而雨未华还是没回话,太一倒是呆了一样看着雨未华。

    「又不是肥皂剧你也太夸张了吧。」

    听到太一话后雨未华更是像撒娇的孩子抱住了梓的腰,然后嘀咕道:

    「我不要,志保要把小梓抢走了。」

    「等一下,哪有这么严重啊。而且你不觉得对我很失礼吗,就这么不愿意我教你?」

    听到太一怄气的话,雨未华赶紧抬起脸来,吸了吸鼻子擦了擦眼后说:

    「啊,对不起,这种说法对的场君确实挺失礼的。」

    「也没想让你道歉啦。」

    看到难搞的气氛终于解开了,梓挠了挠头后,特意用夸张的动作拿开雨未华的手。转过身去和雨未华面对面,雨未华不知所措得东张西望。梓就像要看穿她的内心一样把脸凑过去盯着她的眼,饱有意味地说:

    「以后别突然说这种话,怎么会丢下你。」

    「嗯,抱歉小梓,说法错了。」

    雨未华害羞似地笑了笑,梓也跟着露出了笑容。背后的太一则无奈耸了耸肩,看来雨未华的反应让他有点不那么舒心。

    「我说啊。」

    一直沉默着的志保开口了。她的语调带有平时不具有的冷漠,平常整齐梳好的中分刘海这时也出现了分岔。她的右脚尖踩着自己的左脚,手握拳,把拇指藏在了拳头里

    「我不干了。」

    志保透明的眼镜倒映着平时练习的长长的走廊,走廊的一端放着四个放长号的箱子,还有水瓶。而在这熟悉得不得了的景色中,只有无表情志保的显得特别异类

    「你又说什么鬼。」

    太一的话滑过寂静。志保紧闭唇,舌头舔了舔干燥的嘴唇,抬了抬眼镜,然后她直直看着雨未华。雨未华因为她锐利的目光而抖了一下。

    「我说我不干了笨蛋!」

    她说完便大步走了出去,然后她的手一个突然抓住了梓将她拉走。

    「梓你来一下。」

    「哈?」

    志保没有理会搞不清情况的梓,依旧抓住她的手。在另外两人呆然的目送下,梓说:

    「你们两个先自己练下。」

    志保期间没有停下脚步,互相的距离越拉越远。走过没有人烟的走廊,志保把梓代入一见无人的教室。连灯都没有开,就直直往教室深处走去。在黑暗包围的教室中可以看见从门缝射入的一束光。

    看到二年三班的牌子,梓担心没经前辈的许可就随便进来好吗。黑板上还留着讲课的内容,上面有几条梓还没学到的公式。早已过了放学时间,周围没有一个人影。吹奏乐部是有许可的,所以可以时不时听到乐器声

    「没事吧。」

    梓的声音融入了空气。在眨几次眼睛后,终于适应黑暗了。定眼凝视,发现了在桌子和墙壁中间蜷缩的志保。在一片忧郁的世界中,可以感觉志保的嘴微微动了动。梓在她身边坐下,没说一句话

    「抱歉。」

    「抱歉什么。」

    「把你拉进来了。」

    志保觉得难堪,把脸埋入了臂弯里。抱起身体坐着的她,看起来比平时更要小。

    「你真的要退出吗。」

    志保指尖抽动了一下

    「啊—」

    发出不明所以的声音后志保把脸看了过来,她的刘海乱糟糟的,眼睛也红彤彤的。不知她是怎么理解梓的眼神,她取下了眼镜。没有见过她不带眼镜的样子,梓好奇地盯着瞧

    「银镜方便是方便,不过有点烦。」

    志保苦笑着把眼睛放在桌子上,眼镜脚碰到了桌子,发出小小的响声。

    「梓你是戴隐形的?」

    「嗯,在家是眼镜。」

    「这样啊。」

    「嗯。」

    梓觉得有点尴尬地垂下视线。可以听到志保的脚有节奏踏着地板,窗外的灯光稍稍照到了自己的脚边,教室染上了淡淡的夜色。

    「我太不像话了。」

    志保大大地叹了口气自嘲道,她伸直脚,累了似的靠在了墙上。

    「为什么这样说?」

    「对人乱发脾气。」

    梓反复斟酌她的话。志保用两手捂住脸,再次发出了叹息。

    「反正是梓你,我可以说些真心的话吗。」

    「你这么特意,我觉得不要咯。」

    「没啦,也没多大的事。」

    志保端正把双手放到膝盖上脱力地笑着,然后故意不以为然地说:

    「对雨未华,我喜欢不起来。」

    接着是沉默,梓也抱住自己的腿,把下颚放到膝盖上。

    「我知道。」

    「你原来知道了。」

    「嗯。」

    志保再一次闭紧唇。觉得心有点塞塞的,梓慢慢地摆正姿势。梓的屁股没什么肉,这样直接坐在地板上感觉有点痛,为了舒服些,把右手垫在了屁股下面,透过肉摸到了骨头,感觉有点不舒服。

    「我也知道这样不好,也想对其他人一样平常地对待她和她搞好关系。我当然知道她本性不坏,并非我故意挑她骨头,但就是不行,无论如何都喜欢不起来。」

    听着志保一股脑说出来的话,梓没有否定,梓一开始就知道雨未华那种类型的人不会受所有人的欢迎。

    「你不喜欢雨未华哪点。」

    「以弱卖弱这点,她不是很懂撒娇吗。」

    志保像征求梓的同意一样看过来。

    「嗯,让人放不下心。而且对好的事也好坏的事也好她都直直接下。」

    「我就绝对不行,就算有事也不想麻烦别人没法像她一样马上就表现出来,所以有时会逞强说一些狂妄的话。而且不懂也不觉得雨未华那样的女生比较可爱。」

    梓也懂,被人拜托是无所谓,但麻烦别人就很难做出手。志保再次苦笑,从稍稍张开的唇中看到了她的虎牙。

    「不过最讨厌的还是输给她的自己。」

    她的手指在自己的大腿上游弋。看到她把指甲剪得这么短,隐约觉得这是她心理障碍的一种体现。志保忍不了一点点长出来的指甲,只要一看到,便会马上剪掉。

    「因为sing,我很久前就想来立华了。虽然自己说有点那个,不过我自认为自己挺有实力的。中学那时我是声部长,的场那时是老二,我是老大,solo都是我来的。因为中学那时我是第一,所以我就想就算立华的人都挺有实力,我至少还能排个中上。」

    「我也觉得志保你挺有实力的,虽然舞奏有点不行,不过坐奏真的不懒。」

    「怎么说还是比不上你吧。梓你就没想过除了未来前辈外有人能超过了你。」

    「是这样没错啦......」

    说真的,梓连未来都不想输,在长号方面,自己不想输给任何人。不过梓没有将自己的真心表现出来,过强的自负在别人眼中只会觉得碍眼,在吹奏这种强调团体的活动中,让周围人不舒服绝不是一件好事。

    「我挺妒忌梓的,朋友多,又能干,我没有的你都有。的场那家伙也是,明明我练习量比他多,他随便做做就比我好了,还整天被他当笨蛋。来立华之后我发现我竟然整天都在嫉妒别人,我想都没想过原来自己是这样的人。我好害怕,因为这样下去原本快乐的音乐本身都变成一种负担了。」

    「......所以你才说想退出吗。」

    「嗯,我很傻吧。」

    志保笑了,透过薄薄的T恤,可以看到她的肩在微微颤抖。志保取下头发的橡皮筋,然后甩了甩头,头发带动的风吹到了梓的脸上。

    「我懂,有时候我也会这样。」

    「你也是?」

    「不然你把我当圣人哦?」

    她发出的笑声有几分干涩,头发就算取下了橡皮筋,也依旧保留着原来中分的发型。她用手指滑过自己拿有些硬的头发,几根头发留在了她的手上。梓伸直腿,就把身体往前倾,做出体前屈的动作。梓的柔软性还不错,轻易的就够到了脚尖;这点志保则不然。

    恢复姿势后,梓端正把手放在了腿上,闭上眼,许多以前的事情便浮现出来。梓明白只有抛弃所有讨厌的,极力想忘记的事情,眼看前方不沉溺于过往的泥沼,接下来才能活得更加轻松。梓的第六感告诉她,现在必须将自己那些苦涩的回忆拿出来,和她促膝而谈,否则便无法挽留她。

    犹豫一下后,梓开口了,第一个跳出来的便是sun fes那时偶遇的中学友人黄前久美子的脸。殃及池鱼这个成语形容她或许很合适。老好人的性格让她经常变成那条鱼。中学时候她就是上低音号,而且听说她是从小学开始就是上低音号了,梓在入部那时候,对她很是敬重。

    「中学那时有个上低音号吹得很好的女生,她现在去北宇治了。然后我们中学实力还算可以,选拔也挺残酷的,那个女生顺理成章的去了A组,不过这就意味着她把一些前辈踩在了脚下,加上她为人也挺好欺负的,那时可不得了。」

    「啊—......确实经常有,没办法,吹奏乐部女生多。」

    「我和那个女生是朋友,所以那个时候为她打气之类的和她聊了很多。不过呢,我没有和前辈理论的勇气,能做的只有在她身边为她鼓气。其实也怪不得我,那个时候我很怕无端端趟一浑水。现在想起来,还真是觉得那个时候的自己真讨厌。」

    想成为一个做事正确的人,想选择自己觉得正确的事情做,想成为能做到以上两点的人,然而梓终归知道那不过是理想,无法实现

    「来立华之后我发现大家都很有上进心,没有各种因夺位的明争暗斗,看到别人做得好也会发自内心去认同赞可,所以我真的松了一口气。而且前辈们也是值得尊敬的人,觉得自己好幸运。」

    志保听着梓的话移开了视线。梓不给志保逃避的机会,抓住她的手把她拉到自己面前,两人眼神直直相碰。志保抽到了下喉咙,此时因没有戴眼镜她皱起眉头的表情看起来更加年轻。

    「不光只有志保才会这样想,吹奏乐部有这么多人,有一两个自己的不喜欢的人也是很正常的,所以不要认为自己讨厌别人就是心理异常。不过要记住一点,吹奏是团体的,只有幼稚的人才会把讨厌某人和排挤某人的概念混淆。听好了,自己不喜欢,可以不用勉强自己和那个人搞好关系,只要那个人和自己的目的一样是为了吹奏而来的,那自己也要尽全力去配合他的节奏吹好。看到别人比自己强妒忌也是正常的,我认为如果能好好承认自己能力不足比不上,那即便有妒忌之心,我也不觉在人格方面有任何问题。」

    梓用力的手指陷入了志保的肉中。

    「唯一不能做的就是逃避,如果不想输,那不应为此努力吗?自己前瞻后望的时候,别人可是要超过去了。既然有时间烦恼,还不如抓紧时间练习。」

    「我......」

    志保把到嘴边的话又咽下去了。她犹豫地咬着唇。梓也用力握住她的手,直直看着她的眼睛,然后静静地催促她说下去。

    「......我不像梓那样强,而且也不能直视落后于他人的自己,我知道自己那无谓的自尊心,也很清楚只能努力加油。但还是好怕,一想到拼尽全力去做了最后还做得不好,就感觉自己动不了了。」

    「那要我陪你到你能做到为止吗?」

    志保吞了一口气,她的眼睛闪过动摇的光

    「这些事情梓你为什么能说得这么轻松,雨未华那时候你也是。我越看到这样为别人尽心尽力的你,我就越是觉得自己卑鄙。」

    她一口气说出来的这些,毫无疑问就是她的真心话。她的脸因为强忍泪水而扭曲,透过毛玻璃射进来的灯光显得模糊飘渺,仿佛无论如何伸出手,都无法抓住分毫。光把没有开灯的教室分成了明暗两个世界。被灯照亮的走廊有其他部员走过,他们没有注意到自己。处于灯光下的人是无法发现暗藏于黑暗中的另一面的。

    「我才不是那样的好人。」

    梓摸索着用手指碰到了志保的手腕。

    「我是为了自己才帮别人的。」

    ——我喜欢的是在别人眼中靠谱的自己。

    梓说后,有意识地露出亲切的笑容。

    ——嗯。志保依旧别开视线,从窗透进来的光照亮了她的侧脸,及肩的头发因为橡皮筋取下后没有弄顺所以到处都跳了起来。

    ——这样啊。志保像是理会到了什么再小小地说了句。她看了下梓的手,然后把猫起来的背挺直,垂下的脸颊也慢慢抬了起来,她那别有意思的视线看穿了梓那一看便知虚假的笑容。

    「感觉有点懂你了。」

    微笑着的志保,看起来安心了几分。

    ——————

    梓和志保回去的时候,其他练习的部员已经少了很多。看到在第二视听室的太一和雨未华,梓松了口气。而太一看到志保,则是一脸后悔,看来雨未华说了他一顿。

    「......那个。」

    太一挠着头向志保走出了一步,不过志保无视了他直接走向了雨未华。无视太一在后面的抗议,志保向雨未华说:

    「我为你是新人所以看不起你的事道歉。」

    雨未华是怎么也想不到志保冲自己而来,一时间反应不过来,慌忙摇着双手。

    「啊,嗯,没事没事,我一点都不在意。」

    志保一度回头看着自己,在她身后可以看到一脸疑惑的雨未华。

    「和梓说了一些之后,觉得自己之前都在纠结一些无聊的事。剩下的时间也不多了,就集中精力好好练习吧。我也没时间去幻想一个做得比自己好的人独自怄气了。」

    「嗯,好好练习吧。」

    雨未华开心地点了下头,在她背后,是一脸吃瘪不愉快的太一。

    「你在搞哪一出,把我的反省还给我,都怪你之前乱说话说不干了。」

    「哈?我什么时候说不干了,我只是说不想干而已。」

    「强行夺理。」

    「算了算了,都happy end了。」

    「才不算,我可是真的担心了耶。」

    「好好,谢谢你哦。」

    「为什么对雨未华和对我的态度相差这么大,不公平!」

    志保哼了一声,像看傻子一样看着踩着地的太一。光羡慕着雨未华的太一不知道,在他人看来志保对他的态度明显要好于雨未华。志保不喜欢雨未华,即便是这样,也尽量以尊重的态度对她,和她交往,梓就喜欢志保这种不夹私心的性格。

    「嗯?这么快就和好了。」

    突然露脸的是二年级的杏奈。她的头上绑着看起来很好吃的炒面发饰。

    「啊,杏奈前辈。」

    四个人匆忙摆好姿势,杏奈则是尴尬地点了点头后大步走了过来。

    「我听别人说你们吵架了所以过来看看,已经和好了吗?」

    「是,劳烦前辈担心了。」

    在梓说的时候,旁边的太一小小推了推志保,而志保了踩了太一的脚,看来他们还要吵一阵子。杏奈看了看时间,然后烦恼似的手放在脸一侧说:

    「我听志保和雨未华有点练习麻烦,因为梓和太一你们两个都属于感觉型的,我觉得还是让我这个原本也吃了苦头的过来人教她们会更好。」

    「前辈教她们吗?」

    「嗯,如果她们愿意的话,多少可以给点意见。」

    接着杏奈用有点顾虑的表情看梓和太一。

    「而且你们两个也光是教她们没什么自主练习的时间吧,我就想要不要帮你们减下负担。」

    「没有这回事,而且教人的时候也能学到原本没发现的东西。」

    梓反射性地否定了。不知杏奈有没有发现这是自己顾虑雨未华她们才冲口而出的话,只见她呼呼地笑着。

    「总之这就交给前辈我吧,毕竟sing我练得比你们久。」

    既然前辈说到这个份上也没有理由拒绝了。太一和梓互相看了看后点了头。

    「那她们就麻烦前辈你了。」

    听到太一的话,志保和雨未华赶紧向杏奈鞠躬。刚才志保放下来的头发现在也好好地重新用红色的橡皮筋绑了几圈。

    「交给我吧。」

    杏奈笑着说。在场的人中,唯独梓从头到尾没说一句话。

    ——————

    梓回去的时候已经错过高峰期,车站没什么人。坐在墙边的凳子上,梓大大叹了口气。不锈钢凳子特别凉,梓尽量往前坐减少接触面。旁边的雨未华也一动不动累坏了。

    「电车还没来。」

    「嗯。」

    雨未华对梓的话点了下头,然后她张大口,打了哈欠。从她无神垂下的眼皮中可以看出从早到晚的练习实在是一件很累的事。

    不久,广播声响起,同时电车驶入了对面月台。透过绿色车身电车的玻璃,可以看到绑起头发身穿各式各样浴衣的精心打扮的女生们。她们下了电车,手中满是像巧克力香蕉还有水气球这样祭典中买的小东西。

    「原来今天是县祭哦。」

    听到少女木屐走路的声音,雨未华一下子睁大了眼睛

    「对哦,今天是县祭。浴衣好可爱我也想穿啊。嗯?那边的是不是我们学校的。」

    「可能吧,我们班的人也有说要去的。」

    「哎......如果没有社团我们也会去吧。」

    可以看到有情侣亲密地牵着手走出了车站。梓没什么特别含义地看着女生那绑成蝴蝶结的红色丝带,心想那女生也是为了今天下了番功夫。

    「小梓,你有男朋友吗。」

    「你突然问些什么。」

    「就是突然想问问。」

    雨未华摇着腿,把落到侧脸的头发别到耳后,看着这边。

    「那雨未华你呢。」

    「我?我没有交过男朋友,我从小时候就很怕男生。」

    「那的场呢?」

    「嗯......,的场君也有点可怕,不过还算是个好人。」

    雨未华晃着腿,她的鞋尖檫到了一下地面。此时,广播传来了电车到站的通知。从远方的铁轨射来一束炫目的白光,梓为此眯起了眼睛。

    「小梓你有交过男朋友吗。」

    「没交过。我对那些又不是很感兴趣。」

    梓喜欢雨未华,也喜欢志保,的场也算是喜欢。不过那种喜欢,和恋爱的那种喜欢不一样,毕竟恋爱那种喜欢所追求的东西和梓的这种喜欢所追求的不一样。

    梓对于恋爱不是很懂,因为不懂,所以自然也没有兴趣。就算没有恋爱那种香辛料,自己身处的日子也已经充满了刺激和新鲜感。所以梓一次也没想过为了谈恋爱而谈恋爱。

    梓站起身后用手抚平裙子,和雨未华没有说一句话, 只是静静地等乘客下车。看到那些精疲力竭的上班族都下完了,梓她们也走进了车里。车里没什么人,空位有很多。梓她们就在中间坐下,坐下的时候不禁呼了一口气。

    「原来小梓对这些没兴趣。」

    雨未华好像还想继续这个话题。她的包放在了膝上,双手就放在包的上面,然后她的手指像是弹钢琴一样有节奏地动着。

    「不过也懂,毕竟小梓一心在长号上。」

    「我这个人就是一旦决心要做一件事,别的就不管那么多了。」

    「那很好啊,小梓你一直吹乐器也是多亏这种性格。」

    「雨未华你中学那时怎么样。」

    快到六地藏站,电车减速了。梓只用脚后跟碰地,稍稍把脚从鞋子里露出来了一点。吊手配合着电车的减速大幅度地摇动。

    「不记得了。」

    雨未华就就像说着别人的事情一样。

    「那时候我就是得过且过,一件感觉自豪的事情都没做过。」

    「没参加社团?」

    「姑且进了科学部。不过还是上学放学,回家,一直线。所以现在感觉很充实。」

    雨未华害羞地笑道:

    「有一个目的拼命努力的感觉真好。」

    「是吗。」

    「嗯,现在每天都很充实。而且也有小梓在。」

    雨未华天真的眼睛映出了梓的面容。被人直接这样直吐心中的好意实在是怪不好意思的,梓不好意思看雨未华,坐直了身体,脚也端正地摆好,视线放在了吊手的广告上。以淡色调的背景中,穿着制服的当红偶像喝着饮料,微微出汗让她的头发贴住前额。这就是大人所框出的青春照,清爽光彩亮人,不见任何的污点。

    「虽然现在我还差的远,不过我马上就会追上去,小梓你就等着吧。」

    「好好,我洗干净脖子等你。」

    「可恶,你都没当真,不和你玩了。」

    雨未华鼓起脸。梓依旧以敷衍的口吻回应雨未华,同时想起了栞前辈的话。

    ——我当然早就知道新人不可能永远是新人。

    如果雨未华有一天比自己吹得更好,如果雨未华只限于现在比自己差,如果她有潜力像未来前辈一样将来崭露头角,那个时候自己还能像现在一样面对她吗?

    「啊,黄檗到了。」

    雨未华用手指了指窗外。梓听后赶紧拿起包站起来。停车的时候车有点摇,但梓没有抓吊手,而是靠自己站稳,梓甚至没想到还有抓吊手这回事。

    雨未华抬头看自己,她摇了摇手,说了声明天见。梓看着她小小的手,心想那么小的手,到底能抓住些什么呢。

    从电车下来,月台上没有一个人。梓就自己一人慢慢走向月台的一边。而与夜色不相符的白色灯光,则照亮了整个月台,仿佛自身处于舞台一样。梓就在灯光下看着自己的手掌。光是看到自己的手比雨未华的要大,就有一种打从心底的安心感。

    ————————

    第二天开始,杏奈就来帮手了。她的说明非常仔细,连梓和太一无意识就把握住的细节都一一地向志保和雨未华说明了。

    「抬起右腿的时候你们两个不是都不稳吗,那是因为你们两个连支撑的脚都站不稳就马上做下一个动作的原因。首先是你们两个脚张得不够开,要再张开点。而且你们两个不要动的时候连脸也跟着动,保持住视线,对,就这样。光动腿就行了。上半身不乱动的话就不会不稳了吧。那注意我说的做动作试试。」

    她们两个就这样一一听着杏奈前辈的指导一个个动作去纠正。太一看到杏奈这么教,佩服地说:

    「不愧是前辈,原来还能这么教,换我来绝对不行。」

    「前辈中学的时候不是很会教人,看来上了高中这方面提升不小。」

    「就是说啊。」

    在一脸表情微妙的太一面前,放着黑色的乐谱架。这种折叠式的架子是入部的那时买的,上面放着下次演奏会要用的乐谱。

    「话说的场你不用练吗?被未来前辈说了吧。」

    「我知道啊,但是我不想练嘛。步法就已经够辛苦的了,还要边跳边吹会死人的好不好。」

    「反正到夏天大家都能边跳边吹,你还是赶快习惯比较好哦,好啦,动起来动起来。」

    「别催我啦。比赛的时候竟然要以坐奏的水平吹还要一边动。啊—,舞奏好可怕啊。」

    看着发牢骚的太一,梓耸了耸肩。按了下排水键,把乐器里面凝结的水给放出来。水就这样滴下把水泥地染成了黑色。

    「我倒是挺期待的。因为happy contest的时候只是一年级的做舞奏。好希望舞奏比赛的时候能和未来前辈一起站在舞台上,好想现在就开始舞奏的练习啊。」

    「你这个人到底是有多乐天,换我的话只能想到每天被南前辈训个半死。」

    太一一脸萎靡地叹了口气,梓是想不明白为什么就这么讨厌努力?因为努力练习就能慢慢地做好,今天比昨天好,明天比今天好,看着这样一天天成长的自己,难道就不是乐事一件?

    太一架好了乐器,比起梓吹的音,太一吹出来的明显更具力量感。明明如果他努力,实力足以胜任第一音程到第三音程,不过他偏偏不努力,这就是他最大的缺点。

    「好难,这超难的。」(原文:キツイ,有辛苦和难这类意思,大家自行选择吧)

    刚吹到第一节的地方,太一就把乐器放了下来。看到他挠头挠脸的样子,梓不知为何有一股安心感,不禁露出了笑声。

    ——————

    在杏奈悉心的指导下,雨未华和志保那之前见不得人的动作也终于是有点样子了。特别是雨未华的进步特别大,现在她可以一边拿着乐器一边跟上其他人的动作了。

    「虽然雨未华理解得很慢,不过一旦理解了,就很快能跟上来了。那孩子才刚接触乐器倒是挺努力的。」

    梓听到杏奈这么说不知道要怎么接话,就只好默默地点了点头。而志保好像还是挺吃力的,不过她至少终于习惯了步法,动作看起来也比以前灵活。

    「志保原来你在这里。」

    突然在本该是一年级练习的地方听到了栞前辈的声音,吃惊的四人都匆匆把乐器放下向前辈打招呼。

    「前辈好。」

    「啊,嗯,你们好。」

    栞应付了一下招呼后,就大步走向志保。志保不清楚自己被叫的原因,露出一副疑惑的样子。

    「请问有什么事吗?」

    「志保你之前不是在吹第二音程和第三音程的吗,就是根据不同的曲子吹不同音程这样的。」

    「是这样没错。」

    「其实我们想这次演奏会结束之后就把负责低音长号的人给定下来。」

    「什么?」

    志保的眼睛睁得大大的,太一和梓也是一脸狼狈互相对视。只有雨未华搞不清状况歪着头。

    志保疑惑地问:

    「低音长号吗?」

    「嗯。现在的是三年级的凛音和二年级的杏奈。我们就想每个年级都选一个,而且想到志保应该和她们合得来。」

    长号的乐谱根据不同的音程划分。通常是第一,第二,第三音程,而有时候根据情况还能划分为更多。因此,就算大家的乐器是相同的,但负责的部分却各不一样。第一音程是高音,第二大部分是低音。因为第一音程大部分是旋律和高音部分,所以重视个人的技巧,而solo也大部分是由第一音程的人担当的。

    第二音程则是负责谐音,第三音程则是负责支撑整体。第一第二部分所吹的大部分是共同的,而只有第三音程干着貌似完全不相关的事情。

    第三音程有时也会像上上低音号和大号一样负责旋律,不过总体而言是打造支撑乐曲的基础。

    长号有几种,立华用的是高低音长号和低音长号。高音长号就是大家印象中的长号,特点是管子细长,专门负责高音,特别强调吹奏者的技术和体力。而立华这里用F管的长号,就是高低音长号。

    长号是通过改变拉管长度而调音的乐器,而想做到随心所欲发出合适的音,准确把拉管移动到相应的位置却不是一件易事。特别是对于身材矮小的人,他们的手根本够不到。所以为了解决这个问题,就有了lever(レバー),只要按下就能换成F管,这样手不用伸的很长也能发出低音。

    如果是低音长号,有的低音长号会有两个lever。低音长号的导管和喇叭看起来比高音长号大,所以低音很强,而且吹嘴也是又大又深的形状。

    「请问为什么选我呢?」

    志保的声音中带着不满。梓明白志保为何不满,因为第一音程负责的高音显眼而且谱面也满满当当,可以说是吹奏乐部的主角。相反的,第三音程则默默无闻,完全就是幕后人员。是正常人都想当显眼的部分。

    栞直直看着志保,志保则被那视线压住,小小缩了身体。紧紧抓住高音长号,志保像是后悔一样闭上了眼睛。

    栞说道:

    「因为一年级之中,能把低音吹得又大声又好的是志保你。」

    志保大大吞了口口水,眼睛也反应出了她的动摇,不停摇曳着。而栞则继续平静地说下去

    「低音是乐曲的重要部分,这部分还不能给交新手的雨未华负责。梓的话是擅长高音,所以也排除。太一虽然也可以,不过他吹得不是很稳。志保你除了吹得稳低音也不错,而且也有相应的水平,所以把这个职位交给你是最合适的,理由就是这些。」

    栞虽然最后语气不强地问了句「你可以负责吗?」,不过在梓听来,她的语气中包含了对方肯定会接受的确信。透过体育服,可以看到栞的膝盖有点弯。而被包在袜子里的脚裸则细细的,包裹住整只脚的鞋大概练习得太多,看起来烂烂的。

    志保吸了一口气,她有那么一瞬间看向了雨未华,闭眼犹豫了一下后,便慢慢睁大了眼睛,那黑色的瞳孔,闪耀着光。

    「我知道了,就由我来负责。既然前辈都这样说了,我会努力的。」

    听到志保的回应,栞露出了笑容。她那看起来软软的手拍了拍志保的肩。

    「要拿出干劲不要偷懒咯,大家都在看着哦。」

    从栞的话中听出了那确实的前辈的体贴。就像其他前辈都关照新手雨未华一样,栞也在看着志保。通常来说,这种体贴在同为老手的人听来有点别有意味,不过对于如今的志保,这是必须的。栞看透了志保现在正为自身能力不足而缺乏自信的问题,所以她特意选了这样的话来鼓励志保。

    志保听了栞的话后,大大点了点头。

    「我清楚了!」

    如此回答的志保的脸上,多了几分的光彩。

    ——————

    「那准备再测试。」

    未来对着并排的四人说道。未来的脖子上挂着黑色的毛巾,透过白色的T恤,可以看到包裹住她那不大胸部的运动内衣的形状。关于未来对此无神经的这点,栞整天都叫她好好穿安全内衣。

    「从指定的地方开始。我数数后你们从平常的地方开始。」

    梓拿着乐器大大吸了一口气。往里面吹一口气后,喇叭发出回响。听到未来敲鼓发出的无机质的声音,梓挺直了背。

    「一、二、一、二、三、四。」

    配合着节奏,梓和太一吹出sing的旋律。然后志保和雨未华像是谐音似的哼出歌声。手,脚,身体,每一个部分都要在相同的时间动起来。踢腿的时间,伸手的时间,听觉和视觉的统一,当这些都融为一体后,焕发出了一种特别的协调感。

    当吹完指定的章节后,四个人都在同一时间回过神来。大概是一直抵着吹嘴的缘故,薄薄的皮肤有一阵阵的刺痛。梓用舌尖舔了舔唇,同时听到了雨未华急促的呼吸声。为了抑制住那急促短暂的呼出气息,梓紧紧地闭上了唇。

    未来放开了鼓棒,深深地呼了口气。一年级的四个人都吞了口唾液,不放过未来任何一丝细微的动作。未来来回看了看四个人后,终于说道:

    「合格。」

    听后,大家都长吁一口气,绷紧的神经也一下子放松了。那时的自己没有当场倒下,肯定是多亏了平时训练。

    「终于拿得出手了。在比赛的时候也请维持这个水平。」

    「是。」

    「志保和雨未华在这么短时间里进步很大。也多谢太一和梓你们两个辅导她们。从下一次训练开始一年级的也要和其他年级的一起合练了。happy con结束之后也差不多到演奏会和舞奏的训练了。虽然你们觉得这次测试很辛苦,不过和后面的比起来这只是小菜一碟,所以还请你们加把劲。」

    「是!」

    四个人都精神抖擞地回应未来。

    「那今天就到这里,辛苦你们了。」

    一改刚才严肃的表情,未来露出了笑容。每当梓看到未来这种巨大的反差,都会有种心脏被抓住的感觉。未来认真时帅气的侧脸自不必说,连她那种仿佛少年般亲切天真的笑容也极富魅力。在哪还能找到这么优秀的前辈呢?

    「谢谢前辈指导。」

    大家都异口同声说道。交代下来的任务总算是完成了。看到雨未华按捺不住高兴那傻傻笑着的样子,梓也不自觉整个身体松了下去。

    ——————

    「动作快点,没有时间咯,那边的,快点动起来。」

    「是!」

    听到部长翔子的命令,大家都无心思考只顾快手快脚动起来。happy con的当天,在立华第一体育馆里大部分的一年级生都听从前辈的指示布置会场。为了让客人不换鞋就能进入体育馆,首先的工作是用厚厚的垫子把地板都给铺起来。要铺得不留空隙,然后还要把椅子给摆好。同样的把乐器从音乐室搬过来的也是一年级生。此间,前辈则练习今天要吹的曲子和指示一年级生。

    「小梓,定音鼓上这个奇怪的踏板是什么。」

    在两人搬稍小型定音鼓的时候,雨未华问了一个单纯的问题。定音鼓为打击乐器的一种,在半圆形的鼓上附有支架,可以说是一种大型的太鼓。以皮敷在鼓上,而演奏者则是使用两个鼓棒进行演奏。和通常的太鼓不同,通常定音鼓都由4到5个鼓组成,特点是可以改变音程。

    「这是调整音程的时候用的。如果换成是铜管乐器,不是调音的时候都有调音管吗。所以定音鼓就是用这个踏板来调音的。」

    「我都不知道原来定音鼓也要调音。」

    从衷心点头的雨未华旁边,西条姐妹超了过去。

    「小姐们加油哦。」

    如此笑着说的花音手上拿着几面旗。今天她们两个就是正式上场时候的旗手。可惜的是每当她们走出一步而从翻舞的裙子中露出的不是大白腿,而只是黑色的运动裤。

    「立华的制服真的是超级可爱,姐我已经死而无憾了。」

    花音如此一语后,就原地来了个360的回旋。立华舞台的服装是贴合身体曲线的连衣裙。因为颜色是让人联想到天空的蓝色,所以「水色恶魔」这个外号也因此而来

    立华服装的裙子长度有着非常严格的规定。坐下的时候是刚好看不到内裤,为的是尽量露出大腿,再加上黑色过膝长袜和白色绑头发的丝带,这就是立华给外人的印象。而在比赛之外,在这套舞台装的下面都是穿黑色运动服的。

    「可爱是可爱,不过裙子太短了。」

    「你就不懂了,就是这种绝妙长度才可爱。而且比赛的时候也有穿安全裤,被看到了也不怎样。」

    「就算是安全裤,该害羞还是要害羞啊。」

    「所谓的专业人士就是不知廉耻之人啊少女!」

    —嘿!花音就这样把裙子给拉了起来,然后看到了里面的黑色运动裤。梓则是无语地耸了耸肩。

    「你是不是傻。」

    「哎哟,别夸人家啦羞死个人了。」

    「没人在夸你。」

    在梓和花音说着没营养的话的时候,雨未华和美音聊着天。

    「步法没问题了吗?」

    美音就这样直接问雨未华。大概是为了不影响表演她用发夹别住了前面的头发。她们的脸本来就长得不错,再加上这对双胞胎留着相同的发型,就像是哪里来的偶像一样。

    「没问题了,因为小梓整天留下来陪我练。」

    「这样就好。」

    「嗯。」

    看到笑着的美音和雨未华,花音突然抬头看天发出「嗯嗯」意义不明的声音。

    「啊,好可爱好像在天国,梓啊伯你是不是都这样觉得。」

    「对不起我被你吓到了没听清你在说什么。」

    「美音和雨未华很可爱,超可爱的,啊,当然梓你也是安心吧我没有偏袒任何人。不过论可爱嘛,最可爱的当然还是我。」

    「花音你就不能听人话吗。」

    看到梓无可奈何的样子,花音咯咯咯地笑了。而面对这样的姐姐,走在后面的美音则是一副没眼看的样子。

    「对了,听说志保决定负责低音长号了?」

    美音突然想起这件事转头看着自己,而花音则是惊讶不知情的样子。

    「还真意外,她给人的感觉还以为她会揪住第一音程不放的。」

    「因为她整天都把梓当竞争对手,有时候还会想她还真是来真的哦。」

    「嗯嗯,和梓比根本不够格。」

    被夹在不停说着话的两个人的中间,梓不知如何回应,只好露出暧昧的笑容。而雨未华则是像想看透梓的心情般不时往自己这边张望,而梓是重新调整了下拿定音鼓的姿势,装作没注意到

    美音和花音还是继续说着:

    「不过那种不能客观看待自己的人也不少见。」

    「你在说自己?」

    「Nono,我看自己超客观的好吗。我只是冷静看待自己后,发现自己原来是一名才华美貌于一身的女子而已。」

    「别自恋了。」

    「才没有,我实话实说的好吗。」

    呼呼呼,不知为何花音得意地笑起来。被花音这么一带,梓也笑了。梓就是喜欢花音这一点。

    ——————

    今天的happ con由三部分组成。第一部分是坐奏,乐曲是大家熟悉的流行乐协奏曲。第二部分是舞奏,二三年级舞奏的乐曲是周末演奏会和最近比赛要用乐曲;所以只有一年级独自表演乐曲sing sing sing。因为舞台上站不了那么多人,所以一些前辈们会在舞台下坐奏。最后的第三部分则是俗成的和中学的学生一起演奏,这点和梓中学那时一点没变。

    「嗯,是佐佐木吗?」

    搬乐器途中突然有人叫到了自己。梓抬起脸,发现体育馆的入口处站着自己中学时的顾问藤城老师。梓赶紧走过去,猛地低下头打招呼。

    来立华之后梓慢慢地习惯了这种像体育系一样的打招呼方式,最初的时候还觉得适应不了,不过经过几个月的磨炼,这样的方式已经渗进骨子里了。

    「我现在才想起来佐佐木你中学那时就一直说着要去立华,终究还是给你去了。练习怎么样?」

    「虽然很辛苦,但还是可以应付的。」

    「这就好。其实现在有个中三的人也在说想来立华,我就对ta说你没有拿到推荐所以学习得加把劲啊。」

    哈哈,藤城老师晃着身子愉快地笑道。中学的时候,虽然老师在练习的时候很严厉,不过就总体印象而言还是一个温和亲切的人。虽然也有因为放羊主义而没弄清部内人际关系这样的缺点,可作为一名指导,还是很优秀的。

    「我记得佐佐木你那届还有高坂,黄前,冢本吧,大家都还有在吹乐器吗?」

    「他们三个都去北宇治了,乐器还有在吹,而且之前sunfes也见到久美子——啊,也见到黄前同学了。据她所说现在还挺充实的。」

    「毕竟今年北宇治的顾问换人了。」

    「是的,好像是叫泷老师。」

    隐约记得花音说是个大帅哥。而且听别人评价今年北宇治的实力和去年相比大有不同,所以都在想是不是新老师的功劳。

    藤城老师摸着下巴,口中喃喃自语「泷老师吗」。梓对此感到不解。

    「是认识的人吗。」

    「不,只是觉得这个名字很怀念。」

    ——「请问是怎么一回事呢?」

    在梓开口问之前,就听到大大的声音从后方传来。

    「这不是藤城老师吗?这么早就来了吗?」

    一边笑着一边走过来的是顾问熊田老师。因为今天是正式的日子,所以她穿的衣服也比平时要华丽。一想到是大人间的谈话,所以梓便告辞了。

    「那我先告辞了。」

    「哦,社团加油哦。」

    藤城老师挥了挥手,听到他那优缓的声音,梓不禁想起了中学那时候的事,刚才还鼓足劲的身体,此刻仿佛像漏气的气球一样慢慢泻力。

    ——————

    演奏会以《宝岛》作为开场。这是日本乐器乐队tea square的代表曲,于1986年发行,对于吹奏乐来说是一首相当熟悉的乐曲。以桑巴开头的乐曲因为节奏活泼所以挺受人欢迎的。

    「呃—,虽然已经开始了,不过还容我说几句话。」

    站在坐奏部员前方的熊田老师手拿麦说着话。在吹奏乐行内的人对熊田老师的场间话已经很熟悉了,再加上熊田老师在外人口中是一名能说善道的顾问,所以也有不少来演奏会期待着熊田老师的场间话。

    「因为立华和北中离得近,所以这里不少人之前也是北中吹奏乐部的。所以我想你们大概都会找找前辈们在哪里吧。不过你们可别吓到咯,因为今天你们的前辈漂亮得不得了。如果连你们都在疑惑之前北中是不是有这个人,那妈妈们见到了也肯定会吓一跳吧。」

    观众席的家长听了便笑了起来。happy con的观众也有北中学生的家长。看到穿着北中制服紧张得到处东张西望的吹奏乐部的学生,梓想起了中学的自己。虽然自己才脱下那件绀色的西服没多久,但一股强烈的怀念感直冲自己而来。

    中学的时候,梓有很多朋友,小学的时候也有很多,高中的现在也是,所以梓身边一直都有很多的人,大家看到梓,就会笑眯眯地向自己走来。梓喜欢和别人说话,因为自己喜欢说话的性格所以看起来是一个外向的阳光少女。

    因为朋友多,所以听到的事情也多,把她们口中的只言片语合起来,也大体能把握周围的人际关系。比如A喜欢B,C和D在交往,E讨厌F之类的。在学校属于高级机密的情报,梓知道的总比别人多,然后这些情报,又将更多的人带到了梓的身边。

    学校地位等于朋友数量,而梓至今为止都没有从这个地位上掉下来过。班里面也好社团里面也好梓都有很好谈的朋友。无论是谁,梓都能和ta搞好关系,无论是谁,梓都能和ta说上话。在如此充实的学校里,梓连一次一个人独处的机会都没有。所以当那个人将那句话直直往梓投来,梓顿时不知所措。

    ——「我觉得佐佐木你是不是有一种的病。」

    梓至今依旧清晰地记得被那么说的那天的事情。那时是中三的春天,刚开学没多久。那天梓结束吹奏乐部的放学练习后,去教室拿忘记带的东西。放学的铃声已经打响了,所以学校里也基本不见人影。梓背着装着长号的箱子,打开了教室里的门。

    走廊的空气微温,夕阳染红了走道,像是从这个红色的世界逃脱似的梓走进了教室。奶油色略透明窗帘翻飞如同蜕变的蛹充满了整个空间。然后她就在那里。长长的黑发随风飘舞,长度过膝长裙贴着大腿,在只到脚裸的白色袜子和裙子遮住的下半身之间,只能看到那些许露出的水嫩肌肤。她叫柊木芹菜。

    「啊。」

    对自己无意识发出的声音有所回应,她以极缓慢的动作转过头来。前面的头发已经把她的眼睛都遮住了,在梓看起来这样的有点可怕。她的校服外套有点大,从中露出的纤细的手,则更加强调了她身材的娇细。

    梓和芹菜不熟,她属于那种脱离团体的人,而那种不和别人说话也不和别人交往的人也时不时可以看到。

    「芹菜同学你还没走吗,是留下来自习吗?其实我也很想自习的,不过还有社团都没什么机会。」

    梓的话没有什么特别的含义,只是觉得她自己一个人怪可怜的,所以什么都没想和对待其他人一样向她搭了搭话。

    「你问了又怎么样。」

    隔了几拍后芹菜才回自己。这下轮到梓苦恼了,因为这种话就是单纯填充尴尬用的场面话,并没有特别的含义,而且梓对芹菜私底下干些什么也没有兴趣。

    「没什么,只是想问问而已啦。」

    梓亲切地笑道,对此,芹菜则是用鼻子哼笑了一声。

    「我看你整天都这么多话说,反正只是碰到尴尬的人所以才借此打发过去吧。」

    「没觉得尴尬啦。」

    「别骗人了。」

    芹菜丝毫不领自己的情,接着一口气说道:

    「我看到像佐佐木你这样的人就觉得很烦,你是不是想当然地用那样高高在上的姿态向我搭话我就会像其他人那样理你。我说你啊,理我这样的人有什么好处吗?难不成以为自己是渡世观音,看到一个孤独的人就觉得她好可怜,然后以一副施舍的姿态向她搭话来寻求自我满足?」

    梓拼命地思考她这连串话的含义,努力想接受她的话,可最终还是无法接受她所说的。

    「你说这话是什么意思,我一点都没想过那些,你也太过自卑了吧。」

    「满嘴谎言。」

    「我说你啊,还是好好想一下对方的真心后再说话吧,你这样的性格怎么可能交的上朋友。」

    梓一边说一边走向自己的位置,然后找到了作业要用的英语笔记本。粉色的笔记本表面有朋友画的画。比如用彩色笔大大方方画的猫,然后在一角画有心形。在画的好与画的不好的之间,还有其他朋友的画。所谓的回忆就是这些无意义的东西所组合起来的事物吧。在此之上梓用指尖摩挲着。

    「我觉得佐佐木你是不是有一种的病。」

    芹菜毫无感情冷淡地说。梓拉开包的拉链,把笔记本胡乱塞了进去。回过头,发现芹菜往自己走近了一步。

    「什么?」

    「就是说你是不是有一种想和谁都搞好关系,不想让自己处于不利地位的努力装成是八面玲珑的人的病。」

    她那声音寒冷透彻,包含着叱责对方的口吻,从中听不出她一点的真心,她只是若无其事地揭露对方的内心,对此,梓无意识地后退了一步。自己不擅长应付这种人,因为这种人所追求的不是表面的关系,她们所追求的关系是更深更密切的。

    梓把包挎在肩上,挤出笑容道:

    「不行吗?有朋友总比没有强,而且你在学校自己一个人孤零零的不寂寞吗?难得来学校,高兴地过有何不可?」

    「我又没说不行,只是——」

    她停顿了一下。此间,包含夕阳颜色的春风从窗吹了进来。窗帘也呼呼地舞动着,温和的风拂过芹菜的侧脸,撩起了她那长长的前发,眼睛也随之露了出来。看到她那长睫毛,黑色的瞳孔中映着鲜艳红光的眼睛,梓不禁吞了口气。芹菜就直直看着自己,嘴角上翘,如同讥笑般双唇弯曲。

    「我看你一点都不高兴就是了。」

    听到她那含有讽刺意味的话,梓感觉血一下子冲上了脸。这还是第一次被人这么直接地倾述恶意,那如同刀尖的话刺痛着梓的心。换做平常本应能从容应对的自己,此刻头脑一片空白。梓小小呼了口气,努力想以成熟的姿态装作不在意,不理会,然而此刻的梓无法做到,背着的长号,也显得格外沉重。

    「用不着你对我指手画脚。」

    一番努力后所挤出的话仅此一句,然后梓说完后逃也似的离开了教室。感觉芹菜还在看着自己,所以梓一溜烟跑下了楼梯。拜此所赐一天的心情都没有了。

    「梓你好慢啊,干什么去了?」

    从换鞋地方的死角处,传来了几分不看气氛的呆呆的声音。说话的人是同属吹奏乐部的黄前久美子。她看到梓的脸,就吓了一跳似地大声说道:

    「怎么了?脸色好差。」

    看到慌慌张张的朋友,梓慢慢地冷静下来。调整了下呼吸后,梓露出夹带苦笑的表情说:

    「没有啦,就是碰到了一个讨厌的人,搞得都没心情了。不过看见久美子就好多了。」

    从鞋箱里拿出自己的皮鞋,在换着鞋的梓的身后,听到了久美子短短地露出了惊讶。

    「这还是我第一次听到梓你说一个人讨厌。」

    「就是有那么讨厌的人我才说的啊。啊,想起来整个人都不好了,如果那时候没忘带笔记本就好了。」

    ——唉

    久美子看到梓发脾气地往地板撒气,就觉得好笑一样眯起了眼睛。久美子的头发有点翘,就这样弯弯地垂到了肩上。梓摸了摸自己绑成一束长马尾的头发,硬硬的触感从指间滑过。

    ——她的头发应该很软吧。

    不知为何脑中浮现出了芹菜那随风飘舞的长长黑发。一想到这,梓就皱起眉头,为什么自己会想到她呢?就在梓暗暗怪责自己大脑奇怪的运作机理的这一瞬间,已经开始在意起芹菜的事情来了。

    ——————

    「接下来就是大家期待已久的曲子了。其实今天是一年级生初次表演sing,所以因担心而来的一年级生家长也应该不在少数吧。二年级和三年级的前辈们也会在舞台下演奏为他们加油的,所以还请大家给一年级生们多多鼓励。那接下来请欣赏,sing sing sing。」

    熊田老师话音刚落,鼓就奏响了。刚才还沉浸在回忆中的梓,此刻绷紧神经,用力地握了握手。大家发出高亢的尖叫声,一边挥手一边以笑脸动了起来(表演的人发出尖叫,不是观众。请大家还去B站搜:京都橘高校)

    旗手在观众面前站成一列,一齐挥动起水色的旗帜。铜管乐器的人一起架起乐器,原地转了一圈。长笛和竖笛的人如同表演魔法般拿着乐器踏着舞步,梓让双腿与肩同宽,大大吸气。时不时可以看到苏萨号巨大的白色本体。和低鸣的低音一同,小号和长号一同迸出爵士乐的旋律。然后就听到了那熟悉的旋律。

    无数次的练习让身体不用思考也能跟得上乐曲的节奏。右脚,左脚交替踢向空中。然后双腿交叉,配合乐曲的停止了动作。之所以被人称为水色恶魔,就是立华的人可以在骤静与骤动这巨大的反差间仍可以笑容轻松应对。

    动作华丽让人目不暇接,部员都留意着不要和其他人撞在一起。因为长号的拉伸管很长,所以必须要注意不要撞到其他人的头了。同时还要注意横排和竖排,留意地上打的点站好自己的位置

    一开始是小号,之后是圆号,华丽的旋律超越了单个乐器本身把全部的乐器都连接了起来。梓左右摇晃身体不停拉着拉伸管。长长的乐器映出了旁边的人的动作。就像是铜管乐器之间的谈话,部员一边看着对方一边摆动了起来(swing)。和旁边的雨未华目光接触了,刚才还吃力地做着动作的雨未华,此刻像是松了一口气一样眼角也没那么紧绷了。依旧保持着嘴抵住吹嘴,梓也用眼神往她送去了笑意

    就算没有语言,音乐本身也能将彼此相连,这就是音乐的乐趣所在。此刻,只剩下打击乐的声音,大家排成一横排,进入了最终的环节。在同一时间,大家举起乐器,金色的喇叭一齐朝向空中,在全部人的动作完美同步的一瞬间,就是这首乐曲最让人激动的地方。

    在舞台的一边,领队南以认真的表情挥舞着指挥棒。以余光确认后,梓向长号吹了一口气。大腿此时也高高抬起,左右脚不停交替。以碎步前进,在调整队伍的时候,中途退场的旗手们手持不同于刚才的旗帜再次登场了。花音她们以笑容大大回旋着深蓝色和黑色的旗帜。整个舞台顿时被张开的旗帜给覆盖了。

    乐曲进入了变调,旋律更加激烈了起来。梓仿佛要燃尽身体里的最后一丝力气,直到结束的那一刻为此都不敢放松大意。现场的气氛原来越高昂,一体感也越来越强。接着所有喇叭都高高抬起吹响最后的一个音符,同时梓的情绪也来到了最高潮。

    「hai!」

    与欢呼声一同部员停止了动作。维持着双手展开的姿势,梓就这样停了下来。然后是观众激烈的鼓掌。梓大大吸了一口气调整呼吸,然后也听到了旁边的雨未华长吁了一口气。

    ——————

    happy con完美落幕了,家长们都对此赞不绝口。

    「立华就是厉害。」

    「其他学校完全不能比。」

    中学生也兴奋地互相谈论。而那之中,也有梓中学时候的后辈。说不定明年她们也会和梓,杏奈一样从北中升上立华。

    演奏会结束后,由一年级生负责收拾现场。在前辈开会商量今后日程的时候,后辈就收拾起椅子来。一把轻轻的椅子,拿多了也会重。不过把椅子放上推车,很快就可以搞定了。

    「呼,辛苦了。」

    刚才还默默收拾着绿色垫子的雨未华终于松了口气,她身体往右弯,拉了拉手。大家在水色的裙子底下都穿着运动服,所以都不会纠结裙子,快手快脚地行动着。只有穿着灰色短裤的太一,害羞似的拉着自己衣服的下摆。

    「辛苦了。」

    如此精神的是花音,后面跟着的美音则是一边打哈欠一边向这边走来。

    「累死了,好困。」

    「毕竟今天起得很早嘛。从一大早就绷紧了神经,现在终于可以松一下了。」

    为了消除肌肉的紧张,梓转着肩。在前面,美音把旗子和旗杆卷在一起收好。

    「下周就是来真的咯,也没有一种担子完全放下的感觉就是了。幸好吹奏大赛的练习是坐着吹,感觉可以轻松一下。」

    「啊~,吹奏赛也要来了啊。我们这边练吹奏的时候舞奏的训练也不能停啊。要死人啦。」

    「不过一年级能到A组的人也不多,所以也不用那么担心啦......不过梓你的话倒很可能去A组哦。」

    「没有这回事,吹长号的人也挺多的,也太难竞争了。而且反正还是前辈优先。」

    「不过听前辈说还是会以实力决胜负哦。好像是校内合宿的时候熊田老师会进行选拔。虽然舞奏那边成员是前辈决定就是了。」

    「选拔吗。」

    梓看了看空旷旷的体育馆,小声嘀咕道。可能是刚才搬椅子手有股铁的气味。梓往后仰抬起脚后跟,然后为了让脚尖碰地弓起了脚。花音这时拍了拍梓的肩。

    「我说你们长号还有小号那边有没有人对旗手感兴趣的?你们那边应该人数有多出来的吧,所以就想在舞奏前拉几个人过去。」

    「你们那边还不够人吗?花音你之前也拉过木管那边的人吧。」

    「那边的人拒绝了。前辈说还要一个一年级的,别人一听要被桃花前辈从零教起就可怕得不敢去了。」

    「也是,被桃花前辈教太惨了。」

    ——呃

    美音一听就皱起了眉。桃花就是副部长的小山桃花,乐器是大管,舞奏的时候担任旗手的训练。

    「什么,美音你也怕桃花前辈吗?」

    梓这么一问,美音就不停地点头。

    「因为那个人性格太冲了,而且打扮也夸张的。」(ぶりっ子、大意应该是装乖,但好像和文意不符)

    花音也挺出身体同意道:

    「就是就是,我们旗手被她教的时候都是心惊动魄的。亏她那种性格还能当上副部长呢,难道就不惹人厌吗?」

    「肯定惹人厌啦。不过她像个母夜叉一样都没人敢反对的。」

    看来大家对她的意见不小。很久前就听说她的可怕之处了,没想到竟然弄得没人敢去当旗手。

    「花音花音。」

    一直没说话的雨未华此时拉了拉花音的袖子。

    「怎么了。」

    「招人什么时候截止。」

    「嗯,大概配好位置之前吧。大约在吹奏大赛京都大会结束的那个时候吧。」

    「这样哦。」

    「雨未华你难道对旗手有兴趣吗?超欢迎你来哦。」

    花音的脸一下子亮起来。雨未华还害羞一样磨蹭着自己的指尖,然后像下定决心似的抬起脸来看向了梓。她那像小孩一样小小的软软的手仿佛寻求依靠般抓住了梓的手指

    「小梓你怎么想。」

    「什么怎么想。」

    「就是我去当旗手。」

    花音和美音都互相看着对方,梓有那一会被眼前这左右镜像的样子给吸引了。

    ——小梓

    雨未华摇着梓的手,她的手比平时的要热,大概这就是所谓的小孩温度吧。

    「既然雨未华你有兴趣那就去吧。毕竟你才刚接触乐器,这种时候各种东西都试下说不定能找到适合自己的。而且这次步法到最后你不是跳好了吗,所以如果你想去做那试一下也未尝不可。」

    「这样吗。」

    「不过感觉你和桃花前辈合不来耶。长号的前辈都挺好人的,不过你也听花音说了吧,桃花前辈的性格很冲哦。要是你因此开始讨厌吹奏了我就觉得不是很好咯。好不容易才喜欢上吹奏了,要是因为一个人的缘故又讨厌了我是觉得不是很值。」

    「嗯,说的也是。」

    「我中学那时候身边有人和前辈关系不好,所以这样的事我都见得过了。我不想让雨未华你也有这样经历。而且在长号部里我还能照顾你,你要是出去了就什么都得亲力亲为咯,我就是不放心你这点。」

    梓把想到的都说出口后,雨未华就一副微妙的表情像接受了一样点了点头。而在一旁看着的花音和美音终于忍不下去了大笑了起来。花音就一边笑着一边拍梓的背。

    「梓你也是瞎操心,又不是整天围着孩子转的家长。」

    「梓妈你好。」

    听到她们开自己玩笑,梓嘟起嘴:

    「才没有瞎操心好吗,雨未华呆呆的谁知道她什么时候弄出个大麻烦。」

    「我懂我懂。一看到雨未华就让人忍不住去照顾她。」

    美音也搭了自己的腔。而雨未华则害羞地不说话。花音则笑盈盈地挽住雨未华的左手。

    「哎哟你这孩子太幸福了,有梓这样优秀的妈妈感激都来不及呢。」

    「我很感激啦,要是没有小梓我绝对走不到这一步的。小梓一直以来谢谢你了。」

    「不用谢啦,是我自己愿意做的。」

    看到雨未华对自己露出天真的笑脸,梓别开了视线。

    ——我喜欢靠得住的自己

    那一天对志保所说的话,毫无疑问是梓的真实感受。

    仿佛不让梓逃跑似的,雨未华紧紧地抓住了梓的手腕。她那无暇的笑容,让梓的自尊心痒痒了起来。

    「小梓果然很温柔。」

    ————————

    换上了和平时一样的体育裤和T恤后,梓坐上了电车。雨未华的手提包里塞满了今天比赛用的服装。已经过了高峰期,电车里没什么人。梓和雨未华就并排坐在了空空的位置上。

    「哈——」

    雨未华张大嘴,发出了似叹气又似呻吟的声音。往常她那充满好奇心一直闪闪发亮的眼睛,今天也光是盯着对面窗户暗寂的夜景,看来她很累了。梓也伸直腿,晃着脚。梓今天很早就起来了,现在困得不得了。雨未华小声嘀咕。

    「哎,明天还有练习吗。」

    「嗯。」

    「每天每天都有练习。」

    「嗯。」

    「不过happy con完美落幕太好了。」

    「嗯。」

    「小梓你有在听吗。」

    「嗯。」

    「你完全没听嘛。」

    ——可恶。

    雨未华小小捅了捅梓的侧腹。发呆的梓才缓缓反应过来。

    「抱歉抱歉,有点累了。」

    「今天真的很累,我一直都在担心舞奏的时候步法有没有出错。」

    「嗯,我也是。怎么说,感觉要是冷静下来想之后要怎么做就觉得会做错动作。」

    「就是啊。我都出了一身冷汗。」

    「哈哈,不过幸好没做错。我小时候就很顿,转左就弄成转右。那时候和前辈对了一下眼感觉心都停了,吓死人了。」

    「小梓你也会弄错啊。」

    「都是很久以前的事了。」

    电车到了六地藏站。说起来,这个站是离北宇治最近的,在想会不会碰到熟人呢梓往门口瞄了一眼,就看到了一张认识的面孔。

    瞬间,喉咙窜过了空气,那冰冷的厌恶感抓住了自己的心脏。从额头留下来的汗滑过太阳穴。眼和她对上了。对面也有那么一瞬间吃惊地睁大了眼睛,接着她往自己走来,她的嘴唇弯成了弦月的形状。

    「这不是梓嘛。」

    这是梓已经听惯的怀念的声音。梓无意识地抱紧了包,毕业之后,这还是第一次和她见面。

    「......芹菜。」

    听到自己挤出来的声音,雨未华好奇似的看了看这边。

    「小梓的熟人吗。」

    「嗯,中学的朋友。」

    芹菜就这样笑着回答雨未华。这和梓当初碰见她时冷淡的态度截然不同。那时候遮住前面眼睛的长发,现在也剪短了,变成了斜刘海。她的头发染成不会被教导主任训的明亮颜色,裙子变短了,大腿也露了出来。笑着的她的睫毛也长长弯弯的,仔细看的话她是戴了假睫毛。

    虽然芹菜看了看周围,可最后还是站在了梓的面前。即便周围还有几个座位,可她并没想着去坐。

    「哇,小梓你的朋友好漂亮哦。」

    「谢谢。你们两个社团刚结束?还是吹奏乐部吗?」

    「嗯,练习刚结束,现在才回去。」

    ——小梓?

    雨未华催梓回答,梓只是暧昧地笑了笑。梓现在没有那个心情,在芹菜面前,梓怎么都冷静不下来。

    「我叫柊木芹菜,你叫什么。」

    「我叫名濑雨未华。」

    「哦,雨未华吗。听你不是说关西话,是关东来的?」

    「嗯,从东京搬来的。」

    「哦,所以你才不是说关西话。这边有交到朋友吗?」

    「嗯,交到很多哦。都多亏了小梓。」

    雨未华拉住了梓的手,两人的距离因而拉近了。而朝这边往下看的芹菜的眼神,一下子冰冷起来。眉头皱起,脸也拉起来的表情多了些许的端正,而一种魄力也因此散发出来。芹菜从以前开始就是比起开心的时候不开心的时候显得要更美。

    「雨未华你接触乐器有多久了。」

    芹菜一边问,一边若无其事地把脚叠在了梓的脚上。她就保持着若即若离的距离,把自己的脚和梓的脚重合在了一起。梓知道她的意思,把雨未华的手拿开了。芹菜对此满意地弯起了嘴角。而雨未华丝毫没有注意到这边的一举一动。

    「我是今天春天才开始的,还是新人呢。」

    「哦,这样啊。」

    「小梓对我很温柔,如果没有小梓我都不知道怎样。」

    从笑着的雨未华的脸上,可以看出她一点都没有话中有话的样子。雨未华就是所想即所说,然后她的这份直率,反而刺激了芹菜的神经。

    ——我说啊。

    芹菜的声音带有呛人的甘甜,而她的眼里至始至终只有梓一个人。她的双眸浮出了依恋和憎恶交织的感情,她此时已经完全没把雨未华放在眼里了,她现在的眼里只有梓一个。

    修剪整齐的芹菜的手指,温柔地拿起了梓的手。被用即离即若的距离摸着指间,梓紧闭唇尽力忍耐这骚痒。接着芹菜把脸凑近了低着头的梓的耳边,在她吐出的气息碰触耳朵的一瞬间,一股电流窜过梓的后背。

    芹菜愉快地说道:

    「梓还是一点都没变呢。」

    而这句话背后的含义,毫无疑问是轻蔑的侮辱。
    \section{紧张的滑步}
    暑假日程表上最初写的,就是「校内合宿」这简单的四个字 。结业式之后便正式进入了暑假。虽然暑假的第一周还有补课之类的麻烦事,不过一周之后就是彻底解放了。不过对于吹奏乐部的人来说,休息之类的字眼和自己完全无缘。

    一边盯着房间里的全身镜,梓一边准备行李。彩色的盒子里面塞满了各式的内裤,然后梓就随便挑了一个。虽然母亲有提醒自己说要穿适合自己胸型的内衣,不过梓对这方面倒不是很在意。中学三年级那时候买的内衣已经不合身了,那件内衣中间有白色的褶皱,还有小小的绀色丝带,还亏自己挺喜欢那一件的。梓不高兴地嘟起嘴,穿上了运动内衣。

    梓的胸比其他人的要大,而这一直都是梓的自卑之处,所以梓不是很喜欢自己的身材。因为在运动的时候,胸就碍手碍脚的,特别是立华舞奏的时候,上下晃的胸部实在是非常麻烦。

    「久美子就好了。」

    这时久美子的脸浮现了出来。久美子是苗条的体型,就算穿衣服,也不用担心因为胸部的鼓起而破坏图案的设计。

    穿好体育裤,在运动内衣上再套一件黄色的T恤,扭了扭上半身后,发现衣服有点短了,粉色的内衣稍微露了出来。梓拿起梳子,用粗鲁的动作将睡乱了的头发梳齐。压住跳起来的头发,口叼着皮筋,然后紧紧把头发扎成一束。确认没有绑偏后,再用黑色的发夹把其他的头发弄整齐。在耳后别上发夹后,梓终于弄好了。

    「......OK。」

    左转右转确认没问题后,梓用手拍了拍自己的脸。地上放着合宿装东西的旅行包,里面装有三天两夜的衣服。而接下来三天,就是立华校内合宿的时间了。

    ——————

    一百个人挤在音乐室里,空间就显得有点局促了。平时上课用的椅子密密麻麻地放着,因为是按正式比赛时候那样放的,所以人的密度高,温度自然也上去了。墙上贴着节约用电的提示,熊田老师打开了空调。

    「今天很热吧,在这样的天里还要大家像挤饺子一样真是辛苦大家了,不过还是请忍耐一下。」

    现在的音乐室,是完全按照正式比赛的那样布置的,用了许多的垫脚架,形成了一,二三层。第一层是摆满了圆号和上上低音号,第二层是小号和长号。如果平常训练是不会这样大费周章的,不过暑假后就没有必要每次练习后都要收拾,所以暑假的训练形式就常以这种形式进行。

    「那首先就来说说选拔你们要注意的地方吧,我们就一边合奏一边解释吧。」

    「是。」

    「上午是答疑时间,15:00后是个人练习。因为明天是选拔,所以我想大家都有自己要练的地方。」

    熊田老师用指挥棒敲了敲谱面。这只白色修长的指挥棒是熊田老师喜欢的东西,不过有时候演奏的太过激烈,时不时会敲断。

    梓架好长号。熊田老师所说的选拔,是为了在130名部员中选出55名A部门的人去参加吹奏大赛,就是说有超过一半的人没法去到A部门。

    立华是驰名全国的高校,而全国都有有能之士来到立华,所以竞争的激烈就可想而知了。在这里,能上场的人和辈分完全没有关系,只有实力够强的人才能登上舞台。这条简单明瞭的规则,大家都熟知于心

    ——————

    「吹奏大赛整天听到A部门B部门的,到底是什么意思。」

    听到雨未华的提问,梓放下乐器。银色的吹嘴反射着阳光闪闪发亮。用挂在脖子上的毛巾擦了擦汗,这时贴在谱上的便签脱落了。

    合奏完之后,部员就到各自的场地自练。这天长号所分到的是一年三班的走廊,前辈们在稍远的地方练习。为了让温度下降些,大家把全部的窗户都打开了。空气流通后,刚才积聚着热量的走廊也慢慢地凉爽了起来。把椅子搬到了走廊后,把乐谱架整齐地摆好。用文件夹夹好的乐谱是自由曲和课题曲的乐谱

    「啊,好像还没跟你说这个吧」

    梓把乐器放回箱子内。志保现在在练低音长号,吹着课题曲的一个章节。志保原本就很擅长坐奏,特别是她丝毫没有停止的那拉伸拉管的动作更是让梓惊叹连连。自从转成低音长号,志保的功底比以前更好了。

    「所谓的A部门,就是所谓的大编成。因为吹奏大赛出场的人数是55名。之前翔子前辈也说了吧,晋级顺序是府大会,关西大会,全国大会。然后A部门就是按照这条路线走。而B部门即所谓的中编成,则只到地区大会和都道府县大会,其他的还有C部门这种小编成的。因为A部门是能到全国的,B部门的话拿个金银铜奖就结束了,没有说还能晋级什么的,而且也只有自由曲。立华的话每年都会参加A,B部门,所以去不了A组的人就自然去B组了。」

    「原来B部门没有课题曲哦。」

    「嗯。不过其他学校早就开始练吹奏了,只有我们学校到暑假才开始练的,毕竟立华的重点是舞奏也没办法。怎么说呢,我们学校关注的是舞奏,而吹奏就听天由命的样子。」

    「毕竟我们一直练的都是舞奏嘛。不过最近坐奏的训练也多了,轻松点也不错。」

    「还是一门有一门的难吧。不过这么热的天是练坐奏倒是挺幸运的。」

    从椅子下拿起水瓶,梓大喝一口。听到透明的液体咕噜噜灌进身体里的声音和冰冷感直达胃的身处,梓才停下了喝水的口。到了夏天,好像怎么喝都喝不够,刚喝完马上又渴了,梓的包大部分的空间都让给了水瓶。

    梓翻看乐谱的一页,课题曲《为了吹奏乐的行进》(吹奏楽のためのマーチ)是宇乡晴义所作的曲子。曲风明快,在今年的课题曲里面也是一首受人欢迎的曲子。吹奏大赛的课题曲是5首里面选一首,所以在比赛时,审查员会听到同一首曲子好几次。

    在发表5首课题曲的那一天,梓把全部都听了一遍。而在此之中,梓最喜欢的就是这首。理由也很简单,因为这首曲子有长号solo的部分

    而这首曲子的第一音程能演奏很多富有魅力的旋律。高音的部分也很多难度自然也上去了,相对的,吹的时候也很高兴。看到那需要高演奏技术的谱面,梓觉得自己应该可以从容应对。然后自由曲是《宇宙的音乐(music of the sphere)》,这首曲子也是大家熟悉的曲子。这首经常在吹奏大赛上出现的曲子,是由英国作曲家飞利浦斯巴克所写。而上上低音号的独奏曲《哑剧(pantomime)》也是他作的曲子。

    《宇宙的音乐》是为低音乐团所作的曲子,之后经作曲家本人进行了吹奏乐了改编。而依曲名所示,曲子的内容是用宇宙为题材的,所以用7个连续的主题呈现,名字分别是《t=0》《大爆炸》《孤独的行星》《小行星带和流星群》《宇宙的音乐》《harmonia》《未知》。

    而以古希腊数学家毕达哥拉斯的理论为基础做成的这首曲子,时长是15分钟30秒,算得上是首大曲子。因为吹奏大赛有时间的限制,所以就以这个为编曲演奏。

    「不知道选拔是怎么样的呢。」

    雨未华的手放在了窗框上,看着那些在足球场上奔跑的人。

    「听说是一个人一个人去,然后不是有指定的部分吗,就是让你吹那个然后看你够不够格这样的。」

    「这就紧张了。」

    「长号的人很多啊。轮到一年级也得花不少时间吧,而且三年级有四个,二年级有五个,一年级也有四个,一个说不好可能就光是三年级的去A组了。」

    「说真的长号会选多少个。」

    「通常就四五个吧。」

    「那光是通过选拔就好难了。」

    雨未华耸了耸肩,而她身后睡了午觉的太一这时起来了,他直接是躺在走廊上的,所以背后满是灰尘。

    「其实我就想反正我都去不了A组,干脆直接偷懒练B组的就好了。」

    「你又说这种话了。」

    一直在练习的志保听到后就无语地看着太一。

    「你还是好好练吧,这是选拔耶。」

    「但这样效率太低了。」

    「什么太低了?」

    「就是说反正都不行,就不要浪费时间了。」

    志保听到太一那种说法皱起了眉头。不过太一本人却没感到任何不妥,盘着腿,他抬头看梓。

    「我知道佐佐木你练得很认真,而且我也认为你有可能去A组。不过怎么说呢,其他人就不怎么可能去A组了,那样还不如直接练B组好了。」

    「A的人不是还没定下来吗?那努力到选拔的那时候又有什么不好的?而且的场你认真练也有可能去A组啊。」

    「就算你这么说我还是有自知之明的。而且能去A组的,我想就只有西条姐妹还有佐佐木你了吧。」

    确实花音和美音在一年级中也是吹得格外的好,而且推荐入学的人实力也不会差到哪里去,不过即便如此,要是问是不是一定能去A这个问题本身就有些不合理之处。而非推荐入学的人之中也有实力不错的人,不过在那么多人的情况下,大部分的一年级生都不怎么可能去A。

    不过梓是不赞同这成为太一所谓不用努力也行的理由。梓的想法是,要干就尽力去干,不行就那时候再说。

    「你整天偷懒总有一天会吃大亏的。而且你有没有认真干未来前辈一眼就看出来了。」

    「这我倒不信你,像你这样一直拼命做,我觉得总有一天会跨。你们也这么觉得吧。」

    太一像寻求同意看了看志保和雨未华。志保是无语地叹了口气,雨未华则是低下眼露出不予评价的表情。

    志保没有发出声音,手一直动着拉伸管。低音长号的管比高音长号的要粗上不小,而应在管子上的她的脸,被拉得又细又长。

    「不觉得,而且我想马上练好。」

    「哈?」

    「现在练了说不定来年也有用呢?我是不明白什么是有用没用,我的天赋没你好,能做的就只有一步步来而已。就算在你眼里我很傻,不过交代给自己的任务我想一件件做好。」

    志保淡淡说的这句话,比起对太一说,倒不如是志保对自己说的。志保变了,自从志保决定负责低音长号的那一天起,就再也没说一句讨厌雨未华的话。可能是栞前辈的那句简单的「要拿出干劲不要偷懒咯」,成为了志保改变的契机。

    太一没想到志保会这么认真的反驳自己,尴尬地挠了挠头。他那薄薄的双眼皮,像是隐藏他那低迷的心情,不断眨着。说起来他也挺怪的,明明整天把自己不像样的一面展示出来,而在一些关键的地方又经不起他人的那么一说。

    「我们赶快练习吧。」

    是雨未华开的口。还是像平时一样,她露出纯真的笑容拉了拉梓的袖子。

    「小梓教教我,我还有不懂的地方。」

    「行啊,哪里不懂。」

    「这里。」

    她指着的地方是课题曲小号的部分。这里是以快节奏改变担任主旋律乐器的部分,确实挺难的。

    「你先吹一次看看。」

    「好。」

    用节奏器调好节奏,小小的摆子噶恰噶掐地小小发出愉悦声音。雨未华首先用力震动嘴唇做热身运动,然后才拿起乐器,往吹嘴里吹了一口气,让空气通过乐管。

    「好了。」

    「那三、四。」

    配合着节奏器的节奏,雨未华吹响长号。从E开始的旋律,各个音程的乐谱都不相同。而选拔的时候大家要吹的是第一音程的乐谱,另外还有几个指定要吹的地方,而熊田老师就是根据这个选出A的人。而低音长号的人则是吹第三音程的乐谱。

    「呼哈。」

    吹完指定部分的12小节,雨未华放下了乐器。梓就叉着手看着乐谱。

    「雨未华你有听cd吗,就是模板的那个。」

    「听了,不过一些细节听不出来。」

    由于长号是通过拉管改变音程的,所以自由度很高,相反的也是一种很难维持音程稳定的乐器。对于雨未华这样的新手来说,这是一个难点。

    「不过在说音程之前,你最大的问题还是合不上节奏,你首先把这点改一下。哼一下试试。」

    「ta--,tatata--,tatanta。」

    「停,那里是ta--,tatata--,你要停一下,每句的结尾你都合着上一句听起来很怪。」

    「这有什么不同吗?」

    「那里有八分休止符,雨未华你就直接跳过去了。好,和我一起来。」

    梓一边在桌子上敲着节奏,一边哼出曲子。把节奏器的节奏放慢,数次一起哼节奏后,雨未华终于是掌握正确节奏了。

    「那一起吹,你要配好节奏哦。」

    「好。」

    一开始放慢节奏,不停吹同一个地方直到吹好,然后慢慢提升速度。雨未华放松的时候吹出的音很漂亮,通透稳定。不过合奏的时候由于她太在意自己的音程,所以音色也萎靡了下去。而当前雨未华的目标,就是和梓两人和奏的时候也能吹出她自己一个人吹的水平。

    「嗯,终于行了。」

    吹了几次后,雨未华慢慢能行了。听到梓的称赞,雨未华也松了口气露出笑容。

    「嗯,终于能吹好了。小梓谢谢啦。」

    这时志保把椅子移了过来,她看了看梓和雨未华的脸,不好意思的说。

    「我也要确认一下第三音程,一起吹吧。」

    雨未华听后眼睛发亮,露出大大的笑容后用力点了点头。

    ——————

    23点后部员就在体育馆的一侧铺上被铺。因为是第一次在学校过夜,所以一年级的人都很兴奋。二,三年级的前辈因为明早还有事情要干所以早早就去睡了。

    「明天就是选拔了。」

    躺在被子上的花音嘀咕。她穿着粉色的运动服白色的卫衣,然后把脸埋在厚厚的被子里,手脚不停的乱动。

    「好紧张啊,我超紧张啊!」

    「你好吵。」

    而她正面在抚平床单褶皱的美音则皱起眉头抱怨。不过花音没理妹妹的话,就这样爬到了梓的被子上。

    「梓你在干嘛。」

    「嗯?就在看乐谱。夜晚也不能吹,就意念训练。」

    「好认真咯你。」

    「我就是认真你来打我啊。」

    「哎,别这么无聊啦,和我玩啦。」

    花音就这样抱住梓的腰,呼呼呼地摇着头,梓无奈只好停下翻着乐谱的手。一回头就见一脸叹气的美音。

    「我姐姐给你添麻烦了。」

    「没事啦,已经习惯了。」

    被夹在两个如此正式说话的人中间,花音嘟起嘴。

    「我才不麻烦,梓你也很高兴吧。」

    「是是,超高兴的。」

    「哇竟然这样对我,你好坏的。」

    花音撒起娇来也就这样了。梓苦笑摸了摸她的头,手指从她如丝般的黑发中穿过,感觉很舒服,梓就不自觉地一直摸了下去。

    「雨未华去哪了?」

    美音躺在没有一丝皱的被子上问。梓旁边雨未华的被子已经铺好了,不过却不见她的人。花音也抬起头来看了看周围。

    「对耶,不在喔。」

    「可能去哪里了,反正很快就回来了吧。」

    「不过栞前辈她们也不在,而且部长她们也是。」

    「对哦,都不在。」

    花音听美音说后挺起身来,三年级那边确实空旷旷的很显眼。

    「栞前辈她们大概在排舞奏的位置吧。」

    「哇,都这么晚了还要弄,太辛苦了。」

    「嗯,画位置确实很辛苦。」

    花音和美音都说着自己的感想。梓忍不住睡意,打了个大大的哈欠。把皮筋拿下来后,绑成一束的头发也散开落到了肩上。绑得太久,头发有些变弯了。

    「到熄灯时间了,请大家回各自的位置。」

    体育馆内听到了二年级吹竖笛的前辈的声音。说着话的人也赶紧钻进了被子里。美音和花音也回去了,梓也钻进了被子里,之后把脸埋在枕头里。定耳聆听,可以听到大家的呼吸声在体育馆中回荡。听到了从窗吹进来的风声,紧急出口的灯光也爬上了墙,靠着墙上的反光,梓定眼看可以见到有走来走去的学生。

    然后梓仰面躺好,慢慢闭上了眼睛。手就放在肚子上,深呼吸,感觉肺鼓了起来。虽说开了空调,不过夏天的晚上还是很难睡着。怎么都睡不着,意识就朦朦胧胧的。

    中学的时候也有合宿。那时候大家都在和室里铺被子,说一些恋爱之类没营养的话。只是一年前的事,现在想起来却感觉过了很久。抬起沉重的眼皮,只见天花板静静地看着自己。漠然地想,哦,现在自己在立华。

    「小梓。」

    突然右边传来声音,只把头向右转,就看到了刚从外面回来的雨未华正钻进被子里。她穿着桃色渐变成黄色的毛绒睡衣。

    「你去哪里了?」

    为了不吵到别人,梓小声地说。雨未华用被子把自己卷起来后,整个人转向这边,眨着大大的眼睛。蓝白色的光照在了她的脸上。这个染成深蓝色的世界正静静地呼吸。其他人呼吸的声音,像是海浪一样上涨便而褪去。

    「和前辈商量一些事。」

    雨未华就说了这么一句。

    「哪个前辈?」

    「桃花前辈。」

    听到意料之外的人,梓有点吓到了。事情到底要怎么发展她们两个才有交点呢?

    「你们谈了什么。」

    「也没什么。」

    雨未华说后就打了个大大的哈欠。然后挤出来的眼水就流过了她那圆圆的脸。梓看到那如同流星一般滑过夜空的水珠吞了口气。梓还想多看看她那美丽的水灵灵的双眸。

    「都聊到这种时间了,不可能没聊什么吧。」

    「嗯,就稍微谈了下。不过关于那件事没怎么聊就是了。怎么说,就漫无边际的。」

    「话题吗?」

    「嗯,话题。」

    ——嘿嘿

    雨未华挠了下头。因为她平时的话也不怎么抓得到重点,所以梓也没多问了。

    「那开心吗?」

    雨未华轻轻地笑了。

    「不知道,就紧张。」

    「这样啊。」

    「嗯。」

    雨未华的手伸了过来抓住了梓被子的一端。从袖口伸出来的手很纤细,看起来不怎么靠得住。小小声叫了她的名字,不过她没有回应。她的唇闭成一条线,刘海也偏向了一边,露出了她的前额。仿佛忍耐着什么心酸一样,她皱起了眉头,抓住被子的手在微微发抖。

    「小梓你会不会觉得我很麻烦?」

    「突然间说些什么?桃花前辈对你说些什么了吗?」

    「......不是,就有点在意。」

    放开被子,雨未华的手有些犹豫地伸到了梓的脖子上。被人碰到了敏感的地方,梓抖了一下。为了忍耐这种瘙痒,梓握紧了手。雨未华一直盯着自己

    「小梓,我给小梓添麻烦了吗。」

    梓吞了唾液,用舌舔了舔干燥的嘴唇,感觉有点刺刺的。

    「这种事一次都没想过。反而因为有你在所以我每天都很努力。」

    「真的?」

    「真的。」

    梓点了点头。听到了头发和布料发出的摩擦声。刚才还绷着脸的雨未华,此时终于露出了笑容。像是那条静静的神经终于松了下来,她的嘴也柔和了。梓用被子把自己的脚卷起来,用力握住了雨未华的手,想通过自己的力道,把自己的心意传过去。

    「雨未华你保持这样就好了。尽量拜托我就好了。」

    她点了下头,不知为何,她那笑脸,看起来像倒是像哭泣。

    ——————

    早上的世界属于梓自己一个人。在太阳升起来之前,体育馆里的空气清爽安静。周围还没有人醒来。梓一下子坐起来,然后就这样把身体往前弯。往旁边看雨未华把脸埋在枕头里睡着。用手摸了摸她的头发,她就更用力把脸埋了进去。

    头发都没梳,梓就离开了体育馆。在太阳还没升起来的这一段时间,空气有种说不清的透明感。拧开操场旁的水龙头,细细的水流就从水龙头里流了出来。洗手,然后也利落把脸洗了。凉凉的水拍在火热热的脸上很舒服。用毛巾擦了擦脸,感觉整个人都很精神。

    在操场前的楼梯坐下,梓静静伸直了腿。抬头,看到透过围栏太阳慢慢升了起来。那微暗也被朝阳驱逐,阳光就在无人的操场上延伸开来。以阳光为界的夜与白天,夜色的蓝也被赶到了一角,然后完全消失。

    梓喜欢早上,看到这个无人的世界,有一种整个世界都是自己的感觉。那并排的课室,延伸的走廊,只有这一刻,是完全属于自己的

    「嗯~~」

    拉了拉身子,梓震动自己的嘴唇。然后把嘴张开闭上,深呼吸。空气清新,梓的心情也更加愉悦。把包在手帕里的吹嘴拿出来,抵在了嘴唇上。注意不要太过用力把嘴压到变形,梓往里面吹气。低音慢慢的呼气,而高音则需急促地吐气,通过这样的方法可以让吹嘴发出的声音变化。

    这样的时间相当悠闲,现在整个空间里只有自己发出的声音。用鞋磨了磨楼梯,可以听到砂砾的声音。在脑中浮现乐谱,梓只是用吹嘴吹出了课题曲的旋律。

    第一,二,三音程,就算同是长号,各个人的负责也不同。梓当然知道每个音程都是乐曲的必须部分,也知道每个音程间没有说谁高誰低。不过梓就是想吹第一音程,自己想吹那华丽的旋律

    「梓你起得好早。」

    听到背后有人叫自己,梓马上转过头去。未来前辈就站在自己的身后。她的头发还挂着水珠,看来是刚刚洗完脸。用挂在脖子上的毛巾擦了擦,未来露出了洁白的牙齿

    「早,早上好。」

    未来用手示意梓不用着急站起来,自己也坐在了楼梯上。她穿着T恤和短裤,肌肤露出来很多。从灰色的短裤中可以看到她那晒黑了的大腿。优美曲线的小肚腿和紧致的脚跟,穿着黑色才刚及脚裸的短袜,袜子上有用粉色的线小小绣着。

    「这么早就开始练习了?」

    「对,不过这种时间也不能吹,只能用吹嘴代替了。」

    「也是,现在还很早,会吵到人的。」

    未来笑着摇了摇身体,然后她抱住自己的大腿换成了体育坐的姿势。把下巴抵在膝盖上,然后打了个哈欠。

    「我以前就觉得梓你好可怕。」

    「哪里可怕了。」

    「怎么说......就是你的上进心。你就是那种努力天才,永远不会喊辛苦的。」

    「这不是未来前辈吗,我听栞前辈说了,未来前辈你也很努力。」

    「栞她这样说了?真是的,尽说些让人害羞的话。」

    梓静静地看着未来的侧脸。平时她那清爽的笑容,今天看起来有点僵硬。仿佛像逃过自己的视线,她垂下了眼睛。她那薄薄的眼皮之下,到底看到些什么了呢。梓不知道,梓和她是不同世界的人。

    未来说:

    「梓你来了之后,我才第一次对那种事有了实感。」

    「那种事?」

    未来依旧垂着眼,她的唇有些自嘲地弯曲了。

    「就是看到别人努力就觉得好可怕。」

    梓不懂未来的意思歪着头。握在手中的吹嘴已经变温了。为了不划伤,梓小心用手帕把吹嘴包了起来。刚还闪着光的吹嘴,一下子就被埋在了厚厚的布里。

    「你知道我是上了高中后才开始吹乐器的吗。」

    「知道,我听栞前辈说了。那时候我还不相信前辈竟然是上了高中后才开始的。」

    「因为那时候我完全不知道这间学校原来这么强,也不知天高地厚就入了吹奏乐部,一下子就被那练习量吓到了。一开始我完全跟不上,那时候可惨了。」

    「不过现在不是第一了吗,我真心觉得前辈很厉害。」

    「第一吗。」

    ——哈。

    未来叹了口气,把腿直直伸了出去。很少有机会可以看到平时无懈可击的她此刻竟然这么漏洞百出。想到自己还纠结礼节也挺奇怪的,梓也把脚伸了出去。明明两人的身高没差多少,可就不知为什么未来的脚比自己的长。

    「栞那时就负责教新手的我。就像现在梓你教雨未华一样,全部从零开始,不行的地方栞会尽力陪我练到行,读谱的方法也好,拉管的位置也好,全部都是栞教我的。」

    ——不过嘛。

    未来叹了口气。这种带有自暴自弃的口吻,让她看起来比平时更加幼稚。

    「二年级的时候,不知什么时候我就比栞吹得还要好了。接过前辈递过来的乐谱,我就看了看,想了想就吹了。然后我永远忘不了栞那时候吹不出来而一片煞白的脸。她那时候只说了句‘我自己去练下’就离开教室了。」

    ——那时候我还不知道发生了什么事。未来苦笑道。

    「我喜欢练习,不过比起练习我更喜欢看到不断成长不断超越别人的自己。就有那种咚一声露出头的感觉。而且那时候我不是新人吗,所以更能感觉自己吹得越来越好了。像是「哦,能吹出音了」「能吹乐谱了」「步法也OK了」「能和别人合奏了」这样的。看着自己能做的事情越来越多,就越来越快乐,而且大家还会表扬自己,一来二去就停不下来了。」

    「那未来前辈果然很努力。」

    「比起说是努力还不如单纯说是喜欢吹得越来越好的自己,说到底就是不觉得努力是一件辛苦的事情......对于我来说,栞一直都是我的榜样,被栞赞了,就很开心,然后就练得更加多。不过在栞看来,这一切尽是痛苦的事吧。」

    「我倒是不觉得有。」

    「有的,那种事情是有的。」

    未来打断了梓的话断言道。从她的断言中得知,未来是真的这么觉得的。

    「在我追赶别人的时候,完全没有意识到被人追到底是一件多么可怕的事。因为追别人的时候,只需眼看前方,像一个笨蛋一样不断前进就可以了。不过我现在终于明白为什么栞那时候会露出那样的表情。她之所以对我温柔,肯定是她认为自己比我吹得好,就是她认为我不足以对她造成威胁,才会不求回报地仔细教我。不过就在那天以后一切都变了,栞已经不再教我乐谱了,也不知什么时候我就成为了大家口中的第一,而且周围的人也理所当然地接受了这种说法。回过神来,我就从追逐者变成被追的人了。」

    未来十指交叉,把自己的脸埋进了手中。闻到了她黑色头发传来的柠檬香气。而从她T恤露出来的纤细的手腕,留下了用力抓后的红色印记

    「——好可怕啊。」

    她那沙哑的声音宛如把胃里最后一点残留的感情都挤出来一样。听到这预料之外的话,梓感觉心脏被用力捅了一下。不知为何,此刻只有脚底的感觉异常明晰,为了缓解紧张,梓不停弯曲脚趾然后再伸直

    「原来前辈也有害怕的时候。」

    过了好一会,梓才挤出这句话。

    ——害怕啊。她重复了一句。

    「选拔也可怕。去年还不这么觉得,可今年一想到得拿出点成果,就觉得一切都可怕起来了。」

    一口气说完后未来一下子站了起来,如同甩开这种忧郁的心情,她用力拍了拍自己的脸。有点太大力,她的脸红红的。接着她看向梓,眼神一下子柔和了下来

    「抱歉抱歉,一大早就让你看到不成体统发恼骚的前辈了。」

    「啊,没,没关系。」

    梓用力挥了挥手让她不用在意。未来往天伸直了手,顺便拉了拉身体。像起立蹲下那样弯起膝盖,一下子站了起来。梓就在旁边呆呆看着。乍眼一看她的脸宛如一名美少年,剪得短短的头发给人的感觉很干练。梓无意识地卷起了自己的头发

    「那个。」

    口一张开就不自觉发出了声音。

    ——嗯?

    未来转头看自己,而她的脸上,已然没有刚才柔弱的表情。

    「刚才的话是因为是我才对我说的,还是我恰好在这里就对我说的?」

    未来像猫一样眯起眼,用手拍了拍裤子的灰尘,把上半身歪向一边。

    「这话是我想跟梓说的。」

    未来以若无其事的口吻说,然后她的嘴愉快地上扬。

    「其实我是想告诉你,就算有的前辈看起来很坚强,其实也有意外软弱的地方。我不觉得把自己软弱的一面露出来有什么不可的,而且也不觉得拜托别人是一件不行的事。梓你不是很会照顾别人吗?但相对的你就不懂拜托别人。」

    被未来说中了自己的性格,梓一下子禁言了。就感觉很羞耻。原来在前辈的眼里,自己这种性格竟然有缺陷到要让前辈特别关照自己。这可不行,还得继续努力,还得干得更加好,不能让前辈担心。梓紧紧握住手帕,露出自己最好的笑容

    「哎呀,都被前辈看透了,前辈真厉害。」

    「对吧。」

    「能和前辈说上话太幸运了。」

    「嗯,我也是。」

    未来轻轻拍了拍梓的背。她挺直躯干,双眼直直看着梓。而刚才那吐露不安的嘴唇,此刻发自内心愉快地说道:

    「选拔加油哦。」

    梓用力点了点头。未来转身离开了操场,世界再次寂静了下来。阳光耀眼,梓不禁眯起了眼睛。

    ——————

    在饭堂吃完早餐,大家都到各自的地方练习了。选拔的顺序是竖笛,木管,铜管,打击乐器。在被前面的人叫到为此大家都练习着指定的章节。

    长号在小号后面。一想到现在选拔在家庭教室进行,就有点静不下心来。梓坐在地上抬起脚让脚离地,往腹部用力,然后作空气单车的动作。心也慢慢收了回来。

    在稍远的地方,未来在练solo的部分。而旁边的栞也在练solo。无论是二年级还是三年级的前辈都是冲着solo而去的。每个人都想去A,都想登上正式的舞台。不过在13人中,到底有多少个能被选上呢?而被选上的人里,究竟有没有自己的名字呢?为了压住自己猛跳的心脏,梓满满吸了一口气。然后将空气从吹嘴吹进,喇叭就发出了宏亮的音色。低音到高音,慢慢地吹响长音。没问题的,在老师面前谁都是一样的,只要按平时那样来就行了。

    梓就像为自己打气,用力吐出一口气。这时,教室的门打开了,那里的是一年级小号的人。

    「小号结束了,请从高年级到低年级去家庭教室。」

    前辈都站了起来,而用手夹着乐谱的未来,表情看起来多了几分僵硬。

    家庭教室的门外摆着五张凳子。大家都端正坐在那里,静静等轮到自己。二年级的前辈从课室走了出来,接下来就轮到梓了。梓合上乐谱站了起来。

    「小梓加油。」

    雨未华小小声对自己竖起了拇指。紧绷的身体也松了一下。双手都拿着东西空不出来,梓就朝雨未华挥了挥。手搭上门把,大吸一口气,待肩起伏两次后,梓拉开了门。

    「打扰了。」

    听到顾问说请进,梓拿着长号走了进去。平时那些巨大的桌子现在也被摆到了墙角,中间就空出了一大块地方,只摆着乐谱架和椅子各一。然后椅子的对面,就是熊田老师。

    老师先看了看自己手边的纸,然后伸手示意坐下。

    「请坐。」

    「失礼了。」

    因为紧张自己的声音有点走调。把乐谱放上乐谱架后,张开脚只坐了椅子小小一部分。越到这种时候越是担心有没有调音这种早就做完的事情。

    「名字是什么。」

    「佐佐木梓。」

    「嗯,你好。」

    带着老花镜的熊田老师印象和平时不同,看起来要比平时可怕一些。她手握铅笔,静静地翻着乐谱,纸在翻动的时候,发出干燥的摩擦声。

    「那从指定的E开始,觉得准备好了就可以开始了。」

    「是。」

    拿好乐器,右手拿住支柱,梓的唇抵住了吹嘴。越过鞋底,梓弯曲脚趾紧紧抓住地面,挺直背,挺起胸,作出最放松的吹乐器的姿势,然后一边有意识地保持这个姿势,往吹嘴吹进一口气

    选拔通常要几分钟,梓吹出的音滑过谱面。那高难度连续高音的部分和凸显长号的部分,都是顾问专门选出来的重视长号的章节。可能每个声部都是要吹各自而言的重要章节吧。

    「好,那么请吹一下课题曲solo的部分。」

    熊田老师把笔放在桌上笑眯眯地说。听到solo这个词,梓吞了唾液。

    「能行吗?」

    熊田老师问道。梓用力地点了点头。

    「那请吧。」

    梓大大吸一口气,震动嘴唇。边意识要发出响亮透彻的音边慢慢地吹出旋律。从喇叭中传出的响亮华丽的音充满了整个教室。在短短的四小节里,梓把自己的全部实力都发挥出来了

    「——呼。」

    离开吹嘴,梓静静呼了一口气。熊田老师一边「嗯嗯」一边在纸上写着什么。然后她突然看向这边。

    「站起来。」

    梓有一阵子不明白她的意思,因为自己只听说选拔只要吹指定的部分就可以了。难道要加赛?熊田老师以认真的表情,把手放在了节奏器在,缓慢的嘎查嘎查声响了起来

    「从B开始,每个长音四拍你站着吹一下。」

    「啊,是。」

    听到意料之外的话梓的回答慢了一拍,还以为老师会叫自己吹指定之外的地方呢。梓搞不清老师的意识,站了起来。脚与肩同宽,放松,做出适合吹奏的姿势。为了不让紧张的嘴唇绷得太紧,梓动了动嘴唇

    「请。」

    梓大大吸了口气。让吹的部分既不是快节奏的部分,也不是难吹的旋律,也不是高音的部分,只是单纯的细细吹出长音的一部分。听着节奏器的节奏,从低音B开始,慢慢地提高频率,直直的,像是要让声音传到彼方那样的悠扬长音。把自己吹出的音和自己理想中的重叠。

    听到窗外的蝉叫,夏天的阳光也从窗的空隙中射了进来,时间的流动好像变慢,世界看起来也柔和起来,室内那带有热量的空气,也被梓吹出的声音充满了。

    「到这里就行了,谢谢。」

    老师合上双手,节奏器也停了下来,刚才的空气一下子就消失了。离开吹嘴,从唇间漏出了空气。不知道老师会怎么自己的长音呢,再加上不懂对方的意识,梓有些不安。

    「那请叫一下下一个人。」

    「是。」

    心脏跳得很剧烈,梓慌忙地把乐谱拿好,行了一礼后便离开了教室。刚出教室,其他人的视线便一同射向梓。梓还没能放松表情,就这样以僵硬的表情拍了拍志保的肩。

    「志保到你了。」

    「嗯。」

    志保吞了口气,拿起了放在大腿上的低音长号。她像芭蕾舞演员抬起脚后跟,把脚尖往前伸。

    「我去去就来。」

    伸直背后,她稍稍抬起了视线。梓从中可以看出她的干劲,所以只是默默的点了下头。志保的身影接着就消失在了教室里。然后透过窗户,听到了志保那凛然的低音。而梓则仿佛要从那音色中逃跑一样赶快离开。现在的自己还没有强到能从容听他人选拔的声音。

    ——————

    选拔的结果,将在当天晚上发布。合宿第二天,女生洗完澡正在用毛巾茶头发。而那少数的男生例如太一之类的,则被赶到了墙角。而那些聚在一起的男生部员,看起来也比往常缩了下去。

    「哇,终于要来了。」

    花音看了眼表,就哇哇地叫了出来。在地板上盘着腿坐的她看起来有一种不像是一年级生的大阵势。现在是22:00,大家吃完晚饭后就在体育馆集中。而铺好床铺的体育馆,还是那么得宽敞。

    「不知结果是怎样的好紧张啊。」

    不过那样的花音表情和平时完全一样,完全看不出紧张就是了。花音就这样盘着腿,翻着乐谱。总谱的音符看起来格外地多,看到那密密麻麻的音符,梓马上就跪了。

    「不过老师那时候突然叫我吹长音还真是吓一跳了。杏奈前辈她们之前都没说要叫吹长音吧。」

    梓坐在地上,往旁边的雨未华问道。雨未华好像不懂自己说的话,眨着大大的眼睛。

    「我就没有哦,就只是吹指定的地方然后就结束了。」

    「那有叫你吹solo吗。」

    「没有,我很快就结束了。」

    听到雨未华的话,梓不禁绷紧了脸。死了,自己的和别人的不同。如果是这样的话,大概可以从选拔要做的事情大概推一个结果。看来这话还是不要向志保和前辈说好了。

    不知道自己声部里有多少人被叫了吹长音,而那样做的结果,熊田老师又看清了什么东西呢?为了抹去这个疑问,梓把意识放到了别的事情上。

    「入不了A组的人之后去哪里。」

    雨未华认真的问道,看来她打从一开始就不认为自己能进A组。

    「大概去B组吧,或者直接开始练舞奏吧。」

    「哦,那B组的曲子是什么。」

    「那个啊,是《玫瑰肉》哦。」

    花音刚一问,美音就马上答到。哦哦,雨未华拍了拍手点头。

    「就是《玫瑰的谢肉祭》吧,之前前辈吹的那个。」

    「嗯嗯,你还记得真清楚。」

    梓摸了摸雨未华的头, 雨未华就像是被主人赞的狗狗,哼地自豪地挺起了胸。

    序曲《玫瑰的谢肉祭》,是意大利作家奥利弗·多蒂于1947年为吹奏乐所作的曲子。吹法多变,是吹奏大赛的课题曲。美音把烦人的刘海拿开,然后转向了这边

    「吹奏大赛A部门的话马上就来了吧,B是A的几天后?」

    「三天后。」

    ——哦哦,美音翻着日程表。

    「吹奏大赛关西大会是八月下旬,舞奏的京都预选赛是九月初。吹奏去不了关西大会也有好处呢,这样就可以把精力集中在舞奏上了。」

    「也是。而且舞奏到关西级别的话竞争就很激烈了。」

    就日程来说美音说的很对。不过梓不喜欢直接赞同那种观点,所以只是轻轻咬着唇。

    「小梓。」

    雨未华突然往这边伸出了手,然后她的食指就按住了梓的眉心。

    「别皱起眉,怪可怕的。」

    「啊,抱歉抱歉。」

    看来自己无意中把感情露出来了,这就是自己控制不了自己感情的证据。

    ——这可不行,梓闭紧了唇,手也握紧,指甲陷阱了自己的肉里。那刺痛,让梓恢复了冷静。疼痛是梓让自己冷静的方法之类,梓小时候就不怎么怕痛,治疗蛀牙那时也好,打针的时候也好,自己一次都没想过要逃避。

    「啊,翔子部长来了。」

    花音看了体育馆入口,就慌慌张张地站了起来。

    「大家集合。」

    听到话,刚才还聊着天的大家都一下子动起来。根据不同的声部分列站好。因为平时训练开了,所以整列的时间花不了多久。

    部长翔子,副部长桃花,领队南三人就在大家面前站成一列。翔子看了眼大家,慢慢走出了一步。

    「熊田老师接下来要公布A的名单,之后请A的人赶紧到音乐室集中。其他B组的人在这里铺好床铺后开会。明早A组的人直接到音乐室进行合奏,B组的人到第一视听室听副顾问城谷老师的指示。」

    「是。」

    「明天开始就有一段时间是专门练吹奏的,舞奏的正式训练是京都大会之后,请在那之前把乐谱记下来。」

    「是。」

    「明天气温有点高,注意不要中暑多喝水。如果感觉不舒服也不要顾虑是前辈还是后辈要和周围的人说。身体才是革命的本钱。请你们还是要在不累跨自己的范围里累跨自己。」

    听到她这貌似自相矛盾的话,有人发出了笑声。刚才还紧张的大家表情都稍微放松了。翔子看后满意地笑了笑,不愧是部长,在这方面就是有一手。

    「在熊田老师来之前大家前待机。」

    翔子刚说话,熊田老师就现身了。穿着黑色运动裤和卫衣的她那脖子上,挂着一个size有点大的秒表。她脚穿着胶拖鞋,每走一步就发出塔塔ta的声音。熊田老师站在了大家的面前。而她夹着的文件夹,不用说内容都知道是什么

    「晚上好。」

    熊田老师露出了平时爽朗那样笑容。而大家的回话则比平时更加紧张。

    「吊你们胃口也不好,就直接公布A的人吧,虽然课题曲和自由曲都有各自solo的部分,不过那我都一次性说吧。那首先是打击乐。」

    熊田老师往文件夹看了一眼,她那有皱纹的左手无名指戴着有些氧化的银戒指。

    「神田南。」

    「是。」

    第一个人,就是领队的南。然后老师一个个把名字念了出来。老师的声音淡淡的,并没有什么特别的感情。竖笛,长笛,萨克斯......念完木管后,就到铜管了。虽然名字大部分是三年级的,不过时不时也有二,一年级的人。当花音的名字出现的时候,周围有些吵闹。在长笛声部,她是与三年级生拉开差距第三名。而双簧管的美音,则到最后都没听到她的名字。

    「接下来长号。」

    小号发表完后,终于到长号了。梓握紧拳头,深吸一口气让自己冷静下来,心脏咚咚咚地在胸腔内剧烈跳着,而自己的耳朵仿佛泡在了水里,周围的声音都朦胧了起来。自己急促的呼吸声听起来很吵,梓用力弯脚,拼命压住自己发出的声音。

    熊田老师翻页,发出翻页的干燥声。

    「第一音程,濑崎未来。」

    「是。」

    在梓稍远的位置,听到了未来坚定的声音。不自觉往那边看后,就看到了栞的背影。从这边都可以清楚看到她那紧紧握着的拳头颤抖着。

    「佐佐木梓。」

    这个名字出来的瞬间,周围都骚动起来。梓吞了口水,大声回道:

    「是!」

    涌起来的感情,比起是喜悦,倒不如说是安心。一二三年级的人都看着这边。其中的感情各种各样,梓受不住低下了头。惊愕,接受,称赞,嫉妒,那感情的漩涡就直直压在梓的肩上。就在自己的呼吸都要停止的时候,一只温柔的手搭上了自己。抬起脸,雨未华的手包住了自己的手。和她对上视线,她也注意到自己,静静低下眼,她那双眸的深处,发出如日光灯一样的光芒,她黑曜石般的瞳孔映射出来的是祝福。

    「第二音程,高木栞。」

    「是。」

    「丹波惠。」

    「是。」

    「第三音程,尾上凛音。」

    「是。」

    「以上五名。」

    除了梓以外,全部都是三年级的。梓偷偷看了眼杏奈,她那长长的睫毛抖动着,而眼睛也泛着一层水膜,而为了隐藏自己的感情,她闭上了眼睛。看到了她的脸留下了一条泪痕。看到这的一瞬间,梓觉得自己的心脏猛跳了一下。在觉得可怜的同时,也认为这是理所当然的,因为杏奈还没有A组的水平。

    熊田老师的眼睛离开了文件,往这边看了过来,是看南还是看在后排的自己呢。判断不了,梓的喉咙窜了口气,感觉口特别干。

    「课题曲solo,濑崎未来。」

    听到了未来的回应。那话中充满了对自己信心。梓小小吐了口气,手还在微微地发抖,指尖没有触觉了,抓住自己的裤子,感觉手指被套了一层看不见的薄膜,世界就这样和自己隔离了开来。

    ——solo是未来前辈啊。

    用唇说出不带声音的话语。指尖的热量也渐渐消退了。鼻子深处有一股肿胀感,梓赶紧低下头,眼睛很热,浮现出来的感情,就只有那说不完说不清的悔恨。压住想叫出来的冲动,梓紧紧握住拳头,指甲在皮肤上留下了玄月痕迹。

    「接下来是圆号。」

    熊田老师继续说了下去,第一个被叫到的是部长翔子。圆号四名,大号四名,低音提琴二名,念出来的人个个都是有实力的,所以没有任何人发表不满的声音。可以听到在队伍的一边传来了不知是谁的抽泣。是被后辈拿了自己位置的三年级生,还是没有得到任何结果的二年级生呢。而与流露感情的前辈相比,一年级的人都一副事不关己的表情。

    「A组的人就是以上55名。离大赛没什么时间了,明天开始就要正式训练。特别是三年级,这是最后一次了,不要让自己留下遗憾拼命去干吧。」

    「是。」

    「那A组的人15分钟后到音乐室集合。之后是就寝时间,请大家坚持到最后。」

    「是。」

    「那先解散。」

    老师拍了下手,大家就和自己的朋友聊起天来。刚才还寂然无声的体育馆一下子就吵闹起来。还呆呆站着的梓后背被志保拍了一下。

    「恭喜了。」

    她的第一句就是这个,梓不知作何反应,旁边的太一也说:

    「我一开始就觉得你会去A组了。」

    「小梓果然厉害。」

    从雨未华笑眯眯的表情,很容易就可以看出她是真心为梓能去A组而感到由衷的高兴。

    「别这么说,真的会不好意思的。虽然只是好彩被选上,不过能上场真的好高兴。」

    这般学霸的发言自动就从梓的口中说了出来。挠头,作出不好意思的样子,向周围的人表示自己被选上的理由只是好运而已。越过这三人,可以看到二年级的前辈表情复杂。今年去A组的,只有四个三年级的和梓自己。因为梓,她们少了一个位置,所以对梓她们不可能有好的态度,所以梓也不能嚣张,要做出一副毫无敌意的样子。当然这是不能直接用说的,只能通过动作暗示。

    「梓妹~!」

    肩被人一下子抓住了,不出意外,是花音。

    「我们都去了A组,要加油哦!」

    而跟着她的美音则不服气地嘟起嘴。

    「要不是我是双簧管我肯定也去A组了。只能说双簧管的名额太少,没办法只有一个嘛,选不到也正常。」

    听到美音的话,花音愉悦地笑道:

    「美音你还是不服输,直说没选上不甘心就好了嘛。」

    「哈?我才不认为输给花音你呢,而且我绝对接受不了光是你去A组我没去。」

    「嚯嚯嚯,俗话不是有说吗,这个世界上不存在比姐姐的妹妹。」

    「双胞胎才不分什么姐姐妹妹。」

    「可事实就是我去A组了啊。」

    「啊!!看到你高高在上就气,下年我肯定去A组你到时哭还来不及呢。」

    「反正来年我也是A组~。」

    看到吐舌头的花音,美音气不打一处拼命拉她的脸。就连平时一直冷静的美音这时也露出了与年龄相符的一面。看到这对吵吵闹闹的双胞胎周围的人都笑了。

    露骨表现出不甘的美音和拼命晒自己技术的花音。如果梓换成是花音,绝对做不出她那样的举动,如果是自己,肯定会顾虑美音然后说些安慰的话吧。不过花音并没有那样做,梓觉得正是这样,才是一种心心相印的表现。花音也确信无论自己说什么美音都会接受吧。梓就做不到那样的,梓还没到能信人到那个地步

    「差不多要去音乐室了,而且都是前辈,还是不要太大牌。」

    花音用纤细的手指抓住了梓的手。

    「好咯,走咯走咯。」

    花音直直指向体育馆门口,然后就直直走了出去。梓也笑着追了上去。回过头,大家都已经把注意力放到别的地方了。而被志保搭话的雨未华,则露出害羞的笑容。到梓走出体育馆为止,雨未华一次都没有看向自己。

    ——————

    「呃——,好尴尬。」

    踏入音乐室花音就小声嘀咕了一句。只摆了55张椅子,和平时相比看起来空空的。想也是,立华吹奏乐部有130人,现在只有55人,有一半的人都不在这里。这里的人大部分是三年级的,二年级的人也显得有些难堪,就更不用说为数不多的一年级生了。

    「梓,这边这边。」

    就站在门口的梓听到了斜前方传来声音,是未来向自己招手。她拍了拍自己旁边的空位,那时第一音程的人要坐的位置,而那个位置,是夹在未来和栞之间唯一的位置

    其他长号的前辈也一同看了过来,被那些视线压到,梓不禁后退了一步,而花音则大力拍了拍梓。

    「那加油咯。」

    「嗯,你也加油。」

    梓对花音的鼓励大大点了下头。花音对此满意地翘起了嘴角。接着她就直直往长笛的位置走去了。梓也吸了一口气,然后用力拍了拍自己的脸为自己鼓劲。

    「哦,挺有干劲的哦。」

    未来开玩笑的说道,把脚放到了木质的舞台上,梓也笑着说:

    「是,因为我是认真的。」

    「很好。」

    平时坐惯的椅子今天坐起来有一种陌生的感觉。把手放在膝盖上看了看周围,右边和花音和前辈不知道在说些什么,看来她们已经搞好关系了。

    在梓观察其他人的时候,后面的门用力地打开了。那里站着的就是熊田老师。

    「好,会议开始。」

    她拍了下手,大家都安静了下来,老师就坐在正面的位置上,教室笼罩在了寂静中。对这种完全静滞的空气梓觉得有些可怕。熊田老师是优秀的指导者,能抓住每个人的能力因材施教。不过也没有任何证据表明老师所说的就是对的。大家都只是尽全力跟上老师的要求,梓也同样如此。可时不时会对自己的这份盲目觉得可怕。

    「这里的都是A的人。嗯,三年级30,二年级18,一年级7,嗯,今年一年级也挺努力的。在选solo的时候也想了一阵子呢,幸好三年级的前辈也把自己的实力拿出来了,不然都不知道该怎么选。」

    老师把文件放到了桌子上。

    「现在暑假,里吹奏大赛只有不到两周了。所以重点是在剩下不多的时间里能干多少。所以大家请拿出觉悟努力去练。」

    「是。」

    「而且你们不要忘记了,这里的55个人是从130个人中挑出来的,除了你们之外,还有很多的人想来A组,所以你们得拿出成果让他们看,要让他们听得耳朵流油拼命去练,知道了吗。」

    「是!」

    大家整齐的声音在夜晚的音乐室里回荡,老师满意看了一圈后笑了笑。

    「那明天9点开始,明天也是合宿的最后一天了,今晚早点睡,好好为明天准备。」

    ——————

    合宿的第三天是听着雨声醒来的。揉了揉眼睛,用手梳了梳乱糟糟的头发,看了下表,4点,现在离早上还有一段时间,周围也没有人醒来的。梓站了起来,然后推开了体育馆厚重的门,然后坐在了上面有露出来部分遮头的楼梯上,呆呆看着滴落的雨滴。雨把水泥地染成了黑色。

    「哈。」

    之所以泻出叹息是因为感觉透不过气来,可能是气压低吧。不知怎么的心咚咚咚地跳着,梓不喜欢下雨天,一到下雨天想起来的光是不愉快的事情。

    「早上好。」

    听到头上有声音,梓抬起了头。栞关上了厚厚的门,向自己露出了微笑。她平时露出额头的发型,现在也放了下来(大家可以想象律队的样子)。她把头发别到耳后,在梓的旁边蹲了下来。

    「前辈早上好。」

    栞用手示意梓不用站起来,她手掌上的纹路有些许的汗,反射着光。

    「梓你起得好早。」

    「不知怎么就醒了。」

    「你昨天也起得很早和未来在操场说话了吧。」

    「你听到了吗?那么直接过来也没关系啊。」

    「没啦,只是碰巧看到而已。没听到你们在说什么。」

    栞摇了摇头。因为地方小,所以两人的距离比平时要近。才看她的侧脸,直到刚才还呆呆看着前面的她突然转头看自己,长长的睫毛动了动。

    「恭喜你去A组了,第一音程。」

    梓一下子不知怎么反应。

    「啊,没......谢谢前辈。那个,也恭喜前辈去A组了。」

    「嗯,谢谢。听到去A组后才送了口气。」

    栞的手用力抓住了自己的膝盖。她樱色的指甲仔细地修剪过,不过皮肤有点点起刺。

    「不过还是不甘心啊,去第一音程的是梓而不是我,我也想吹第一音程啊。」

    「前辈想在未来前辈旁边吗?」

    「哈?」

    看到她皱起眉发出低沉的声音梓的脸一下子绷紧了。梓心想不好连忙补充道:

    「不,怎么说,我想是不是我坐在了未来前辈旁边有些不好这样的。」

    「没啦,而且要是为这个就生气我该是多喜欢未来啊。」

    「不喜欢吗。」

    「不是,喜欢当然是喜欢的,不过没喜欢到那种程度。」

    栞把手指交叉,手往前伸,然后手就暴露在了雨中,雨一下子就润湿了她的皮肤,栞对此毫不在意。

    「我想成为未来。」

    ——嗯

    梓暧昧点了点头。不过说实话,梓完全不懂这话的意思。栞看了过来。

    「你不懂吧。」

    一下子就被看透了,梓也只好双手投降。

    「对不起,完全不明白。」

    「也对,梓也是那样类型的。毕竟梓你没想过会输给同年级的人吧。」

    「才没有这回事。」

    梓不假思索否定了,同时脑中浮现出了中学那时候一个人的脸。高坂丽奈,北中的小号吹奏者。她的父亲是职业的小号吹奏家,所以自小她就开始接触乐器了。记得第一次听到她吹出来的声音,梓感觉就像被人打了一拳那样。

    「这还真意外。」

    栞是真的挺惊讶地睁大了眼。究竟自己在她眼中是怎样的一个人呢?自己貌似是一名看起来值得称赞的后辈。

    「对我而言我输给的人是未来,所以我想变成未来。」

    「哦,原来如此。」

    自己是报以认真的态度回答的,可栞却不满地眯起了眼,嘴也嘟了起来,她用手指弹了下梓的额头,看来她对弹额头没什么心得,听起来很大声,却不怎么痛。

    「梓你都没认真听我说话。」

    「我有听,不过听不懂。」

    「为什么,我还以梓你会懂的。」

    「因为我一次都没想过要变成别人啊,我就想以我自己超过未来前辈,一次都没想过也变成未来前辈。」

    这是梓的真心话,不像平时那样经过加工,而是原原本本的,发自内心的,没有其他任何别的含义的真心话。

    栞听后,喉咙抽动了一下,可以看到从衣服领口处露出来的她的喉咙小小动了一下。两人间落下寂静,雨势加大,水滴也更加大力地撞击地面。而那喧嚣个不停的雨声,唯独此时听起来让人愉快

    「你这孩子究竟怎么回事嘛。」

    栞把脸埋在自己的手中用力摇着头。梓看到她这意料之外的反应吓得不知道要说些什么。

    「怎,怎么了?」

    「没事,因为梓你真的太强了。」

    「没啦,就很普通而已。」

    「不不不,你真的不得了。我都不知道你要以自己超过未来,哈,你真是了不得。」

    「前辈你是不是在玩我?」

    因为对方是前辈所以有些顾忌,不过到这种份上梓也忍不住了。看到梓鼓起脸的样子,栞不知为何笑了。

    「没有玩你。只是觉得我自己很傻。」

    「其实前辈你自己也不知道自己说什么对吧。」

    「你这孩子怎么说话的,我只是单纯佩服你而已。因为你刚才说‘要以自己胜过未来’,这就很厉害了」

    对于栞的话, 梓还是不明所以。她究竟在意些什么呢,她和未来都是三年级的,就算直接宣布自己视未来为对手,也应该没有人会说流言蜚语才对。

    「直接说不出来不就好了吗,我觉得想超过比自己吹得好的人这种事很正常啊。」

    「是这样没错,不过虽然我看起来这样不过我可是很胆小的。一想到或许有人会在背后说我就那种实力竟然还想超过未来就觉得可怕。」

    「原来前辈在担心这个吗。」

    「当然啊。」

    ——哎,自己也真是自我意识过剩。

    她耸了耸肩。梓觉得这时说些奉承的话也挺怪的,所以点了点头。

    「怎么说......就那个吧。」

    「嗯,那个呢。」

    也不清楚「那个」究竟是哪个,梓一时想不出具体的东西,只好顺着点了点头。不过严肃不过三秒,梓一想到都不知自己说些什么,就不禁笑了出来。栞也跟着笑了起来。

    「都不知道我们在说些什么傻话。」

    「就是啊。」

    梓就坐着看着栞站了起来、她黑色的运动裤在膝盖的位置皱得很厉害,而裤子上那两条线,也随着她的动作弯曲。

    「虽然我那个,不过梓你也差不多。」

    「什么。」

    「就是自我意识过剩这点。」

    她的眼弯成了玄月形。虽然她是以开玩笑的口吻说的,不过之中可以听出她那对自己说法的确信。梓不自觉摸了摸自己的头发。那是长长的,硬硬的,直直的,不懂变通的头发。

    「我想你现在的立场也挺辛苦的,毕竟比前辈吹得更好嘛,不过至少我现在还没有把梓你看做是碍眼的人,所以你不用那么紧张也行。而且因为音乐之外的事情而劳神,也怪可惜的。」

    「劳什么神?」

    「谁知道,不过看到梓你就有那种感觉。」

    看到栞说完后低下眉头的样子,梓不禁想起昨天未来的话。

    ——我并不觉得把自己软弱的一面展示出来有何不可的。

    难道说两个前辈虽然表达的方式不一样,但却想对梓转达同一个意思吗?无论是未来还是栞,都是为后辈着想的前辈,对雨未华也好志保也好,她们两个都在支持着她们。说不定在她们两个的眼里,梓也是其中应该支持的后辈之一。

    梓笑了

    「没问题的,我和其他人不一样。」

    自己的话听起来比平时的要兴奋。

    ——这样啊。

    栞垂下眼,她的嘴唇稍微震动了一下仿佛有什么想要说的,不过雨将之抹去了。梓到最后还是无法推测她的心情到底是如何。

    ——————

    合宿结束后,练习也转成了吹奏大赛的模式。练习的时间也更加长了。从早上六点到晚上十点都是满当当的安排。梓与花音和前辈一起为大赛努力练习着。同时其他B组的人也为比赛准备着,并且舞奏的曲子她们也开始练了。

    今年舞奏的主题是「we love music,we are rikka」(爱音乐,立华人),这由四首曲子组成。乐谱已经全部发给大家了,在舞奏的正式训练开始前,必须要把乐谱全部记下来。梓是乐谱发下来的那一天就全部记好了,梓挺擅长记乐谱的,所以记起来不太辛苦,不过雨未华和太一就不是了。对于那些不怎么记得住的人来说把乐谱刻进脑子里这件事除了痛苦,就什么都不是了。看到雨未华一边发出呻吟一边死劲盯着乐谱的样子,梓也只好为她献上微笑。

    今天是难得的声部练习。暑假大家也不用在意会吵到别人了,所以平时的练习地点也从中庭和校舍背面换成了课室。气温高的话会对乐器的音调造成影响,所以能在校舍内练习可以说是万幸。

    「小梓,有个地方想让你教教我。」

    到了只有一年级练习的时候,雨未华和以前一样拿着乐谱走了过来。她的文件袋里也有玫瑰肉第三音程的乐谱。谱面上也是写满了演奏中要注意的地方,不过她写字的力气很浅,看起来挺费劲的。

    「哪里?」

    「就是inside blue。」

    雨未华指着舞奏的第三首曲子的中途部分。因为是爵士乐改成吹奏乐,所以有长号solo的部分。她给自己看的是第三音程的乐谱。立华舞奏出场的人是前辈决定的。第一音程这些也是由前辈分配好的。梓拿到的是第一音程的谱子,未来和栞也一样。

    「嗯,确实挺难的,你等等,我们一起吹看看。」

    梓停下课题曲的练习,拿出了舞奏用的文件袋。虽然是开学时刚买的,不过早就被乐谱塞得满当当的了。梓想着是时候清理一下了,把一些放在家里吧。在梓思考的时候,雨未华习惯地坐在了梓的前面。

    「就是这里,总是跟不上节奏。」

    「你用节拍器试过了吗。」

    「嗯,就像之前小梓教的那样从慢的练起,不过怎么都跟不上。」

    「我懂了,先吹吧。」

    「嗯。」

    雨未华点了点头,她以认真的表情拿好了乐器,然后吹起来。看到她鼓起来的脸,梓心想不是早就叫她改一下这个坏习惯了吗。一边听着雨未华吹出的声音,梓的眼睛从乐谱上扫过。说起来因为A组和B组的训练是分开的,所以最近都没怎么教过雨未华。梓反思着最近都没怎么理雨未华的事,同时用手敲了敲乐谱

    「这里再来一次。」

    「啊,果然很奇怪吗?」

    「嗯,你不会三连符吗。」

    「比起三连符,不如说我运舌不怎么行。这种慢慢的不是很行,反倒是那种快的还好。」

    所谓运舌是管乐器的一种演奏方法。演奏的时候通过舌头控制空气,以制造音的区隔和音的骤升。虽然这是基本的技巧,不过掌握好可不简单。

    「运舌也有很多种,那你刚才是什么感觉?」

    「什么感觉是什么感觉?」

    雨未华不懂歪着头,梓笑着说

    「tatata这样的是单运舌(single),然后toktok这样的是双运舌(double)。」

    「哦,是这个啊,我的是toktok,小梓你之前教我的那个。」

    「如果你会双运舌就没问题。三连符的时候用三运舌(triple)就好了。tokto,tokto这样的。」

    梓做了一次示范,就是让舌头数次轻触牙齿然后松开的样子。

    「哦,原来是这样。」

    雨未华看后,嗯嗯地点头。

    「雨未华你也试试。」

    「嗯。」

    雨未华拿起乐器,发出了鼓鼓的声音。

    「雨未华你是不是发不出k这个音。to的音很清楚,可到了k的时候就没有了,这样听起来就很奇怪。你接下来只用吹嘴试试。」

    雨未华把吹嘴拔了下来,把长号递给梓,然后嘴抵上了吹嘴。

    「也有边说banana边吹这种方法啦,不过最重要的还是中间出音的时候要意识到舌的动作。一开始慢慢来没关系,总之先把音好好吹出来。」

    雨未华就toktok地用吹嘴练着,梓听后嗯嗯地点头

    「这不是吹出来了吗,那你试下对着乐谱吹,注意点基本都是一样的。」

    「知道了,谢谢了小梓。」

    被雨未华这样满脸道谢感觉并不坏,自己也回了句「不用客气」。就在这出看着痒的交谈后,志保走了过来,太一也跟着走了过来。他们两个一起的样子很少见,所以梓不解歪头,雨未华也一样

    「你们两个怎么了。」

    「没,就有些话想说。」

    太一说完就闭嘴了。志保抬了抬细细的眼镜框,眼睛不安地眨了几下,仿佛犹豫着说接下来的话。然后沉默笼罩了四个人。梓受不了这种沉默,于是特意用开朗的声音说

    「有什么要说的吗?」

    太一和志保互相看了看,明明他们两个平时整天吵架,偏偏今天同一个鼻子出气。梓有种不祥的感觉,皱起了眉,梓的两手拿着自己和雨未华的长号空不出来

    「其实这个我们早就想跟你说了。」

    太一为难地抓了抓头发。他们不惜占用练习时间都要说的话究竟是什么呢。怎么说都不可能是交往宣言吧。梓为了看透太一在想什么就盯着太一瞧。他T恤上的米老鼠正向自己招手。太一吸了口气

    「我想说名濑你还是不要这么依赖佐佐木了。」

    听到这话,梓的身体僵住了。而雨未华也睁大了眼睛。梓赶紧反驳:

    「为什么,我没觉得有什么啊?」

    「虽然你是这么说,不过佐佐木你是A组的吧,还是集中精神练好自己的更好。而且名濑也和户川一样是第三音程的,让户川教也更好吧。」

    「可能我之前没说让你们多想了,不过真的没关系的。而且教新人的时候也能学到东西。」

    太一用大大的叹气盖过了梓一连串的话。

    「佐佐木你还把名濑当新人吗?现在都已经7月了哦,新人期早就结束了。」

    「可——」

    「当然到非得佐佐木你帮忙的时候还是会叫你的。不过现在不是那种非得你出场的时候吧。而且你现在是A的第一音程,有必须要在大赛上拿出些成果的义务吧,而且你是挤掉其他前辈的位置才上去的,不努力点可不行。」

    「你的意思是我没在努力吗?」

    自己的声音低沉了下去,一听就知道心情不好。而且梓不想被不努力的太一说自己不努力。梓自认为自己比任何人都要努力,而且因此,自己获得的成果也一个接着一个。

    太一无奈地呻吟说

    「你就适可而止吧,我知道你想教雨未华,也知道你想帮朋友的心情,不过现在你是A组的人应该没这个余力才对。而且我也可以教雨未华吧,户川也可以,杏奈前辈也可以。所以说如果你有时间教雨未华,还不如自己练好自己的。」

    「我不就说了我有在练吗?我是处理完自己的事情后才教雨未华的,所以没有理由被你说。」

    感觉脸一阵发热,涌起来的感情说是愤怒倒不如说是急躁。听到梓这样冲的话,雨未华抖了一下。她的眼睛看着太一,平时柔和笑着的唇此时也紧闭,脸色也不怎么好看

    「我是不是给小梓添麻烦了?」

    「雨未华你干嘛把的场的话当真?我有说麻烦了吗。」

    梓刚想站起来,肩就被志保按住了。志保就像安慰自己一样按住自己的肩,而梓也无意甩开她的手,只好又静静地坐下了。双手也拿着长号,不可能乱来。

    听到雨未华这么一说,太一摇了摇头。

    「对佐佐木你可能不麻烦,可对我们是麻烦。」

    「怎么说?」

    「因为这样下去,雨未华没佐佐木你就不行了。」

    ——有什么不行的?

    梓没想过从雨未华旁边离开,而且至今为止不是什么问题都没有吗?

    「雨未华你不用在意他们的话,我会一直帮你的。」

    梓故意让太一听清楚,所以一个字一个字念得清清楚楚的。雨未华往太一和志保看了一眼,最后还是点了点头。

    「......嗯,我知道了。」

    太一和志保无奈地耸了耸肩。志保拉住太一的手,像是放弃了一样摇了摇头,而蹙起的皱纹就清楚地刻在她的眉间。

    「没办法了。」

    她那冷冷的声音里,带着几分失望。

    ——————

    「今天真有点吓到了。」

    练习结束,终于踏上了回家的路。因为合奏的时候一直坐着浑身酸痛。有意无意地躲开路灯旁飞舞着小虫子,梓和雨未华并排着走在街上。

    「他们肯定是关心小梓才那样说的,小梓现在去了A组,在我身上花那么多时间也不好意思。」

    「没事,我又没在意。」

    「可他们在意吧。」

    入学的时候,梓的步幅比雨未华的要大,而现在已经调整为和雨未华相同的了。雨未华不知什么时候把五米八这个行奏的基本步幅刻进了脑子里,她走路的时候也和行奏的时候一样,先后跟着地,然后到脚尖着地为止把脚底像划半圆一样滑着踏出一步。而梓以前走路的时候也有这个怪癖。

    「志保也好的场君也好,肯定是觉得非说不可才说的。」

    「我知道,不过说实话他们真的多管闲事。」

    「小梓你是这么认为的?」

    「雨未华不是吗?」

    雨未华没有回答,只是停下了脚步。鞋底擦过泊油路,发出不愉快的声音。通过的车射出的灯光,把眼前照地煞白。

    「雨未华?」

    回过头梓叫了叫她的名字。温暖的夜风拂过梓的脖子。站在那里的雨未华,手抓住自己的半身裤,裤子也被拉得左右张开了。雨未华的裤子size有点大,可以看到她那没什么肉的膝盖露出了出来。裹住她脚裸的粉色袜子缝着色彩多样的图样。她黄色的运动鞋上,印着圆圆的星星图案。

    「小梓。」

    她用不甚明晰的带有甘甜的声音说出了自己的名字,如同将之后的话压回肚子里,雨未华走出一步。或许是新买的她鞋子上的绳子洁白无瑕。

    「我有想和你一起去的地方,现在有时间吗?」

    「现在?都快11点了。」

    「很快的。要是晚了我叫爸爸送你回家吧。」

    梓若无其事暗示时间不早了,可雨未华直接否认了,而且她还说了父亲会送自己,这下也不好拒绝。

    「好,就一下哦。」

    「谢谢。」

    雨未华双手夹着自己脸开心地说道。看到她这纯真的表情,梓也放松了力气。雨未华不会说谎,也不懂所谓的周旋,正因为她是表里如一的女孩子,所以梓才喜欢她。

    「就这里。」

    雨未华所说的是京阪宇治站的月台,也就是平时雨未华下车的站,她住的地方就是在这再往北走的公寓。

    「啊, 肚子饿了。」

    「去便利店买点东西吧。」

    走上楼梯,雨未华就走进了便利店,梓走向了卖点心的地方。泡芙,可丽饼,布丁,虽然都很好吃,不过自己现在想吃点饱肚的东西。

    「小梓选好了吗。」

    还在想吃什么,雨未华就从旁边伸出头来。

    「还没。」

    「我也不知道吃什么,小梓你帮我选吧。」

    雨未华就拉着梓往收银台走去。

    「肉包披萨包馅包......选哪个呢。」

    「你自己选吧。」

    「没事啦,就小梓选。」

    「哎。」

    在保温箱里除了肉包外,还有炸鸡,薯仔饼这样的熟食。可能是夏天,熟食放得比冬天少。梓想了好一会,才对店员说:

    「不好意思,肉包披萨包各一个。」

    「好,请稍等。」

    ——小梓也吃?

    梓付钱后接过袋子。刚走出店,夏天的空气就罩住了自己。梓就扇着自己的衣服往里面送气。此间雨未华一直盯着袋子瞧。

    「小梓付你钱。」

    「不用啦,请你。」

    「这不好意思啦,是我说饿才买的。」

    「那回去的时候不是雨未华父亲送我吗,就当的士钱好了。」

    「嗯,好吧......」

    雨未华还是一脸不好意思,才120円也不用那么在意吧。看到雨未华还有什么想说,梓直接换了话题。

    「雨未华你说想去的地方是哪里?」

    「啊,就在这里哦,塔岛。」

    「就是有石塔的那里对吧。」

    「嗯,梓也去过吗?」

    「我有朋友住那边。」

    雨未华对自己没想太多的话趣味津津。她抓住梓的包抬头看梓的侧脸。

    「朋友是那个人?」

    「哪个人?」

    「就是之前电车里遇到的那个好看的人......嗯,是叫柊木吧。」

    「你说芹菜啊。」

    雨未华用力地点头。

    时间不早了,周围没有人影。在那孤零零竖着的路灯照亮的地方之间,有几台自动贩卖机像是钻这个缝隙一样放在了那里。梓一看到季节限定的商品就忍不住去看,所以一直惦记着那大大写着宇治茶的饮料。

    「柊木同学说她是宇治站下车的,好像就住我附近。」

    「我黄檗下车后你们还聊了那些啊。」

    「嗯。在分开的时候她还叫我向你问好,不过——」

    对于这个不自然的停顿,不知道雨未华在想些什么,见她蹙起眉头。

    「不过?」

    梓催她继续说下去。在一片寂静的空间里,只有宇治川的潺潺流水静静作响。一排的茶屋早已关门,外面的木凳也没人在坐。回头,看到宇治桥连接着的道路有几台车从上面走过。到了早上,那些上班的人肯定会挤满那里。车能如此顺畅通过那里也只限于现在了。

    雨未华把话说下去:

    「总觉得小梓和柊木同学在一起的时候有种奇怪的感觉,觉得不太喜欢。」

    「奇怪的感觉?」

    「我也说不清,不过看到小梓和柊木同学说话的时候露出我一次都没见过的表情,感觉胃被人抓住了,搞得对柊木同学也没什么好感。」

    ——我觉得对小梓的朋友这样不好.

    雨未华说后就像跳起来一样大大跨出一步,她的鞋底和地面接触发出tata这样的轻快声音。毫无顾虑地踩在枯萎的杂草上,她转头看着这边。

    「柊木同学和小梓是什么关系。」

    梓低下头,手勾着的袋子被风一吹发出沙沙的 声音。雨未华把长长蓬松的头发别到耳后,再一次走到了梓的前面。

    「和她在冷战。」

    「可不像是吵架了。」

    「就各种各样的原因。」

    「各种各样?」

    雨未华一口气跑上桥的阶梯。她的手滑过银色的扶手,再一次跳了上去。这座红色的桥叫朝雾桥,连接着中之岛和宇治神社,如果面朝下游,则可一览爱宕(dang)山。太阳升起的那一刻更是绝景。朱色的栏杆夜的黑,目随彼方跟又追。梓若无其事地说:

    「就各种各样的。」

    「这样啊。」

    「嗯。」

    梓还是第一次对雨未华的问题这么敷衍过去,梓无论何时都想对雨未华坦诚相待,可唯独和芹菜的事情,不想对其他人说太多。和芹菜一起渡过的那段时间,说不上好也说不上坏,就是一段特别的时光。

    「这样啊。」

    雨未华耸了耸肩,她薄薄的唇有什么想说似的上下动着。梓就静静等着她的话,不过雨未华最后还是没深究,只是露出笑脸而已。

    「啊,是鵜。」

    她指着的是养着鵜的笼子。人工样的鵜是候鸟,即海鵜这个品种。养鵜人抓住野生的鵜后再训练。传统的服饰是风折鸟帽加腰蓑衣。宇治川在平安时代就已经有养鵜业了,不过之后受佛教影响不推崇杀生就消停下去了。然后西大寺的僧侣睿尊借由命令全面禁止养鵜的太政官符,将浮岛一带的渔具渔船全都埋掉,然后建起日本最大的13重石塔以供鱼灵。现在的养鵜业是大正15年之后才再次兴起的,所以也作为宇治川夏天的特别风情。

    「雨未华见过活着的鵜吗。」

    「之前和家人出去玩的时候见过。那时候就坐着船,原来养鵜人也有女的。」

    「嗯,那还挺有名的。」

    「电视看过一次,不过那时候还是第一次面对面看到的。好漂亮啊,那时候火光就射在河面上。」

    雨未华就想再现火光的样子挥舞着手。看到她傻傻的样子,梓笑了。

    鵜看起来很可爱,不过它们抓鱼的时候都是用它那锋利的爪子和嘴巴。养鵜的时候要用绳子把它们的脖子绑起来。这是为了鵜抓到鱼的时候吃不下去,而养鵜人就这样把鱼从它们的嘴中拿出来。而绑的松紧也很讲究,绑紧了鵜就呼吸不了,绑松了鵜就把鱼吃下去,吃饱后它们就自然不工作了。

    梓就时不时看向雨未华,雨未华对笼子里面很感兴趣,把鼻子凑到都快贴上去了。因为她低头所以头发都往前跑了,露出了她细细的脖子,梓也没多想,就用手指摸了摸,雨未华一下子就跳了起来。

    「突然干嘛啊小梓!」

    雨未华涨红了脸双手盖住自己的脖子。梓就一脸没事地说:

    「没,脖子就在那里。」

    「这是什么理由,别有空就开人家玩笑嘛。」

    说完雨未华又走了起来。看到她耳朵红起来,大概是相当害羞。

    ——抱歉抱歉。

    梓口上说着追了上去。雨未华看起来还在生气,大步走在了沙路上。而跟上那样的步伐,对梓而言则是恢复自己原本的速度而已。河边养鵜的游船已经停止营业了。虽然梓对雨未华说了关于鵜这样那样的话,不过梓还没在现实中看到过鵜,全部都是纪录片看来的。想起了电视中女养鵜人用绳子绑鵜的样子。不能松也不能紧,对于这种需把握程度的事梓很是苦恼。

    「这里,坐吧。」

    她指的是岛一端的木质凳子。白天经常可以看到在这里歇脚的人,而晚上怎么说也不会有那样的人了。

    「雨未华你想来的就是这里。」

    「嗯。」

    「为什么要特意过来。」

    「我喜欢这里。」

    雨未华坐了下来,梓也坐在了她的旁边。她小小的手往自己伸了过来,她看着的是梓拿着的袋子。

    「我要吃肉包。」

    「哎呀,都忘了。」

    梓把肉包拿了出来。然后分一半,把大的给雨未华。

    「分一半,我想两个都试试。」

    「小梓真醒目。」

    「对吧。」

    雨未华看起来真的挺饿的,手才刚拿到包肚子就叫了起来。梓假装没注意咬了一口包。白胖胖的包子料很足,刚下口肉汁就挤出来了。

    「小梓披萨包。」

    「已经吃完了?吃得太快会变猪哦。」

    「不会啦,每天社团都有训练。」

    「也是,从早练到晚,怎么可能会胖。」

    披萨包表面有点点的红色,分开时可以看到芝士拉成了一条丝。雨未华拿过后就大大咬了一口,她鼓起脸吃东西的样子就像仓鼠一样。

    「雨未华你为什么喜欢这里。」

    抬起头,便见黑色的水面。梓不喜欢夜晚的河边,因为看不到水底,感觉定不下心。

    「夜晚的河很漂亮,我夜晚经常自己一个这样散步。」

    「你还是不要这样啦,夜晚自己一个人多危险啊。」

    「可是这边挺有趣的,黄昏的时候也经常见到其他学校的人在这里练乐器。」

    「那你黄昏的时候散步好了。」

    「嗯,不过黄昏的时候人很多,就不太喜欢」

    雨未华把鞋子脱掉,摆好鞋子后就把脚踩上了凳子作出了体育坐的姿势,她把下巴放到了自己的膝盖上。看着缩起来的她梓默默咬了口包。茄酱和芝士混在一起的味道很怀念

    「之前我也说了吧,中学那时我就得过且过,都不怎么记得那时候的事情了。」

    「是说过,不过你说你那时是科学部的,那至少也有什么活动吧。」

    「......什么都没有啦,我那时是幽灵部员。」

    她转头看向自己,脸的一侧压在了膝盖上,紧紧地抱住自己的大腿。

    「我以前是一个不怎么中用的人,不喜欢去学校,朋友也没几个,又怕生不敢和别人说话,做事迟钝光是给别人添麻烦。」

    她的眼睛看着地面,像是拂去掩盖过去记忆的灰尘般不断眨着。

    「中学二年级的时候运动会上有班级间的百人一列比赛,你知道百人一列吗?」

    「嗯,就是百人一足的纵向版本吧,大家就踩在带子上像轮子一起向前走的那个。」

    「嗯,就是那个。那个时候我可是超拖后腿的,记不清摔倒几次了。虽然大家没有明说,不过我还是知道私底下他们都在说如果我不出场就好了。想也是,我要是摔倒了也会连累其他人摔倒还可能会受伤。在那之后我因为害怕惹人嫌就不敢看别人的眼睛了,就一个人呆着以此逃避。」

    梓想不到要说什么,只好咬着包。而雨未华也没想过要得到自己的看法,便又说下去:

    「想起来那时候也真是单调,起床,上学,上课,回家,百年如一日,回忆什么的完全没有。做个透明人静静地呆在一边光等着一天结束,那样的日子怎么会有愉快的记忆嘛。」

    她嘴角弯曲自嘲地说道,有那么一瞬间她露出成熟的表情,梓看呆了。梓原本以为自己已然熟知雨未华的全部事情,可看到她竟然也会有那样的表情,不禁觉得为自己的自大感到羞耻

    「爸爸工作地点改成京都也是那个时候。所以我就认为这是个机会。我想好好抓住这个机会做一个新的自己,我再也不愿意像以前那样在一角静静呆着了,我讨厌那样的自己。所以我就模仿受欢迎的人装成活泼开朗。看到有人是吹奏乐部的,所以我也跟着入吹奏乐部了。」

    「至今为止的都是演技?」

    梓感觉口干,说话的声音也有些抖,听到雨未华说的这个,自己有了几分动摇。雨未华眼神柔和把自己的嘴埋在了膝盖上。

    「一开始是。那时候就强迫自己装开朗。不过现在已经没法像一开始那样了。自己不清楚自己到底是怎样的了,也想不起原本的自己来了。只好把那样的自己当原本的自己了。」(竹田千爱和永濑伊织的结合,不过程度和前两人相比有点弱,只是到了那两人的程度,已经是病态的了)

    雨未华害羞地笑了笑,还是一如往常的纯真笑容。

    「一进吹奏乐部就感觉自己搞砸了,谁知道立华吹奏乐部竟然是那么辛苦的。不过幸好有小梓在,所以就慢慢喜欢上吹奏了。舞奏步法的那个时候也很担心自己搞砸了怎么了,幸好正式上场的时候没出问题。那个时候我才真切地觉得自己改变了,所以小梓,真的谢谢你一直在我身边。」

    说到这她的脸扭曲了,她痛苦似地短短吐出一口气。她的手紧紧抓住自己胸口,脚尖也高高抬了起来,喉咙中泻出了哽咽。

    「小梓是我第一个真正的朋友,不过就像志保说的,我是不是给小梓添麻烦了,我还是不要缠着你比较好吧。」

    梓没想太多一把就抓住雨未华的肩把她拉向了自己,放手的袋子直直落到了地上。

    「才没那回事,别人说什么你都不用管,我一点都没觉得雨未华你给我添麻烦了,你就尽管在我身边就好了。」

    「可是——」

    「为什么你这么在意别人的看法?我说行不就行了吗。我说了不麻烦就是不麻烦。你有什么做不到的,我就陪你到你做到为止,或者我可以替你去做。所以雨未华你什么都不用担心。」

    「这样不行,这样好可怕。」

    「什么可怕?」

    「如果小梓你不在我该怎么办。」

    梓的喉咙冲过一股气流,那一点点涌上来的热量蚕食着身体,那动摇的感情塞住了气管,胃像是灌进了一大口冰水打从心里凉透了。梓就静静看着倒在自己怀中的雨未华的发旋,垂下了眼睛。雨未华双手掩脸继续说道:

    「这样不行的,这样下去我就没法以自己的力量活下去了。每天回家躺在被子里的时候我都在想如果小梓不在了我不就活不下去了吗。一想到这个就害怕得不得了。我就整天在想如果我一直给小梓添麻烦,那有一天小梓不耐烦了我该怎么办。因为我都是被给予的,我一点东西都没给过小梓。」

    「你不用在意这个,我也不是为了你给我什么东西才帮你的。」

    不知自己的话是否正确,雨未华没有回答。梓摸了摸她的脑袋,她慢慢抬起头来,眼睛红肿,玻璃般黑色的眼睛闪着泪光。

    梓喜欢雨未华哭的样子,喜欢雨未华寻求帮助时抓住自己的那双柔弱的手。对梓而言,她发自内心需要自己的这个事实就是对自己的一种救赎。雨未华这样就好,她没有要以自己的力量走下去必要。

    「我们是朋友吧,那就不用想那么多了,按至今为止那样来就行了。是不是?」

    梓对雨未华笑了笑,为了让她安心自己揉了揉她的头发,听到了雨未华吸鼻子的声音。

    ——————

    雨未华父亲送自己回到家的时候已经稍稍过了12点。雨未华的父亲和雨未华很像,都是看上去和蔼的人。在白色的轿车里因为摆有角色的抱枕所以显得有点窄。雨未华不好意思地把那些抱枕都扔进尾箱。那些有点旧的抱枕貌似是她父亲的兴趣。

    「我回来了。」

    家里没人,母亲回来的时候大多数是半夜了。在放热水的时候梓把碗给洗了。戴上粉色的手套,把食物的渣拿起来放到三角框里(日本洗碗池的一角有小小三角的塑料框,专门放这些东西)

    虽然雨未华和她父亲没怎么说话,不过从气氛可以看出他们的关系很好。梓没有父亲,理解不了家里有异性的感觉。对梓而言家人就是指梓和母亲而已。

    「妈妈还没回来吗。」

    自己的嘀咕静静落下,洗碗池的水面微微泛起了涟漪。

    ———————

    随着吹奏大赛的日子一天天临近,训练也一天天加码。A组和B组的训练是分开的,所以梓这一阵子都不怎么见到其他一年级的人。声部训练的时候也是和A组的三年级前辈一起。花音和梓唯一能和朋友碰面的机会,也只剩下吃午饭的这种时候了。

    「啊—......」

    可能太累了吧,花音发出一声叹息。虽然她的筷子一直往嘴里送,可便当却没怎么动过。

    「花音你吃西北风?」

    「哇,都没发现。」

    花音就直接用筷子刺进了小番茄,她已经放弃用夹的了。梓从包里拿出瓶子,瓶子里只装了一半冻水,把瓶子按在额头上,感到一阵阵冰凉,瓶子表面已满是水珠了。

    「小梓没问题?」

    梓笑着回雨未华:

    「没问题,和舞奏那时比坐奏轻松些。」

    「对霍,啊~不过吹奏大赛结束后就是地狱mode的舞奏了,死人了要。」

    花音萎了下来,不见平时的活力。见此在一旁吃着炒面的美音坏坏地说:

    「看你一副夏困没胃口的样子要不要我帮你解决午饭。」

    「免了。」

    「别客气嘛,妹妹的好意你接受就好了嘛。」

    「你也只有瞄准我炸鸡的时候才装出一副妹妹样。」

    「谁叫妈妈的炸鸡那么好吃。」

    梓感觉有一股视线穿过吵吵闹闹的双胞胎向自己射了过来。看那边,就见志保看着自己,在眼神即将对上的那一刻,志保别开了视线。自从为雨未华那件事争论过的那天起,志保好像就一直躲开自己。

    「雨未华之后志保有没有跟你说什么。」

    问了雨未华,她摇了摇头。明明在地板上,雨未华却以正坐的姿势坐着。

    「不过志保和的场君好像一直都在想小梓的事情哦。」

    「的场也打算弄点什么鬼吗。」

    「没有啦。他是担心小梓在想怎么办才好。」

    「是吗。」

    看到雨未华为他们两个说话,梓有点不愉快。不过还是自我催眠自己也不是小孩子了,才不会为了这些小事而不高兴呢。

    「有什么麻烦事尽管和我说就行。」

    「嗯,小梓谢谢。」

    雨未华眼角垂下,她那表情看起来高兴的同时,也有那么一种被逼到绝境的样子。

    ——————

    「从课题曲C开始。」

    这天合奏专门是练课题曲。梓右边是未来,左边是栞。坐在未来旁边后才又真切感受到未来吹出的音既笔直又通透。她既能吹出正统的优美音色,也能吹出符合爵士乐那富有力量的凛冽的音色。不愧是第一,表现的领域就是大。

    「木管这里再来一次。」

    「是。」

    熊田老师用指挥棒敲了敲谱面。这个部分已经第五次吹了。看来现实的声音和老师理想中的声音还是有出入。

    「萨克斯声音太大了,长笛声音再靠前点。」

    「是。」

    「萨克斯管再大声点,单簧管就......先小声点看看。」

    「是。」

    「那再来一次,低音提琴也一起。」

    老师手翻着总谱,那本厚厚的本子记录着所有乐器的音符。和自由曲不同,课题曲不能由指挥者自行改变。课题曲讲究的就是在规定内完成度到底有多高。

    「啊,单簧管,声音再调回去吧。」

    「是。」

    「全部人一起来。」

    大家听指示架起乐器。教室左边看到大号竖了起来,大号有四架,低音提琴有2台,那边都是大型乐器,看起来特别挤。

    「三、四。」

    随着老师的指挥棒,柔和的旋律流了出来。课题曲的C部分主要以木管为中心,除了负责低音的铜管乐器外别的人都没什么要干的。梓前面的上低音号吹着副旋律,而梓则瞄准他们的缝隙拉动拉伸管。因为长号要往前伸,所以得注意不要撞到其他人的头了。

    「停,从铜管切入的位置再来一次。」

    「是。」

    「光铜管就可以。」

    木管的人把乐器放了下来。把长号架在肩上,梓吹起来。圆号,小号,长号,三种乐器一起把华丽的音色往前推。

    ——停。

    熊田老师把手一横。

    「光长号来。」

    「是。」

    梓像泵一样,把肺里的空气一口气吹了进去。

    「嗯,是第二音程,第二音程的音合不上。」

    ——是。

    三年级负责的两人短短回话。

    「长号音程必须得合上,好好吹。」

    「是。」

    「要好好听第一和第三音程的来。那再一次。」

    除了一口气吹完整首曲子的练习,吹奏的训练大体就是这样单调的一个个揪错误。把每个人的错误挑出来从而使55个人的音合为一体要花大量的时间,而反映到比赛上,这样数不清的时间就被压缩成短短12分钟。

    「行,C开始,全部人一起来。」

    全员对此大声回应。

    ——————

    练习结束后梓也习惯自己留下来继续练。这是中学留下来的习惯,如果自己练得不够别人多就感觉不安。梓想吹得更好,想超过未来前辈。

    三年级也有很多人留下来,不过梓不是很想和前辈们一起,因为就这样练自己想练的部分,可能会产生不必要的误会。从中学那时梓便得知吹奏乐部这个地方的可怕之处。越多人的地方各种小纷争也越多,被卷进去的几率也越大。所以对此要多少留意一下是团体生活的必须。

    一个劲只扑在音乐上而不顾他人的感受说到底只是单纯的任性而已。梓早有这种想法,并将之付诸实践,也多亏这个,梓至今没有直接被卷入大麻烦。

    拿起折叠式的乐谱架,梓往校舍深处的安全出口走去。这里基本没人,很适合练习。开灯后,窗上反射着模糊的白光。从外面看,应该可以看到校舍一角孤零零亮起来。

    乐谱放在乐谱架上,梓脚与肩同宽,也不在意音色就光是把气吹进去。喇叭低鸣,发出的声音震动着空气,而那不怎么动听的声音把梓内心堆积这的乱糟糟的心情带了出去。梓心情不好的时候都是这样做的,把肺里空气全部挤出去,光是这样心情就舒畅不少。

    从弱音(ピアノ,这里不是钢琴)到强音(フォルテ),慢慢改变音量吹起了长音。这次不像刚才那样,留意着音色,让自己吹出的音和未来前辈的音重合。能吹出的音和能用上的音完全是不同的概念。不管吹得多么大声,要是有损音色本身,那也无济于事。

    「呼——」

    各八拍的长音吹完后梓才放下了长号。梓喜欢基础练习,喜欢乐曲的练习,只要关于长号的,梓全都喜欢。翻开课题曲的谱,可见从E开始的solo部分用铅笔画了大大的一个叉表示这个部分不用自己吹。这是选拔结果公布那一天梓自己亲手画上的。solo的是未来。自己明白也理解了这个事实,不过对于这个部分自己放不下来。因为只要练习,机会总会轮到自己的,自己的机会并非是0。

    梓就这样吹起了4小节的solo。这个慢节奏的章节很美,长号那悠扬的声音如同在长笛和竖笛的旋律中穿梭。这一段已经在CD中听过无数次了,已经完全记下来了。

    从课题曲到自由曲,从开始到结束,自己就这样完整地过了一遍。弄清自己不行的部分,反复练习必须要注意的部分,梓最近每天都在重复着这样的工作。随着日历的格子一个个被划掉,感觉离比赛越来越近了。看到那渐渐逼近的日期,梓吞了口唾液。那高扬无法抑制的心情,让心脏鼓动。

    ——————

    「嗯?今天濑崎休息吗。」

    熊田老师看了眼坐得密密麻麻的部员,失望地说道。在大赛临近的日子未来之所以缺席是因为她有考试的说明会要去。老师的手摸了摸总谱,眼睛扫了眼长号的人,然后停在了梓身上。

    「佐佐木代替solo。」

    「是。」

    梓一开始就是这样打算的。梓有自信除了未来之外,吹得最好的就是自己。是满意自己的回答吗,熊田老师的眼笑了笑。她看了眼在课室后方的副顾问山田老师,她今天是为了测时间而来的。

    「那从头开始,准备。」

    「是。」

    按排水键,把管里的水放到水桶里。手放在拉伸管上,梓大吸一口气。课题曲从铜管和木管的协奏开始,同时加有棒敲铜拨的乐音。大号流出低鸣的旋律,圆号的声音付于其上。长号solo是曲子的后半段,在那里整体节奏欢快的曲子一口气转成缓慢的旋律。

    梓长号发出的音如同融入了竖笛的声音中,在只有木管构筑起的基台上长号的声音踏了上去。注意每一个音都得细心地,美丽地吹出来,如此一来,自己的声音便和未来的相差不大了。

    课题曲结束直接过渡到自由曲,在自由曲也吹完熊田老师的指挥棒停下来的那一刻,副顾问按下了秒表。

    「11分钟23秒。」

    「比平时快了11秒,又是那里抢调了吗」

    老师表情并不好看,往纸上写着些什么。老师所说的那里恐怕就是课题曲的G部分。因为那里是快节奏的部分,所以每次节奏都快了。老师挠了挠头,看了眼表。

    「我把注意的地方说一下,那从课题曲开始说起,镲一开始——」

    ——————

    合奏结束,梓去洗手台洗吹嘴。拧开水龙头水就哗哗流了出来。这里水压有点大,水在吹嘴上反弹溅到了梓的T恤上。

    「啊,好烦。」

    黑色的T恤被水染成一块块的,梓为了快点干扬着衣服。

    「辛苦了。」

    栞也在洗着吹嘴。她只是洗着没看自己,梓感觉不是很舒服。

    「前辈也辛苦了。」

    栞没理低下头的梓,只是用手指摩擦吹嘴的表面。流出来的水在栞长长的食指上左右流淌。

    「今天你替未来吹solo了。」

    「啊,是。」

    「嘛,想也是。」

    梓喉咙动了一下。梓从没想过栞竟然会对自己说这些,越过毛巾,梓紧紧抓住了包在里面的吹嘴。

    「如果比赛那天未来缺席了吹solo的肯定是你,大家都知道除了未来之外就你最强了。不过还请你不要忘记其他三年级的人也想solo。」

    「前辈。」

    「大家都很不甘,只是碍于面子没说出来而已。」

    说完,她就用毛巾擦干了吹嘴,然后紧紧拧紧了水龙头。

    「我想说的就是这个。」

    这时栞终于抬起头看梓,她那眼中的,是类似于嫉妒的神色。被她的气势压倒,梓后退了一步。栞之后也没说什么,直直走过梓走向了乐器室,梓唯有目送她离开。入学以来,这还是第一次被前辈这么露骨地吐露恶言。就算栞说了认同梓,也承认梓是第二名,但也还是忍不住向自己投来那带刺的视线,仿佛不想承认这个事实一样。

    连讨厌谁都不允许的这种人际关系只是单纯纯粹的麻烦而已。栞也是接受现实的,可即便如此,她还是忍不住向后辈吐出自己的心声。梓脑袋中冷静的那部分理解了栞被自己抛在身后的现实。而自己也乐观地对自己说到了明天栞肯定又会像平时那样对自己了吧。可梓仍然无法消除内心的不安,只是不知所措地呆呆站在原地。

    这天的最后,栞都没向自己露出过笑脸。

    ——————

    做梦梦见了小学时候和朋友在操场玩的样子。

    「来玩鬼抓人吧。」

    一个朋友说到。看不起她的脸,只看到模糊的轮廓和她笑起来的小小的嘴。梓小时候就很会躲,所以玩鬼抓人很强。

    「好啊。」

    「谁做鬼。」

    「猜拳吧。」

    ——剪刀石头布!

    看到有人输了,其他都立马跑了出去,梓也一样跟着跑了出去。鬼换人了,可没人追梓,因为梓跑得太快了,所以大家都想追了也白追。梓很无聊,于是故意放慢速度,把速度调成和别人一样。

    世界扭曲了,周围的人也长大了,她们的脸也慢慢变成了中学的朋友,在梓还来不及反应的时候,那张脸又变了,从关系很好的朋友变成了立华的前辈,前辈变成了抓人的鬼,追在梓的身后。梓逃了,可一旦认真跑起来,前辈就抓不住梓了。梓不想被抓住,可一想到被前辈抓住了,前辈也会高兴吧。

    「你不跑吗?」

    突然,前方转来声音。梓看到那长长的黑发,马上就想到是高坂丽奈。中学时同时吹奏乐部的,和她关系不是特别好,不过梓很是佩服她。丽奈就用不可思议的表情看着梓。

    「不快点就要被抓住咯。」

    「我知道,可这样对前辈不好吧。」

    「为什么?」

    丽奈蹙眉不理解梓的话。

    「为什么,那时因为前辈想抓住我嘛。」

    「抓不住也只能怪她们自己能力不够吧,我们只要跑自己的不就好了。」

    说完,丽奈又跑了出去,边跑她边回头看自己。

    「梓你跑不跑。」

    梓看了眼后方,已经不见前辈的身影了。无论前方还是后方都不见一人,这里只剩梓一个了,丽奈也消失了。

    「也只能全力跑了啊。」

    自己的声音被吸进了空旷旷的操场上。梓脱下鞋,就这样光脚站在了地上。跑的时候,钝痛的感觉不断转来,不过只要下定决心,梓就能一定跑下去了。鬼什么的怎样都好,自己已经下定决心了。前进,前进,前进进。

    ——哔哔哔。

    手机的闹钟响了。梓像平时那样伸出手,可这次却挥空了,抬起重重的眼皮,才发现自己不是睡在床而是在沙发上。看来回来躺了会就睡着了。累得连衣服都没换,就直接穿着训练时的运动服睡了。感觉梦到了什么,不过睡得太熟都不怎么想得起内容了。抬起身毯子就滑了下去。

    「早,睡那种地方可休息不好哦。」

    厨房换上套装的母亲已经在坐料理了。梓揉了揉眼睛,看着自己的妈妈。看到梓没有平常的样子,母亲问道:

    「怎么了,干嘛在那里发呆。」

    「没,不过妈你什么时候回来的。」

    「一点左右吧。一回来就看到你躺在那里,叫醒你也怪可怜的。」

    「可这样很碍事吧。」

    「可爱的女儿哪有什么碍不碍事的。」

    母亲给自己送了个飞吻,梓动作夸张躲开,然后站到了母亲旁边。厨房都是鸡蛋烧的香味,梓大大一了一口气,感觉满满的幸福。

    「熊猫眼都出来了梓你有这么累吗。睡眠不足可不行哦会变丑的。」

    把脸凑过来盯着自己的母亲一看到自己的脸就绷起了脸,她擦了膏霜的拇指摸了摸梓的眼袋。

    「没啦,就想了点东西。」

    「别想太多了,梓你整天都勉强自己。」

    「我知道啦。」

    梓心想又不是小孩子了嘟起了嘴,母亲看到呼呼的笑了。不过又马上担心地说道:

    「就快到大赛了吧,练得怎样了。」

    「就那样。」

    「答得这么不清不楚的。」

    「没......怎么说,就有点难。」

    「这回答不像你呢。」

    「是吗。」

    「嗯,不过有烦恼才是一件正常事。」

    然后母亲就把鸡蛋烧夹了起来放进了自己口里。里面加了紫菜,咬下去的时候有一股海的味道。自己默默吃着的时候母亲边把便当盒拿出来边问;

    「要不要泡澡。」

    「都早上不了,洗一下就可以了。」

    「嗯,不过都快到比赛了,要好好注意身体,露出肚子睡一个不小心就感冒了。」

    「嗯,谢谢。」

    母亲对走出厨房的梓开玩笑似地说;

    「阿拉阿拉,我们家的孩子还会这么坦率地道谢真可爱。」

    「好好。」

    敷衍一下后梓就走向了浴室。脱下袜子踩在地上有股冰冷贴在脚板底,感觉不错,梓就不断把脚开合开合。

    现在精力充沛,感觉无所畏惧。

    ——————

    「昨天抱歉了。」

    一进音乐室坐在位子上的未来就对自己就双手合掌。现在离训练还有一段时间,时间还早,没有多少人。未来这么早来也是少见。

    「没关系......不过前辈怎么了这么早来。」

    「昨天休息了一下感觉生疏了就早点来补一下。梓你都是这么早来的?」

    「通常是这个时间。」

    「雨未华呢。」

    「嗯?」

    突然听到雨未华的名字梓一脸不解。未来像是强调自己纤细的脚,把右脚笔直往前伸。因为未来在台上,梓得抬头看她。眼睛刚对上,她的嘴角就抬了起来。从她黑发之中,可以看到露出来的有点大的耳朵。

    「你不是和雨未华一起来的吗。」

    「开始练A组后就没有了。」

    「嘛也是,A组和B组的内容完全不同。」

    ——你不坐吗

    未来拍了拍自己旁边的位置,梓走上了舞台的第一层,这层是上低音号的。穿过座位的缝隙,梓才到了自己的位置。

    「话说昨天练习怎么样,solo的是谁。」

    「姑且是我。」

    「哦,果然是梓吗。」

    未来的声音听起来有几分奚落。她挺直背,把放下的乐器架在了肩上。因为在舞台上,所以音乐室整体看得很清楚,不过旁边前辈的侧脸因为坐得太近加上乐器的阻挡不太明晰。感觉有点尬,梓只好假装翻乐谱,乐谱上的solo部分画着一个大大的叉,梓摸了摸那部分。

    「练习一起吗。」

    梓大大地眨了下眼。

    「是。」

    看到自己没有犹豫点了点头,未来满意地笑了笑。

    ——————

    「明天就是比赛了记住千万别迟到了,早上还要把乐器搬上卡车,所以这点千万注意。」

    大家都精神饱满回应。立华为了节约经费,都是各自去会场的,因有担任接送的家长,这种方法也能实现。开卡车的是熊田老师,她有大型车的驾照。

    「明天就是了啊,感觉一转眼就到了。」

    刚解散,花音就走过来了。把运动服卷得高高的她忍不住眼困,打了个大大的哈欠。

    「我们能去关西吗。」

    「我们的重点是舞奏,吹奏怎样都好吧。」

    「也是。」

    在梓收起长号的时候,花音也没有走开。花音就像观察实验的孩子一样,好奇地看着梓收乐器的样子。

    「不过还是想去关西大会。」

    听到这,梓的手停了下来。用软布擦了擦金色的表面,就呈现了最初的光辉。

    「我想去关西。」

    看到难得认真的花音,梓睁大了眼睛。注意到了自己看着她,花音不好意思地笑了笑。

    「梓也想去吧。」

    「能去当然想去啊,能去多远就想去多远。」

    「哇,梓你超帅的。」

    花音拍手,可因为她是笑着拍手的,所以看起来像是开自己玩笑。梓苦笑,把乐器盒的盖子合上。

    「梓你忘了东西。」

    一个影子盖住了梓,眼前突然暗了下去,眨了眨眼睛。

    ——啊,前辈。

    同时听到花音慌张的声音。拿着调音器的栞就站在那里。

    「你就放在椅子上了,这个可不能忘哦。」

    她递出来的粉色调音器确实是梓的东西。梓心想不是自己忘了,只是放在那里待会回去拿而已。不过直说也不好,梓有意识地露出笑容说;

    「谢谢前辈。」

    「没事,不用谢。」

    栞挠了挠脸,不知为何她都交完东西了还是站着没走。花音和自己一样不解地看了看。栞把手放在额头上,流下的汗把运动服染成一块块的。她的口刚张开,便又犹豫似的闭成了一条线。那修剪整齐的细眉弯了起来

    「明天加油。」

    梓不知说些什么,只好点头,担心自己是不是被说什么坏话而在意着的时候,力气突然松了下去。栞破颜而开,那是梓见惯了的笑起来有些害羞的表情。

    「我想亲口好好说的就是这个。」

    然后她就嗒嗒嗒地跑开了。梓和花音看了一眼,泻出了毫无意义的声音,接着笑声混杂其中。花音翘起嘴抱住梓的肩。

    「有奸情,栞前辈喜欢你?」

    「没有啦,不如说恰好相反。」

    「可不会有人专门来和自己讨厌的人说话吧。」

    「就说没有嘛,怎么说......和好那样的。」

    花音一听,不解地说:

    「梓你和栞前辈吵架了?」

    「差不多吧。」

    解释也麻烦,梓就随便糊弄过去。看到缩了缩脖子的梓,花音笑地更加深了。她那眯起眼的样子,就像车站前的流浪猫一样。

    「嘛,不打不相识嘛。」

    「这我倒不想咯,不过为什么时不时有这种事呢。」

    「没办法,优秀的后辈惹人嫌。」

    花音轻松说后就放开了梓的肩。走廊那边B组的人练习完了走过过来,其中当然有雨未华和美音。花音看到熟人摇了摇手。梓看着她的背问:

    「花音有觉得对不起美音吗。」

    「一点都没。」

    花音脸色动都不动地说,她回答也太过干脆,梓一下就被逗笑了。

    「花音就是这样的人嘛。」

    「嗯?有没有迷上我啦。」

    「你傻哦。」

    花音就像调戏耸耸肩的梓一样一脸愉悦。

    ——————

    比赛当天,部员早上到校集中,在熊田老师的指挥下把乐器搬上卡车。做完后,部员再各自自行前往会场。

    「今天麻烦你了。」

    「没事没事别在意。」

    今天是雨未华的父母送梓。雨未华虽然今天不用出场,不过她的父母都很热情,想着A组B组都去看。坐在后排,把乐器箱放在腿上,梓小小呼了一口气。握紧拳头,深呼吸了几次,不然心中的这份躁动就安静不下来。

    「小梓你一年级就A组了真厉害呢。」

    「对啊,小梓很厉害的。」

    「雨未华她啊在家里就一直小梓小梓的,看到她每天社团都这么开心做母亲的也安心了。」

    「中学那时就完全不说社团的事,幸好高中后整个人都变开朗了,做爸爸的我也很开心。」

    「爸妈你们说太多了啦,别在小梓面前爆这些事出来啦。」

    雨未华家人之间气氛很和睦话一直没停。梓没怎么听过别人家是怎么说话的,饶有兴趣地听着。对自己太多话的双亲不好意思,雨未华红着脸。

    「小梓今天加油哦,回去的时候爸爸妈妈会送你回去的。」

    「真是谢谢了,不过这样好吗。」

    「没事没事,爸妈都挺感谢小梓的。」

    雨未华有点兴奋,说话的气息也有点乱。梓依旧对她这种表里如一的感谢感到不好意思。

    ——那个

    雨未华害羞地把手伸进自己的口袋,然后拿出深蓝色的手绳。

    「我做了这个,希望小梓比赛的时候能顺顺利利的。」

    雨未华手上的那个是用不知多少根细细的线交织编起来的,颜色是美丽的大海蓝。雨未华就红着脸把那个递了过来。

    「我没能去A组,所以我希望用这个代替我在小梓身边为小梓加油。」

    听到雨未华这一连串的话梓感觉鼻子有点酸酸的。压住涌上来的感伤,梓笑着小声说道:

    「雨未华谢谢。」

    「我可以帮你戴上吗。」

    「当然可以。」

    拉起衣袖,梓把手伸到了雨未华面前,雨未华以认真的表情把绳子戴到了梓的手上。那小小的手指灵活地把绳子打结,长出来的则用剪刀剪掉,弄完后雨未华满意地说:

    「这样就行了。」

    她的手松开,在自己有点晒黑的手上多了一圈鲜艳的蓝色。把袖子拉回去,那蓝色也便不见了。

    「这样就看不见了,没关系吗。」

    「没事,我就是故意绑紧的,要是从袖子里露出来会影响比赛的。」

    「原来你还想了这么多。」

    「对啊,别看我这样我也很会想的。」

    看到挺胸一副自豪表情的雨未华,梓摸了摸她的头。

    「呼呼呼。」

    用手透过衣服摸了摸,可以清楚感受到绳子的存在,梓为了感觉更加真切,用手抓住了自己的手腕。

    「谢谢啦雨未华。」

    雨未华听后满脸笑容。

    ——————

    下车,其他人已经集合好了。和坐电车来的人一起把乐器搬下来。B组的人在搬打击乐器忙得不可开交,这时梓也拿出自己的乐器做最后的调整。

    「这10分钟里把自己的乐器调好。」

    「是。」

    听到翔子的指示,大家一同行动起来。梓从西装外套拿出调音器,把那个放在耳边确认调音。比赛的时候没有服装的指定要求,立华是用冬装制服上场的。水色的西装加深灰色的裙子,而裙子太短会影响比赛,所以长度控制在膝盖之下。绀色的袜子加黑色的皮鞋是部里指定的。确认胸前的黑色蝴蝶结绑紧后,梓闭紧了嘴巴。刚想发音,就听到了在远处的地方传来尖叫。

    「在吵什么。」

    未来问后,栞苦笑地说:

    「好像是北宇治的男顾问。」

    「就是那个帅哥?」

    「因为顾问很少有年轻人,所以人气一下子就起来了。」

    「哦,不过实力还是我们的熊田老师强。」

    看到嘟嘴的未来,其他三年级的人笑了。

    ——这有什么好争的。

    ——争也可以理解啦。

    ——毕竟熊田老师实力没的说。

    听到前辈这样话,就可以看出她们是多么看得起熊田老师了。

    「......北宇治吗。」

    梓的朋友黄前久美子和高坂丽奈去了那里。说起来,丽奈拒绝了立华的推荐而去了北宇治的谜题至今未解。想起sunfes那时和久美子说的话,就连中学那些不愉快的记忆也一同带了出来。

    不好不好,梓赶紧停了下来。自己不是那种故意挖难堪的记忆虐自己的M。

    「梓你已经调好了?」

    栞看到梓呆呆站着便走了过来。梓赶紧摇头,露出平时的笑脸。

    「对不起还没,我现在开始调。」

    「我帮你拿调音器吧。」

    「没关系吗。」

    「嗯。」

    栞伸出手,把黑色的调音器放到梓的喇叭前,梓一边回想刚才调音器发出的音调,一边发音。因为温度高,音调有点高了。

    ——————

    在场外调好后,大家就进入了彩排室。经过几个可以出声的地方,这是出场前的最后一个可以出声的地方。平常一直穿运动服的熊田老师此时也换上了西装,她胸前的口袋也别着白玫瑰的胸针。她手放在腰上,看了眼大家。

    「调音OK了吗?」

    「是。」

    「准备好了吗?」

    「是。」

    「好,那比赛大声吹就行了。」

    熊田老师愉悦地摇着身体,和平时对学生说话的时候一样,她此时通过此安抚大家紧张的情绪。少见的她涂着睫毛膏的睫毛有力地上下动着,涂了口红的唇也无所畏惧地翘了起来。

    「虽然在这里的不是全体立华的吹奏乐部成员,不过也别忘记B组帮我们搬乐器的人期待着我们最出色的发挥,所以今天大家要鼓起劲,好好干。」

    「是!」

    「嗯,不错。其实老师我也很期待这天,我们绝对要笑着回去。」

    「是!」

    此时,女性的工作人员打开了门。

    「立华的各位,时间到了。」

    大家一下子绑紧了表情,紧张的空气也灌满了整个房间。工作人员把门推开后,空调的冷气吹了进来。大家就列队离开了彩排室。会场和彩排室以一条光线微暗的走廊连接,走廊很窄,梓边费心不让长号撞上墙壁边努力跟上未来的步伐。到达舞台幕后,可以听到在暗幕的一端其他学校在演奏,为了抑制兴奋梓吸了一口气。把视线落到拿着乐器的手上,可以看到西装中露出来的白色衬衫,那之下就是雨未华给自己紧紧系上的手绳。

    这已经是梓第四次出场了。小学的时候也有演奏会,不过那时候并不是大赛。中学入学时买的长号已经陪自己挺久的了。那金色的乐器,不管何时都这样静静地呆在梓的身边。从第一次看到立华表演到现在到底过了多久呢。自己现在就在圆着那小时候的梦想,也证明着自己的选择是正确的。

    「——接下来是32号,立华高校的演奏。」

    灯灭了,通知也响了起来。未来小小声说了句「上场了」,梓无言点了点头。

    坐在舞台上可以清楚地看到观众席。众多到场的观众都直直看着舞台,而其中也可以看到已经吹奏完的其他学校学生的身影。梓不想输给她们,不其实说真的,梓不想输给在场的所有人

    「演奏曲,飞利浦斯巴克作曲,《宇宙的音乐》,指挥,熊田祥江。」

    灯光一下子照在了舞台上,强烈的灯光充满了视野,听到旁边的未来吸气的声音。梓张开腿,让脚牢牢踏在地面上。熊田老师行了一礼后掌声响起,她站上指挥台,那眼神,那手,大家都紧紧盯着。没有紧张,只有兴奋的热量流遍全身。老师嘴角上翘,那手指小小动了起来,大家一同架起乐器,吸气。透过吹嘴,可以听见呼吸的声音。

    随着老师的手挥下,大号一同发出的低鸣沉重的低音。在此之上铜管和木管也搭了上去,糅合成完美的和音。一开始还暗含在各种声音之下的定音鼓此时也愈发明显,圆号也在此时发出开朗的旋律。在铜管奏出的乐声间,长笛和竖笛的声音穿插其中并从中穿出。

    低音以一定的旋律律动着,竖笛那闪闪发亮的音符在其上跳跃。小号旋律在其上奔跑,木管在其后伴奏。音量越来越高,在上低音号和大管流出音符的同时长号也跟了上去。指挥棒大幅度上下挥动,音量也随而越发膨大。然后是唐突的寂静。在微动的世界里,双簧管轻轻奏出缓慢的旋律。铜管鸣响而后潜伏,木管代替而上落下音的团块。而后冲破这的是长号那突鸣的撕哄。大号的低喃也跟了上来,接着是圆号的同音。沿着定音鼓的节奏,梓吹响长号,那是轰鸣的骤变调音,在滑动拉伸管的下一刻,高音的音符就在面前等着自己。往腹部用力,注意不要把嘴唇绷太紧,梓用力吹出一口气,喇叭发出的音与自己预料中一样丝毫不乱。

    乐曲进入后半段,刚才还冲冲向前的调子急速下降转成了慢节奏。从强音急速收弱,像是在萨克斯和长笛柔和的音色中游泳一样,未来长号的solo响起。梓放下乐器,盯着谱面。奏响这四小节优美旋律的人不是梓而是未来。这个事实,直到现在依旧扰乱梓的心。未来的音色一直都是那么澄澈高昂,那一个个的音节每个都把观众的视线吸引过来,这就是立华的第一名。

    那甘甜的乐音融在了现场富含热量的空气中,那安稳的旋律即将迎来结束,低音响起,毫不留情地打破了这份宁静。每当触动低音提琴的琴弦,会场的空气也随之一颤。

    solo完毕后,未来嘴离开了吹嘴,在旁边的梓听到了呼气露出来的声音。她有那么一刻看向了自己,那是叫自己架起乐器的信号。梓轻轻咬唇,与她的动作一同把长号架在了肩上。

    盛大的铜拨之后,是木管的连续环节。曲子的气势回来了,小号和长号的旋律也不断往前冲,低音也跟在后面追了上来。音量渐渐变大,最终达到极限,进行曲那轻快的音色盈满了现场,跟着熊田老师把指挥棒挥上的手势,最后一音飞了出来。

    会场顿时无声,课题曲结束,大家一同翻过乐谱,打击乐的人移动到别的乐器上,看到翔子gu地吞了口气,自由曲《宇宙的音乐》是以她圆号的solo开始的。

    完全的寂静中夹带了低音鼓的那微微的响动。在风的低吟中唯有圆号的声音。那孤独的声音还有风钟的陪伴。翔子富有感情的乐声充满了现场。台下是一片寂静。

    梓把弱音器(ミュート)插进了喇叭中,接在圆号solo之后,各个乐器的音乐叠了上去。突然膨胀,爆炸。木管急促的快板,大家的手指都在乐器上飞快舞动,不断按下乐器的按键。这时小号和长号也赶紧把消音器从喇叭中拿出来,依照着乐谱跟了上去。定音鼓的声音震动着空气,然后铜拨那迫人的声音加了上去。纠缠交合的旋律,这需要高超的技术,乐谱也是高难度的谱子。唯有把一个个音都完美配合好,才能奏出这一首曲子。

    圆号的骤变调如同波纹向四周辐射,接着那无法抓住规律的旋律在波纹上跳动。紧接着长号的声音将之全部清空。这首曲子信息量异常大,光听一次是无法全部把握其中内容的。只有各个声部完美配合,那乍看毫无秩序的声音才能和而为一变成音乐。

    中低音拉出壮大的旋律,木管拍手相应。低沉的音色与轻快的旋律交合,声音的规模也渐渐扩大,不久那激烈的声音便爆发出来。每当拉动拉伸管,现场的空气宛如海浪般随着自己的动作上涨消退。接着又是寂静,竖笛甜美的调子充满了其中,之后大管紧随而上,慢慢的,一点点的,各种声音附上了主体的旋律,合成雄大的和调。那光芒十足的调子不久就消失了,然后是钟琴(グロッケン)的乐粒。大家同呼吸,长号发出一个又一个音。神秘的旋律慢慢转变,不久后进入了优雅的壮大乐章。指挥棒犹如带着一团火,兴奋也反应在了音乐的高涨上。大家一同喇叭齐鸣,临近终盘的时候,大家的力气都差不多了。那种气竭的感觉让梓蹙起了眉,还有一点点,还有一点点就结束了。

    气氛一转,回到了冒头那不安稳的曲调。初耳一听像无秩序的旋律木管毫不留情一个个加了上去。这时是长号发挥重要作用的时刻。梓挺胸,大吸一口气。长号所有音程的声音一同堆叠,把旋律拉着跟自己走。这是梓最喜欢的,感觉最帅的部分。

    低音鼓鼓动,在此乐曲进入了最终部分。面对接连出现的连符,鞭乐器(スラップスティック)仿佛要撕碎连符接连发出破裂音。长号和圆号也不断发出骤变调音。高难度的旋律就不断在会场内回荡,不久,终于迎来了结束。看准指挥棒停止的那一刻,梓也闭上了气门。(这一段真是翻得兴奋,一边听交响一边翻,感觉停不下来。可惜有一些专业名词经常打断,不然真是畅快淋漓)

    熊田老师放下指挥棒,看到这,观众发出巨大的欢呼。她有皱纹的手向这边伸了出来。大家放下乐器,然后起立。老师行了一礼,拍手的声音再次热烈响起。为了调整混乱的呼吸,梓不断深吸着气。听到旁边的未来短短吐了口气,灯光暗了下来,梓她们为了下个上场的学校让出地方。把乐谱夹在腋下,梓想象着暗幕的那一边。下一个就是北宇治了,可惜梓没有机会欣赏她们的演奏。

    「加油久美子。」

    无声地,梓清楚地说道。结束了,在有这个实感的瞬间,胸口顿时松了不少。正因为这种清爽的感觉,才能一直吹停不下来。能给自己带来胜过结束这一刻高昂感的事情,梓至今一次都没经历过。

    ————————

    「立华高校的集合,排成两列。」

    「是。」

    吹奏大赛的结果张贴公布的。这和中学那时没有一丝变化,所以梓有种既视感。把裙子弄好,梓紧紧握住了自己的手腕。摸到雨未华给自己的手绳,感觉冷静了下来。翔子确认着人数。每个学校都各自列队,在稍远的地方,看到了绀色的水手服,那边大概就是北宇治吧。

    「果然顾问是个大帅哥。」

    「这倒否认不了。」

    前面小号的前辈说着悄悄话。大概就是传闻中的那个吧。sun fes的时候梓瞥了眼,是帅没错,可没到要叫出来的程度。未来像是把想着事情的梓的注意力拿回来,用她的手指捅了捅梓的肩,她笑眯眯地看着这边。

    「好紧张啊。」

    「我也紧张,所以不怎么喜欢这种等待的时间。」

    「我懂我懂。既然结果都定了,赶快公布不就好了,还卖什么关子呢,光吊人胃口。我的心好痛。」

    未来脱下了西装外套,作出用手抓住自己胸口的动作。苗条的她没有一点赘肉。把不必要的东西全部去掉,大概就是她这样的吧

    「啊,来了。」

    在一边,看到抱着巨大纸张的男工作人员走了过来。听到梓一说,未来要咬住了唇。梓抓住自己手腕,妄图拭去不安。心跳个不停,把口中积聚的口水吞下,发出了大大的吞咽声。工作人员慢慢把纸张开。自己学校是32号,而自己就找着自己学校的号码,很快就找到了

    ——32号 立华高校 金奖 京都府代表

    穿透耳膜的是自己的欢呼。

    「是关西啊!梓我们去关西了!」

    旁边的未来兴奋得又跳又蹦,一把抱住了梓。梓也不敢把手放到前辈背后,之后不知所措地上下挥舞。未来这时也加大了抱住梓的力气。

    「太好了!」

    「我也好高兴。」

    「嗯嗯!超高兴的!今天超赞的!」

    未来的手就搓着梓的头,绑得整整齐齐的马尾被弄乱了,可梓也没空管那个。透过白色的衬衣,可以听到未来那咚咚咚跳着的心跳。梓松了一口气,原来前辈也同样不安

    「是,注意一下!」

    翔子拍了拍手。大家听后马上安静下来。翔子以严肃的表情看着大家,看到她那有力的眼神,梓有点害怕。看到大家都看过来后,翔子低头,大大吸了一口气,接着咻声把头抬起来,她的嘴愉悦弯曲着

    「我们去关西了!我们做到了!」

    听到她的话,大家再次欢呼。周围B组的人和家长也拍起了手。

    「开心归开心,明天开始就是舞奏的正式训练了,去B组的人比赛结束后就有地狱的舞奏合宿等着你们。我们真正的战斗现在才开始,我们得向全国证明我们立华是NO.1!」

    大家听后都精神鼓舞地大声回应:

    「是!」

    这带有热量的空气让梓的喉咙动了一下。京都吹奏大会结束了,不过立华的重点不是这里。立华水色恶魔之名的,响遍全国的,自己的身影闪闪发亮的那个地方,是——

    「舞奏大赛我们绝对要赢!」

    旁边的未来静静说。梓用力点了点头。正片,终于来了。

    ——立华高校的,进军全国的一战。
    \setcounter{secnumdepth}{-2}
    \section{终章}
    「还真是厉害,虽然不怎么看吹奏,不过现场听魄力果然完全不同。」

    「就是,因为平时也没去过演奏会,如果不是雨未华还真没机会来这种地方。」

    「就是就是,幸好雨未华进吹奏乐部了,开心的事也多了不少。」

    车里雨未华的双亲兴奋说着刚才的演奏会。梓的乐器箱放在腿上,眼睛看着窗外。外面已经完全黑了,可以看到三日月。高速公路对面混杂不堪,那长长的车龙不见一丝移动。刹车的红灯到处闪烁。

    「能去关西太好了。」

    旁边的雨未华开心地说。梓面对她率直的祝福,不好意思挠了挠头。

    「我还没想到坐奏竟然能去关西,不过练习出了成果果然很高兴。」

    「吹得很好啊,大家都说今年立华吹得不错。」

    「这样就好。因为吹的时候就用尽全力了根本没有余力顾及自己吹得怎么样。」

    「毕竟曲子很难嘛,我看到那乐谱头就大起来了。」

    看到雨未华苦涩的样子,梓笑了。和雨未华聊天的时候心情会不知不觉好起来,是因为和她聊天不用担心太多。不管是优秀的前辈还是关系好的朋友对梓都或多或少怀有嫉妒。而那平常收起獠牙的敌意,时常会有那么一刻露出坚唇利齿。和雨未华在一起的时候则不用担心这个,因为新手的她不会把梓当成是对手。她说出的话,都是表里如一的真心话,所以梓喜欢雨未华,对她不用提心吊胆

    「明天就是舞奏的训练了吧,如果能和小梓一起出场就好了。」

    「我也是。不过长号的人挺多的,雨未华你进正式队员挺难的,而且志保也瞄着正式队员去。」

    舞奏也有人数限制,上限是80,所以100多个人肯定有些是出不了场的。雨未华紧紧抓住安全带,重重地开口道:

    「那个啊小梓。」

    「嗯?」

    「我有话得向小梓说。」

    梓有股不好的预感。导航传出了指示音。雨未华的父母没把注意力放在这边,而是指着对面车道在说什么。导航出现了变线的提示。前方有两条路,一条是往前的宽敞直道,一条是出口的细路。导航用红色的箭头指向出口,仿佛提醒避开前面塞车。

    「啊,塞车了。」

    雨未华母亲困扰地说。周围的车减速,最终停了下来。雨未华父亲叹了口气说这下得等够久的了。

    雨未华伸手,把看向窗外的梓的注意力拉回自己身上。她的手抓住了梓的衣服。

    「我一直都在思考自己这样下去究竟好不好。」

    听到这,梓不禁皱眉。

    「之前我不是说了吗,雨未华这样下去就行了。」

    刚说,就响起了ki——这样的金属摩擦音,听到这样的声音梓更是绷起了脸。耳朵如同套了层膜,左耳听得不是很清。自己说话的声音就在耳里回荡,光这一点就让人不愉快。不过下定决心的雨未华没注意到自己的异样,像挤出声音一样,雨未华吞吞吐吐地说:

    「因为小梓很温柔才会对我那样说吧。不过我讨厌给别人添麻烦的自己,所以我决定了。」

    「决定?决定什么?」

    因为耳鸣自己什么都听不见,吵死了,就不能消停一会吗。如果此时能堵住耳朵那该多么幸福,可梓做不了。自己才不会那样撒娇。所以梓装出一副冷静的样子直直看着雨未华,用没被干扰的右耳,努力不放过雨未华的任何一个字。

    「我要去做旗手。」

    梓吞了一口气,一股冲击贯穿了全身,直接把梓思考的时间给抽空了。雨未华拿起梓的手,手指和手指交合,紧紧地握住了自己。越过皮肤感受到了她的体温,她的手仿佛燃烧一样滚滚发烫。

    「我会试着一个人努力的。」

    雨未华为了不让自己担心,对自己笑了笑。那纯真的笑容直直抓住了梓的心脏。雨未华用她那柔软的嘴唇,残酷地把现实直直推到了梓的面前。

    「所以,就算小梓不在我身边也没问题了。」
\end{document}